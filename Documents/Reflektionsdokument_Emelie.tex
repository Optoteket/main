\documentclass{article}
\pagestyle{headings}

% some of the packages used are
%\usepackage[utf8]{inputenc}
\usepackage{latexsym}
\usepackage{eepic}
\usepackage{makeidx}
\usepackage[dvips]{graphicx}
\usepackage[utf8]{inputenc}  %för ä ö å ?
\usepackage[english]{babel}
%\usepackage{times}
\usepackage{amssymb}
\usepackage{fancybox} 
\usepackage{textcomp}
\usepackage{float}
\usepackage{xcolor,colortbl}
\usepackage{tabularx}
\usepackage{enumitem}

\newcommand{\mc}[2]{\multicolumn{#1}{c}{#2}}
\definecolor{Gray}{gray}{0.85}
\definecolor{Orange}{rgb}{1,0.88,0}
\definecolor{LightCyan}{rgb}{0.7,0.9,0.9}
\definecolor{bluegray}{rgb}{0.4, 0.6, 0.8}
\definecolor{buff}{rgb}{0.94, 0.86, 0.51}
\definecolor{corn}{rgb}{0.98, 0.93, 0.36}
\definecolor{coralred}{rgb}{1.0, 0.25, 0.25}
\definecolor{coral}{rgb}{1.0, 0.5, 0.31}
\definecolor{applegreen}{rgb}{0.55, 0.71, 0.0}
\newcolumntype{a}{>{\columncolor{Gray}}c}
\newcolumntype{b}{>{\columncolor{white}}c}
\newcolumntype{g}{>{\columncolor{corn}}c}
\newcommand{\colcell}{\cellcolor{bluegray}}
\newcommand{\colcelltwo}{\cellcolor{coralred}}
\newcommand{\colcellthree}{\cellcolor{applegreen}}

% the fancy header/footer
% consult http://research.cmis.csiro.au/gjw/tex/docs/fancyhdr.pdf
% or some other fancy header documenatation for more info
\usepackage{fancyhdr}
\pagestyle{fancy}




\usepackage{blindtext}
\usepackage[utf8]{inputenc}
 
\title{Reflektionsdokument}
\author{Studerande: Emelie Karlsson \\ Personnummmer: 910429-1480}
\date{\today}
\begin{document}
 
\maketitle

 
\section*{Reflektion över hur examensarbetet relaterar till de mål som finns för programmet}

Exjobbet, som har utförts vid Matematiska institutionen på LiU inom ämnesområdet optimeringslära, har bidragit till en fördjupad matematisk förståelse, speciellt inom området optimering. Processen med att få ett komplext problem och sedan lösa det från början till slut har gett gedigna problemlösningsförmågor och en insikt i det arbetssätt som tillämpas både inom industrin och i den akademiska världen. Exjobbet har dessutom varit ett sätt att testa de kunskaper som erhållits under de tidigare studiåren och se hur de håller i praktiken. 

Eftersom exjobbet utförts vid en extern institution så har det bidragit till utveckling av kommunikativa förmågor, likväl som de tekniska. Det har krävts stor lyhördhet, samarbetsförmåga samt självständighet för att få samarbetet att fungera. Denna typ av extern kontakt har annars fattats i utbildningen.

Jag tror att exjobbet har bidragit till att utveckla även affärsmässiga färdigheter, såsom hur man planerar ett arbete som sträcker sig över en längre period, hur man (i vårat fall var vi två exjobbare) fördelar arbete samt hur man ser till att upprätthålla enhetligheten.

\section*{Reflektion över eget arbete}

\subsection*{Planering}
\subsection*{Genomförande och rapportskrivning}

\section*{Reflektion över det ämnesinnehåll, kunskaper, färdigheter och förhållningssätt som var till mest nytta för examensarbetets genomförande}

\end{document}

