\documentclass{article}
\pagestyle{headings}

% some of the packages used are
%\usepackage[utf8]{inputenc}
\usepackage{latexsym}
\usepackage{eepic}
\usepackage{makeidx}
\usepackage[dvips]{graphicx}
\usepackage[utf8]{inputenc}  %för ä ö å ?
\usepackage[english]{babel}
%\usepackage{times}
\usepackage{amssymb}
\usepackage{fancybox} 
\usepackage{textcomp}
\usepackage{float}
\usepackage{xcolor,colortbl}
\usepackage{tabularx}
\usepackage{enumitem}

\newcommand{\mc}[2]{\multicolumn{#1}{c}{#2}}
\definecolor{Gray}{gray}{0.85}
\definecolor{Orange}{rgb}{1,0.88,0}
\definecolor{LightCyan}{rgb}{0.7,0.9,0.9}
\definecolor{bluegray}{rgb}{0.4, 0.6, 0.8}
\definecolor{buff}{rgb}{0.94, 0.86, 0.51}
\definecolor{corn}{rgb}{0.98, 0.93, 0.36}
\definecolor{coralred}{rgb}{1.0, 0.25, 0.25}
\definecolor{coral}{rgb}{1.0, 0.5, 0.31}
\definecolor{applegreen}{rgb}{0.55, 0.71, 0.0}
\newcolumntype{a}{>{\columncolor{Gray}}c}
\newcolumntype{b}{>{\columncolor{white}}c}
\newcolumntype{g}{>{\columncolor{corn}}c}
\newcommand{\colcell}{\cellcolor{bluegray}}
\newcommand{\colcelltwo}{\cellcolor{coralred}}
\newcommand{\colcellthree}{\cellcolor{applegreen}}

% the fancy header/footer
% consult http://research.cmis.csiro.au/gjw/tex/docs/fancyhdr.pdf
% or some other fancy header documenatation for more info
\usepackage{fancyhdr}
\pagestyle{fancy}




\usepackage{blindtext}
\usepackage[utf8]{inputenc}
 
\title{Reflektionsdokument}
\author{Studerande: Emelie Karlsson \\ Personnummmer: 910429-1480}
\date{\today}
\begin{document}
 
\maketitle

 
\section*{Reflektion över hur examensarbetet relaterar till de mål som finns för programmet}

Exjobbet, som har utförts vid Matematiska institutionen på LiU inom ämnesområdet optimeringslära, har bidragit till en fördjupad matematisk förståelse, speciellt inom området optimering. Processen med att få ett komplext problem och sedan lösa det från början till slut har gett gedigna problemlösningsförmågor och en insikt i det arbetssätt som tillämpas både inom industrin och i den akademiska världen.

Eftersom exjobbet utförts vid en extern institution så har det bidragit till utveckling av kommunikativa förmågor, likväl som de tekniska. Det har krävts stor lyhördhet, samarbetsförmåga samt självständighet för att få samarbetet att fungera. 

Jag tror att exjobbet har bidragit till att utveckla även affärsmässiga färdigheter, såsom hur man planerar ett arbete som sträcker sig över en längre period, hur man (i vårat fall var vi två exjobbare) fördelar arbete samt hur man ser till att upprätthålla enhetligheten. Därför tycker jag att exjobbet överstämmer väl med de mål som finns för programmet.

\section*{Reflektion över eget arbete}

\subsection*{Planering}
Planeringen som upprättades i början av arbetet var till stor hjälp. Framförallt hjälpte det oss att, vid det tillfället, få en översikt över arbetsprocessen. Även om inte alla detaljer var tydliga, så hjälpte planeringen oss att få en känsla för när vi borde vara klara med de olika delarna i examensarbetet. Vidare så gav planeringen en insikt om när de olika delarna i rapporten skulle skrivas. Med tiden frångicks planeringen och exjobbet drog över tiden. Då blev planeringen också mindre betydelsefull. Jag tror att vi la ner tillräckligt med tid på planeringen, däremot var exjobbet av sådant att vi själva inte visste den exakta problemformuleringen vid examensarbetets påbörjande. Det försvårade planeringen.

När det gäller arbetsinsatsen som lades ned på arbetet så spenderades mer tid på examensarbetet än poängen motsvarande. Anldeningen till detta var dels att examensarbetet var mer omfattande än vi från början trott och dels att vi valde att implementera arbetet i ett programmeringsspråk vi båda var obekanta vid. Det var därför ett aktivt val att lägga ner mer tid på arbetet än vad som krävdes.

\subsection*{Genomförande och rapportskrivning}

Examensarbetet genomfördes av mig och en till student. Vi valde att i början hjälpas åt att skapa en modell för problemet som vi studerade. Vi avsatte halva exjobbstiden för detta. Den andra halvan avsattes för att implementera varsin lösning till problemet. Detta tror jag var en bra disponering av tiden. Det som blev problematiskt var att den första delen gick väldigt bra, medan den andra delen drog ut på tiden. Det var svårt att förutse från början att det skulle bli så. Den första delen hade vår handledare mer insikt om, eftersom det handlade om att utveckla en matematisk modell som var ett känt område för handledaren. Att däremot sedan implementera modellen med hjälp av C++ var svårare att förutse hur lång tid det skulle ta, eftersom området var mindre känt för handledaren. Det akn ha varit en orsak till förseningen.

Samarbetet mellan mig och min exjobbspartner har överlag fungerat mycket bra. Det har varit till stor hjälp för mig att ha någon att bolla idéer med, då handledaren inte är på plats. Framförallt har det gjort exjobbet roligare, eftersom man är två som är insatta i samma problem. Jag tror också att kvalitén på exjobbet hade varit lägre om jag inte haft någon att diskutera idéer med. Att vi sedan fick en varsin implementation att genomföra gjorde också att arbetet blev mer självständigt. Då kunde vi också hjälpa varandra, eftersom vi är bra på olika saker. Därför var samarbetet viktigt även här.

Gällande rappporten så gick det förvånansvärt bra att skriva den, även om det kändes som att den blev väldigt omfattande. Det var inga större problem med att formulera akademisk text på engelska och vår handledare var nöjd. Jag är också väldigt nöjd med arbetet som vi utfört och tycker att vi har uppnått bra resultat och en intressant rapport.

\section*{Reflektion över det ämnesinnehåll, kunskaper, färdigheter och förhållningssätt som var till mest nytta för examensarbetets genomförande}

Det som varit till mest nytta i examensarbetet har främst varit grundkursen i optimeringslära. Den gav både ett intresse för matematisk optimering och nödvändiga kunskaper för att kunna identifiera det studerade problemet som ett optimeringsproblem. Utöver detta så har alla analyskurser varit till nytta eftersom de gett en grundläggande förståelse för matematik. 

CDIO projektet som genomfördes på hösten var även det väldigt bra eftersom det gav en bra insikt i hur ett större projekt genomförs. Även detta projekt var inom optimeringlära, vilket väckte ett intresse för området. Det var också i det projektet som jag lärde mig metoder för versionshantering (GIT) för att alla i gruppen skulle kunna arbeta på samma projekt samtidigt. Det hade dock varit väldigt nvändbart om detta ingick som ett obligatoriskt moment i CDIO projektet. 

Jag känner mig väl förberedd inför arbetslivet och tycker att utbildningen på LiU har gett gedigna ingenjörskunskaper. Något som jag saknar är en bredd i form av andra, mer mjuka, ämnesområden, exempelvis historia eller filosofi. Dessa skulle kunna vara valbara. (MTS-kurserna motsvarar inte helt detta eftersom de är väldigt teknikfokuserade och i många fall främst handlar om olika miljöaspekter.) Programet skulle också kunna innehålla fler praktiska moment. 

\end{document}

