% Review of previous work in the field

Staff and machine scheduling problems are mathematical optimization problems which have been studied since the 1950s and concern creating feasible and satisfactory schedules for workers or machines performing tasks. One of the most extensive overviews of the area of staff scheduling is provided by \citet{ernst_2004}. They state that, although the complexity of scheduling problems has not increased in recent years, the mathematical models used to solve scheduling problems have become more realistic and refined. Modern staff scheduling problems often concern the distribution of tasks and the creation of worker shifts, also taking softer values into account, such as worker satisfaction and worker fatigue. Due to this modelling refinement as well as the development of more powerful computational methods, it is possible today to solve staff scheduling problems in a more satisfactory way than before.

In this chapter, staff scheduling problems are classified into different subcategories which are related to the thesis work. The relevant areas for our work include Personnel Task Scheduling Problems (PTSP), Shift Minimization Task Scheduling Problems (SMTSP), Tour Scheduling Problems (TSP), and a few variations of these. Also problems featuring soft constraints will be studied in an additional section. Models and solution methods of these problems will be discussed under each headline and a summary will be provided at the end, together with a discussion about the relevance of the studied articles to our thesis work.


\section{Personnel task scheduling problem} \label{PTSP}

In many real life situations production managers will face the Personnel Task Scheduling Problem (PTSP) while scheduling service operations. \citet{krishnamoorthy_2001} write in their article that the PTSP occurs when the rosterer or shift supervisor need to allocate tasks with specified start and end times to available personnel who have the required qualifications. It also occurs in situations where tasks of fixed times shall be assigned to machines. Each machine have a certain amount of maintenance workers assigned to keep it in operation. Decisions will then have to be made both regarding which machine the workers are assigned to, as well as the optimal amount of maintenance workers needed at each machine.

There are numerous variants of the PTSP. Studies of these have been made by \citet{krishnamoorthy_2001}, who give a list of attributes that commonly appear in a PTSP, which are listed in Table \ref{PTSP}. Furthermore, there are traits that always appear in a PTSP; tasks with fixed start and end time are to be distributed to staff members that possess certain skills, allowing them to perform only a subset of the available tasks. Start and end time of their shifts are also predetermined for each day.

One variant, which also is the most simple, is mentioned in \citet{krishnamoorthy_2001} and is called the \textit{Feasibility Problem}, where the aim is to merely find a feasible solution. This requires that each task is allocated to a qualified and available worker. It is also required that a worker cannot be assigned more than one task simultaneously, as well as tasks cannot be pre-empted, meaning that each task has to be completed by one and the same worker.

In Table \ref{tab:PTSP} attributes of PTSP variants are shown. The nomenclature of the attributes T, S, Q and O refer to the \textit{Task type}, \textit{Shift type}, \textit{Qualifications} and \textit{Objective function} respectively. 
\begin{table}[H]
\caption{PTSP variants}
\label{tab:PTSP}
\begin{tabular}{|c|c|l|}
%-------------------------------------------------------------------
\hline
\textbf{Attribute} & \textbf{Type} & \textbf{Explanation} \\ \hline
%-------------------------------------------------------------------
T & F & Fixed contiguous tasks \\
& V & Variable task durations \\
& S & Split (non-contiguous) tasks \\
& C & Changeover times between consecutive tasks \\
\hline 
%-------------------------------------------------------------------
S & F & Fixed, given shift lengths \\
& I & Identical shifts which are effectively of infinite duration \\
& D & Maximum duration without given start or end times \\
& U & Unlimited number of shifts of each type available \\
\hline 
%-------------------------------------------------------------------
Q & I & Identical qualification for all staff (homogeneous workforce) \\
& H & Heterogeneous workforce \\
\hline 
%-------------------------------------------------------------------
O & F & No objective, just find a feasible schedule \\
& A & Minimize assignment cost \\
& T & Worktime costs including overtime \\
& W & Minimize number of workers \\
& U & Minimize unallocated tasks \\
\hline  

%-------------------------------------------------------------------
\end{tabular}
\end{table}

Many of the most basic problems and a few more complex ones can be described with this definition of PTSP attributes. It is, however, not possible to describe all of the numerous types of PTSP using these nomenclatures according to \citet{krishnamoorthy_2001}.

By combining attributes it is possible to obtain more complex variants of the PTSP. An example would be the PTSP[F;F;H;A-T-W] mentioned in \citet{krishnamoorthy_2001}, where multiple objectives are used. This problem has fixed contiguous tasks, fixed shift lengths, heterogeneous workforce and three objective functions (A-T-W), which represent assignment costs, work time with overtime included and requirements to minimize the number of workers, respectively. The objective function for this problem is then a linear combination with parameters used to weigh (prioritize) them against each other.

Given the nomenclature above, our problem would be most related to the PTSP[F;F;H;F]. The difference is that the objective function is not empty. We are looking to maximize the number of qualified stand-ins each day as well as maximize employee satisfaction by meeting their recommendations. We have a fixed number of workers, no costs and no unallocated tasks when a feasible solution is found. Therefore, none of the objective function types given in Table \ref{PTSP} are relevant in our case.

Some variants of the PTSP are given specific names in the literature, which is stated by \citet{krishnamoorthy_2001}. An example is when the shifts and qualifications are identical (S=I, Q=I) and the objective function is to minimize the number of workers that are used (O=W). This variant, PTSP[F;I;I;W], has been published as the Fixed Job Schedule Problem and is described in Section \ref{other}.

\subsection{Applications}
An example where the PTSP can be found is when developing a rostering solution for ground personnel at an airport. This is mentioned in the article by \citet{krishnamoorthy_2001}. This problem can be dealt with by first assigning the workers to days in order to satisfy all the labour constraints, followed by assigning the tasks to the scheduled workers.

In the article mentioned above, three further problems of type PTSP, all related to airplanes, are mentioned. They occur when scheduling for either airport maintenance staff, planes to gates or staff that do not stay in one location, such as airline stewards. Scheduling for airport maintenance staff can lead to either PTSP[F;I;H;U-A] or PTSP[F;I-U;H;W], which are similar problems but given two different names: Operational Fixed Interval Scheduling Problem and Tactical Fixed Interval Scheduling Problem respectively. These are both described further in Section \ref{other}. 

Another application, which has been frequently studied, is classroom assignments and is discussed in \citet{krishnamoorthy_2001}. Based on specifications such as the amount of students in a class or the duration of a class, different classrooms have to be considered. Requirements of equipment, e.g. for a laboratory, may also greatly limit the available classrooms to choose from. A majority of the complications of this problem is due to the fact that lessons can span over multiple periods. 

Worth noting for classroom assignment problems is that there are no start or end times for the shifts, as they represent the rooms. The aim in this problem would be to simply find a feasible assignment of classrooms. Therefore, the nomenclature of the problem would be PTSP[S;I;H;F], with the possibility of adding preferences to the objective function. An example of a preference would be to assign the lessons as close to each other as possible on a day, preventing travel distances for teachers and students.



%Papers of interest:
%"The Personnel Task Scheduling Problem", Mohan Krishnamoorthy, Andreas T. Ernst (2001) - probably the most fundamental article
%
%"Task assignement for maintenance personnel": Roberts and Escudero, 1983a, 1983b
%
%"A stochastic programming model for scheduling maintenance personnel" Duffuaa and Al-Sultan, 1999



\section{Shift minimization personnel task scheduling problem}\label{SMTSP}
% Write about PTSP[F;F;H;W]
A close relative to the PTSP is the Shift Minimization Personnel Task Scheduling Problem (SMPTSP), which is a variant where the aim is to minimize the cost occuring due to the number of personnel that are used. The same common traits for this problem are mentioned in the article \citet{krishnamoorthy_2012} as for the PTSP: Workers with fixed work hours are to be assigned tasks with specified start and end times, for which they are qualified.

In the same article as above the authors "... concentrate mainly on a variant of the PTSP in which the number of personnel (shifts) required is to be minimized". In doing so, it is possible to determine the lowest number of staff and the mix of staff a company needs in order to be able to complete the tasks to be operational. They also presumed that the pool of workers is unlimited for each skill group, which is not the case in our problem due to the limitations of librarians and assistants. 

Furthermore, it is said in \citet{krishnamoorthy_2012} that SMPTSP can be applied when there is a large number of workers available with different qualifications and it must be ensured that the tasks for that day are performed. The PTSP and SMPTSP are therefore useful day-to-day operational management tools and commonly occur in practical instances where tasks are allocated on a daily basis.

It is shown in \citet{kroon_1995} that SMPTSP is a complex problem even if the preemption constraint were to be removed. However, it is stated in \citet{krishnamoorthy_2012} that if the qualifications of the workers were identical it would be an easily solvable problem. The difference in publication year between the articles indicates that this problem has become less challenging in recent years.

During the last decade, a couple of heuristics have been implemented to deal with the SMPTSP. One method introduced by \citet{krishnamoorthy_2012} is a Lagrangean relaxation approach that combines two problem specific heuristics: Volume Algorithm (VA) and Wedelin's Algorithm (WA). These heuristics exploit the special structure of the SMPTSP by relaxing some of the harder constraints into the objective function, thus being a problem specific heuristic. What remains after the relaxation is a problem decomposed into several problems which can be solved independently. One way to solve these decomposed problems is discussed further in \citet{krishnamoorthy_2012}.

Another way to solve the SMPTSP is to use a very large-scale neighbourhood search algorithm, as was presented by \citet{smet_2012}. The main purpose of their implemented hybrid local search is to repeatedly fix and optimize to find neighbouring solutions. Using this method on 137 benchmark instances introduced by \citet{krishnamoorthy_2012}, Smet and Vanden Berghe managed to find 81 optimal solutions compared to Krishnamoorthy et al. who only managed to find 67. Though, both methods found feasible solutions for 135 out of the 137 problem instances. This comparison is presented in \citet{smet_2014}.

\citet{smet_2014} introduced a third and most effective method to solve the benchmark instances related to the SMPTSP. By using a versatile two-phase matheuristic approach, solutions to all 137 benchmark instances could be found for the first time. The procedure used in their implementation is to first generate an initial solution by using a constructive heuristic, followed by improving the solution using an improvement heuristic.


% CONCLUSION: By being able to efficiently deploy the workforce it results in an optimized resource occupation, which in turn reduces or eliminates the need of temporal workers. These temporal workers are otherwise needed in the case when no stand-ins are available to cover when a worker cannot show up for work, which results in extra expenses for the library.

 

%Difference: "The only cost incurred is due to the number of personnel (shifts) that are used."
%
%Papers of interest:
%"Algorithms for large scale Shift Minimisation Personnel Task Scheduling Problems" Krishnamoorthy, Ernst, Baatar (2011)
%
%"The shift minimisation personnel task scheduling problem: A new hybrid approach and computational insights" Smet, Wauters, Mihaylov, Berghe (2014)
%
%"Fast local search and guided local search and their application to British Telecom's workforce scheduling problem" Tsang and Voudouris, 1997 - also with travelling costs, investigates two methods.
%
%"A Triplet-Based Exact Method for the Shift Minimisation Personnel Task Scheduling Problem" Baatar et al., 2015


\section{Tour scheduling problem}\label{TSP}

The Tour Scheduling Problem (TSP) is described by \citet{loucks_1991} as a combination of shift scheduling and days-off scheduling. Shift scheduling refers to creating sets of contiguous hours during which a worker is assigned for work. The need for days off scheduling typically occurs when the time horizon for scheduling is weekly or more and when weekend staff is needed. Using the notation in Section \ref{PTSP}, this would be be classified as a variant of the PTSP[F;D;I;W] or PTSP[F;D;H;W], depending on if the workforce is homogeneous or heterogeneous.

According to \citet{loucks_1991}, the vast majority of all tour scheduling problems up to 1991 involve a homogeneous workforce, that is, any worker can perform any assigned task. One such early study of a tour scheduling problem is provided by \citet{thompson_1988}. The problem studied concerns only homogeneous workforces and the task assignment part is not considered.

In the article by \citet{loucks_1991}, the authors study a tour scheduling problem with a heterogeneous workforce. The problem both involves tour scheduling and task assignment, where the latter part is most interesting to us. The problem is studied in the context of fast food restaurants, where certain personnel is qualified only for certain stations in the restaurant. In such industries, the demand of staff differs between weekdays and times of the day. Two worker attributes are considered, their availability for work and their qualifications to perform different tasks. The problem concerns finding shifts for all workers which are to be assigned a length between a minimum and maximum number of hours per day.

The main problem studied in the article involves creating a one-week schedule for 40 workers in a fast food restaurant, available for eight different tasks with a seven-day and 128-hour workweek. Several synthetic problems are studied in the article, all, however, with a minimum shift length of three hours, a maximum shift length of eight hours and a maximum of five work days.

A similar problem to the one descibed by Loucks and Jacobs is studied by \citet{choi_hwang_park_2009}. They focus on a particular fast food restaurant in Seoul, which is made a representative of fast food chains in general. In this study, only two types of workers are available, fulltime and part time workers, with no reference to difference in skill. The different shifts are already given by the restaurant managers and the task is to combine them into a tour. The task assignment aspect is lacking in this article.


In both articles the main objective is to minimize both overstaffing and understaffing, which will both have economic consequences for the fast food chain. This is done by reducing or increasing the workforce. For a problem with a fixed workforce, such as ours, this objective function is not relevant. In the example studied by Loucks and Jacobs there is also a goal to meet staff demand on total working hours. This is modeled as a secondary goal which is similar to ours and somehow models a "soft" value, which is of interest to us.

A more recent TSP concern monthly tour scheduling, as opposed to most literature which concerns only weekly scheduling. Such a study was done by \citet{rong_2010}. The main advantage of monthly scheduling over shorter time periods, as stated in the article, is the possibility to plan a schedule with respect to fairness and balance over a longer period of time. The problem concerns workers with different skills, where each worker also can possess multiple skills. This is referred to as a mixed-skill problem. Thus, the problem is similar to our problem, where mixed-skill is also present. In the study, workers have individual weekend-off requirements. The problem does not involve task assignment, which makes it less relevant for us.

The solution methods used to solve the TSP differ greatly between the articles studied. In the older articles, such as \citet{thompson_1988} and \citet{loucks_1991}, custom made algorithms very similar to the methods used in manual scheduling are proposed to solve the problem. These solution methods involve classifying staff and distributing them according to some rule (for example, the staff with the most scarce skill is assigned first). General commercial solvers are not proposed, due to their lack of efficiency during these times. 

In \citet{hojati_2011}, the same model and data is used as in \citet{loucks_1991}. Also this article states that commercial solvers are insufficient for solving the problem. Instead two methods are proposed, one which decomposes the problem into two problems solvable by commercial solvers and one customized heuristic method based on a Lagrangian relaxation method. This method solves the problem with more satisfactory results than in \citet{loucks_1991}, as explained in the article. In \citet{choi_hwang_park_2009}, a pure integer programming method is used.


\section{Other similar problems}\label{other}
In this section a couple of other problems, similar to our own, will be described. The focus will not be on the variety of methods used to solve these problem; instead, the focus is to give clarity to how closely related many of these problems are.
\subsection{Fixed job schedule problem}
Variations of the task assignment problem relevant to our problem include for example the Fixed Job Schedule Problem (FJSP). The FJSP has been studied since the 1970s in the context of task assignment in processors. According to \citet{krishnamoorthy_2012}, the problem concerns the distribution of tasks with fixed starting and ending times over a workforce with identical skills, such as processing units. Such problems have been solved by \citet{gertsbakh_1977} and \citet{fischetti_1992}.

In the article by \citet{gertsbakh_1977} a situation is studied where \textit{n} jobs need to be scheduled over an unlimited number of processors. The objective function of such a problem becomes to minimize the number of machines needed to perform all tasks. Fischetti solves a similar problem, but adds time constraints, saying that no processor is allowed to work for more than a fixed time \textit{T} during a day as well as a constraint forcing tasks to spread out with time gaps \textit{S} over a processor.

%A lemma in \citet{kroon_1993} states that \textit{"A feasible non-preemptive schedule for all jobs exists if and only if the maximum job overlap is less than or equal to the number of available machines."}. It is a necessary and sufficient constraint to know if a feasible solution to the FJSP exists. If more machines are required than available the problem would become a Maximum Fixed Job Schedule Problem, which instead tries to prioritize the tasks that shall be performed for the day. The Maximum Fixed Job Schedule Problem is not as relevant since in the present problem we have more workers available than tasks and is therefore excluded in the report. 
 	
%PTSP[F;I;I;W] - Gertsbakh and Stern as well as Fischetti et al. or PTSP[F;I;I;U-A] - Arkin and Silverberg
\subsection{Tactical fixed interval scheduling problem} \label{TFISP}
The Tactical Fixed Interval Scheduling Problem (TFISP) is a problem very closely related to the SMPTSP, with the sole difference being that the TFISP concerns workers which always are available, such as industrial machines or processors. The problem is studied in \citet{kroon_1995}. A typical TFISP can be expressed using the nomenclature in Table \ref{PTSP} and written as PTSP[F;I-U;H;W].
%TFISP = PTSP[F;F;H;W]PTSP[F;U;H;W]PTSP[F;I-U;H;W]

Opposed to the FJSP, the TFISP deals with a heterogeneous workforce. Two different contexts are studied by \citet{kroon_1995}. One of them concerns the handling of arriving aircraft passengers at an airport. Two modes of transport from the airplane to the airport are investigated, directly by gate or by bus. The two transportation modes thus correspond to two processing units, which can only handle a number of jobs at the same time.

\subsection{Operational fixed interval scheduling problem}
The Operational Fixed Interval Scheduling Problem (OFISP) is a close relative to the TFISP. Both OFISP and TFISP are restricted by the following: Each machine (worker) cannot handle more than one job at a time, it can only handle a subset of the jobs and preemption of jobs is not allowed. The difference between OFISP and TFISP occurs in the objective function, as described in \citet{kroon_1995}. TFISP tries to minimize the number of workers, while OFISP tries to minimize the operational costs and the number of unallocated tasks using priority indices. In the present nomenclature this would give rise to the problem PTSP[F;I;H;U-A].

Given the problem definition above, working shifts are to be created for the workers and tasks have to be allocated on a day-to-day basis. The OFISP, studied by \citet{kroon_1995}, can therefore be seen both as a job scheduling problem and a task assignment problem.



%Problem defined in: "Algorithms for large scale Shift Minimisation Personnel Task Scheduling Problems" M. Krishnamoorthy
%http://www.sciencedirect.com/science/article/pii/S0377221711010435
%
%Problem: "A metaheuristic for the fixed job scheduling problem under spread time constraints" André Rossi, http://www.sciencedirect.com/science/article/pii/S0305054809002251 (Fixed job)


\subsection{Stochastic job problems} \label{STOCH}
What differs mostly between the problem types described above and the problem studied in this thesis work, is the objective function. The main objective function is often to minimize staff for a fixed number of jobs, not taking stand-in assignment into account.

An area where the need for stand-in personnel appears is in the  maintenance industry, where some jobs can be foreseen and other (emergency) jobs are of a stochastic nature, that is, there is a probability that such jobs will occur a certain hour. The problem combining both foreseen and stochastic maintenance worker scheduling was studied by \citet{duffuaa_1999}, as a continuation of the work by \citet{roberts_1983}. 

In the former article, a fixed heterogeneous workforce consisting of electricians, plumbers and mechanics is studied. The shifts of the personnel are predetermined by their given work times and thus the problem becomes a pure task assignment problem. The goal is to maximize the number of planned and unplanned jobs performed by the workers, by taking into account the probability of unplanned work to occur. Thus, certain workers will be left at the station as stand-ins in the case an unplanned job arises. The commercial solver LINDO was used to solve the problem.


\section{Modelling soft constraints} \label{MSC}
For most scheduling problems, the main objective is to minimize worker-related costs by reducing the number of workers needed to perform a task, or by reducing the working time for part-time employees. Recently, however, many studies have started to focus more on softer goals such as worker satisfaction as an objective function. Such goals are usually considered when scheduling is done manually, but have been forgotten or set aside in mathematical modelling.

 In an article by \citet{akbari_2013} a scheduling problem for part-time workers with different preferences, seniority level and productivity is investigated. In this article, these aspects are reflected in the objective function and weighted against each other. A similar problem was also studied by \citet{mohan_2008}, but for a workforce of only part-time workers. %Write more here!

Other factors which may affect worker satisfaction, and in the long run efficiency and presence at work are fatigue, fairness and boredom. These are discussed by  \citet{eiselt_2008}. Repetitiveness of a job as well as the level of challenge can cause boredom of workers. The variance in the schedule is increased by \citet{eiselt_2008} through providing an upper bound of how many tasks can be performed in a given time span. The article suggests a sort of measurement of the distance between the task requirements and the worker abilities used. This will then be minimized in the objective function.

Another modelling method which is relevant specifically for scheduling problems featuring soft constraints is fuzzy goal programming. The method is discussed by \citet{shahnazari_2013}, who model soft constraints as "fuzzy goals". These goals can become contradictory, for example could a preference of high seniority level workers come in conflict with a preference in working hours by an employee. The article uses fuzzy set theory, and a solution approach involving Li's two-phase method (Li, 1990). Soft constraints are modeled as trapezoid functions and an optimal solution with the best average value of all functions is found.

The solution methods proposed by \citet{akbari_2013} to solve a scheduling problem with soft constraints are two matheuristics: Simulated Annealing (SA) and Variable Neighbourhood Search (VNS). According to \citet{akbari_2013}, SA has been studied as a solution method for the scheduling problem since the 1990s and many studies have shown that it is capable of providing near-optimal solutions in a short time compared to optimizing integer programming models, for a variety of problems. An exponential cooling time was used for the algorithm and it was concluded that it was faster than the commercial solver LINGO in finding a solution.

VNS is the other proposed method by \citet{akbari_2013}. A big difference between this and other methods is that VNS requires very little parameter tuning, while often providing good solutions.
Larger movement sizes are used at higher temperatures, and smaller at lower temperatures. The method uses both a random and a systematic phase. In the random phase, worker schedules are regenerated randomly and better solutions are saved. In the systematic phase, two shifts are swapped. In the article, it was concluded that VNS could solve the problem faster than the implemented SA heuristic.

In the article by \citet{eiselt_2008}, a commercial solver was used. This was also the case in \citet{mohan_2008}, as the article compares these commercial solver results with the results obtained from a branch-and-cut algorithm. Also \cite{shahnazari_2013} finds results using a commercial solver.

\section{Summary}

In order to get an overview over the problems discussed in this chapter it is a good idea to look at general historical trends among the problems. One clear trend is the shift from problems concerning homogeneous to heterogeneous workforces. As stated by \citet{krishnamoorthy_2012}, the problem of heterogeneous workforces was trivial at the publishing year, while in \citet{loucks_1991}, it is introduced as a rather new and challenging concept. Furthermore, the articles concerning the different PTSPs are mainly from the 1980s and 1990s and were challenging in their structure during these times. Some of them, such as FJSP, are related to the scheduling in processors, which was a hot topic at the start of the computer era. All the articles about modelling of soft constraints are presented after the year 2000, probably as a result of better computational powers.

The various solution methods found in the studied articles include commercial solvers, heuristics and matheuristics. Heuristics which have been studied in the articles include SA, VNS and Lagrangian relaxation, with and without VA. Also some local search-based heuristics have been used. Commercial solvers are used increasingly in newer articles such as \citet{hojati_2011} and \citet{mohan_2008}, probably due to the improvement in performance in such solvers. Similarly, matheuristics are also discussed mostly in more recent articles, such as \citet{akbari_2013}. 

\section{Relevance to our problem}

As described in Section \ref{PTSP}, many different types of personnel tasks scheduling problems exist. Only a few of them have been discussed in this chapter, of which some are closer related to our problem than others. In order to get a better understanding for how they are related to our problem, a few connections to them will be presented in this section.

Using Table \ref{tab:PTSP}, the closest classification of the problem studied in this thesis work would be the PTPS[F;F;H;F]. This describes a Personnel Task Scheduling Problem with fixed tasks, fixed shift lengths, a heterogeneous work force and an empty objective function. As stated in Section \ref{PTSP}, the main difference lies in the objective function, as ours is not empty. Thus, many problem types discussed in this chapter concern problems which are similar to ours but with a different objective function. This includes the shift minimization personnel task scheduling problem, the tour scheduling problem, the fixed job schedule problem, the tactical fixed interval scheduling problem and the operational fixed interval scheduling problem. The first two are relevant to our problem as they concern task assignment and, in the case of tour scheduling problem, days off scheduling. However, both involve shift scheduling while we have fixed shifts. The other three problems concern only task assignment, but are otherwise further from our problem since the tasks types are different from ours.

As stated in Section \ref{PTSP}, our problem has an objective function in which the minimal number of stand-in personnel is to be maximized. Such problems often arise in the maintenance job scheduling problem, which is discussed in Section \ref{STOCH}. As a secondary objective function, we are also interested in making the schedule varied and with a repeating pattern over a cycle of five weeks. Modelling of such constraints is discussed in Section \ref{MSC}.


%Kan börjas med syftesformulering. 
%Referera tillbaka i texten, samt lägg till nya reflektioner.









\iffalse


\section{Solution Methods}
Integrate into previous parts!


In many real life situations, the scheduling method used to create worker schedules is a simple matching algoritm between two can do what and when. The process is most often left in the hands of experienced and knowledgable schedulers, who know the capacity of the workforce and how to maximize productivity by meeting task demands as well as employee demands and individual personality traits. This is referred to as the "art" of scheduling \citet{roberts_1983}. However, when personnel forces grow large and there are regulations, task skill requirements or several personnel preferences to take into account, the problem becomes too large to solve manually in a satisfactory manner.

The first computational methods used for solving scheduling problems were in many cases simple heuristics resembling the scheduling process as performed in a manual way. One example of this is the heuristic presented by Loucks and Jacobs which assigns workers to tasks, following certain rules, until all tasks are assigned \citet{loucks_1991}. An overview of solution methods is given by Ernst et al, where almost 30 different methods are presented  and it is not uncommon that special purpose algorithms are used to suite a specific problem \citet{ernst_2004}. Some of the more interesting solution methods with respect to the probelm studied in this thesis are discussed in this section. These include solving with commercial solvers, matheuristic methods such as simulated annealing and variable neighbourhood search, pure heuristic methods, goal programming and fuzzy goal programming.  


%column generation\\
%Lagrangean relaxation\\
%Network models\\
%General integer programming\\
%Heusristics (\citet{eiselt_2008})\\

\subsection{Commersial software}

Commersially available scheduling programs.

\subsection{Mathematical Programming}

Formulating a mathematical model. Objective function and constraints. Solving using commersial solver such as CPLEX or Guroby.

%Using a commercial solver such as CPLEX or Guroby seldom needs the computational insight as does using a customized algorithm. 

%\subsection{TSP with inhom workforce}


%\subsection{Solving the SMPTSP}
\ %citet{smet_2014}
%Hybrid approach, works better for homogeneous workforces. Matheuristical methods presented, can be considered state of the art. Is based on the work by Krishmooty et al \citet{krishnamoorthy_2012}.

%\subsection{Integer Programming}

Stochastic and non-stochastic

\subsection{Simulated Annealing}

Simulated Annealing (SA) has been studied as a solution method to the scheduling problem by researchers such as Brusco and Jacobs in the early 1990s. The method is a metaheuristic method which has the advantage over local search methods that it does not easily get stuck in local optima. The method is a random optimization method designed to find a global optimum solution. The method allows bad moves according to a function ... . 
According to Akbari, Simulated Anneahling 

\subsection{Variable Neighbourhood Search}

Also avoids local optima. 

"A variable neighborhood search based matheuristic for nurse rostering problems" Della Croce et Salassa. "VNS outoperforms exact commercial general purpose solvers"
Matheuristic approach!

Early work by:
Hansen  and  Mladenovic, 2001, Mladenovic and Hansen, 1997

\subsection{Tabu Search}

Commonly used meta-heuristic.

\subsection{Goal programming and Fuzzy Goal Programming}

GP: Used for multiple goals. \\
FGP: Used for contradictory goals.

Bellman and Zadeh's max-min operator!

Fuzzy goal programming.  "Fuzzy goals" = soft constraints. Fuzzy set theory. The basic idea of FGP is to present some of the model parameters as imprecise numbers.
Goal programming: good when combining soft and hard constraints.

Using an average value approach with goals that are contradictory makes it possible to maximize the amount of "goodness" in the solution, by priotritizing one constraint over another, which in total generates the most good.


\fi