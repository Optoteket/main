%A thorough description of the mathematical model in words and in math. More significant and general constraints in equations and less significant in words.
In this chapter the mathematical model implemented to solve this problem will be presented. Prior to the objective function and constraints, the most significant sets and variables will be provided to give the reader some basic knowledge of the implemented model. Section \ref{section:obj} presents the objective function and gives a short description of what it represents. In Section \ref{constraints} the essential constraints will be presented and explained. A complete model with all definitions and the full set of constraints can be found in Appendix \ref{definitions}. %To get a full view of the mathematical problem the reader  A list of the defined parameters can be found in Appendix \ref{definitions}. 
\section{Set and variable definitions} \label{variables}
To solve the problem many sets and variables had to be declared. As mentioned before, there are many unique and individual requirements given by the library that have to be met. An example is that some workers want a day free from outer tasks to be able to attend meetings or perform inner tasks. Another example is that some have two alternating schedules for odd and even weeks. These specific cases have to be modeled and result in a variety of sets and variable definitions. Hence, only the most important ones are listed below. A complete list of the definitions can be found in Appendix \ref{definitions}. \\
\itab{$I$} \tab{Set of workers}\\
\itab{$I_{lib}$} \tab{Set of librarians ($I_{lib} \subseteq I$)} \\
\itab{$I_{ass}$}	 \tab{Set of assistants ($I_{ass} \subseteq I$)}	\\
\itab{$W$}                 \tab{Set of all ten weeks} \\
\itab{$W_5$}	\tab{Set of first five weeks} \\
\itab{$D$}                 \tab{Set of all days in a week}           \\
\itab{$D_5$}	\tab{Set of all five weekdays} \\
\itab{$S_d$}           \tab{Set of shifts available day \textit{d}}         \\
\itab{$S_3$}           \tab{Set of first three shifts on a weekday}     \\
\itab{$J_d$}            \tab{Set of task types available day \textit{d}}   \\

In order to further define the problem we introduce the following variables: Let,
\begin{align}
    x_{iwdsj}&=
    \begin{cases}
      1, & \text{if worker \textit{i} is assigned in week \textit{w}, day \textit{d}, shift \textit{s} to a task \textit{j}}\\
      0, & \text{otherwise}
    \end{cases}
    \\
    H_{iwh}&=
    \begin{cases}
      1, & \text{if worker \textit{i} works weekend \textit{h} (= 1, 2) in week \textit{w}}\\
      0, & \text{otherwise}
    \end{cases}
	\\
	r_{iw}&=
	\begin{cases}
		1, & \text{if worker \textit{i} has its schedule rotated \textit{w-1} steps}\\
		0, & \text{otherwise}
	\end{cases}
	\\
	l_{iwd}&=
	\begin{cases}
	  1, & \text{if librarian \textit{i} is a stand-in week \textit{w}, day \textit{d}} \\
	  0, & \text{otherwise}
	\end{cases}
	\\
	a_{iwd}&=
	\begin{cases}
 		1, & \text{if assistant \textit{i} is a stand-in week \textit{w}, day \textit{d}} \\
 		0, & \text{otherwise}
	\end{cases}
	\\
	y_{iwds}&=
	\begin{cases}
 		1, & \text{if worker \textit{i} works week \textit{w}, day \textit{d}, shift \textit{s} at task type E, I or P} \\
 		0, & \text{otherwise}
	\end{cases}
	\\
	W_{iwd}&=
	\begin{cases}
	 	1, & \text{if a worker \textit{i} is working a shift week \textit{w}, day \textit{d}} \\
	 	0, & \text{otherwise}
	\end{cases}
	\\
	b_{iw}&=
	\begin{cases}
 		1, & \text{if worker \textit{i} works at HB week \textit{w}} \\
 		0, & \text{otherwise}
	\end{cases}
	\\
	f_{iw}&=
	\begin{cases}
 		1, & \text{if worker \textit{i} is assigned to work friday evening week \textit{w}} \\
 		0, & \text{otherwise}
	\end{cases}	
	\\
	M_{wds}&=
	\begin{cases}
	 	1, & \text{if a big meeting is placed on week \textit{w}, day \textit{d}, shift \textit{s}} \\
	 	0, & \text{otherwise}
	\end{cases}
	\\
	m_{wdsD}&=
	\begin{cases}
	 	1, & \text{if a meeting is placed on week \textit{w}, day \textit{d}, shift \textit{s} at department \textit{D}} \\
	 	0, & \text{otherwise}
	\end{cases}
	\\
	d_{iwds}&=
	\begin{cases}
	 	1, & \text{if there is a difference in assignment of tasks at a shift}\\
	 		& \text{ \textit{s}, for a worker \textit{i}, day \textit{d} between week \textit{w} and \textit{w+5}} \\
	 	0, & \text{otherwise}
	\end{cases}
	\\
	l^{min}&= \text{lowest number of stand-in librarians found (integer)} \\
	a^{min}&= \text{lowest number of stand-in assistants found (integer)} \\
	s^{min}&= \text{weighted sum with number of stand-in librarians and assistants}.
\end{align}

Based on the variables defined above it has been possible to solve our scheduling problem. \textit{$l^{min}$} and \textit{$a^{min}$} are the variables of most significance as they represent the number of stand-ins found after a run. 

\section{Objective function} \label{section:obj}
As the problem consists of multiple objective functions it has been necessary to weigh them against each other using parameters. These are shown in Equation \ref{objfcn} below as $M$ and $N$.

\begin{equation} \label{objfcn}
maximize \hspace{0.3cm} M\cdot s^{min} - N \cdot \sum_{i \in I}\sum_{w \in W_5}\sum_{d \in D_5}\sum_{s \in S_3} d_{iwds}
\end{equation}
The two first objective functions represent the lowest amount of stand-in librarians and assistants found. The second objective function is a preference from the library that two weeks with a five-week interval should be as similar as possible (e.g. week 1 and 6, 2 and 7 etc.).

The parameter $N$ prioritizes the similarity of weeks compared to the number of stand-ins. Based on the information given by the library it is a much higher priority to have many stand-ins. Hence, $M \gg N$ in our case.




% % % % GO THROUGH THE REFERENCES TO CONSTRAINT SECTION IN THE DOCUMENT
\section{Constraints} \label{constraints}
To model the present problem it has been of relevance to divide many of the constraints into weekend and weekday constraints. Several help constraints have also been added to avoid multiplication of two variables, making the problem non-linear. These help constraints have been left out of this chapter for simplicity reasons. Instead, these can be seen in Appendix \ref{definitions}.

\subsection{Demand and assignment constraints} \label{section:demand_ass_constraints}
The most crucial constraint is to ensure that the demand of workers is met each day. This can be modeled as:
\begin{equation} \label{eq:demand}
\sum_{i \in I} x_{iwdsj} = demand_{wdsj}, \; \forall w\in W,d\in D,s\in S,j\in J_d
\end{equation}
$demand_{wdsj}$ is an integer representing the number of workers required week \textit{w}, day \textit{d}, shift \textit{s} for a task \textit{j}.

The following constraint says whether or not a worker is assigned a task during a weekday, where the evening shifts and task type \textit{Library on Wheels} are excluded:
\begin{equation} \label{constr:y_assign}
y_{iwds} = \sum_{j \in J_d\backslash \{B\}} x_{iwdsj}, \; \forall i \in I, w \in W, d \in D_5, s \in S_3
\end{equation}
The variable assignment above is used to simplify a couple of constraints later on.

To ensure that no worker is assigned more than one task the following constraint is implemented:
\begin{equation} \label{constr:one_task_constraint}
\sum_{s\in S}\sum_{j\in J_d} x_{iwdsj} \leq 1, \; \forall i\in I, w \in W, d\in D
\end{equation}
However, if we allow a person to have two shifts at the Library on Wheels on a day, Equation \ref{constr:one_task_constraint} has to be slightly modified, which is left out of this chapter.

It is preferred to allow only one PL per week and a maximum of three PL per ten weeks. These are easily modeled with the following constraints:
\begin{equation} \label{constr:one_PL}
\sum_{s \in S_d}\sum_{d \in D} x_{iwdsP} \leq 1, \; \forall i\in I, w \in W
\end{equation}
\begin{equation} \label{constr:three_PL}
\sum_{w \in W}\sum_{s \in S_d}\sum_{d \in D} x_{iwdsP} \leq 3, \; \forall i\in I
\end{equation}
The duration of a PL (in the equations named "P"), which is from 08:00-16:00, is the cause of these preferences; some workers are required to have some alloted time to perform their inner tasks.

Another preference is to have various start times of the assigned tasks each week, so that the more and less desired shifts are evenly distributed. The following equation is implemented to meet such requirements:
\begin{equation} \label{constr:various_start_times}
\sum_{d \in D_5} y_{iwds} \leq 2, \; \forall i\in I, w \in W, s \in S_3
\end{equation}
Equation \ref{constr:various_start_times} allows a worker to have at most two tasks starting at the same hour. Worth noting is that task type \textit{Library on Wheels} is disregarded in the constraint as the variable $y_{iwds}$ is used; see Equation \ref{constr:y_assign} for the definition.

It is desirable to avoid assigning too many tasks to a worker. The reason for this, as mentioned previously, is to let the worker have some time to allot for inner services during weekdays. The equation below models this preference:
\begin{equation} \label{constr:four_weekly_shifts_at_most}
\sum_{d \in D_5}\sum_{s \in S_d}\sum_{j \in J_d} x_{iwdsj} \leq 4, \; \forall i\in I, w \in W
\end{equation}
Equation \ref{constr:four_weekly_shifts_at_most} allows a worker at most four weekday shifts per week. The exception, which is one of the BokB workers, is left out of this equation for simplicity reasons. To model this completely a new subset of \textit{I} needs to be created where that worker is left out. 
%Ändra till max 4 tasks per vecka diff {36}?

\subsection{Weekend and rotation constraints} \label{section:weekend_rot_constraints}
To further model the problem weekends and week rotations have to be considered. The following three constraints are the most basic constraints regarding weekends $H_{iwh}$ and week rotations $r_{iw}$:
\begin{equation} \label{constr:one_rot}
\sum_{w \in W} r_{iw} = 1, \; \forall i\in I
\end{equation}
\begin{equation} \label{constr:max_one_weekend}
\sum_{w \in W} H_{iwh} \leq 1, \; \forall i\in I, h = 1,2
\end{equation}
\begin{equation} \label{constr:rot_weekend}
r_{iw} \geq H_{iw1}, \; \forall i\in I, w \in W
%	\begin{cases}
% 		1, & \text{if $H_{iwh=1} = 1$} \\
% 		0/1, & \text{otherwise}
%	\end{cases}
%	\; \forall i\in I, w \in W \;
\end{equation}

Equation \ref{constr:one_rot} provides all workers with a rotation of their schedule regardless if they are working weekends or not. Equation \ref{constr:max_one_weekend} allows a worker a maximum of two weekends $(h = 1 and h = 2)$ per ten weeks. Lastly, Equation \ref{constr:rot_weekend} in combination with Equation \ref{constr:one_rot} ensures that the schedule rotation is aligned with the first weekend, if the worker is due for weekend work. In case the worker is not due for weekend work, the rotation of the schedule is free.

A worker is supposed to work weekends with a five-week interval. However, in case there are enough workers to satisfy the demand on weekends, it may be enough to work one weekend per ten weeks for some. To avoid problems with the rotation when only the second weekend is assigned to a worker, the following equation is implemented:
\begin{equation} \label{constr:five_week_interval}
r_{i(mod_{10}(w+4)+1)} \geq H_{iw2}, \; \forall i\in I, w \in W
\end{equation}
Worth noting is that if a worker is assigned both weekends then Equation \ref{constr:five_week_interval} provides the same information as Equation \ref{constr:rot_weekend}.

Every workers schedule is able to rotate up to nine times, where the decision variable $r_{iv}$ decides the rotation. Therefore, the parameter $qualavail_{iwdsj}$ has to align with the rotation so that all workers are assigned tasks only when available. This is provided in the following equation:
\begin{equation} \label{constr:qualavail}
x_{iwdsj} \leq \sum_{v \in V} r_{iv}*qualavail_{i(mod_{10}(w-v+10)+1)dsj}, \; \forall i \in I, w \in W, d \in D, s \in S_d, j \in J_d
\end{equation}
Table \ref{tab:mod} gives a better understanding of how the modulus function in the parameter $qualavail_{iwdsj}$ is used. \textit{w} represent the current week in the relative schedule and \textit{v} represent \textit{v-1} rotations to the right from the relative schedule. The resulting week after rotations can be seen in the cells.
\begin{table}[H]
\centering
\caption{Resulting table of the function $mod_{10}(w-v+10)+1$}
\label{tab:mod}
\begin{tabular}{llllllllllll}
    &                         &                         &                         &                         &                         & \multicolumn{2}{l}{w =}                           &                         &                         &                         &                         \\
    &                         & 1                       & 2                       & 3                       & 4                       & 5                       & 6                       & 7                       & 8                       & 9                       & 10                      \\ \cline{3-12} 
    & \multicolumn{1}{l|}{1}  & \multicolumn{1}{l|}{1}  & \multicolumn{1}{l|}{2}  & \multicolumn{1}{l|}{3}  & \multicolumn{1}{l|}{4}  & \multicolumn{1}{l|}{5}  & \multicolumn{1}{l|}{6}  & \multicolumn{1}{l|}{7}  & \multicolumn{1}{l|}{8}  & \multicolumn{1}{l|}{9}  & \multicolumn{1}{l|}{10} \\ \cline{3-12} 
    & \multicolumn{1}{l|}{2}  & \multicolumn{1}{l|}{10} & \multicolumn{1}{l|}{1}  & \multicolumn{1}{l|}{2}  & \multicolumn{1}{l|}{3}  & \multicolumn{1}{l|}{4}  & \multicolumn{1}{l|}{5}  & \multicolumn{1}{l|}{6}  & \multicolumn{1}{l|}{7}  & \multicolumn{1}{l|}{8}  & \multicolumn{1}{l|}{9}  \\ \cline{3-12} 
    & \multicolumn{1}{l|}{3}  & \multicolumn{1}{l|}{9}  & \multicolumn{1}{l|}{10} & \multicolumn{1}{l|}{1}  & \multicolumn{1}{l|}{2}  & \multicolumn{1}{l|}{3}  & \multicolumn{1}{l|}{4}  & \multicolumn{1}{l|}{5}  & \multicolumn{1}{l|}{6}  & \multicolumn{1}{l|}{7}  & \multicolumn{1}{l|}{8}  \\ \cline{3-12} 
    & \multicolumn{1}{l|}{4}  & \multicolumn{1}{l|}{8}  & \multicolumn{1}{l|}{9}  & \multicolumn{1}{l|}{10} & \multicolumn{1}{l|}{1}  & \multicolumn{1}{l|}{2}  & \multicolumn{1}{l|}{3}  & \multicolumn{1}{l|}{4}  & \multicolumn{1}{l|}{5}  & \multicolumn{1}{l|}{6}  & \multicolumn{1}{l|}{7}  \\ \cline{3-12} 
v = & \multicolumn{1}{l|}{5}  & \multicolumn{1}{l|}{7}  & \multicolumn{1}{l|}{8}  & \multicolumn{1}{l|}{9}  & \multicolumn{1}{l|}{10} & \multicolumn{1}{l|}{1}  & \multicolumn{1}{l|}{2}  & \multicolumn{1}{l|}{3}  & \multicolumn{1}{l|}{4}  & \multicolumn{1}{l|}{5}  & \multicolumn{1}{l|}{6}  \\ \cline{3-12} 
    & \multicolumn{1}{l|}{6}  & \multicolumn{1}{l|}{6}  & \multicolumn{1}{l|}{7}  & \multicolumn{1}{l|}{8}  & \multicolumn{1}{l|}{9}  & \multicolumn{1}{l|}{10} & \multicolumn{1}{l|}{1}  & \multicolumn{1}{l|}{2}  & \multicolumn{1}{l|}{3}  & \multicolumn{1}{l|}{4}  & \multicolumn{1}{l|}{5}  \\ \cline{3-12} 
    & \multicolumn{1}{l|}{7}  & \multicolumn{1}{l|}{5}  & \multicolumn{1}{l|}{6}  & \multicolumn{1}{l|}{7}  & \multicolumn{1}{l|}{8}  & \multicolumn{1}{l|}{9}  & \multicolumn{1}{l|}{10} & \multicolumn{1}{l|}{1}  & \multicolumn{1}{l|}{2}  & \multicolumn{1}{l|}{3}  & \multicolumn{1}{l|}{4}  \\ \cline{3-12} 
    & \multicolumn{1}{l|}{8}  & \multicolumn{1}{l|}{4}  & \multicolumn{1}{l|}{5}  & \multicolumn{1}{l|}{6}  & \multicolumn{1}{l|}{7}  & \multicolumn{1}{l|}{8}  & \multicolumn{1}{l|}{9}  & \multicolumn{1}{l|}{10} & \multicolumn{1}{l|}{1}  & \multicolumn{1}{l|}{2}  & \multicolumn{1}{l|}{3}  \\ \cline{3-12} 
    & \multicolumn{1}{l|}{9}  & \multicolumn{1}{l|}{3}  & \multicolumn{1}{l|}{4}  & \multicolumn{1}{l|}{5}  & \multicolumn{1}{l|}{6}  & \multicolumn{1}{l|}{7}  & \multicolumn{1}{l|}{8}  & \multicolumn{1}{l|}{9}  & \multicolumn{1}{l|}{10} & \multicolumn{1}{l|}{1}  & \multicolumn{1}{l|}{2}  \\ \cline{3-12} 
    & \multicolumn{1}{l|}{10} & \multicolumn{1}{l|}{2}  & \multicolumn{1}{l|}{3}  & \multicolumn{1}{l|}{4}  & \multicolumn{1}{l|}{5}  & \multicolumn{1}{l|}{6}  & \multicolumn{1}{l|}{7}  & \multicolumn{1}{l|}{8}  & \multicolumn{1}{l|}{9}  & \multicolumn{1}{l|}{10} & \multicolumn{1}{l|}{1}  \\ \cline{3-12} 
\end{tabular}
\end{table}
An example to better understand the function: Imagine that we are looking at an unrotated relative schedule where everyone is said to work weekend the first week. Say that we look at a worker's first week, $w=1$. If this schedule is rotated two times to the right, $v=3$, then the first week in the new rotated schedule represent the previous ninth week, $w=9$ from the old schedule. The availability to look at in the old schedule is, therefore, week nine.

A worker is supposed to work with the same task both Saturday and Sunday when working a weekend. This can be modeled with the following constraints:
\begin{equation} \label{constr:consecutive_days}
\sum_{j \in J_d} x_{iw61j} + \sum_{j \in J_d} x_{iw71j} = 2*\sum_{h = 1}^{2} H_{iwh}, \; \forall i\in I, w \in W
\end{equation}
\begin{equation} \label{constr:same_tasks}
x_{iw61j} = x_{iw71j}, \; \forall i\in I, w \in W, j \in J_d
\end{equation}

Friday evening is also included in extension to working Saturday and Sunday, unless the worker is assigned to HB. It is, however, not a necessity to perform the same task Friday evenings as during the weekend, thus Fridays are not included in Equation \ref{constr:same_tasks}. Equation \ref{constr:friday_added} below adds Fridays to the weekend:
\begin{equation} \label{constr:friday_added}
\sum_{j \in J \backslash \{L\}}x_{iw54j} = f_{iw}, \; \forall i \in I, w \in W
\end{equation}
"L" in the equation refers to the task Library on Wheels. The variable $f_{iw}$ is, as mentioned in the variable declaration, equal to one if and only if a worker is working weekend as well as not being assigned to HB.

It is of interest to implement a constraint to prevent workers from being assigned to HB more than once every ten weeks to avoid unfairness. The constraint can be modeled as:
\begin{equation} \label{constr:max_one_hb}
\sum_{w \in W}\sum_{d = 6}^{7}x_{iwd1B} \leq 2, \; \forall i \in I
\end{equation}
"B" in this equation refers to HB. The combination of Equations \ref{constr:same_tasks} and \ref{constr:max_one_hb} ensure that once a worker is assigned to HB it will be for two consecutive weekend days, which is the amount of days the last mentioned equation allow a worker every ten weeks. Why this is preferable is described in Section \ref{section:library_tasks}.

\subsection{Objective function constraints} \label{section:obj_fcn_constraints}
To calculate the variable $s^{min}$ used in the objective function, \ref{objfcn}, the following equation is implemented:
\begin{equation} \label{constr:s_min}
s^{min} \leq L\cdot \sum_{i \in I_{lib}} l_{iwd} + A\cdot \sum_{i \in I_{ass}} a_{iwd}, \; \forall w \in W, d \in D_5
\end{equation}
%\begin{equation} \label{constr:l_min}
%l^{min} \leq \sum_{i \in I_{lib}} l_{iwd}, \; \forall w \in W, d \in D_5
%\end{equation}
%\begin{equation} \label{constr:a_min}
%a^{min} \leq \sum_{i \in I_{ass}} a_{iwd}, \; \forall w \in W, d \in D_5
%\end{equation}

$l_{iwd}$ and $a_{iwd}$ are binary variables stating whether a worker is a stand-in or not on a specific day. $s^{min}$ is being maximized in the objective function; therefore, it will assume the lowest value of the sum of stand-in librarians and assistants for any day during the ten weeks. Just as with previous equation, help constraints have been left out for simplicity reasons. In this case regarding how $l_{iwd}$ and $a_{iwd}$ are determined.

If $L < A$ in the model the solver would prioritize assistants over librarians as stand-ins. Librarians are, however, more desired as stand-ins due to their ability to perform all task types. Therefore, it is desired to set $L \geq A$.

As stated in Section \ref{section:obj}, two weeks with a five-week interval should be as similar as possible. Equation \ref{constr:obj_fcn_shifts} in combination with the objective function equation, \ref{objfcn}, provides this preference.
\begin{equation} \label{constr:obj_fcn_shifts}
d_{iwds} = \abs{y_{iwds} - y_{i(w+5)ds}}, \; \forall i \in I, w \in W_5, d \in D_5, s \in S_3
\end{equation}
The decision variable $d_{iwds}$ in this equation states if there is a difference in assignment between two tasks at the same hour and day for two weeks, $w$ and $w+5$. The differences occur if, say, an Exp task is assigned on Monday week one at 08:00-10:00 and no task is assigned the same shift Monday week six. Thus, a variable minimized in the objective function.

\subsection{Meeting constraints} \label{section:meeting_constraints}
At the library there are both library meetings as well as department meetings, where both occur with a five-week interval. Library meetings are set to take place from 08:00-10:00 on Mondays, whereas department meetings are more freely distributed. A few workers are not assigned library meetings, since they are needed at the stations to keep the library running. The set $I_{big}$ in the equation below consists of all workers who are to be assigned library meetings. The constraints modeling library meetings are as follows:
\begin{equation} \label{constr:library_meetings}
\sum_{w \in W_5} M_{w11} = 1
\end{equation}
\begin{equation} \label{constr:library_meetings2}
M_{(w+5)11} = M_{w11}, \; \forall w \in W_5
\end{equation}
\begin{equation} \label{constr:library_meetings3}
\sum_{s=1}^{3} \sum_{j \in J_1 \backslash \{L\}} x_{iw1sj} \leq 1-M_{w11}, \; \forall i \in I \backslash I_{big}, w \in W
\end{equation}
Equation \ref{constr:library_meetings} and \ref{constr:library_meetings2} assign two library meetings with a five-week interval during the ten-week scheduling period. Equation \ref{constr:library_meetings3} makes sure that the workers that do not attend library meetings are not assigned any other task the day of the meeting.

The constraints added to model department meetings are somewhat similar to the library meeting constraints. Just as for library meetings they take place two times during the ten weeks with a five-week interval, which is described by Equation \ref{constr:dep_meetings} and \ref{constr:dep_meetings2}.
\begin{equation} \label{constr:dep_meetings}
\sum_{w \in W_5}\sum_{d \in D_5}\sum_{s \in S_3} m_{wdsD} = 1
\end{equation}
\begin{equation} \label{constr:dep_meetings2}
m_{(w+5)dsD} = m_{wdsD}, \; \forall D = 1, \ldots, 3, w \in W_5, d \in D_5, s \in S_3
\end{equation}
\begin{equation} \label{constr:dep_meetings3}
m_{wdsD} + x_{iwdsj} \leq 1, \; \forall D = 1, \ldots, 3, i \in I_D, w \in W, d \in D_5, s \in S_3, j \in J_d
\end{equation}
\begin{equation} \label{constr:dep_meetings4}
m_{wdsD} \leq \sum_{v \in V} r_{iv}*qualavail_{i(mod_{10}(w-v+10)+1)dsE}, \; \forall D = 1, \ldots, 3, i \in I_D, w \in W, d \in D_5, s \in S_3
\end{equation}
Equation \ref{constr:dep_meetings3} prohibits a worker from multitasking, i.e. attend a meeting as well as work with a task. Lastly, Equation \ref{constr:dep_meetings4} enables department meetings only when everyone in that department is available to be assigned a task; the rotation is taken into the account. "E" in Equation \ref{constr:dep_meetings4} stands for Exp and is used due to it is the only task type everyone is available for and \textit{qualavail} is dependent on the variable \textit{j}. 

