\section{Introduction}

The second heuristic approach tested distributes individual tasks to workers. This method greatly resembles the process of manually placing tasks as is typically done in many practical situations. The heuristic undergoes two phases. In the first phase, weekends, as well as some weekday tasks, are distributed and in the second phase the rest of the weekday tasks are distributed.

The primary method used in this approach is a large neighbourhood search (LNS) together with a simulated annhealing (SA) accept function. Destroying and repairing the solution, as is customary in LNS, helps leading the solution out of local optima or plateaus. Similarly, SA is used in order to allow the solution to move in a less favourable directions to avoid these local optima.

\section{Costs}
The search algorithm is guided using costs. A worker gets has a number of costs, as displayed in Table \ref{tab:worker_costs}. Each cost represents a misplaced task, resulting in an infeasible schedule or unused stand in potential. In the library, different costs are weighed together in three different ways. These are referred to as objective functions, see Table \ref{tab:lib_costs}. 

\begin{table}[!h]
\centering
\caption{Worker costs. Costs are incremented when a worker gets a task which breaks a soft constraint.}
\label{tab:worker_costs}
\begin{tabular}{|l|l|}
\hline
\multicolumn{2}{|l|}{\cellcolor{Gray} \textbf{Stand in cost}} \\ \hline
$C_{SI\_min}$ &  Cost of day with minimum number of stand ins. \\ \hline
\multicolumn{2}{|l|}{\cellcolor{Gray} \textbf{Infeasibility costs}} \\ \hline
$C_{Task\_Day}$ & Cost from all tasks exceeding the daily limit. \\ \hline
$C_{Task\_Week}$ & Cost from all tasks exceeding the weekly limit.  \\ \hline
$C_{PL\_Day}$ & Cost from all PL exceeding the weekly limit. \\ \hline
$C_{PL\_Tot}$& Cost from all PL exceeding the total limit. \\ \hline
$C_{SShift\_Week}$ & \begin{tabular}[t]{@{}l@{}} Cost from the number of task performed at \\ the same shift in a week exceeding the weekly limit. \end{tabular} \\ \hline
\end{tabular}
\end{table}


\begin{table}[!h]
\centering
\caption{Library objective functions and their cost components. Weekend and weekday objective functions are to be maximized while the worker objective function is to be minimized.}
\label{tab:lib_costs}
\begin{tabular}{|l|l|}
\hline
\multicolumn{2}{|l|}{\cellcolor{corn} \textbf{Weekend Objective Function}} \\
\hline 
\rowcolor{Gray} Cost & Cost description \\ \hline
$C_{SI\_min}$ &  Cost of day with minimum number of stand ins. \\ \hline
$C_{Shift\_min}$ & Cost of shift with minimum number of workers. \\ \hline
$C_{Day\_min}$ & Cost of day with minimum number of workers. \\ \hline
$C_{SI\_aver}$ & Average number of stand ins at a day. \\ \hline
$C_{Shift\_aver}$ & Average number of workers at a shift. \\ \hline
$C_{Day\_aver}$ & Average number of workers at a day. \\ \hline
\hline
\multicolumn{2}{|l|}{\cellcolor{corn} \textbf{Worker Objective Function}} \\
\hline
\rowcolor{Gray} Cost & Cost description \\ \hline
$C_{Task\_Day}$ & Cost from all tasks exceeding the daily limit in all workers. \\ \hline
$C_{Task\_Week}$ & Cost from all tasks exceeding the weekly limit in all workers. \\ \hline
$C_{PL\_Week}$ & Cost from all PL exceeding the weekly limit in all workers. \\ \hline
$C_{PL\_Tot}$& Cost from all PL exceeding the total limit in all workers. \\ \hline
$C_{SShift\_Week}$ & \begin{tabular}[t]{@{}l@{}} Cost from the number of task performed at \\ the same shift in a week exceeding the weekly limit in all workers. \end{tabular}\\ \hline
\hline
 \multicolumn{2}{|l|}{\cellcolor{corn} \textbf{Weekday Objective Function}} \\
\hline
\rowcolor{Gray} Cost & Cost description \\ \hline
$C_{SI\_min}$ &  Cost of day with minimum number of stand ins. \\ \hline
\end{tabular}
\end{table}

The weekday objective function is associated with the weekend distribution phase of the problem. Their are three types of costs in the objective function, measuring a certain aspect of a good weekend schedule. Since we want to increase the number of stand ins at the most critical days, the worst days of all aspects are considered, referred to as the min cost. Also, the average of each aspect is considered, in order to distinguish solutions which for which the worst day is identical. All costs in this objective function are calculated using the formula:

\begin{equation}
\label{eq:wend_cost_calc}
C_{type} = W_{lib}*num\_of\_lib + W_{ass}*num\_of\_ass
\end{equation}

The weights are constants used in all weekend objective function costs. Typically $W_{lib}$ is larger than $W_{ass}$ since librarians can perform a greater number of tasks.

After maximizing weekends, the weekday distribution phase is entered. Here, the worker objective function is used to identify infeasibilities in the worker schedules and is simply a sum of all the worker costs over all workers. When this cost is zero, the schedule is feasible. Then, in order to compare feasible schedules, the weekday objective function is used, containing only the minimum stand in cost, calculated in the same way as in the weekend objective function. This objective function is corresponding to the objective function in the Mathematical Model.

\section{Weekend phase}

The weekend phase is the first phase the algorithm enters and is described in detail in Appendix TODO. The LNS component is described by a destroy and repair loop, while the simulated annealing step is marked as the decision to accept solutions which are worse than the current one.

The reason for implementing the weekend phase separate from the distribution of other tasks was because of the realization that the minimum number of stand ins depended to a very large extent on the weekend structure. In particular, it is the week rest associated following upon weekend work which decides how many people are present at a particular shift or day. This can be seen in for example Table \ref{tab:Bob_avail}.

The outer loop is performed a specified number of iterations. This loop is guided by the weekend objective function described in the previous section. Better solutions are always accepted and the globally best solution is saved. The SA component is implemented using exponential cooling, so that a solution is accepted with a probability P:

\begin{equation}
P = exp(-\Delta E /kT)
\label{eq0}
\end{equation}

where:

\begin{equation}
\Delta E = Wend_t - Wend_{t-1}
\label{eq1}
\end{equation}

and

\begin{equation}
T = T_0 \alpha^t
\label{eq2}
\end{equation}

$T$ is the temperature of the SA accept function, which has an initial value of $T_0$ and which cools down with the iteration count  $t$ at a rate $\alpha$ ($0 < \alpha < 1$). The objective function at the iteration $t$ is written $Wend_t$.


The inner loop of the algorithm destroys and repairs the solution. The loop checks if the produced solution is infeasible, that is the difference between the number of available workers at each shift with the number of required workers is negative. Such schedules are discarded and a new destroy and repair is performed. An available worker in this context is a worker who is available for work at the shift and does not have any tasks scheduled at that day.

 % ----------------- %

The algoritms for distributing, destroying and repairing weekends is simply a random function since it is very hard to predict what is a good or bad placement of a specific weekend without placing the whole schedule. Since the weekend objective function is only a measurement of the current number of available workers and stand ins, this number will be greatly reduced when placing the weekday tasks. 


\begin{table}[!h]
\caption{Worker availability placing only weekends. Intensity of red indicates number of workers.}
\centering
\label{tab:num_avail_no_tasks}
\begin{tabular}{|C{1.2cm}
|C{0.6cm}|C{0.6cm}|C{0.6cm}|C{0.6cm}|C{0.6cm}|C{0.6cm}|C{0.6cm}|}
\hline & \multicolumn{7}{l|}{\textbf{Num available assistants}} \\ \hline\rowcolor{Gray} & Mo & Tu & We & Th & Fr & Sa & Su \\ \hline\colcell Shift 1: & {\cellcolor{maroon!50}}8 & {\cellcolor{maroon!60}}10 & {\cellcolor{maroon!60}}10 & {\cellcolor{maroon!60}}10 & {\cellcolor{maroon!40}}6 & {\cellcolor{maroon!0}}0 & {\cellcolor{maroon!0}}0 \\ \hline
\colcell Shift 2: & {\cellcolor{maroon!50}}8 & {\cellcolor{maroon!55}}9 & {\cellcolor{maroon!55}}9 & {\cellcolor{maroon!55}}9 & {\cellcolor{maroon!40}}6 & {\cellcolor{maroon!0}}0 & {\cellcolor{maroon!0}}0 \\ \hline
\colcell Shift 3: & {\cellcolor{maroon!55}}9 & {\cellcolor{maroon!50}}8 & {\cellcolor{maroon!55}}9 & {\cellcolor{maroon!45}}7 & {\cellcolor{maroon!35}}5 & {\cellcolor{maroon!0}}0 & {\cellcolor{maroon!0}}0 \\ \hline
\colcell Shift 4: & {\cellcolor{maroon!25}}3 & {\cellcolor{maroon!20}}2 & {\cellcolor{maroon!20}}2 & {\cellcolor{maroon!25}}3 & {\cellcolor{maroon!0}}0 & {\cellcolor{maroon!0}}0 & {\cellcolor{maroon!0}}0 \\ \hline
\hline & \multicolumn{7}{l|}{\textbf{Num available librarians}} \\ \hline\rowcolor{Gray} & Mo & Tu & We & Th & Fr & Sa & Su \\ \hline\colcell Shift 1: & {\cellcolor{maroon!90}}16 & {\cellcolor{maroon!85}}15 & {\cellcolor{maroon!90}}16 & {\cellcolor{maroon!75}}13 & {\cellcolor{maroon!70}}12 & {\cellcolor{maroon!0}}0 & {\cellcolor{maroon!0}}0 \\ \hline
\colcell Shift 2: & {\cellcolor{maroon!90}}18 & {\cellcolor{maroon!85}}15 & {\cellcolor{maroon!95}}17 & {\cellcolor{maroon!80}}14 & {\cellcolor{maroon!75}}13 & {\cellcolor{maroon!0}}0 & {\cellcolor{maroon!0}}0 \\ \hline
\colcell Shift 3: & {\cellcolor{maroon!95}}17 & {\cellcolor{maroon!80}}14 & {\cellcolor{maroon!90}}18 & {\cellcolor{maroon!90}}18 & {\cellcolor{maroon!75}}13 & {\cellcolor{maroon!0}}0 & {\cellcolor{maroon!0}}0 \\ \hline
\colcell Shift 4: & {\cellcolor{maroon!25}}3 & {\cellcolor{maroon!30}}4 & {\cellcolor{maroon!30}}4 & {\cellcolor{maroon!25}}3 & {\cellcolor{maroon!15}}1 & {\cellcolor{maroon!0}}0 & {\cellcolor{maroon!0}}0 \\ \hline
\hline & \multicolumn{7}{l|}{\textbf{Num available BB-librarians}} \\ \hline\rowcolor{Gray} & Mo & Tu & We & Th & Fr & Sa & Su \\ \hline\colcell Shift 1: & {\cellcolor{maroon!20}}2 & {\cellcolor{maroon!0}}0 & {\cellcolor{maroon!15}}1 & {\cellcolor{maroon!15}}1 & {\cellcolor{maroon!15}}1 & {\cellcolor{maroon!0}}0 & {\cellcolor{maroon!0}}0 \\ \hline
\colcell Shift 2: & {\cellcolor{maroon!0}}0 & {\cellcolor{maroon!0}}0 & {\cellcolor{maroon!0}}0 & {\cellcolor{maroon!0}}0 & {\cellcolor{maroon!0}}0 & {\cellcolor{maroon!0}}0 & {\cellcolor{maroon!0}}0 \\ \hline
\colcell Shift 3: & {\cellcolor{maroon!0}}0 & {\cellcolor{maroon!0}}0 & {\cellcolor{maroon!0}}0 & {\cellcolor{maroon!0}}0 & {\cellcolor{maroon!0}}0 & {\cellcolor{maroon!0}}0 & {\cellcolor{maroon!0}}0 \\ \hline
\colcell Shift 4: & {\cellcolor{maroon!15}}1 & {\cellcolor{maroon!0}}0 & {\cellcolor{maroon!20}}2 & {\cellcolor{maroon!20}}2 & {\cellcolor{maroon!0}}0 & {\cellcolor{maroon!0}}0 & {\cellcolor{maroon!0}}0 \\ \hline
\end{tabular}
\end{table}

\begin{table}[!h]
\centering
\caption{Worker availability after placing weekends as well as evening tasks and BokB for the same week. }
\label{tab:num_avail_with_tasks}
\begin{tabular}{|C{1.2cm}
|C{0.6cm}|C{0.6cm}|C{0.6cm}|C{0.6cm}|C{0.6cm}|C{0.6cm}|C{0.6cm}|}
\hline & \multicolumn{7}{l|}{\textbf{Num available assistants}} \\ \hline\rowcolor{Gray} & Mo & Tu & We & Th & Fr & Sa & Su \\ \hline\colcell Shift 1: & {\cellcolor{maroon!45}}7 & {\cellcolor{maroon!60}}10 & {\cellcolor{maroon!55}}9 & {\cellcolor{maroon!55}}9 & {\cellcolor{maroon!50}}8 & {\cellcolor{maroon!0}}0 & {\cellcolor{maroon!0}}0 \\ \hline
\colcell Shift 2: & {\cellcolor{maroon!45}}7 & {\cellcolor{maroon!55}}9 & {\cellcolor{maroon!50}}8 & {\cellcolor{maroon!50}}8 & {\cellcolor{maroon!50}}8 & {\cellcolor{maroon!0}}0 & {\cellcolor{maroon!0}}0 \\ \hline
\colcell Shift 3: & {\cellcolor{maroon!45}}7 & {\cellcolor{maroon!45}}7 & {\cellcolor{maroon!50}}8 & {\cellcolor{maroon!40}}6 & {\cellcolor{maroon!40}}6 & {\cellcolor{maroon!0}}0 & {\cellcolor{maroon!0}}0 \\ \hline
\colcell Shift 4: & {\cellcolor{maroon!0}}0 & {\cellcolor{maroon!0}}0 & {\cellcolor{maroon!0}}0 & {\cellcolor{maroon!0}}0 & {\cellcolor{maroon!0}}0 & {\cellcolor{maroon!0}}0 & {\cellcolor{maroon!0}}0 \\ \hline
\hline & \multicolumn{7}{l|}{\textbf{Num available librarians}} \\ \hline\rowcolor{Gray} & Mo & Tu & We & Th & Fr & Sa & Su \\ \hline\colcell Shift 1: & {\cellcolor{maroon!80}}14 & {\cellcolor{maroon!65}}11 & {\cellcolor{maroon!75}}13 & {\cellcolor{maroon!65}}11 & {\cellcolor{maroon!70}}12 & {\cellcolor{maroon!0}}0 & {\cellcolor{maroon!0}}0 \\ \hline
\colcell Shift 2: & {\cellcolor{maroon!80}}14 & {\cellcolor{maroon!65}}11 & {\cellcolor{maroon!80}}14 & {\cellcolor{maroon!65}}11 & {\cellcolor{maroon!70}}12 & {\cellcolor{maroon!0}}0 & {\cellcolor{maroon!0}}0 \\ \hline
\colcell Shift 3: & {\cellcolor{maroon!75}}13 & {\cellcolor{maroon!65}}11 & {\cellcolor{maroon!80}}14 & {\cellcolor{maroon!75}}13 & {\cellcolor{maroon!75}}13 & {\cellcolor{maroon!0}}0 & {\cellcolor{maroon!0}}0 \\ \hline
\colcell Shift 4: & {\cellcolor{maroon!0}}0 & {\cellcolor{maroon!0}}0 & {\cellcolor{maroon!0}}0 & {\cellcolor{maroon!15}}1 & {\cellcolor{maroon!15}}1 & {\cellcolor{maroon!0}}0 & {\cellcolor{maroon!0}}0 \\ \hline
\hline & \multicolumn{7}{l|}{\textbf{Num available BB-librarians}} \\ \hline\rowcolor{Gray} & Mo & Tu & We & Th & Fr & Sa & Su \\ \hline\colcell Shift 1: & {\cellcolor{maroon!0}}0 & {\cellcolor{maroon!0}}0 & {\cellcolor{maroon!0}}0 & {\cellcolor{maroon!0}}0 & {\cellcolor{maroon!0}}0 & {\cellcolor{maroon!0}}0 & {\cellcolor{maroon!0}}0 \\ \hline
\colcell Shift 2: & {\cellcolor{maroon!0}}0 & {\cellcolor{maroon!0}}0 & {\cellcolor{maroon!0}}0 & {\cellcolor{maroon!0}}0 & {\cellcolor{maroon!0}}0 & {\cellcolor{maroon!0}}0 & {\cellcolor{maroon!0}}0 \\ \hline
\colcell Shift 3: & {\cellcolor{maroon!0}}0 & {\cellcolor{maroon!0}}0 & {\cellcolor{maroon!0}}0 & {\cellcolor{maroon!0}}0 & {\cellcolor{maroon!0}}0 & {\cellcolor{maroon!0}}0 & {\cellcolor{maroon!0}}0 \\ \hline
\colcell Shift 4: & {\cellcolor{maroon!0}}0 & {\cellcolor{maroon!0}}0 & {\cellcolor{maroon!15}}1 & {\cellcolor{maroon!0}}0 & {\cellcolor{maroon!0}}0 & {\cellcolor{maroon!0}}0 & {\cellcolor{maroon!0}}0 \\ \hline
\end{tabular}
\end{table}


In order to get a better measurement of the number of available workers during this phase evening tasks and BokB tasks are distributed upon repairing weekends. The evening tasks are distributed using the same method as is described in the next section while BokB tasks are placed according to a fixed schedule. The effect of this distribution is visible in Tables \ref{tab:num_avail_no_tasks} and \ref{tab:num_avail_with_tasks}. High intensity of red in a cell indicates a higher number of workers. Thus, since the first table is generally redder, there are fewer available workers in the second table, which more reflect the real number of available workers for the weekend constellation. 

\section{Weekday phase}

When entering the weekday phase, all weekends, evenings and BokB tasks are already placed. In this phase, the rest of the tasks, referred to as "weekday" tasks are also placed. The demand for a week at this point is illustrated in Table \ref{tab:current_demand}. This demand is identical for all weeks.The algorithm for distributing weekday tasks is found in Appendix TODO and greatly resembles the algorithm for distributing weekends.


\begin{table}[!h]
\centering
\caption{Worker demand during a week when entering weekday phase.}
\label{tab:current_demand}
\begin{tabular}{|C{1.2cm}
|C{1cm}|C{1cm}|C{1cm}|C{1cm}|C{1cm}|}
\hline
\rowcolor{Gray} & Exp & Info & PL & HB & BokB \\ \hline
\multicolumn{6}{|l|}{\colcelltwo Monday } \\ \hline
\colcell Shift 1: & {\cellcolor{maroon!25}}2 & {\cellcolor{maroon!25}}2 & {\cellcolor{maroon!25}}1 & {\cellcolor{maroon!0}}0 & {\cellcolor{maroon!0}}0 \\ \hline
\colcell Shift 2: & {\cellcolor{maroon!25}}3 & {\cellcolor{maroon!25}}3 & {\cellcolor{maroon!0}}0 & {\cellcolor{maroon!0}}0 & {\cellcolor{maroon!0}}0 \\ \hline
\colcell Shift 3: & {\cellcolor{maroon!25}}3 & {\cellcolor{maroon!25}}3 & {\cellcolor{maroon!0}}0 & {\cellcolor{maroon!0}}0 & {\cellcolor{maroon!0}}0 \\ \hline
\colcell Shift 4: & {\cellcolor{maroon!0}}0 & {\cellcolor{maroon!0}}0 & {\cellcolor{maroon!0}}0 & {\cellcolor{maroon!0}}0 & {\cellcolor{maroon!0}}0 \\ \hline
\multicolumn{6}{|l|}{\colcelltwo Tuesday } \\ \hline
\colcell Shift 1: & {\cellcolor{maroon!25}}2 & {\cellcolor{maroon!25}}2 & {\cellcolor{maroon!25}}1 & {\cellcolor{maroon!0}}0 & {\cellcolor{maroon!0}}0 \\ \hline
\colcell Shift 2: & {\cellcolor{maroon!25}}3 & {\cellcolor{maroon!25}}3 & {\cellcolor{maroon!0}}0 & {\cellcolor{maroon!0}}0 & {\cellcolor{maroon!0}}0 \\ \hline
\colcell Shift 3: & {\cellcolor{maroon!25}}3 & {\cellcolor{maroon!25}}3 & {\cellcolor{maroon!0}}0 & {\cellcolor{maroon!0}}0 & {\cellcolor{maroon!0}}0 \\ \hline
\colcell Shift 4: & {\cellcolor{maroon!0}}0 & {\cellcolor{maroon!0}}0 & {\cellcolor{maroon!0}}0 & {\cellcolor{maroon!0}}0 & {\cellcolor{maroon!0}}0 \\ \hline
\multicolumn{6}{|l|}{\colcelltwo Wednesday } \\ \hline
\colcell Shift 1: & {\cellcolor{maroon!25}}2 & {\cellcolor{maroon!25}}2 & {\cellcolor{maroon!25}}1 & {\cellcolor{maroon!0}}0 & {\cellcolor{maroon!0}}0 \\ \hline
\colcell Shift 2: & {\cellcolor{maroon!25}}3 & {\cellcolor{maroon!25}}3 & {\cellcolor{maroon!0}}0 & {\cellcolor{maroon!0}}0 & {\cellcolor{maroon!0}}0 \\ \hline
\colcell Shift 3: & {\cellcolor{maroon!25}}3 & {\cellcolor{maroon!25}}3 & {\cellcolor{maroon!0}}0 & {\cellcolor{maroon!0}}0 & {\cellcolor{maroon!0}}0 \\ \hline
\colcell Shift 4: & {\cellcolor{maroon!0}}0 & {\cellcolor{maroon!0}}0 & {\cellcolor{maroon!0}}0 & {\cellcolor{maroon!0}}0 & {\cellcolor{maroon!0}}0 \\ \hline
\multicolumn{6}{|l|}{\colcelltwo Thursday } \\ \hline
\colcell Shift 1: & {\cellcolor{maroon!25}}2 & {\cellcolor{maroon!25}}2 & {\cellcolor{maroon!25}}1 & {\cellcolor{maroon!0}}0 & {\cellcolor{maroon!0}}0 \\ \hline
\colcell Shift 2: & {\cellcolor{maroon!25}}3 & {\cellcolor{maroon!25}}3 & {\cellcolor{maroon!0}}0 & {\cellcolor{maroon!0}}0 & {\cellcolor{maroon!0}}0 \\ \hline
\colcell Shift 3: & {\cellcolor{maroon!25}}3 & {\cellcolor{maroon!25}}3 & {\cellcolor{maroon!0}}0 & {\cellcolor{maroon!0}}0 & {\cellcolor{maroon!0}}0 \\ \hline
\colcell Shift 4: & {\cellcolor{maroon!0}}0 & {\cellcolor{maroon!0}}0 & {\cellcolor{maroon!0}}0 & {\cellcolor{maroon!0}}0 & {\cellcolor{maroon!0}}0 \\ \hline
\multicolumn{6}{|l|}{\colcelltwo Friday } \\ \hline
\colcell Shift 1: & {\cellcolor{maroon!25}}2 & {\cellcolor{maroon!25}}2 & {\cellcolor{maroon!25}}1 & {\cellcolor{maroon!0}}0 & {\cellcolor{maroon!0}}0 \\ \hline
\colcell Shift 2: & {\cellcolor{maroon!25}}3 & {\cellcolor{maroon!25}}3 & {\cellcolor{maroon!0}}0 & {\cellcolor{maroon!0}}0 & {\cellcolor{maroon!0}}0 \\ \hline
\colcell Shift 3: & {\cellcolor{maroon!25}}3 & {\cellcolor{maroon!25}}3 & {\cellcolor{maroon!0}}0 & {\cellcolor{maroon!0}}0 & {\cellcolor{maroon!0}}0 \\ \hline
\colcell Shift 4: & {\cellcolor{maroon!0}}0 & {\cellcolor{maroon!0}}0 & {\cellcolor{maroon!0}}0 & {\cellcolor{maroon!0}}0 & {\cellcolor{maroon!0}}0 \\ \hline
\multicolumn{6}{|l|}{\colcelltwo Saturday } \\ \hline
\colcell Shift 1: & {\cellcolor{maroon!0}}0 & {\cellcolor{maroon!0}}0 & {\cellcolor{maroon!0}}0 & {\cellcolor{maroon!0}}0 & {\cellcolor{maroon!0}}0 \\ \hline
\multicolumn{6}{|l|}{\colcelltwo Sunday } \\ \hline
\colcell Shift 1: & {\cellcolor{maroon!0}}0 & {\cellcolor{maroon!0}}0 & {\cellcolor{maroon!0}}0 & {\cellcolor{maroon!0}}0 & {\cellcolor{maroon!0}}0 \\ \hline
 \end{tabular}
\end{table}

The outer loop of the algorithm is performed a specified number of iterations and simply checks the weekday objective function. Better solutions are saved, others are discareded. 

The inner loop destroys and repairs the solution until the worker objective function is zero, that is until the schedule is feasible. If the loop cannot find such a schedule during a maximum number of iterations, the solution is discarded as infeasible, and the outer loop is entered.

As opposed to the previous phase, the destroy and repair as well as the initial distribution of weekday tasks is following a greedy algoritm. Library tasks are first sorted according to qualification requirement, so that librarian tasks are placed first and assistant tasks second. This guarantees that all librarian tasks can be placed. Secondly, these tasks are sorted according to the differens between the number of available workers and the demand at each task. This makes sure that the most critical tasks are placed first.

The tasks are then staffed one at a time. The process starts with temporarily placing the task at all available workers. This will generate a cost for each worker according to Table \ref{tab:worker_costs}. The cheapest workers are then chosen and permanently placed on the task.

The destroy function identifies the worker with the highest cost. The worst week of this worker is then identified and destroyed. Similarly, a few random workers are chosen, their number depending on the destroy amount specified, for which the same week is destroyed. This makes it possible for the first worker to get a better worst week by changing tasks with the other workers in the repair function.
