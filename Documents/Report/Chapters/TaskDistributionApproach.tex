

The second heuristic approach tested distributes individual tasks to workers and greatly resembles the process of manual scheduling. The implemented algorithm undergoes two phases. In the first phase, weekends, as well as some weekday tasks, are distributed and in the second phase the rest of the weekday tasks are distributed. 

The primary method used in this approach is a Large Neighbourhood Search (LNS) together with a Simulated Annealing (SA) accept function. Destroying and repairing the solution, as is customary in LNS, helps leading the solution out of local optima or plateaus. Similarly, SA is used in order to allow the solution to move in a less favourable directions to move out of local optima regions.

\section{Costs} \label{section:tasks_cost}
The search algorithm is guided using costs, both for worker schedules and the whole library schedule. Each worker has a number of costs, as displayed in Table \ref{tab:worker_costs}. Each of these costs represents a badly placed task, resulting in an infeasible schedule or unused stand-in potential. The costs have different weights, as illustrated in the table. How the weights should be set is discussed in Chapter \ref{chap:res}. 

For the library schedule, costs are weighed together in three different ways. These are referred to as the three library objective functions, as is illustrated in Table \ref{tab:lib_costs}. These costs partly overlap with the individual worker costs.

\begin{table}[!h]
\centering
\caption{Individual worker costs and cost weights.}
\label{tab:worker_costs}
\begin{tabular}{|l|l|p{7cm}|}
\hline
\multicolumn{3}{|l|}{\cellcolor{gray!90} \textbf{Stand in cost}} \\ \hline
\rowcolor{Gray} Cost & Weight & Cost description \\ \hline
$C_{SI}$ & $W_{w\_SI}$ & Cost from having a task on a day where the worker can be a stand-in. \\ \hline
\multicolumn{3}{|l|}{\cellcolor{gray!90} \textbf{Infeasibility costs}} \\ \hline
\rowcolor{Gray} Cost & Weight & Cost description \\ \hline
$C_{Task\_D}$ & $W_{w\_Task\_D}$ & Cost from all tasks exceeding the daily limit. \\ \hline
$C_{Task\_W}$ & $W_{w\_Task\_W}$ & Cost from all tasks exceeding the weekly limit.  \\ \hline
$C_{PL\_D}$ & $W_{w\_PL\_W}$ & Cost from all PL exceeding the weekly limit. \\ \hline
$C_{PL\_Tot}$ & $W_{w\_PL\_Tot}$ & Cost from all PL exceeding the total limit. \\ \hline
$C_{SShift\_W}$ & $W_{w\_SShift\_W}$ & Cost from the number of task performed at the same shift in a week exceeding the weekly limit. \\ \hline
\end{tabular}
\end{table}

\begin{table}[!h]
\centering
\caption{Library objective functions with their cost components and weights.}
\label{tab:lib_costs}
\begin{tabular}{|l|l|p{7cm}|}
\hline
\multicolumn{3}{|l|}{\cellcolor{gray!90} \textbf{Weekend Objective Function}} \\
\hline 
\rowcolor{Gray} Cost & Weight & Cost description \\ \hline
$C_{SI\_m}$ & $W_{SI\_m}$ & Cost of day with minimum number of stand-ins. \\ \hline
$C_{S\_m}$ & $W_{S\_m}$ & Cost of shift with minimum number of workers. \\ \hline
$C_{D\_m}$ & $W_{D\_m}$ & Cost of day with minimum number of workers. \\ \hline
$C_{SI\_a}$ & $W_{SI\_a}$ & Average number of stand-ins at a day. \\ \hline
$C_{S\_a}$ & $W_{S\_a}$ & Average number of workers at a shift. \\ \hline
$C_{D\_a}$ & $W_{D\_a}$ & Average number of workers at a day. \\ \hline
\hline
\multicolumn{3}{|l|}{\cellcolor{gray!90} \textbf{Worker Objective Function}} \\
\hline
\rowcolor{Gray} Cost & Weight & Cost description \\ \hline
$C_{Task\_D}$ & $W_{Task\_D}$ & Cost from all tasks exceeding the daily limit for all workers. \\ \hline
$C_{Task\_W}$ & $W_{Task\_W}$ & Cost from all tasks exceeding the weekly limit for all workers. \\ \hline
$C_{PL\_W}$ & $W_{PL\_W}$ & Cost from all PL exceeding the weekly limit for all workers. \\ \hline
$C_{PL\_Tot}$ & $W_{PL\_Tot}$ & Cost from all PL exceeding the total limit for all workers. \\ \hline
$C_{SShift\_W}$ & $W_{SShift\_W}$ & Cost from the number of task performed at the same shift in a week exceeding the weekly limit for all workers. \\ \hline
\hline
 \multicolumn{3}{|l|}{\cellcolor{gray!90} \textbf{Weekday Objective Function}} \\
\hline
\rowcolor{Gray} Cost & Weight & Cost description \\ \hline
$C_{SI\_m}$ & - & Cost of day with minimum number of stand-ins. \\ \hline
\end{tabular}
\end{table}

The weekend objective function, as referred to in the table, is associated with the weekend distribution phase of the problem. There are three types of costs in this objective function, each measuring a certain aspect of a good weekend schedule. Since we want to increase the number of stand-ins at the most critical days, the worst shift or day of these aspects is to be maximized. This is referred to as the min cost. Also, the average of each aspect is considered, in order to distinguish solutions for which the worst shift or day is identical. All costs in this objective function are calculated using the formula:

\begin{equation}
\label{eq:wend_cost_calc}
C_{type} = W_{lib}*num\_of\_lib + W_{ass}*num\_of\_ass
\end{equation}

The weights $W_{lib}$ and $W_{ass}$ are constants used in all weekend objective function costs. Typically $W_{lib}$ is larger than $W_{ass}$ since librarians can perform a greater number of tasks. The weekend objective function becomes the sum of the weighted costs displayed in Table \ref{tab:lib_costs}. 

After a certain number of iteration the weekend phase is finished and the weekday distribution phase is entered. Here, the worker objective function is used to identify infeasibilities in the worker schedules and is simply a sum of the worker costs for all workers. When this cost is zero, the schedule is feasible and the weekday objective function value is calculated. This objective function contains the minimum stand-in cost, which is calculated in the same way as in the weekend objective function. The weekday objective function corresponds to the objective function in the Mathematical Model and is to be maximized.

\section{Weekend phase}

The weekend phase is the first phase the algorithm enters and it's flow is described also in Appendix \ref{appendix:flow_charts} Figure \ref{fig:weekend_alg}. The LNS component is part of the "destroy and repair loop", while the simulated annealing step is marked as the decision to accept a solution which is worse than the current one.

The reason for implementing the weekend phase separate from the distribution of other tasks is because of the fact that the number of stand-ins in the final schedule depends to a large extent on the weekend-worker constellation. In particular, it is the week rest following upon weekend work which reduces the number of workers at a particular shift or day. 

The outer loop in the flow chart is performed a specified number of iterations. This loop is guided by the weekend objective function described in the previous section. Better solutions are always accepted and the globally best solution is saved. The SA component is implemented using exponential cooling, so that a solution which is worse than the current one is accepted with a probability P:

\begin{equation}
P = exp(-\Delta E/T)
\label{eq0}
\end{equation}

where

\begin{equation}
\Delta E = Wend_t - Wend_{t-1}
\label{eq1}
\end{equation}

and

\begin{equation}
T = T_0 \alpha^t
\label{eq2}
\end{equation}

$T$ is the temperature of the SA accept function, which has an initial value of $T_0$ and which cools down with the iteration count  $t$ at a rate $\alpha$ ($0 < \alpha < 1$). The objective function value at iteration $t$ is written $Wend_t$.


The inner loop of the algorithm destroys and repairs the solution. The loop checks if the produced solution is infeasible, that is if the difference between the number of available workers at each shift and the number of required workers is negative. Such schedules are discarded and a new destroy and repair is performed. The number of available workers is calculated per shift and is the sum of all workers who are available at that shift and who do not have any tasks scheduled at that day.

The algorithms for distributing, destroying and repairing weekends is simply a random function since it is very hard to predict what is a good or bad placement of a specific weekend without placing the whole schedule. Since the weekend objective function is only a measurement of the current number of stand-ins and available workers, this number will be greatly reduced when placing the weekday tasks. 


\begin{table}[!h]
\caption{Worker availability placing only weekends. Intensity of red indicates number of workers.}
\centering
\label{tab:num_avail_no_tasks}
\begin{tabular}{|C{1.2cm}
|C{0.6cm}|C{0.6cm}|C{0.6cm}|C{0.6cm}|C{0.6cm}|C{0.6cm}|C{0.6cm}|}
\hline \cellcolor{gray!90} & \multicolumn{7}{l|}{\cellcolor{gray!90} \textbf{Num available assistants}} \\ \hline\rowcolor{Gray} & Mo & Tu & We & Th & Fr & Sa & Su \\ \hline\colcell Shift 1: & {\cellcolor{maroon!50}}8 & {\cellcolor{maroon!60}}10 & {\cellcolor{maroon!60}}10 & {\cellcolor{maroon!60}}10 & {\cellcolor{maroon!40}}6 & {\cellcolor{maroon!0}}0 & {\cellcolor{maroon!0}}0 \\ \hline
\colcell Shift 2: & {\cellcolor{maroon!50}}8 & {\cellcolor{maroon!55}}9 & {\cellcolor{maroon!55}}9 & {\cellcolor{maroon!55}}9 & {\cellcolor{maroon!40}}6 & {\cellcolor{maroon!0}}0 & {\cellcolor{maroon!0}}0 \\ \hline
\colcell Shift 3: & {\cellcolor{maroon!55}}9 & {\cellcolor{maroon!50}}8 & {\cellcolor{maroon!55}}9 & {\cellcolor{maroon!45}}7 & {\cellcolor{maroon!35}}5 & {\cellcolor{maroon!0}}0 & {\cellcolor{maroon!0}}0 \\ \hline
\colcell Shift 4: & {\cellcolor{maroon!25}}3 & {\cellcolor{maroon!20}}2 & {\cellcolor{maroon!20}}2 & {\cellcolor{maroon!25}}3 & {\cellcolor{maroon!0}}0 & {\cellcolor{maroon!0}}0 & {\cellcolor{maroon!0}}0 \\ \hline
\hline \cellcolor{gray!90} & \multicolumn{7}{l|}{\cellcolor{gray!90} \textbf{Num available librarians}} \\ \hline\rowcolor{Gray} & Mo & Tu & We & Th & Fr & Sa & Su \\ \hline\colcell Shift 1: & {\cellcolor{maroon!90}}16 & {\cellcolor{maroon!85}}15 & {\cellcolor{maroon!90}}16 & {\cellcolor{maroon!75}}13 & {\cellcolor{maroon!70}}12 & {\cellcolor{maroon!0}}0 & {\cellcolor{maroon!0}}0 \\ \hline
\colcell Shift 2: & {\cellcolor{maroon!90}}18 & {\cellcolor{maroon!85}}15 & {\cellcolor{maroon!95}}17 & {\cellcolor{maroon!80}}14 & {\cellcolor{maroon!75}}13 & {\cellcolor{maroon!0}}0 & {\cellcolor{maroon!0}}0 \\ \hline
\colcell Shift 3: & {\cellcolor{maroon!95}}17 & {\cellcolor{maroon!80}}14 & {\cellcolor{maroon!90}}18 & {\cellcolor{maroon!90}}18 & {\cellcolor{maroon!75}}13 & {\cellcolor{maroon!0}}0 & {\cellcolor{maroon!0}}0 \\ \hline
\colcell Shift 4: & {\cellcolor{maroon!25}}3 & {\cellcolor{maroon!30}}4 & {\cellcolor{maroon!30}}4 & {\cellcolor{maroon!25}}3 & {\cellcolor{maroon!15}}1 & {\cellcolor{maroon!0}}0 & {\cellcolor{maroon!0}}0 \\ \hline
\hline \cellcolor{gray!90} & \multicolumn{7}{l|}{\cellcolor{gray!90} \textbf{Num available BokB-librarians}} \\ \hline\rowcolor{Gray} & Mo & Tu & We & Th & Fr & Sa & Su \\ \hline\colcell Shift 1: & {\cellcolor{maroon!20}}2 & {\cellcolor{maroon!0}}0 & {\cellcolor{maroon!15}}1 & {\cellcolor{maroon!15}}1 & {\cellcolor{maroon!15}}1 & {\cellcolor{maroon!0}}0 & {\cellcolor{maroon!0}}0 \\ \hline
\colcell Shift 2: & {\cellcolor{maroon!0}}0 & {\cellcolor{maroon!0}}0 & {\cellcolor{maroon!0}}0 & {\cellcolor{maroon!0}}0 & {\cellcolor{maroon!0}}0 & {\cellcolor{maroon!0}}0 & {\cellcolor{maroon!0}}0 \\ \hline
\colcell Shift 3: & {\cellcolor{maroon!0}}0 & {\cellcolor{maroon!0}}0 & {\cellcolor{maroon!0}}0 & {\cellcolor{maroon!0}}0 & {\cellcolor{maroon!0}}0 & {\cellcolor{maroon!0}}0 & {\cellcolor{maroon!0}}0 \\ \hline
\colcell Shift 4: & {\cellcolor{maroon!15}}1 & {\cellcolor{maroon!0}}0 & {\cellcolor{maroon!20}}2 & {\cellcolor{maroon!20}}2 & {\cellcolor{maroon!0}}0 & {\cellcolor{maroon!0}}0 & {\cellcolor{maroon!0}}0 \\ \hline
\end{tabular}
\end{table}

\begin{table}[!h]
\centering
\caption{Worker availability after placing weekends as well as evening tasks and BokB for the same week. }
\label{tab:num_avail_with_tasks}
\begin{tabular}{|C{1.2cm}
|C{0.6cm}|C{0.6cm}|C{0.6cm}|C{0.6cm}|C{0.6cm}|C{0.6cm}|C{0.6cm}|}
\hline \rowcolor{gray!90}  & \multicolumn{7}{l|}{\cellcolor{gray!90} \textbf{Num available assistants}} \\ 
\hline\rowcolor{Gray} & Mo & Tu & We & Th & Fr & Sa & Su \\
 \hline\colcell Shift 1: & {\cellcolor{maroon!45}}7 & {\cellcolor{maroon!60}}10 & {\cellcolor{maroon!55}}9 & {\cellcolor{maroon!55}}9 & {\cellcolor{maroon!50}}8 & {\cellcolor{maroon!0}}0 & {\cellcolor{maroon!0}}0 \\ \hline
\colcell Shift 2: & {\cellcolor{maroon!45}}7 & {\cellcolor{maroon!55}}9 & {\cellcolor{maroon!50}}8 & {\cellcolor{maroon!50}}8 & {\cellcolor{maroon!50}}8 & {\cellcolor{maroon!0}}0 & {\cellcolor{maroon!0}}0 \\ \hline
\colcell Shift 3: & {\cellcolor{maroon!45}}7 & {\cellcolor{maroon!45}}7 & {\cellcolor{maroon!50}}8 & {\cellcolor{maroon!40}}6 & {\cellcolor{maroon!40}}6 & {\cellcolor{maroon!0}}0 & {\cellcolor{maroon!0}}0 \\ \hline
\colcell Shift 4: & {\cellcolor{maroon!0}}0 & {\cellcolor{maroon!0}}0 & {\cellcolor{maroon!0}}0 & {\cellcolor{maroon!0}}0 & {\cellcolor{maroon!0}}0 & {\cellcolor{maroon!0}}0 & {\cellcolor{maroon!0}}0 \\ \hline
\hline \hline \rowcolor{gray!90} & \multicolumn{7}{l|}{\cellcolor{gray!90} \textbf{Num available librarians}} \\ \hline\rowcolor{Gray} & Mo & Tu & We & Th & Fr & Sa & Su \\ \hline\colcell Shift 1: & {\cellcolor{maroon!80}}14 & {\cellcolor{maroon!65}}11 & {\cellcolor{maroon!75}}13 & {\cellcolor{maroon!65}}11 & {\cellcolor{maroon!70}}12 & {\cellcolor{maroon!0}}0 & {\cellcolor{maroon!0}}0 \\ \hline
\colcell Shift 2: & {\cellcolor{maroon!80}}14 & {\cellcolor{maroon!65}}11 & {\cellcolor{maroon!80}}14 & {\cellcolor{maroon!65}}11 & {\cellcolor{maroon!70}}12 & {\cellcolor{maroon!0}}0 & {\cellcolor{maroon!0}}0 \\ \hline
\colcell Shift 3: & {\cellcolor{maroon!75}}13 & {\cellcolor{maroon!65}}11 & {\cellcolor{maroon!80}}14 & {\cellcolor{maroon!75}}13 & {\cellcolor{maroon!75}}13 & {\cellcolor{maroon!0}}0 & {\cellcolor{maroon!0}}0 \\ \hline
\colcell Shift 4: & {\cellcolor{maroon!0}}0 & {\cellcolor{maroon!0}}0 & {\cellcolor{maroon!0}}0 & {\cellcolor{maroon!15}}1 & {\cellcolor{maroon!15}}1 & {\cellcolor{maroon!0}}0 & {\cellcolor{maroon!0}}0 \\ \hline
\hline \rowcolor{gray!90} & \multicolumn{7}{l|}{\textbf{\cellcolor{gray!90} Num available BokB-librarians}} \\ \hline\rowcolor{Gray} & Mo & Tu & We & Th & Fr & Sa & Su \\ \hline\colcell Shift 1: & {\cellcolor{maroon!0}}0 & {\cellcolor{maroon!0}}0 & {\cellcolor{maroon!0}}0 & {\cellcolor{maroon!0}}0 & {\cellcolor{maroon!0}}0 & {\cellcolor{maroon!0}}0 & {\cellcolor{maroon!0}}0 \\ \hline
\colcell Shift 2: & {\cellcolor{maroon!0}}0 & {\cellcolor{maroon!0}}0 & {\cellcolor{maroon!0}}0 & {\cellcolor{maroon!0}}0 & {\cellcolor{maroon!0}}0 & {\cellcolor{maroon!0}}0 & {\cellcolor{maroon!0}}0 \\ \hline
\colcell Shift 3: & {\cellcolor{maroon!0}}0 & {\cellcolor{maroon!0}}0 & {\cellcolor{maroon!0}}0 & {\cellcolor{maroon!0}}0 & {\cellcolor{maroon!0}}0 & {\cellcolor{maroon!0}}0 & {\cellcolor{maroon!0}}0 \\ \hline
\colcell Shift 4: & {\cellcolor{maroon!0}}0 & {\cellcolor{maroon!0}}0 & {\cellcolor{maroon!15}}1 & {\cellcolor{maroon!0}}0 & {\cellcolor{maroon!0}}0 & {\cellcolor{maroon!0}}0 & {\cellcolor{maroon!0}}0 \\ \hline
\end{tabular}
\end{table}


In order to get a better measurement of the number of available workers during this phase, evening tasks and BokB tasks are distributed after each weekend repair. The evening tasks are distributed using the same method as is described in Section \ref{wday_phase} while BokB tasks are placed according to a fixed schedule. The effect of this distribution is visible in Tables \ref{tab:num_avail_no_tasks} and \ref{tab:num_avail_with_tasks}. High intensity of red in a cell indicates a higher number of workers. It can be seen that the first table is generally redder and thus there are fewer available workers in the second table. This state reflects better the real number of available workers for a certain weekend constellation. 

\section{Weekday phase}\label{wday_phase}

When entering the weekday phase, all weekends, evenings and BokB tasks are already placed. In this phase, the rest of the tasks, referred to as "weekday" tasks are to be placed. The demand for a week at this point is illustrated in Table \ref{tab:current_demand} and is identical for all weeks.The algorithm for distributing weekday tasks is found in Appendix \ref{appendix:flow_charts}, Figure \ref{fig:weekday_alg}, and greatly resembles the algorithm for distributing weekends.


\begin{table}[!h]
\centering
\caption{Worker demand during a week when entering weekday phase.}
\label{tab:current_demand}
\begin{tabular}{|C{1.2cm}
|C{1cm}|C{1cm}|C{1cm}|C{1cm}|C{1cm}|}
\hline
\rowcolor{Gray} & Exp & Info & PL & HB & BokB \\ \hline
\multicolumn{6}{|l|}{\cellcolor{gray!90} Monday } \\ \hline
\colcell Shift 1: & {\cellcolor{maroon!25}}2 & {\cellcolor{maroon!25}}2 & {\cellcolor{maroon!25}}1 & {\cellcolor{maroon!0}}0 & {\cellcolor{maroon!0}}0 \\ \hline
\colcell Shift 2: & {\cellcolor{maroon!25}}3 & {\cellcolor{maroon!25}}3 & {\cellcolor{maroon!0}}0 & {\cellcolor{maroon!0}}0 & {\cellcolor{maroon!0}}0 \\ \hline
\colcell Shift 3: & {\cellcolor{maroon!25}}3 & {\cellcolor{maroon!25}}3 & {\cellcolor{maroon!0}}0 & {\cellcolor{maroon!0}}0 & {\cellcolor{maroon!0}}0 \\ \hline
\colcell Shift 4: & {\cellcolor{maroon!0}}0 & {\cellcolor{maroon!0}}0 & {\cellcolor{maroon!0}}0 & {\cellcolor{maroon!0}}0 & {\cellcolor{maroon!0}}0 \\ \hline
\multicolumn{6}{|l|}{\cellcolor{gray!90} Tuesday } \\ \hline
\colcell Shift 1: & {\cellcolor{maroon!25}}2 & {\cellcolor{maroon!25}}2 & {\cellcolor{maroon!25}}1 & {\cellcolor{maroon!0}}0 & {\cellcolor{maroon!0}}0 \\ \hline
\colcell Shift 2: & {\cellcolor{maroon!25}}3 & {\cellcolor{maroon!25}}3 & {\cellcolor{maroon!0}}0 & {\cellcolor{maroon!0}}0 & {\cellcolor{maroon!0}}0 \\ \hline
\colcell Shift 3: & {\cellcolor{maroon!25}}3 & {\cellcolor{maroon!25}}3 & {\cellcolor{maroon!0}}0 & {\cellcolor{maroon!0}}0 & {\cellcolor{maroon!0}}0 \\ \hline
\colcell Shift 4: & {\cellcolor{maroon!0}}0 & {\cellcolor{maroon!0}}0 & {\cellcolor{maroon!0}}0 & {\cellcolor{maroon!0}}0 & {\cellcolor{maroon!0}}0 \\ \hline
\multicolumn{6}{|l|}{\cellcolor{gray!90} Wednesday } \\ \hline
\colcell Shift 1: & {\cellcolor{maroon!25}}2 & {\cellcolor{maroon!25}}2 & {\cellcolor{maroon!25}}1 & {\cellcolor{maroon!0}}0 & {\cellcolor{maroon!0}}0 \\ \hline
\colcell Shift 2: & {\cellcolor{maroon!25}}3 & {\cellcolor{maroon!25}}3 & {\cellcolor{maroon!0}}0 & {\cellcolor{maroon!0}}0 & {\cellcolor{maroon!0}}0 \\ \hline
\colcell Shift 3: & {\cellcolor{maroon!25}}3 & {\cellcolor{maroon!25}}3 & {\cellcolor{maroon!0}}0 & {\cellcolor{maroon!0}}0 & {\cellcolor{maroon!0}}0 \\ \hline
\colcell Shift 4: & {\cellcolor{maroon!0}}0 & {\cellcolor{maroon!0}}0 & {\cellcolor{maroon!0}}0 & {\cellcolor{maroon!0}}0 & {\cellcolor{maroon!0}}0 \\ \hline
\multicolumn{6}{|l|}{\cellcolor{gray!90} Thursday } \\ \hline
\colcell Shift 1: & {\cellcolor{maroon!25}}2 & {\cellcolor{maroon!25}}2 & {\cellcolor{maroon!25}}1 & {\cellcolor{maroon!0}}0 & {\cellcolor{maroon!0}}0 \\ \hline
\colcell Shift 2: & {\cellcolor{maroon!25}}3 & {\cellcolor{maroon!25}}3 & {\cellcolor{maroon!0}}0 & {\cellcolor{maroon!0}}0 & {\cellcolor{maroon!0}}0 \\ \hline
\colcell Shift 3: & {\cellcolor{maroon!25}}3 & {\cellcolor{maroon!25}}3 & {\cellcolor{maroon!0}}0 & {\cellcolor{maroon!0}}0 & {\cellcolor{maroon!0}}0 \\ \hline
\colcell Shift 4: & {\cellcolor{maroon!0}}0 & {\cellcolor{maroon!0}}0 & {\cellcolor{maroon!0}}0 & {\cellcolor{maroon!0}}0 & {\cellcolor{maroon!0}}0 \\ \hline
\multicolumn{6}{|l|}{\cellcolor{gray!90} Friday } \\ \hline
\colcell Shift 1: & {\cellcolor{maroon!25}}2 & {\cellcolor{maroon!25}}2 & {\cellcolor{maroon!25}}1 & {\cellcolor{maroon!0}}0 & {\cellcolor{maroon!0}}0 \\ \hline
\colcell Shift 2: & {\cellcolor{maroon!25}}3 & {\cellcolor{maroon!25}}3 & {\cellcolor{maroon!0}}0 & {\cellcolor{maroon!0}}0 & {\cellcolor{maroon!0}}0 \\ \hline
\colcell Shift 3: & {\cellcolor{maroon!25}}3 & {\cellcolor{maroon!25}}3 & {\cellcolor{maroon!0}}0 & {\cellcolor{maroon!0}}0 & {\cellcolor{maroon!0}}0 \\ \hline
\colcell Shift 4: & {\cellcolor{maroon!0}}0 & {\cellcolor{maroon!0}}0 & {\cellcolor{maroon!0}}0 & {\cellcolor{maroon!0}}0 & {\cellcolor{maroon!0}}0 \\ \hline
\multicolumn{6}{|l|}{\cellcolor{gray!90} Saturday } \\ \hline
\colcell Shift 1: & {\cellcolor{maroon!0}}0 & {\cellcolor{maroon!0}}0 & {\cellcolor{maroon!0}}0 & {\cellcolor{maroon!0}}0 & {\cellcolor{maroon!0}}0 \\ \hline
\multicolumn{6}{|l|}{\cellcolor{gray!90} Sunday } \\ \hline
\colcell Shift 1: & {\cellcolor{maroon!0}}0 & {\cellcolor{maroon!0}}0 & {\cellcolor{maroon!0}}0 & {\cellcolor{maroon!0}}0 & {\cellcolor{maroon!0}}0 \\ \hline
 \end{tabular}
\end{table}

The outer loop of the weekday phase, or the "weekday task distribution loop", is performed a specified number of iterations and calculates the weekday objective function for each iteration. Better solutions are saved, others are discarded. 

The inner "destroy and repair loop" destroys and repairs the solution until the worker objective function is zero, that is until all worker schedules are feasible. If the loop cannot find such a schedule during a maximum number of iterations, the whole solution is discarded as infeasible, and the outer loop is entered.

When destroying and repairing tasks, the tasks are first sorted according to qualification requirement, meaning that librarian tasks are placed first and assistant tasks second. This guarantees that not all librarians will be used up for assistant tasks. Secondly, in each subgroup, the tasks are sorted according to the difference between the number of available workers and the demand at each task. This makes sure that the most critical tasks are placed first.

The tasks are then staffed one at a time. The process starts with temporarily placing the task at all available workers. This will generate a cost for each worker according to Table \ref{tab:worker_costs}. The cheapest workers are then chosen and permanently placed on the task.


The destroy function identifies the worker with the highest cost. The worst week of this worker is then identified and destroyed. Also, a few random workers are chosen, their number depending on the destroy amount specified, for which the same week is destroyed. This makes it possible for the first worker to get a better worst week by changing tasks with the other destroyed workers in the repair function.
