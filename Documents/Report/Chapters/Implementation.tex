%The two heuristical approaches


In the AMPL implementation, the model was identical to the one described in Chapter \ref{chap:mathmod}. However, in order to implement the heuristics, the model's constraints had to be relaxed. Figure \ref{fig:AMPL_vs_heur} illustrates this process as some constraints considered hard in the original model are softened in the heuristics. The alterations of the model for the two heuristics is described in Tables \ref{tab:task_constraints} and \ref{tab:weekly_task_constraints}. The reason for relaxing the model is to create a larger neighbourhood for the heuristic to search through so that it can move more freely between solutions and increase the chance of finding the optimal solution.

% Define block styles
\tikzstyle{block} = [rectangle, draw=none, fill=white,
    text width=8em, text ragged, rounded corners, minimum height=0em, node distance =1.5cm]
\tikzstyle{line} = [draw, -latex']
    
\newcommand*{\h}{\hspace{18pt}}% for indentation
\newcommand*{\hh}{\hspace{24pt}}% double indentation

\begin{figure}[!h]
\caption{Illustration of the difference between the model used in AMPL and the model  for a heuristic}
\label{fig:AMPL_vs_heur}
\begin{center}
\begin{tikzpicture}[node distance = 2cm , auto, scale=0.7, every node/.style={scale=0.7}]
	%Row 1
	\node[block, node distance= 6cm, text width=4em] (AMPL){\Large \textbf{AMPL}};
	\node[block, right of=AMPL, node distance= 5cm] (Mid){};
	\node[block, right of=Mid, node distance= 6cm] (Heur){\Large \textbf{HEURISTIC}};
	%Row 2
	\node[block, below of = AMPL, text width=15em, node distance =1.5cm](ObjfA) {\hh AMPL Objective Function};
	\node[block, below of = Heur, text width=15em, node distance =1.5cm](ObjfH) { \h Heuristic Objective Function};
	%Row 3	
	\node[block, below of = ObjfA, text width=10em](ConstA) {AMPL Constraints};
	\node[block, below of = Mid, text width=10em, node distance=2.5cm](SConst) {\h Soft Constraints};
	%Row 4
	\node[block, below of = SConst, text width=10em, node distance=1cm](HConst) {\h Hard Constraints};
	\node[block, below of = ObjfH, text width=15em, node distance=1.5cm](ConstH) {\h Heuristic Constraints};
	%Invisible node
	\node[block, right of=SConst,scale=0.05, node distance=4cm](inv){};
	%\node at (ObjfH)[block, left of=ObjfH,scale=0.05,node distance=2.7cm](inv2){};
	\node at (8.4,-1.6)(inv2){};
	\begin{scope}[every path/.style=line]
		\path (ObjfA) -- (ObjfH);
		\path (ConstA.east) -- (HConst.west);
		\path (ConstA.east) -- (SConst.west);
		\path (SConst.east) -- (inv2.west);
		\path (HConst.east) -- (ConstH.west);
%		\path [-,draw](SConst) -- (inv);
%		\path (inv) |- (ObjfH);
	\end{scope}
	\draw [color=gray!70,thick](15,1) rectangle (-3,-5);
\end{tikzpicture}
\end{center}
\end{figure}


\begin{table}[!h]
\centering
\caption{Week block scheduling approach; soft and relaxed constraints}
\label{tab:weekly_task_constraints}
\begin{tabular}{|p{4cm}|p{7cm}|}
\hline
% ---------------------------------------------------------
\multicolumn{2}{|l|}{\cellcolor{gray!90} \textbf{Soft Constraints}} \\
\hline 
\rowcolor{Gray} Affecting constraints & Constraint description \\ \hline
\ref{eq:demand} & The amount of workers needed for every shift and task type in the library.  \\ \hline
\ref{constr:three_PL} & Number of PL per ten weeks restriction. \\ \hline
% ---------------------------------------------------------
\multicolumn{2}{|l|}{\cellcolor{gray!90} \textbf{Relaxed Constraints}} \\
\hline 
\rowcolor{Gray} Affecting constraints & Constraint alteration \\ \hline
Several. $W = W_5$ in all constraints. & Five week scheduling instead of ten week scheduling. \\ \hline
\ref{constr:obj_fcn_shifts} & Any two weeks \textit{w} and \textit{w+5} shall be as similar as possible. \\ \hline
BokB-constraints & BokB placed according to constraints, but the schedule is fixed. \\ \hline
Availability data & Even and odd week workers have availability at each shift according to the stricter of the two sets. \\ \hline
\ref{constr:library_meetings} - \ref{constr:dep_meetings4} & Meetings are not implemented. \\ \hline
\end{tabular}
\end{table}


\begin{table}[!h]
\centering
\caption{Task distribution approach; soft and relaxed constraints}
\label{tab:task_constraints}
\begin{tabular}{|p{4cm}|p{7cm}|}
\hline
% ---------------------------------------------------------
\multicolumn{2}{|l|}{\cellcolor{gray!90} \textbf{Soft Constraints}} \\
\hline 
\rowcolor{Gray} Affecting constraints & Constraint description \\ \hline
\ref{constr:one_task_constraint} & One task per day restriction.  \\ \hline
\ref{constr:four_weekly_shifts_at_most} & Four tasks per week restriction. \\ \hline
\ref{constr:one_PL} & One PL per week restriction. \\ \hline
\ref{constr:three_PL} & Number of PL per ten weeks restriction. \\ \hline
\ref{constr:various_start_times} & Not more than two tasks at the same shift in a week restriction.  \\ \hline
% ---------------------------------------------------------
\multicolumn{2}{|l|}{\cellcolor{gray!90} \textbf{Relaxed Constraints}} \\
\hline 
\rowcolor{Gray} Affecting constraints & Constraint alteration \\ \hline
Several. $W = W_5$ in all constraints. & Five week scheduling instead of ten week scheduling. \\ \hline
\ref{constr:obj_fcn_shifts} & Any two weeks \textit{w} and \textit{w+5} shall be as similar as possible. \\ \hline
BokB-constraints & BokB placed according to constraints, but the schedule is fixed. \\ \hline
Availability data & Even and odd week workers have availability at each shift according to the stricter of the two sets. \\ \hline
\ref{constr:library_meetings} - \ref{constr:dep_meetings4} & Meetings are not implemented. \\ \hline
\end{tabular}
\end{table}
