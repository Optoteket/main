Results will be presented in this chapter followed by a discussion regarding the performances of the implemented heuristics. 

\section{Results}
In the following sections results will be presented for both of the heuristics.

\subsection{Weekly scheduling approach}
Text



\subsection{Task distribution approach}
Text


\section{Discussion}
Text


\subsection{Weekly scheduling approach}
Table \ref{pros_cons_weekly_scheduling} lists pros and cons with the implemented weekly scheduling approach.
\begin{table}[!h]
\caption{Pros and cons with the implemented weekly scheduling approach}
\label{pros_cons_weekly_scheduling}
\begin{tabularx}{\linewidth}{>{\parskip1ex}X@{\kern4\tabcolsep}>{\parskip1ex}X}
\toprule
\hfil\bfseries Pros
&
\hfil\bfseries Cons
\\\cmidrule(r{3\tabcolsep}){1-1}\cmidrule(l{-\tabcolsep}){2-2}

%% PROS, seperated by empty line or \par
The same amount of week block appearances will exist for five and ten weeks.\par
Quick iterations when destroying and repairing.\par

&

%% CONS, seperated by empty line or \par
Weekends needs to be assigned in a more systematically way in order to achieve reasonable results regarding lowest amount of stand-ins through the days.\par
The amount of unique block appearances grows exponentially in case more task types are added, such as meetings.\par
The solution time can vary considerably as several random generators have been used.\par
A great deal of costs are needed (some correlated), where each of them affects the solution procedure.

\\\bottomrule
\end{tabularx}
\end{table}

 Weekends, solution time and costs are no major issues as they can be avoided by a few smarter implementations. For instance, weekends can be improved by assigning values when each worker is available on a day and from those numbers create an even distribution of possible stand-ins and therefore increase the lowest value of stand-ins through the days. This shall implicitly decrease the solution time as less iterations will be required. However, to always be able to create a pool of week appearances regardless of problem size can easily become a major issue. Just by adding meetings and the assignment of Library on Wheels tasks to the problem makes it grow considerably.
 

\subsection{Task distribution approach}
Text