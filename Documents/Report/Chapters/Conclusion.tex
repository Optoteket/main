In this thesis work, a mathematical model for the staff scheduling problem at the Library of Norrköping was developed according to the demands given by the library. The model was solved by the commercial optimization solver CPLEX, which generated an optimal schedule, where three librarian stand-ins and zero assistant stand-ins was found on the worst day (or days). This is better than the current schedule at the library, where only one librarian and zero assistants were assigned as stand-in on the worst day.

When developing the two heuristics for solving the problem, the conclusion was reached that the weekend distribution is essential for the stand-in distribution. The heuristic that focused on making weekend scheduling separately from the rest of the problem, thus performed better. Thus, the first heuristic would probably work better if it took weekend allocation into consideration.

Although good results were obtained both by using the AMPL model and in one of the heuristics, there is room for improvement in the modeling of the problem. One thing which could be done in a more robust way, is the modelling of exceptions and preferences among the staff members. In the current model, these are simply modelled as hard constraints. If these could be modelled in a more general way or separately from the rest of the problem, the model would not be so sensitive to changes in exceptions or in the staff preferences.

Several components of the model could be simplified. For example, the library on wheels could be modelled separately from the rest of the problem. Furthermore, staff members schedules which differ between odd and even weeks also add complexity to the problem. One way of handling this could be to produce schedules which are not entirely feasible according to the problem description, but which can be used as basic schedules for which manual adjustment is needed.

In the heuristics, there are several components missing in order for them to fully correspond to the mathematical model. Although these features are not crucial, they could be implemented in order to make the heuristics more accurate. Furthermore, a more systematic framework for testing and evaluating the heuristics could be developed, in order to be able to measure the results more fairly. If one heuristic would be chosen for further development it would be the second one, since, in the first method, a very large number of blocks would have to be created if more task types were added.


% % OLD CONCLUSION % %
%In this thesis work, a mathematical model for the given staff scheduling problem at the Library of Norrköping was developed. The model was run on the commercial optimization solver CPLEX, which generated an optimal schedule with three librarian stand-ins and zero assistant stand-ins at the worst day. This is better than the current schedule at the library, where only one staff member was assigned as stand-in on the worst day.

%When developing the two heuristics for solving the problem, the conclusion was reached that the weekend distribution is essential for the stand-in distribution in the final schedule. The heuristic which focused on weekend scheduling separate from the rest of the problem thus performed better. It was also concluded that the problem is not very difficult to solve, however it is challenging to model. In order to find a more robust scheduler for the problem, which can adapt to changes in the work force, a better model must be developed.


% To CLaes: is it possible to focus on weekends using your implementation?
