In this thesis work, a mathematical model for the staff scheduling problem at the Library of Norrköping was developed. The overall goal was to find schedules that are robust with respect to disturbances by evenly distributing stand-ins, that is, staff members that are not assigned to any tasks during a day. The model was solved by the commercial optimization solver CPLEX, which generated an optimal schedule with three librarian stand-ins and zero assistant stand-ins at the worst day. This is better than the current schedule at the library, where only one staff member was assigned as stand-in on the worst day.

When developing the two heuristics for solving the problem, the conclusion was reached that the weekend distribution is essential for the stand-in distribution in the final schedule. The heuristic which focused on making weekend scheduling separately from the rest of the problem thus performed better. 

Although good results were obtained both by using the AMPL model and the second heuristic, there is room for further development in the modelling of the problem. One of the things which could be done in a more robust way is the modelling of exceptions and preferences among the staff members. In the current model, these are simply modelled as hard constraints. If these could be modelled in a more general way or separately from the core problem, the model would not be so sensitive to changes in exceptions or the staff preferences.

Several components of the model could be simplified. For example, the library on wheels could be modelled separately from the rest of the problem. Another suggestion is to give the same availability for all weeks despite if the staff member is working different hours during odd and even weeks. This would make it easier to model as the staff member's ten-week schedule would consist of two five-week schedules instead. Also, a more general way of describing meetings would simplify their distribution. One way of doing this could be to produce schedules which are not entirely feasible according to the problem description, but which can be used as a basis for manual adjustments.

Regarding the implementations, there are several components missing in order for the heuristics to fully correspond to the mathematical model. Although these features are not crucial, they could be implemented in order to make the heuristics correspond to the complete model. Due to lack of time, these missing components were not implemented. Furthermore, a more systematic framework for testing and evaluating the heuristics could be developed, in order to be able to measure the results more fairly.



% % OLD CONCLUSION % %
%In this thesis work, a mathematical model for the given staff scheduling problem at the Library of Norrköping was developed. The model was run on the commercial optimization solver CPLEX, which generated an optimal schedule with three librarian stand-ins and zero assistant stand-ins at the worst day. This is better than the current schedule at the library, where only one staff member was assigned as stand-in on the worst day.

%When developing the two heuristics for solving the problem, the conclusion was reached that the weekend distribution is essential for the stand-in distribution in the final schedule. The heuristic which focused on weekend scheduling separate from the rest of the problem thus performed better. It was also concluded that the problem is not very difficult to solve, however it is challenging to model. In order to find a more robust scheduler for the problem, which can adapt to changes in the work force, a better model must be developed.


% To CLaes: is it possible to focus on weekends using your implementation?