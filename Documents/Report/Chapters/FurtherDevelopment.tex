Although good results were obtained both by using the AMPL model and the second heuristic, there is room for further development in the modeling as well as the implementation of the heuristics. One of the things which could be done in a more robust way is the modeling of exceptions and preferences among the staff members. In the current model, these are modeled as hard constraints or by costs which are pushed down to zero for feasibility. If these could be modeled in a more general way or in separation from the core problem, the model would not be so sensitive to changes in exceptions or staff preferences.


Several components of the model could be simplified. For example, The library on wheels could be modeled separately from the rest of the problem. Another suggestion is to simplify every other week staff memebers' schedules so that the person gets two five week schedules instead. Also, a more general way of describing meetings would simplify their distribution. One way of doing this could be to produce schedules which are not entirely feasible according to the problem description, but which can be used as a basis for manual scheduling. One interesting topic to study in order to resolve this problem could be fuzzy goal programming.

Regarding the implementations, there are several components missing in order for the heuristics to fully correspond to the mathematical model. Although the features are not crucial, these could be implemented in order to improve the heuristics. Furthermore, a more systematic framework for testing and evaluating heuristics could be used for measuring the results of the implementations.


\iffalse
Although the constraints were difficult to model, the actual problem was not very difficult to solve, neither for CPLEX nor for the second heuristic.  


For both of the heuristics the first step is to generalize the problem to ten weeks instead of five. In doing so, the heuristics will have the same time span as the AMPL implementation. Furthermore, if meetings are implemented and five-week separated work weeks look alike the heuristics will solve the same problem as the AMPL implementation. 

By looking at the results from the heuristics it deems possible to solve the complete problem by spending more time implementing. However, regardless if such results were reached it would only work for this specific library. A general solver would not be a feature that could be implemented. The reason is due to the uniqueness of requirements in the library. In case they were similar someone would, most likely, already have created a general solver for libraries or any other service institution.
\fi
