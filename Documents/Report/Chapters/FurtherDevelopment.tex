Although good results were obtained both by using the AMPL model and the second heuristic, there is room for further development in the modeling of the problem. One of the things which could be done in a more robust way is the modeling of exceptions and preferences among the staff members. In the current model, these are simply modeled as hard constraints. If these could be modeled in a more general way or in separation from the core problem, the model would not be so sensitive to changes in exceptions or staff preferences.

Several components of the model could be simplified. For example, the library on wheels could be modeled separately from the rest of the problem. Another suggestion is to simplify every other week staff memebers' schedules so that the person gets two five week schedules instead. Also, a more general way of describing meetings would simplify their distribution. One way of doing this could be to produce schedules which are not entirely feasible according to the problem description, but which can be used as a basis for manual scheduling.

Regarding the implementations, there are several components missing in order for the heuristics to fully correspond to the mathematical model. Although these features are not crucial, they could be implemented in order to make the heuristics correspond to the full problem. Furthermore, a more systematic framework for testing and evaluating the heuristics could be developed in order to be able to measure results which are more comparable.

% To CLaes: is it possible to focus on weekends using your implementation?