One of the main challenges in this thesis work has been modeling of exceptions and preferences in the work force. This was true both for the AMPL model and the heuristics. Therefore, one way of preceeding with this problem would be to try to generalize or reduce the effect of the exceptions. This would create a more robust model, as the current one is very sensitive to  changes in worker preferences. One example of how this could be done could be to schedule the Library on Wheels independently. Another way would be to simplify every other week workers' schedules so that all weeks are identical, using the most restrictive of the two weeks. A third simplification would be to consider also meeting as tasks. 

Although the constraints were difficult to model, the actual problem was not very difficult to solve, neither for CPLEX nor for the second heuristic.  


For both of the heuristics the first step is to generalize the problem to ten weeks instead of five. In doing so, the heuristics will have the same time span as the AMPL implementation. Furthermore, if meetings are implemented and five-week separated work weeks look alike the heuristics will solve the same problem as the AMPL implementation. 

By looking at the results from the heuristics it deems possible to solve the complete problem by spending more time implementing. However, regardless if such results were reached it would only work for this specific library. A general solver would not be a feature that could be implemented. The reason is due to the uniqueness of requirements in the library. In case they were similar someone would, most likely, already have created a general solver for libraries or any other service institution.
