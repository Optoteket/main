% Define block styles
\tikzstyle{decision} = [diamond, draw, fill=blue!20, 
    text width=4.5em, text badly centered, inner sep=0pt]
\tikzstyle{block} = [rectangle, draw, fill=blue!20, 
    text width=5em, text centered, rounded corners, minimum height=4em]
\tikzstyle{line} = [draw, -latex']
\tikzstyle{cloud} = [draw, ellipse,fill=red!20, node distance=3cm,
    minimum height=2em]

%The two heuristical approaches

Apart from the implementation of the AMPL model we have also created two separate solvers of the problem using heuristics. One method was decided to be based on creating a pool of all week appearances that exists. Each week appearance contains a unique set of tasks during all seven days. After creating the pool of blocks these are assigned to the workers based on costs, one for each week. The other method ...

\section{Weekly scheduling approach} \label{Weekly}
In Figure \ref{flow_chart}

% % FLOW CHART
\begin{figure}[!h]
  \caption{A flow chart of the implemented heuristic}
  \centering
	\scalebox{0.85}{ \label{flow_chart}
		\begin{tikzpicture}[node distance = 2cm, auto]
		    % Place nodes
			\node [block] (blocks) {Create all blocks};
			\node [block,below of= blocks] (workers) {Create all workers};
			\node [block,below of= workers] (sort) {Sort blocks to workers};
			\node [block,below of= sort] (assign_rot) {Assign rotations};
			\node [block,below of= assign_rot] (assign_low) {Assign LoW schedule};
			\node [block,below of= assign_low] (init_sol) {Create initial solution};
			\node [block,right of= init_sol, node distance = 4cm] (destroy) {Destroy workers};
			\node [block,below of= destroy] (repair) {Repair workers};
			\node [decision, below of= repair, node distance= 2.5cm] (feasible) {Is solution feasible?};
			\node [block,below of= feasible, node distance = 6cm] (print) {Print solution to file};
			
			\node[decision,right of= feasible, node distance=3.5cm] (close) {Close enough?};
			\node[block,below of= close, node distance = 3cm] (final) {Enter final phase};
			\node[decision, right of= final, node distance = 4cm] (solved) {Solved in given number of iterations?};
			\node[block,right of= assign_low, node distance = 11.5cm] (failed) {Failed iteration, run again};
			\node[right of= assign_low, node distance = 5.5cm] (text) {LNS};
			
			%Invisible node, useful later
%			\node[right of= destroy, node distance=2cm,scale=0.01](inv){};
			
		    % Draw edges
		    \path [line] (blocks) -- (workers);
		    \path [line] (workers) -- (sort);
		    \path [line] (sort) -- (assign_rot);
		    \path [line] (assign_rot) -- (assign_low);
		    \path [line] (assign_low) -- (init_sol);
		    \path [line] (init_sol) |- (feasible);
		    \path [line] (destroy) -- (repair);
		    \path [line] (repair) -- (feasible);
		    \path [line] (feasible) -- node[right]{yes}(print);
		    \path [line] (feasible) -- node[above]{no}(close);
		    \path [line] (close) |- node[right]{no}(destroy);
		    \path [line] (close) -- node[right]{yes}(final);
		    \path [line] (final) -- (solved);
		    \path [line] (solved) |- node[right]{yes}(print);
		    \path [line] (solved) -- node[right]{no}(failed);
		    \path [line] (failed) |- (blocks);
%			\path[-,draw] (feasible) -| node[right]{no} (inv.north);
%		    \path[line]{} (inv.north) |- node{} (destroy);
			\draw [color=gray!70,thick](2.3,-8.5) rectangle(6,-17);
		\end{tikzpicture}
		}
\end{figure}

\subsection{Block creation} \label{block_creation}
A big part of this heuristic is to create the big pool of unique week appearances. These are then filtered for each of the worker based on its availability. The workers availability are generalized into three categories: \textit{Weekend week}, \textit{weekrest week} and \textit{weekday week}, where weekday week occurs three times during a five week period, see Table \ref{tab:Bob_avail}. 


One week block contains seven days and up to four shifts where each shift can contain up to three task types. Table \ref{Generalized weekblock} below is a representation of a general week block with all possible tasks for each day \textit{d} and shift \textit{s}. \textit{I} in the table represents "No task that day", \textit{D} represents "Desk task" meaning either Exp or Info, \textit{PL} represents "Fetch list" and \textit{HB} represents "Hageby". 

\begin{table}[!h]
\centering
\caption{A generalized weekblock with all existing tasks}
\label{Generalized weekblock}
\begin{tabular}{cccccccc}
                         & Mon                         & Tue                         & Wed                         & Thu                         & Fri                         & Sat                         & Sun                         \\ \cline{2-8} 
\multicolumn{1}{c|}{8-10} & \multicolumn{1}{c|}{I,D,PL} & \multicolumn{1}{c|}{I,D,PL} & \multicolumn{1}{c|}{I,D,PL} & \multicolumn{1}{c|}{I,D,PL} & \multicolumn{1}{c|}{I,D,PL} & \multicolumn{1}{c|}{I,D,HB} & \multicolumn{1}{c|}{I,D,HB} \\ \cline{2-8} 
\multicolumn{1}{c|}{10-13} & \multicolumn{1}{c|}{D}      & \multicolumn{1}{c|}{D}      & \multicolumn{1}{c|}{D}      & \multicolumn{1}{c|}{D}      & \multicolumn{1}{c|}{D}      &       \\ \cline{2-6} 
\multicolumn{1}{c|}{13-16} & \multicolumn{1}{c|}{D}      & \multicolumn{1}{c|}{D}      & \multicolumn{1}{c|}{D}      & \multicolumn{1}{c|}{D}      & \multicolumn{1}{c|}{D}      &       \\ \cline{2-6} 
\multicolumn{1}{c|}{16-20} & \multicolumn{1}{c|}{D}      & \multicolumn{1}{c|}{D}      & \multicolumn{1}{c|}{D}      & \multicolumn{1}{c|}{D}      & \multicolumn{1}{c|}{D}      &       \\ \cline{2-6} 
\end{tabular}
\end{table}

Every day must contain exactly \textit{one} task from either of the four shifts when creating a week block. The tasks \textit{I} and \textit{PL} are ranging more than one shift. The duration of a PL is three shift and the I refers to the entire day. Hence, they are both placed in the first shift. When creating the combinations of block appearances there are additional conditions that have to be met:
\begin{enumerate}  
\item At most two tasks can be assigned the same shift and week.\label{first_item}
\item No more than two evenings are allowed each week, one of which is required to be a Friday. \label{second_item}
\item At most one PL is allowed in a weekblock. \label{third_item}
\item Saturday and Sunday shall always contain the same task type, as well as if they contain a Desk task then Friday at shift four must also contain a Desk task.\label{fourth_item}
\item No more than four tasks are allowed during the weekdays leaving at least one day without tasks. \label{fifth_item}
\end{enumerate}

If item \ref{first_item}, \ref{second_item}, \ref{third_item} and \ref{fifth_item} is disregarded there exists $6^5*3 = 23,328$ unique weekblocks. However, if Exp and Info were to be used instead of the combination of the two, the possible combinations would be significantly higher, namely $10^5*4 = 400,000$. By applying all conditions the total amount of unique block appearances for this implementation are 4,175.

An illustration of one of the 4,175 existing block can be seen in Table \ref{block_example} below.
\begin{table}[!h]
\centering
\caption{Illustration of one of the unique block appearances}
\label{block_example}
\begin{tabular}{cccccccc}
                           & Mon                                            & Tue                                             & Wed                    & Thu                                            & Fri                    & Sat                                             & Sun                                             \\ \cline{2-8} 
\multicolumn{1}{c|}{8-10}  & \multicolumn{1}{c|}{}                          & \multicolumn{1}{c|}{\cellcolor[HTML]{FCFF2F}PL} & \multicolumn{1}{c|}{I} & \multicolumn{1}{c|}{}                          & \multicolumn{1}{c|}{I} & \multicolumn{1}{c|}{\cellcolor[HTML]{FCFF2F}HB} & \multicolumn{1}{c|}{\cellcolor[HTML]{FCFF2F}HB} \\ \cline{2-8} 
\multicolumn{1}{c|}{10-13} & \multicolumn{1}{c|}{}                          & \multicolumn{1}{c|}{\cellcolor[HTML]{FCFF2F}}   & \multicolumn{1}{c|}{}  & \multicolumn{1}{c|}{}                          & \multicolumn{1}{c|}{}  &                                                 &                                                 \\ \cline{2-6}
\multicolumn{1}{c|}{13-16} & \multicolumn{1}{c|}{}                          & \multicolumn{1}{c|}{\cellcolor[HTML]{FCFF2F}}   & \multicolumn{1}{c|}{}  & \multicolumn{1}{c|}{\cellcolor[HTML]{FCFF2F}D} & \multicolumn{1}{c|}{}  &                                                 &                                                 \\ \cline{2-6}
\multicolumn{1}{c|}{16-20} & \multicolumn{1}{c|}{\cellcolor[HTML]{FCFF2F}D} & \multicolumn{1}{c|}{}                           & \multicolumn{1}{c|}{}  & \multicolumn{1}{c|}{}                          & \multicolumn{1}{c|}{}  &                                                 &                                                 \\ \cline{2-6}
\end{tabular}
\end{table}

If a worker is assigned this block the worker will work in the Expedition desk or Information desk Monday and Thursday at 16-20 and 14-16 respectively. Also the worker will work with the Fetch list from 8-16 on Tuesday and be in Hageby during the weekend.


\subsection{Block percolation}
After creating all existing week combinations they are percolated to each of the workers based on their availability in each of the three categories mentioned in Section \ref{block_creation}. Table \ref{blocks_available_per_worker} below shows the results after this percolation has been made for all workers.

\begin{table}[!h]
\centering
\caption{Number of assignable block combinations for the workers based on their availability.}
\label{blocks_available_per_worker}
\begin{tabular}{|c|ccc|}
\hline
Worker & Weekend & Weekrest & Weekday \\ \hline
1      & 532     & 347      & 1580    \\ \hline
2      & 1580    & 1580     & 1580    \\ \hline
3      & 1063    & 347      & 1580    \\ \hline
4      & 557     & 165      & 779     \\ \hline
5      & 261     & 130      & 531     \\ \hline
6      & 532     & 130      & 1580    \\ \hline
7      & 261     & 130      & 531     \\ \hline
8      & 261     & 29       & 531     \\ \hline
9      & 115     & 12       & 247     \\ \hline
10     & 532     & 130      & 1580    \\ \hline
11     & 9       & 8        & 8       \\ \hline
12     & 1063    & 347      & 1580    \\ \hline
13     & 771     & 92       & 1190    \\ \hline
14     & 265     & 29       & 489     \\ \hline
15     & 51      & 18       & 120     \\ \hline
16     & 495     & 130      & 843     \\ \hline
17     & 237     & 69       & 267     \\ \hline
18     & 532     & 279      & 1580    \\ \hline
19     & 495     & 47       & 843     \\ \hline
20     & 532     & 130      & 1580    \\ \hline
21     & 227     & 227      & 227     \\ \hline
22     & 2       & 1        & 1       \\ \hline
23     & 3       & 2        & 2       \\ \hline
24     & 11      & 5        & 47      \\ \hline
25     & 1063    & 279      & 1580    \\ \hline
26     & 5       & 4        & 4       \\ \hline
27     & 213     & 106      & 425     \\ \hline
28     & 2       & 2        & 2       \\ \hline
29     & 426     & 106      & 1281    \\ \hline
30     & 127     & 126      & 455     \\ \hline
31     & 495     & 130      & 843     \\ \hline
32     & 261     & 29       & 531     \\ \hline
33     & 72      & 45       & 306     \\ \hline
34     & 425     & 425      & 425     \\ \hline
35     & 91      & 20       & 221     \\ \hline
36     & 55      & 27       & 101     \\ \hline
37     & 1063    & 347      & 1580    \\ \hline
38     & 3       & 1        & 1       \\ \hline
39     & 2       & 1        & 1       \\ \hline
\end{tabular}
\end{table}

All of these values are a subset of the total amount 4,175. Knowing the structure of the problem, one can see that the amount of available weekrest blocks are always less than or equal to the available weekday blocks, as it should be. The only difference between the two mentioned block categories is when a worker is free from work due to its weekrest. Therefore, using this information one can interpret that when they are equal the worker is never working weekends.


\begin{table}[!h]
\centering
\caption{Typical availability for a generalized worker. Yellow signifies that the worker is available. In parenthesis, the weekend shift hours.}
\label{typical_availability}
\begin{tabularx}{\textwidth}{|X|l|l|l|l|l|l|l|X|}
\hline
%-------------------------------------------------------------------
\textbf{Weekend week}& \colcell \textbf{Mon} & \colcell \textbf{Tue} & \colcell \textbf{Wed} & \colcell \textbf{Thu} & \colcell \textbf{Fri} & \colcell \textbf{Sat} & \colcell \textbf{Sun}
\\ \hline 
%%------------------------------------------------------------------- 
%\rowcolor{Gray} 
\colcell 08:00-10:00 (11:00-16:00) & \colcelltwo & \colcelltwo & \colcelltwo & \colcelltwo & \colcelltwo & \colcelltwo & \colcelltwo
\\ \hline 
%%-------------------------------------------------------------------
%\rowcolor{Gray} 
\colcell 10:00-13:00 & \colcelltwo & \colcelltwo & \colcelltwo & \colcelltwo & \colcelltwo &   & 
\\ \hline 
%%-------------------------------------------------------------------
%\rowcolor{Gray} 
\colcell 13:00-16:00 & \colcelltwo & \colcelltwo & \colcelltwo & \colcelltwo & \colcelltwo & &
\\ \hline 
%%-------------------------------------------------------------------
%\rowcolor{Gray} 
\colcell 16:00-20:00 & & & \colcelltwo & & \colcelltwo & &
\\ \hline 
%%-------------------------------------------------------------------
\end{tabularx}
\begin{tabularx}{\textwidth}{|X|l|l|l|l|l|l|l|X|}
\hline
%-------------------------------------------------------------------
\textbf{Weekrest week}& \colcell \textbf{Mon} & \colcell \textbf{Tue} & \colcell \textbf{Wed} & \colcell \textbf{Thu} & \colcell \textbf{Fri} & \colcell \textbf{Sat} & \colcell \textbf{Sun}
\\ \hline 
%%------------------------------------------------------------------- 
%\rowcolor{Gray} 
\colcell 08:00-10:00 (11:00-16:00) & \colcelltwo & \colcelltwo & \colcelltwo & & & & 
\\ \hline 
%%-------------------------------------------------------------------
%\rowcolor{Gray} 
\colcell 10:00-13:00 & \colcelltwo & \colcelltwo & \colcelltwo & & & & 
\\ \hline 
%%-------------------------------------------------------------------
%\rowcolor{Gray} 
\colcell 13:00-16:00 & \colcelltwo & \colcelltwo & \colcelltwo & & & &
\\ \hline 
%%-------------------------------------------------------------------
%\rowcolor{Gray} 
\colcell 16:00-20:00 & & & \colcelltwo & & & &
\\ \hline 
%%-------------------------------------------------------------------
\end{tabularx}
\begin{tabularx}{\textwidth}{|X|l|l|l|l|l|l|l|X|}
\hline
%-------------------------------------------------------------------
\textbf{Weekday week}& \colcell \textbf{Mon} & \colcell \textbf{Tue} & \colcell \textbf{Wed} & \colcell \textbf{Thu} & \colcell \textbf{Fri} & \colcell \textbf{Sat} & \colcell \textbf{Sun}
\\ \hline 
%%------------------------------------------------------------------- 
%\rowcolor{Gray} 
\colcell 08:00-10:00 (11:00-16:00) & \colcelltwo & \colcelltwo & \colcelltwo & \colcelltwo & \colcelltwo & & 
\\ \hline 
%%-------------------------------------------------------------------
%\rowcolor{Gray} 
\colcell 10:00-13:00 & \colcelltwo & \colcelltwo & \colcelltwo & \colcelltwo & \colcelltwo &   & 
\\ \hline 
%%-------------------------------------------------------------------
%\rowcolor{Gray} 
\colcell 13:00-16:00 & \colcelltwo & \colcelltwo & \colcelltwo & \colcelltwo & \colcelltwo & &
\\ \hline 
%%-------------------------------------------------------------------
%\rowcolor{Gray} 
\colcell 16:00-20:00 & & & \colcelltwo & & & &
\\ \hline 
%%-------------------------------------------------------------------
\end{tabularx}
\begin{tabularx}{\textwidth}{|X|l|l|l|l|l|l|l|X|}
\hline
%-------------------------------------------------------------------
\textbf{Weekday week}& \colcell \textbf{Mon} & \colcell \textbf{Tue} & \colcell \textbf{Wed} & \colcell \textbf{Thu} & \colcell \textbf{Fri} & \colcell \textbf{Sat} & \colcell \textbf{Sun}
\\ \hline 
%%------------------------------------------------------------------- 
%\rowcolor{Gray} 
\colcell 08:00-10:00 (11:00-16:00) & \colcelltwo & \colcelltwo & \colcelltwo & \colcelltwo & \colcelltwo & & 
\\ \hline 
%%-------------------------------------------------------------------
%\rowcolor{Gray} 
\colcell 10:00-13:00 & \colcelltwo & \colcelltwo & \colcelltwo & \colcelltwo & \colcelltwo &   & 
\\ \hline 
%%-------------------------------------------------------------------
%\rowcolor{Gray} 
\colcell 13:00-16:00 & \colcelltwo & \colcelltwo & \colcelltwo & \colcelltwo & \colcelltwo & &
\\ \hline 
%%-------------------------------------------------------------------
%\rowcolor{Gray} 
\colcell 16:00-20:00 & & & \colcelltwo & & & &
\\ \hline 
%%-------------------------------------------------------------------
\end{tabularx}
\begin{tabularx}{\textwidth}{|X|l|l|l|l|l|l|l|X|}
\hline
%-------------------------------------------------------------------
\textbf{Weekday week}& \colcell \textbf{Mon} & \colcell \textbf{Tue} & \colcell \textbf{Wed} & \colcell \textbf{Thu} & \colcell \textbf{Fri} & \colcell \textbf{Sat} & \colcell \textbf{Sun}
\\ \hline 
%%------------------------------------------------------------------- 
%\rowcolor{Gray} 
\colcell 08:00-10:00 (11:00-16:00) & \colcelltwo & \colcelltwo & \colcelltwo & \colcelltwo & \colcelltwo & & 
\\ \hline 
%%-------------------------------------------------------------------
%\rowcolor{Gray} 
\colcell 10:00-13:00 & \colcelltwo & \colcelltwo & \colcelltwo & \colcelltwo & \colcelltwo &   & 
\\ \hline 
%%-------------------------------------------------------------------
%\rowcolor{Gray} 
\colcell 13:00-16:00 & \colcelltwo & \colcelltwo & \colcelltwo & \colcelltwo & \colcelltwo & &
\\ \hline 
%%-------------------------------------------------------------------
%\rowcolor{Gray} 
\colcell 16:00-20:00 & & & \colcelltwo & & & &
\\ \hline 
%%-------------------------------------------------------------------
\end{tabularx}
\end{table} 

Looking at Table \ref{typical_availability} one might think that there shall be more weekend blocks available than weekday blocks as the availability, almost in all cases, are higher for weekend blocks. This is not the case as all blocks without weekend tasks are removed in the percolation. This means that all combinations with "No task" on weekends are removed, as well as the case that is shown in Table \ref{Friday_percolation}, which occur when a worker is assigned weekend Desk tasks and therefore can not be assigned any other task that Friday. "X" in Table \ref{Friday_percolation} means any task that day and shifts colored in black means no other task can be assigned that day. However, there are a few cases when a worker's weekrest occurs at two consecutive Fridays starting with the weekend week leading to less availability during a weekend week than a weekday week. % % % % % % % % % % Korrekturläs och ändra...

\begin{table}[!h]
\centering
\caption{My caption}
\label{Friday_percolation}
\begin{tabular}{cccccccc}
                                 & Mon                    & Tue                    & Wed                    & Thu                    & Fri                                            & Sat                                            & Sun                                            \\ \cline{2-8} 
\multicolumn{1}{c|}{08:00-10:00} & \multicolumn{1}{c|}{X} & \multicolumn{1}{c|}{X} & \multicolumn{1}{c|}{X} & \multicolumn{1}{c|}{X} & \multicolumn{1}{c|}{\cellcolor[HTML]{000000}}  & \multicolumn{1}{c|}{\cellcolor[HTML]{FCFF2F}D} & \multicolumn{1}{c|}{\cellcolor[HTML]{FCFF2F}D} \\ \cline{2-8} 
\multicolumn{1}{c|}{10:00-13:00} & \multicolumn{1}{c|}{}  & \multicolumn{1}{c|}{}  & \multicolumn{1}{c|}{}  & \multicolumn{1}{c|}{}  & \multicolumn{1}{c|}{\cellcolor[HTML]{000000}}  &                                                &                                                \\ \cline{2-6}
\multicolumn{1}{c|}{13:00-16:00} & \multicolumn{1}{c|}{}  & \multicolumn{1}{c|}{}  & \multicolumn{1}{c|}{}  & \multicolumn{1}{c|}{}  & \multicolumn{1}{c|}{\cellcolor[HTML]{000000}}  &                                                &                                                \\ \cline{2-6}
\multicolumn{1}{c|}{16:00-20:00} & \multicolumn{1}{c|}{}  & \multicolumn{1}{c|}{}  & \multicolumn{1}{c|}{}  & \multicolumn{1}{c|}{}  & \multicolumn{1}{c|}{\cellcolor[HTML]{FCFF2F}D} &                                                &                                                \\ \cline{2-6}
\end{tabular}
\end{table}


\subsection{Rotation assignment} \label{rotation}


\subsection{Assignment of Library on Wheels} \label{LoW_assignment}
In order to avoid creating enormous amounts of block combinations, LoW tasks are assigned manually. To assign LoW manually also lead to a fix weekend rotation for the five LoW-workers. This slightly reduces the degrees of freedom of the problem as the LoW-workers' rotation will remain unchanged. However, as there are only a couple different set of feasible LoW assignments it shall not be the deciding factor for the quality of the solution.

Without a manual assignment of LoW the number of existing tasks in a weekblock would increase significantly. The number of tasks would increase from 36 to 43, as there are seven LoW tasks during a week, resulting in much more than 4,175 unique week appearances. % %ändra sista

\subsection{Initial solution} \label{initial_solution}
The initial solution is created in a similar fashion as the repair function in this heuristic. Based on a greedy heuristic the best weekblock for a random worker and week is found and inserted. This is done until every worker have one weekend block, one weekrest block and three weekday blocks assigned to them.

To find the best weekblock several costs have been introduced to measure whether a block is good or bad to assign a worker. Say, if the library demands two librarians at the Information desk Monday at 08:00 and currently there are one assigned, then it would be good to assign another one. Such an assignment will, therefore, be rewarded using a cost. Good assignments will provide negative costs and bad assignments will provide positive costs to the objective function. Table \ref{block_to_evaluate} together with Figure \ref{fig_costs} shows an increment where different cost parameters have to be considered when evaluating a block before assigning it to a worker. 

% Please add the following required packages to your document preamble:
% \usepackage[table,xcdraw]{xcolor}
% If you use beamer only pass "xcolor=table" option, i.e. \documentclass[xcolor=table]{beamer}
\begin{table}[]
\centering
\caption{A block example to be evaluated using costs}
\label{my-label}
\begin{tabular}{llllllll}
\multicolumn{1}{c}{}             & \multicolumn{1}{c}{Mon}                         & \multicolumn{1}{c}{Tue} & \multicolumn{1}{c}{Wed}                        & \multicolumn{1}{c}{Thu} & \multicolumn{1}{c}{Fri}                        & \multicolumn{1}{c}{Sat} & \multicolumn{1}{c}{Sun} \\ \cline{2-8} 
\multicolumn{1}{c|}{08:00-10:00} & \multicolumn{1}{c|}{\cellcolor[HTML]{FCFF2F}PL} & \multicolumn{1}{c|}{I}  & \multicolumn{1}{c|}{}                          & \multicolumn{1}{c|}{I}  & \multicolumn{1}{c|}{}                          & \multicolumn{1}{c|}{I}  & \multicolumn{1}{c|}{I}  \\ \cline{2-8} 
\multicolumn{1}{c|}{10:00-13:00} & \multicolumn{1}{c|}{\cellcolor[HTML]{FCFF2F}}   & \multicolumn{1}{c|}{}   & \multicolumn{1}{c|}{}                          & \multicolumn{1}{c|}{}   & \multicolumn{1}{c|}{\cellcolor[HTML]{FCFF2F}D} & \multicolumn{1}{c}{}    & \multicolumn{1}{c}{}    \\ \cline{2-6}
\multicolumn{1}{c|}{13:00-16:00} & \multicolumn{1}{c|}{\cellcolor[HTML]{FCFF2F}}   & \multicolumn{1}{c|}{}   & \multicolumn{1}{c|}{\cellcolor[HTML]{FCFF2F}D} & \multicolumn{1}{c|}{}   & \multicolumn{1}{c|}{}                          & \multicolumn{1}{c}{}    & \multicolumn{1}{c}{}    \\ \cline{2-6}
\multicolumn{1}{c|}{16:00-20:00} & \multicolumn{1}{c|}{}                           & \multicolumn{1}{c|}{}   & \multicolumn{1}{c|}{}                          & \multicolumn{1}{c|}{}   & \multicolumn{1}{c|}{}                          & \multicolumn{1}{c}{}    & \multicolumn{1}{c}{}    \\ \cline{2-6}
                                 &                                                 &                         &                                                &                         &                                                &                         &                         \\
                                 & Mon                                             &                         &                                                &                         &                                                &                         &                         \\ \cline{2-2}
\multicolumn{1}{l|}{08:00-10:00} & \multicolumn{1}{l|}{\cellcolor[HTML]{FCFF2F}PL} &                         &                                                &                         &                                                &                         &                         \\ \cline{2-2}
\multicolumn{1}{l|}{10:00-13:00} & \multicolumn{1}{l|}{\cellcolor[HTML]{FCFF2F}}   &                         &                                                &                         &                                                &                         &                         \\ \cline{2-2}
\multicolumn{1}{l|}{13:00-16:00} & \multicolumn{1}{l|}{\cellcolor[HTML]{FCFF2F}}   &                         &                                                &                         &                                                &                         &                         \\ \cline{2-2}
\end{tabular}
\end{table}



    


\subsection{Costs}
In order to find a feasible solution many of the implemented costs must be carefully chosen. Presently, there are 16 existing costs where some of them are correlated. 

\section{Individual task scheduling approach}\label{Individual}
Text
