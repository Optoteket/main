% Master Thesis Report
% Created by: Claes Arvidson and Emelie Karlsson 
% Linköping University Spring 2016
% First version: 2016-01-27

\documentclass[a4paper, 10pt, twoside, openright]{book}
\pagestyle{headings}

% some of the packages used are
\usepackage[round, authoryear]{natbib}
\bibliographystyle{unsrtnat}
\usepackage{xcolor, colortbl}
\usepackage{tabularx}
\usepackage{latexsym}
\usepackage{eepic}
\usepackage{makeidx}
\usepackage{graphicx}
\usepackage[utf8]{inputenc}  %för ä ö å ?
\usepackage[english]{babel}
%\usepackage{times}
\usepackage{amssymb}
\usepackage{fancybox} 
\usepackage{textcomp}
\usepackage{float}
\usepackage{amsmath}
\usepackage{mathtools}

% the fancy header/footer
% consult http://research.cmis.csiro.au/gjw/tex/docs/fancyhdr.pdf
% or some other fancy header documenatation for more info
\usepackage{fancyhdr}
\pagestyle{fancy}
\lfoot{\nouppercase{\rightmark}}
\rfoot{\nouppercase{\leftmark}}
\fancyfoot{} 
\fancyhead[LO]{\nouppercase{\small{\rightmark}}}
\fancyhead[RE]{\nouppercase{\small{\leftmark}}}
\fancyhead[LE,RO]{\small{\thepage}}
\fancyhead[C]{}
\renewcommand{\footrulewidth}{0pt}
\renewcommand{\headrulewidth}{0.2pt}
\fancypagestyle{plain}{ \fancyhf{}  
  \fancyfoot[RE,LO]{\nouppercase{\small{\putegotext}}}
  \fancyfoot[LE,RO]{\small{\thepage}}
  \renewcommand{\headrulewidth}{0pt}  \renewcommand{\footrulewidth}{0.2pt} }

\newcommand\mydots{\hbox to 1em{.\hss.\hss.}}
% ------------------------------------------------------------------------
% BEGINNING of abstract, author, examiner, etc
% here you must fill in your own text. Remember to respect the { and the }.
%
% the *-marked % in 
%                                   *
%      \newcommand{\putabstract}[0]{% 
%      some text}
%
% is made to avoid a blank space in the beginning of the abstract,
% author, date etc.
%
\newcommand{\putabstract}[0]{%
Here is where you can write your abstract. It may be very long, or it
may be very short, the reason you have an abstract is for people not
to be forced to read lots of crap.

But still, they will have to read your abstract. After all, the
abstract is what everyone reads\ldots}

\newcommand{\putkeywords}[0]{%
Keyword One, Chemostat, Another Key-Word, Key, Clé, Mot de cle,
Nyckelhål, XBOX, Dagens viktigaste nyckelord, and Keywords.}

\newcommand{\putthemonth}[0]{June }

\newcommand{\putshortdate}[0]{2016}

\newcommand{\putmydate}[0]{\putthemonth \putshortdate}

\newcommand{\putauthor}[0]{Claes Arvidson, Emelie Karlsson}

\newcommand{\putegotext}[0]{Arvidson, Karlsson, 2016.}

\newcommand{\puttitle}[0]{Work Distribution of a Heterogeneous Library Staff - A Personnel Task Scheduling Problem}

% some kind of administrator will give you this number
\newcommand{\putregnumber}[0]{LiTH - MAT - EX - - 04 / 04 - - SE}

\newcommand{\putexaminer}[0]{E. Rönnberg}

\newcommand{\putsupervisor}[0]{T. Larsson}

% for example Applied Mathematics
\newcommand{\putdepartment}[0]{Optimeringslära}

% send an email to ep.liu.se to make sure it is correct before you print
% this number is the one I had
\newcommand{\putmyurl}[0]{http://www.ep.liu.se/exjobb/mai/2004/tm/004/}

% Please do not change into University of Linköping or something, the
% international name of Liu is Linköpings Universitet
\newcommand{\putliu}[0]{Linköpings Universitet}

%Create the command \tab
\newcommand{\tab}[1]{\hspace{.2\textwidth}\rlap{#1}}
\newcommand{\itab}[1]{\hspace{0em}\rlap{#1}}
\DeclarePairedDelimiter\abs{\lvert}{\rvert}
\DeclarePairedDelimiter\norm{\lVert}{\rVert}
%
% END of abstract, author, examiner, etc
% ------------------------------------------------------------------------

\begin{document} 

\frontmatter
\raggedbottom
% some new commands I used, you won't need them I guess- just erase it
\newcommand{\ala}[0]{{\alpha}_{1}}
\newcommand{\alb}[0]{{\alpha}_{2}}
\newcommand{\nc}[0]{{N}_{C}}
\newcommand{\tCh}[0]{the Chemostat }
\newcommand{\Ch}[0]{Chemostat }
\newcommand{\ch}[0]{chemostat }
\newcommand{\MMk}[0]{Michaelis-Menten kinetics }
\newcommand{\KI}[0]{{K}_{I} }
\newcommand{\KN}[0]{{K}_{N} }
\newcommand{\mmu}[0]{{\mu}_{max} }
\newcommand{\KMAX}[0]{\mmu}
\newcommand{\AAA}[0]{\mathbf{A} }
% stop erasing here

%\usepackage[first=0,last=9]{lcg}
%\newcommand{\ra}{\rand0.\arabic{rand}}

% Colors
\newcommand{\enc}[0]{utf8}
\definecolor{bluegray}{rgb}{0.4, 0.6, 0.8}
\definecolor{darkcyan}{rgb}{0.0, 0.55, 0.55}
\definecolor{corn}{rgb}{0.98, 0.93, 0.36}
\definecolor{coralred}{rgb}{1.0, 0.25, 0.25}
\definecolor{Gray}{gray}{0.85}
\newcommand{\colcell}{\cellcolor{Gray}}
\newcommand{\colcelltwo}{\cellcolor{corn}}
\newcommand{\colcellthree}{\cellcolor{darkcyan}}
\newcolumntype{g}{>{\columncolor{corn}}c}

% Special cells
\newcommand{\specialcell}[2][c]{%
  \begin{tabular}[#1]{@{}l@{}}#2\end{tabular}}
\newcommand{\specialcelltwo}[2][c]{%
   \begin{tabular}[#1]{@{}l@{}l@{}}#2\end{tabular}}



% don't mind these
\title{\puttitle}
\author{\putauthor}

% this command makes the first page, and the second page, handle with care
% You really don't have to change anything here I guess.
\renewcommand{\maketitle}{%
              \newpage%
              \pagestyle{empty}
              \null%
              \hfil\hspace*{-3mm}
              \begin{minipage}{150mm}
                \center 
                {\Large{Master thesis}} \\\vspace*{4mm}
                {\vbox to 22mm{\vfil\Large\textbf{\puttitle}}}
                \vspace*{5mm}
                 {\Large{\putauthor}}       \\ \vspace*{4mm}
                  \putregnumber\\
                \end{minipage}\hfil
              \clearpage%\hspace*{0mm}\thispagestyle{empty}
              \clearpage%\thispagestyle{empty}
              \null%
              \hfil\hspace*{-3mm}
              \begin{minipage}{150mm}
                \center
                {\vbox to 48mm{\vfil\Large\textbf{\puttitle}}}
                \vspace*{5mm}
                  \putdepartment, \putliu \\ \vspace*{4mm}
                  \textbf{\putauthor}       \\ \vspace*{4mm}
                  \putregnumber
                  \vspace*{100mm}\\
                \flushleft
                
                Exam work:\hspace*{3pt}
                \begin{minipage}[t]{70mm}
% but you might want to change this 30
                  \textbf{30 hp}
                \end{minipage} \\ \vspace*{4mm}
                
                Level:\hspace*{3pt}
                \begin{minipage}[t]{70mm}
% and you might want to change this A 
                  \textbf{A} 
                \end{minipage} \\ \vspace*{4mm}
                
                Supervisor:\hspace*{3pt}
                \begin{minipage}[t]{120mm}
                  \textbf{\putsupervisor}, \\\putdepartment, \putliu
                \end{minipage} \\ \vspace*{4mm}
                
                Examiner:\hspace*{3pt}
                \begin{minipage}[t]{120mm}
                  \textbf{\putexaminer}, \\\putdepartment, \putliu
                \end{minipage} \\ \vspace*{4mm}
                
                Linköping:
                \begin{minipage}[t]{70mm}
                  \textbf{\putmydate}
                \end{minipage} \\ \vspace*{4mm}
              \end{minipage} \\ \hfill
}

% now the \maketitle command is changed and we use the new one
\maketitle

% I'm not sure this phantom is needed, but I leave to make sure
% a phantom leaves empty space corresponding to the size of what is in it.
\phantom{crap}
\thispagestyle{empty}
\pagestyle{empty}

\setlength{\unitlength}{1mm}


\setlength{\unitlength}{1pt}

% now turn on fancy headers
\pagestyle{fancy}

% ..................chapter ...............................................
\chapter*{Abstract\index{Abstract}}
% your abstract
\putabstract

% enters keywords
\begin{description}
\item[Keywords:]{%
\putkeywords
}
\item[URL for electronic version: ]{\hfill%\quad\\%
%\phantom{\qquad} http://urn.kb.se/resolve?urn=urn:nbn:se:liu:diva-77777\\
%where 77777 should be replacd by an appropriate number.
\begin{center}
http://urn.kb.se/resolve?urn=urn:nbn:se:liu:diva-77777
\end{center}
%where 77777 should be replaced by an appropriate number.
}
\end{description}


% here or on the next page you put an abstract in swedish
%\section*{Abstract in Swedish: Sammanfattning}
%Det finns många typer av bioreaktorer som tillämpas\ldots

\chapter*{Acknowledgements\index{Acknowledgements}}

I would like to thank my supervisor, I would like to thank my
supervisor, I would like to thank my supervisor, I would like to thank my supervisor\ldots

I also have to thank, I would like to thank my supervisor, I would
like to thank my supervisor, I would like to thank my supervisor, I
would like to thank my supervisor\ldots

My opponent NN also deserves my thanks, I would like to thank my
supervisor, I would like to thank my supervisor, I would like to thank my supervisor\ldots





% optional I guess
\chapter*{Nomenclature\index{Nomenclature}}

Most of the reoccurring definitions, symbols and abbreviations are described here.

%----------------NEW_SECTION--------------------------------------------------
\section*{Definitions\index{Definitions}}
\begin{tabular}{ll}
Plocklista & Text\\
Library on wheels & Text\\
Rostering & Text\\
Matheuristic & Text\\
Initial schedule (?)
Basic schedule (?)
\end{tabular}

\section*{Symbols\index{Symbols}}

%\begin{tabular}{ll}
%$Y_0$      & The amount of the variable $Y$ inserted into a system.\\
%$\hat Y$& The unit-dimension of the variable $Y$, for example $\hat t=1s$ .\\
%$\bar Y_i$ & A steady state (number $i$) value of Y.\\
%\phantom{a}& \phantom{b} \\
%$K_i$ & Constants used in kinetic expressions, for example $K_I$.\\
%\phantom{a}& \phantom{b} \\
%$\AAA$     & The system matrix. \\
%\end{tabular}

\section*{Abbreviations\index{Abbreviations}}

\begin{tabular}{ll}
Exp		& Text\\
Info	& Text\\
PL		& Text\\
PTSP	& Text\\
SMPTSP	& Text\\
\end{tabular}


\tableofcontents 
\begin{samepage}
\listoffigures
\let\clearpage\relax
\listoftables
\end{samepage}


\mainmatter

%##########################################################
% -------------------CHAPTERS -----------------------------
%##########################################################

\chapter{Introduction}

\inputencoding{\enc}
% Introduction to Master Thesis


%\section{Background} 
%Schemaläggning av personal vid Norrköpings bibliotek.
\section{Background}
At a library absence can cause problems, since the qualifications required to perform tasks varies. If a worker were to be unavailable a day due to a meeting or simply being ill it would require for a stand-in to fill the vacancy. Therefore, it is of great interest to have a schedule with as many skilled stand-ins as possible to overcome such disturbances. 


\section{Problem description}
The goal of this thesis is to distribute given tasks to the heterogeneous workforce at the library of Norrköping. Each task is either classified as an outer or an inner service where an outer service is when a librarian needs to interact with visitors. Inner services can in some rare cases require a predetermined person to be assigned to a specified time or day.

Demands and requests are to be fulfilled to the furthest extent possible. Weekends are included in the scheduling problem, which adds more constraints regarding the number of contiguous working days. However, the librarians are permitted a few exceptions from these laws regarding days of rest.

The main purpose of the thesis is to create a schedule robust enough to withstand absence, such that outer services always are assigned to a qualified and available worker. This is visualized as having a list of available stand-ins for each shift. 

There are a limited number of workers at the library and they make the resources that are to be distributed. Each individual has a set of \textit{skills} and \textit{competences}. Competences refer to the capability of being assigned the different outer services; Expedition, Norpan, Information desk, Library on wheels and Hageby as well as different inner services. The set of skills an individual can possess are described in Table \ref{int:1}. In total there are 39 workers available.

The outer services can be seen as assignments which requires available workers to be assigned to them. Each outer service is specified to a certain station, time and date. They also have a fix length and occur on a regular basis every ten weeks, which makes it possible to create a periodic schedule with a period of ten weeks. 


%Examensarbetet går ut på att lägga ett arbetsschema för personalen vid Norrköpings bibliotek. Problemet går i grund och botten ut på att fylla alla uppgifter på de stationer som tillhör bibliotekets utåtriktade verksamhet (så kallade yttre tjänst)  med personal av rätt kompetens samt samtidigt som personalen får tid över till övriga uppgifter (så kallad inre tjänst). Schemat som tas fram ska även uppfylla de regelverk och önskemål som finns kring personalens individuella scheman, till exempel de arbetstider som ingår i de olika tjänsterna. Då det även ingår helgarbete i personalens arbetsuppgifter ska lediga dagar fördelas enligt arbetsmiljölagen och de undantag från dessa lagar gällande veckovilan.

%Utöver detta ska schemat även medföra en robusthet så att störningar i den yttre tjänsten, i form av att personal blir sjuk eller uppbokad annonstädes, ska gå att avhjälpa med en reservlista. Denna reservlista består av de bibliotekarier och assistenter som inte har något pass tilldelat sig samt är tillgängliga under dagen. 

%Personalen på biblioteket är begränsad och utgör de resurser som finns att tillgå. Varje enskild personal har en uppsättning \textit{egenskaper} och \textit{kompetenser}. Kompetenser syftar på personalens förmåga att arbeta vid någon av de yttre stationerna; Expedition, Norpan, Informationsdisk, Bokbuss och Hageby samt några av de inre stationerna; inköp, katalogisering med mera. De egenskaper som identifierats hos en personal finns beskrivna i tabell \ref{int:1}. Totala arbetskraften består av 39 stycken arbetare på biblioteket.

%De yttre och inre uppgifterna kan ses som behov av personal som måste täckas av den personal som finns att tillgå. De olika yttre uppgifterna som behöver utföras inkluderar arbete vid olika stationer vid olika tidpunker och datum. Varje uppgift har en bestämd längd och återkommer regelbundet inom ett 10-veckorsinterval, vilket gör att ett rullande schema kan skapas med en period om tio veckor. 

\begin{table}[h]
\centering
\caption{Personnel}
\label{int:1}
\begin{tabular}{|l|l|}
%-------------------------------------------------------------------
\hline 
\textbf{Skills} & \textbf{Description} \\ \hline
%-------------------------------------------------------------------
Work degree & 0-100 \% 
\\ \hline 
%-------------------------------------------------------------------
Type of employment & Librarian/Assistant
\\ \hline 
%-------------------------------------------------------------------
Competence & Inner and outer services the worker is qualified for  
\\ \hline 
%-------------------------------------------------------------------
Weekly rest & Which days the worker has requested after working a weekend
\\ \hline 
%-------------------------------------------------------------------
Other requests & Does not work evenings etc.
\\ \hline 
%-------------------------------------------------------------------
\end{tabular}
\end{table} 

Furthermore, outer and a few inner services can be characterized by different properties, which are represented in Table \ref{int:2}. \\

\begin{table}[!h]
\caption{Outer and inner services}
\label{int:2}
\begin{tabular}{|l|l|}
%-------------------------------------------------------------------
\hline
\textbf{Outer service} & \textbf{Property} \\ \hline
%-------------------------------------------------------------------
 & Start time, end time, week and duration \
\\ \hline 
%-------------------------------------------------------------------
 & Station
\\ \hline 
%-------------------------------------------------------------------
 & Number of qualified librarians demanded
\\ \hline 
%-------------------------------------------------------------------
 & Number of qualified assistants demanded
\\ \hline 
%-------------------------------------------------------------------

\textbf{Inner service} & \textbf{Property} \\ \hline
%-------------------------------------------------------------------
 & Start time, end time, week and duration \
\\ \hline 
%-------------------------------------------------------------------
 & Type
\\ \hline 
%-------------------------------------------------------------------
 & Number of qualified librarians demanded
\\ \hline 
%-------------------------------------------------------------------
 & Number of qualified assistants demanded
\\ \hline 
%-------------------------------------------------------------------
\end{tabular}
\end{table}

In addition to the properties mentioned above, there are several requirements that have to be met. These can be divided into job, robust and other requirements and are listed in Table \ref{int:3} below. 

%Utöver de ovan nämnda resurserna och behoven, finns ett antal krav som ställs på hur schemat får utformas. Dessa kan delas upp i arbetsvillkor, robusthetskrav samt övriga krav och finns representerade i tabell \ref{int:3}.

\begin{table}[H]
\caption{Requirements}
\label{int:3}
\begin{tabular}{|l|l|}
%-------------------------------------------------------------------
\hline
\textbf{Job requirements} & \textbf{Description} \\ \hline
%-------------------------------------------------------------------
& A maximum of one outer service is to be distributed to each person and day.
\\ \hline 
%-------------------------------------------------------------------
& Remaining work time is individually distributed on assignments such as Övrig arbetstid distribueras självständig med uppgifter såsom exempelvis bokplock eller bokuppsättning.
\\ \hline
%-------------------------------------------------------------------
 & Helgarbete ska fördelas rättvist mellan de i personalen som är tillgängliga för helgarbete. 
\\ \hline 
%-------------------------------------------------------------------
 & Helgarbete innefattar arbete under fredag kväll, påföljande lördag och söndag.
\\ \hline 
%-------------------------------------------------------------------
 & Högst ett kvällspass per personal i veckan bortsett från den vecka helgarbete ska utföras.
\\ \hline 
%-------------------------------------------------------------------
 & Schemat ska upprepa sig var 10e vecka.
\\ \hline 
%-------------------------------------------------------------------
 & Varje arbetsvecka ska ha liknande struktur i största möjliga mån.
\\ \hline 
%-------------------------------------------------------------------

\textbf{Robust requirements} & \textbf{Description} \\ \hline
%-------------------------------------------------------------------
 & För varje yttre uppgift ska det finnas minst en reserv.
\\ \hline 
%-------------------------------------------------------------------
 & Reserverna ska vara av rätt kompetens för uppgiften de är reserver till.
\\ \hline 
%-------------------------------------------------------------------
 & Prioritet ligger i att det lägsta antalet reserver för en uppgift ska vara så hög som möjligt.
\\ \hline 
%-------------------------------------------------------------------

\textbf{Other requirements} & \textbf{Description} \\ \hline
%-------------------------------------------------------------------
 & Stormöte och avdelningsmöten ska vardera hållas en gång var femte vecka.
\\ \hline 
%-------------------------------------------------------------------
\end{tabular}
\end{table}
\medskip

%Den optimala schemat ska inte endast uppfylla kraven ovan, utan även 



\inputencoding{\enc}

\chapter{Literature review}

\inputencoding{\enc}
% Review of previous work in the field
Emelie

The scheduling problem has been studied since the 1950's as a mathematical optimization problem and involves creating a feasible and satisfactory schedule for either workers or machines performing tasks. According to Ernst et al. the complexity of the scheduling problem has not in itself become more advanced with time. However, the mathematical models used to solve the scheduling problems have become more realistic and refined. This together with more powerful computational methods, makes it possible today to solve scheduling problems in a more satisfactory way, taking into account softer values such as worker satisfaction and worker fatigue \cite{Ernst_2004}.

In the paper \cite{Ernst_2004} the scheduling problem is classified into different subcategories. A few relevant areas for our work include task based demand scheduling, days off scheduling, shift scheduling, tour scheduling and task assignment. 

Task based demand scheduling involves the process of distributing a fixed number of tasks which need to be performed over a workforce. The workforce can either be fixed, as in our case, or subject to minimization in the objective function.
	
Days off scheduling involves scheduling staff and assigning a days off as required by work time regulations. This problem is often found together with shift scheduling which involves choosing the most suitable shifts for a workforce. The combination of the two is called tour scheduling and will be discussed later in this report. The big difference between our problem and tour scheduling is that in our problem we are not allowed to choose the free days of the workers, only in what week to assign them, and there is no shift work.

The problem which is most similar to our problem is, however, task assignment. This problem and different variations of it will be discussed in section \ref{PTSP}.



\section{Tour Scheduling Problem with a heterogenous work force}\label{TSP}
Emelie

The Tour Scheduling Problem (TSP) involves creating work shifts with days off for a work force. According to Loucks and Jacobs, the vast majority of all tour scheduling problems up to 1991 were with a homogeneous workforce, that is under the assumption that any worker could perform any assigned task \cite{loucks_1991}. The authors discuss a tour scheduling problem where the objective is to construct weekly schedules for each worker which also show the specific task assignments. The problem is studied in the context of fast food restaurants, where certain personnel is qualified only for certain stations in the restaurant. In such industries, the demand of staff differs between different weekdays and different times of the day. Two things differ between workers, their availability for work and their qualification. Furthermore, the workers working times are not fixed, but are composed of a number of consecutive tasks assigned during a block of time during a day.

The representative problem studied in the article involves creating a one-week schedule for 40 workers in a fast food restaurant, available for eight different tasks with a seven-day, 128-hour workweek. Several synthetic problems are studied in the article, all, however with minimum shift lenght three hours, maximum shift length eight hours and five maximum number of work days.

A similar problem to the one descibed by Loucks and Jacobs is studied by \cite{choi_hwang_park_2009}. The article focuses on a particular fast food restaurant in Korea, which is made a representative of fast food chains in general. One big difference between this study and the previous one is the identification of part-time and full-time workers, between which the ratio of scheduled personnel should always be 6:4. Although a tour is scheduled also in this problem, the properties of the tour is different from Louck and Jacobs as the shifts are already divided into periods 8:00-12:00, 12:00-14:00, 14:00-21:00 and 21:00-23:00 while the latter schedules on an hourly basis. Further, a tour is defined as working five consecutive days, as opposed to the previous article. The task assignment dimension is lacking in this article, making it less similar to the problem described in this report.

In both articles the main objective is to minimize overstaffing and understaffing, which will both have severe consequences for the fast food chain. This is not relevant to our problem as we have a fixed work force. In the example studied by Loucks and Jacobs there is also a goal to meet staff demand on total working hours. This is modeled as a secondary goal and is similar to our goal of creating even and fair schedules.


The greatest difference between the problem studied by Loucks and Jacobs and our problem is the composition of shifts. Both problems have heterogeneous worker qualification and availability and both deal with task assignment for schedules with a fluctuating worker demand. Since our problem only concerns librarians and assistants, there are fewer skill groups. Compared to the problem studied by \cite{choi_hwang_park_2009}, there is more similarity in the shift design as the library also has four different shifts. However, our problem is a task asssignment problem and does not affect working times. 

In some cases, a problem can be a combined tour scheduling and task assignment problem or can be divided into these two solution stages, as is the case in \cite{keylist}. "An integer linear programming-based heuristic for scheduling heterogeneous, part-time service employees" , 2011


\section{Personnel Task Scheduling Problem} \label{PTSP}
Claes

In many practical instances production managers will face the Personnel Task Scheduling Problem (PTSP) while scheduling plant operations. It occurs when the rosterer or shift supervisor need to allocate tasks with specified start and end times to available personnel who have the required qualifications. Furthermore, it also occurs in situations where tasks of fixed times have been assigned to machines. Decisions will then have to be made regarding the amount of maintenance workers needed and which machine the workers are assigned to. \cite{krishnamoorthy_2001}

There are several variants to the PTSP. These have been studied in an article by \cite{krishnamoorthy_2001} who gives a list of attributes that commonly appear in a PTSP and are listed in Table \ref{PTSP} below. There are furthermore traits that always appear in a PTSP; tasks with fixed start and end time are to be distributed to staff members that possesses certain skills, allowing them to perform a subset of the available tasks. The start and end time of their shifts are also predetermined for each day.

One variant, which also is the most simple, is mentioned in \cite{krishnamoorthy_2001} and is called the \textit{Feasibility Problem} where the aim is to just find a feasible solution. This requires that each task is allocated to a qualified and available worker. It is also required that a worker can not be assigned more than one task simultaneously as well as tasks can not be pre-empted, meaning that each task has to be completed by one and the same worker.

In Table \ref{PTSP} one can see attributes of PTSP variants. The nomenclature of the attributes T, S, Q, O refer to the \textit{Task type}, \textit{Shift type}, \textit{Qualifications} and \textit{Objective function} respectively. 
\begin{table}[H]
\caption{PTSP variants}
\label{PTSP}
\begin{tabular}{|c|c|l|}
%-------------------------------------------------------------------
\hline
\textbf{Attribute} & \textbf{Type} & \textbf{Explanation} \\ \hline
%-------------------------------------------------------------------
T & F & Fixed contiguous tasks \\
& V & Variable task durations \\
& S & Split (non-contiguous) tasks \\
& C & Changeover times between consecutive tasks \\
\hline 
%-------------------------------------------------------------------
S & F & Fixed, given shift lengths \\
& I & Identical shifts which are effectively of infinite duration \\
& D & Maximum duration without given start or end times \\
& U & Unlimited number of shifts of each type available \\
\hline 
%-------------------------------------------------------------------
Q & I & Identical qualification for all staff (homogeneous workforce) \\
& H & Heterogeneous workforce \\
\hline 
%-------------------------------------------------------------------
O & F & No objective, just find a feasible schedule \\
& A & Minimise assignment cost \\
& T & Worktime costs including overtime \\
& W & Minimise number of workers \\
& U & Minimise unallocated tasks \\
\hline  

%-------------------------------------------------------------------
\end{tabular}
\end{table}

With this definition of PTSP attributes many of the most basic problems and a few more complex ones can be described. It is, however, not possible to describe all of the numerous types of PTSP using these nomenclatures.

By combining attributes it is possible to obtain more complex variants of the PTSP. An example would be the PTSP[F;F;H;A-T-W] mentioned in \cite{krishnamoorthy_2001} where multiple objectives are used. This problem has fixed contiguous tasks, fixed shift lengths, heterogeneous workforce and three objective functions; assigment costs, work time with overtime included and requirements to minimize the number of workers respectively. This objective function is then a linear combination with different parameters used to prioritize them against each other.

Our problem would be most related to the PTSP[F;F;H;F]. The difference is the objective function, since we are looking to maximize the number of qualified stand-ins each day as well as maximize employee satisfaction by meeting their recommendations. This can not be described with the type of attributes given in Table \ref{PTSP} above because we have no costs, a fix number of workers and no unallocated tasks when a feasible solution is found. 

Different variants of PTSP are given names in the literature. One example is when the shifts and qualifications are identical (S=I and Q=I) and the objective function is to minimize the number of workers that are used (O=W). This variant, PTSP[F;I;I;W], has been published as the \textit{"fixed job schedule problem"} and is described in Section \ref{FJSP} below \cite{krishnamoorthy_2001}.

\subsection{Applications}
This type of problem can be found when developing a rostering solution for ground personnel at an airport. Such a problem can be dealt with by first assigning workers to days to satisfy all the labour constraints, followed by assigning the tasks to the scheduled workers.

Similar problems of type PTSP related to airplanes can also be found when scheduling for either airport mainteance staff (leading to either PTSP[F;I;H;U-A] or PTSP[F;I-U;H;W]), staff that do not stay in one location, such as airline stewards, or planes to gates. 

Another application, which has been frequently studied, can be found in classroom assignments. Based on demands such as the amount of students in a class or the duration of the class, different classrooms have to be considered. Requirements of certain equipment, e.g. for a laboratory, may also greatly limit the available rooms to choose from.

For classroom assignment there are no start or end times for the shifts, as they represent the rooms. The aim would be to find a feasible assignment of classrooms and therefore the type of problem would be PTSP[S;I;H;F] with the possibility of adding preferences to the objective function. An example of a preference would be to assign the lessons as close to each other as possible on a day, preventing travel distances between classes for teachers and students.



%Papers of interest:
%"The Personnel Task Scheduling Problem", Mohan Krishnamoorthy, Andreas T. Ernst (2001) - probably the most fundamental article
%
%"Task assignement for maintenance personnel": Roberts and Escudero, 1983a, 1983b
%
%"A stochastic programming model for scheduling maintenance personnel" Duffuaa and Al-Sultan, 1999

\section{Shift Minimisation Personnel Task Scheduling Problem}\label{SMTSP}
Claes

A close relative to the PTSP is the Shift Minimisation Personnel Task Scheduling Problem (SMPTSP) and is a special case in which the aim is to minimize the cost occuring due to the number of personnel (shifts) that are used. The same traits are valid in this problem as in the PTSP; workers with fixed work hours are to be assigned tasks, with specified start and end times, they are qualified for.

In article \cite{krishnamoorthy_2011} they "concentrate mainly on a variant of the PTSP in which the number of personnel (shifts) required is to be minimised."


Difference: "The only cost incurred is due to the number of personnel (shifts) that are used."

Papers of interest:
"Algorithms for large scale Shift Minimisation Personnel Task Scheduling Problems" Krishnamoorthy, Ernst, Baatar (2011)

"The shift minimisation personnel task scheduling problem: A new hybrid approach and computational insights" Smet, Wauters, Mihaylov, Berghe (2014)

"Fast local search and guided local search and their application to British Telecom's workforce scheduling problem" Tsang and Voudouris, 1997 - also with travelling costs, investigates two methods.

"A Triplet-Based Exact Method for the Shift Minimisation Personnel Task Scheduling Problem" Baatar et al., 2015

\section{Other similar problems}\label{other}

Variations of the task assignment problem relevant for our problem include the fixed job schedule problem and the flexible job scheduling problem. The fixed job schedule problem (FJSP) has been studied since the 1970s in the context of task assignment in processors. The problem concerns the distribution of tasks with fixed starting and ending times over a workforce with identical skills, such as processing units \cite{krishnamoorthy_2011}. Such problmes have been solved by I. Gertsbakh, H.I. Stern \cite{Gertsbakh_1977} and Fischetti et al. \cite{fischetti_1992}. In the article by Gertsbakh, a situation where n jobs need to be scheduled over an unlimited number of procesors is studied. The jobs have a specified starting time and duration. The objective of such a problem becomes the minimization of the number of machines needed to perform all tasks. Fischetti solves a similar problem, but adds time constraints, saying that no processor is allowed to work for more than a fixed time \textit{T} during a day as well as a spread time constraint forcing tasks to tasks to spread out with time gap \textit{s} over a processor.	

Another type of problem is the tactical fixed interval scheduling problem. This is a problem very closely related to the SMPTSP problem with the only difference being that the TFISP concerns workers which are always available, such as industrial machines or processors. The problem is studied by for example Kroon et al. \cite{kroon_1997}. As opposed to the FJSP, this problem deals with a heterogeneous workforce. Two different contexts are studied by Kroon et al. One of them concerns the handling of arriving aircraft passengers at an earport. Two modes of transport from the aeroplane to the airport are investigated; directly by gate or by bus. The two transportation modes thus correspond processing units which can take only a number of jobs at the same time.

OFISP


Problem: "A metaheuristic for the fixed job scheduling problem under spread time constraints" André Rossi, http://www.sciencedirect.com/science/article/pii/S0305054809002251 (Fixed job)

\section{Work load allocation and worker satisfaction} \label{WLA}
For most scheduling problems, the main objective is to reduce worker-related costs by reducing the number of workers needed to perform a task, or by reducing the working time for part-time employees. Equivalently, the goal in production industries is to reduce the number of machines needed. However, what has been studied more in recent years is also scheduling problems which take into account worker satisfaction. In an article by Akbari from 2012 a scheduling problem for part-time workers with different preferences, seniority level and productivity is investigated. \cite{akbari_2012}

Trötthet och uttråkad. Något vi borde ta med i litteraturen enligt Torbjörn, fast inte leta källor på det. Mer källor?

Source: "Employee positioning and workload allocation", Eiselt, Marianov, 2006


"Scheduling part-time personnel with availability restrictions and preferences to maximize employee satisfaction" Srimathy Mohan 2008

\section{Methods}
\subsection{TSP with inhom workforce}

Solution methods to compare (similar problems):

"Task assignment and tour scheduling": Loucks and Jacobs, 1991


"Scheduling Restaurant Workers to Minimize Labor Cost and Meet Service Standards" Choi, Hwang and Park, 2009

"An integer linear programming-based heuristic for scheduling heterogeneous, part-time service employees" Heterogenous work force, tour scheduling. Using two objective functions Hojati and Patil, 2010

for another definition as PTSP[F;I;I;W], see "The Personnel Task Scheduling Problem" by Krishnamoorty and Ernst, 2001




\inputencoding{\enc}

\chapter{The mathematical model}

\inputencoding{\enc}
%A thorough description of the mathematical model in words and in math. More significant and general constraints in equations and less significant in words.
In this chapter the mathematical model implemented to solve the ten week scheduling problem described in Section \ref{problem_description} will be presented. Prior to the objective function and constraints, the most significant sets and variables will be provided. Section \ref{section:obj} presents the objective function and gives a short description of what it represents. In Section \ref{constraints}, the essential constraints will be presented and explained. A complete model with all definitions and the full set of constraints can be found in Appendix \ref{appendix:mathmod}. It should also be noted that the abbreviations for the different tasks in the mathematical model are: E for Exp, I for Info, P for PL, H for HB and B for BokB.%To get a full view of the mathematical problem the reader  A list of the defined parameters can be found in Appendix \ref{definitions}. 
\section{Set and variable definitions} \label{variables}
To solve the problem, many sets and variables have to be declared. As mentioned before, there are many unique and individual requirements given by the library that have to be met. An example is that some staff members have two alternating schedules for odd and even weeks. The specific cases have to be modelled and result in a variety of sets and variable definitions. Below, we focus on the common and general cases in order to keep the presentation simple and easily accessible. A complete list of the definitions can be found in Appendix \ref{appendix:mathmod}. The most important sets include: \\
\itab{$I$} \tab{Set of staff members}\\
\itab{$I_{lib}$} \tab{Set of librarians ($I_{lib} \subseteq I$)} \\
\itab{$I_{ass}$}	 \tab{Set of assistants ($I_{ass} \subseteq I$)}	\\
\itab{$I_{G}\{g\}$}	 \tab{Set of workers in the group g}	\\
\itab{$W$}                 \tab{Set of all ten weeks} \\
\itab{$W_5$}	\tab{Set of the first five weeks} \\
\itab{$D$}                 \tab{Set of all days in a week}           \\
\itab{$D_5$}	\tab{Set of all five weekdays} \\
\itab{$S_d$}           \tab{Set of shifts available on day \textit{d}}         \\
\itab{$S_3$}           \tab{Set of the first three shifts on a weekday}     \\
\itab{$J_d$}            \tab{Set of task types available on day \textit{d}}   \\
\itab{$G$}	 \tab{Set of groups}	\\
\itab{$V$}	 \tab{Set of possible week rotations (shifts the week by 0-9 steps forwards)}	\\

In order to further define the problem we introduce the following variables.
\begin{align}
    x_{iwdsj}&=
    \begin{cases}
      1, & \text{if staff member \textit{i} is assigned to a task \textit{j} in week \textit{w}, day \textit{d}, shift \textit{s}}\\
      0, & \text{otherwise}
    \end{cases}
    \\
    H_{iwh}&=
    \begin{cases}
      1, & \text{if staff member \textit{i} works weekend \textit{h} in week \textit{w}}\\
      0, & \text{otherwise}
    \end{cases}
	\\
	r_{iw}&=
	\begin{cases}
		1, & \text{if staff member \textit{i} has its schedule rotated \textit{w-1} steps forwards}\\
		0, & \text{otherwise}
	\end{cases}
	\\
	l_{iwd}&=
	\begin{cases}
	  1, & \text{if librarian \textit{i} is a stand-in week \textit{w}, day \textit{d}} \\
	  0, & \text{otherwise}
	\end{cases}
	\\
	a_{iwd}&=
	\begin{cases}
 		1, & \text{if assistant \textit{i} is a stand-in week \textit{w}, day \textit{d}} \\
 		0, & \text{otherwise}
	\end{cases}
	\\
	y_{iwds}&=
	\begin{cases}
 		1, & 
 		\parbox[t]{.7\textwidth}{if staff member \textit{i} works at task type E, I or P in week \textit{w}, day \textit{d}, shift \textit{s}} \\
 		0, & \text{otherwise}
	\end{cases}
	\\
	W_{iwd}&=
	\begin{cases}
	 	1, & \text{if a staff member \textit{i} is working a shift in week \textit{w}, day \textit{d}} \\
	 	0, & \text{otherwise}
	\end{cases}
	\\
	b_{iw}&=
	\begin{cases}
 		1, & \text{if staff member \textit{i} works at HB in week \textit{w}} \\
 		0, & \text{otherwise}
	\end{cases}
	\\
	f_{iw}&=
	\begin{cases}
 		1, & \text{if staff member \textit{i} is assigned to work Friday evening in week \textit{w}} \\
 		0, & \text{otherwise}
	\end{cases}	
	\\
	M_{wds}&=
	\begin{cases}
	 	1, & \text{if a big meeting is placed in week \textit{w}, day \textit{d}, shift \textit{s}} \\
	 	0, & \text{otherwise}
	\end{cases}
	\\
	m_{wdsg}&=
	\begin{cases}
	 	1, & \text{if a meeting is assigned for group \textit{g} in week \textit{w}, day \textit{d}, shift \textit{s}} \\
	 	0, & \text{otherwise}
	\end{cases}
	\\
	d_{iwds}&=
	\begin{cases}
	 	1, & \text{if there is a difference for a staff member \textit{i} in assignment of}\\
	 		& \text{ tasks at shift \textit{s}, day \textit{d} between weeks \textit{w} and \textit{w+5}} \\
	 	0, & \text{otherwise}
	\end{cases}
	\\
	l^{min}&= \text{lowest number of stand-in librarians found at a day (integer)} \\
	a^{min}&= \text{lowest number of stand-in assistants found at a day(integer)} \\
	s^{min}&= \text{weighted sum of numbers of stand-in librarians and assistants} \\
	\delta&= \text{total number of shifts that differ for all staff members (integer)}
\end{align}

Based on the variables defined above it has been possible to model and solve our scheduling problem. The variables \textit{$l^{min}$} and \textit{$a^{min}$} are the ones of most significance, as they represent the number of stand-ins found at the worst day after a run. 

\section{Objective function} \label{section:obj}
As the problem contains multiple objective functions, it has been necessary to weigh them against each other using parameters. These weights are shown in Equation \ref{objfcn} as $M$ and $N$.

\begin{equation} \label{objfcn}
\text{maximize} \hspace{0.3cm} M\cdot s^{min} - N \cdot \delta
\end{equation}
The first part of the objective function represents the lowest amount of stand-in librarians and assistants found at a day, while the second part minimizes the difference between the first and last five weeks. This means that week 1 and 6, 2 and 7 and so on should be as similar as possible. The parameter $N$ prioritizes the similarity of weeks compared to the number of stand-ins. Based on the information given by the library, there is a much higher priority to have many stand-ins. Hence, $M \gg N$, where $N > 0$ holds in our case. The exact relation between $M$ and $N$ is not of high importance, as we have seen that multiple optimal solutions exist, regarding the stand-ins. In our case, we set $M$ to be a hundred times bigger than $N$.




% % % % GO THROUGH THE REFERENCES TO CONSTRAINT SECTION IN THE DOCUMENT
\section{Constraints} \label{constraints}
In order to model the problem it is of relevance to divide many of the constraints into weekend and weekday constraints. Several auxiliary constraints and variables are also added to avoid multiplication of two variables, which would make the problem non-linear. These auxiliary constraints are needed as the solver CPLEX can not handle non-linear constraints, as it displays error messages in case they occur. The auxiliary constraints are left out of this chapter for simplicity reasons. Instead, they can be seen in Appendix \ref{appendix:mathmod}.

\subsection{Demand and assignment constraints} \label{section:demand_ass_constraints}
The most crucial constraint is to ensure that the demand of staff members is met each day. This can be modelled as
\begin{equation} \label{eq:demand}
\sum_{i \in I} x_{iwdsj} = demand_{wdsj}, \;   w\in W,d\in D,s\in S,j\in J_d
\end{equation}
Here, $demand_{wdsj}$ is an integer representing the number of staff members required in week \textit{w}, day \textit{d}, shift \textit{s} for a task \textit{j}.

The following constraint says whether or not a staff member is assigned a task during a weekday. Evening shifts and library on wheels tasks are excluded.
\begin{equation} \label{constr:y_assign}
y_{iwds} = \sum_{j \in J_d\backslash \{L\}} x_{iwdsj}, \;   i \in I, w \in W, d \in D_5, s \in S_3
\end{equation}
This variable assignment is used to simplify a couple of constraints later on.

To ensure that no staff members are assigned more than one task the following constraint is implemented.
\begin{equation} \label{constr:one_task_constraint}
\sum_{s\in S}\sum_{j\in J_d} x_{iwdsj} \leq 1, \;   i\in I, w \in W, d\in D
\end{equation}
However, if we allow a person to have two shifts at the library on wheels on a day, Equation \ref{constr:one_task_constraint} has to be slightly modified. This is left out of this chapter, but is allowed in the complete model.

For most staff members, it is preferred to allow only one PL per week and a maximum of three PL per ten weeks. These are easily modelled with the following constraints.
\begin{equation} \label{constr:one_PL}
\sum_{s \in S_d}\sum_{d \in D} x_{iwdsP} \leq 1, \;   i\in I, w \in W
\end{equation}
\begin{equation} \label{constr:three_PL}
\sum_{w \in W}\sum_{s \in S_d}\sum_{d \in D} x_{iwdsP} \leq 3, \;   i\in I
\end{equation}
The long duration of a PL (in the equations named "P"), is the reason for these preferences. Some staff members are required to have some time free from outer tasks, to be able to perform their inner tasks.

Another preference is to have varying work hours when it comes to the assignment of tasks. Then the more and less desired shifts will be fairly distributed. The following equation is implemented to meet such requirements.
\begin{equation} \label{constr:various_start_times}
\sum_{d \in D_5} y_{iwds} \leq 2, \;   i\in I, w \in W, s \in S_3
\end{equation}
Equation \ref{constr:various_start_times} allows a staff member to have at most two tasks starting at the same hour every week. Worth noting is that library on wheels is disregarded in this constraint as the variable $y_{iwds}$ is used, see Equation \ref{constr:y_assign} for the definition.

It is desirable to avoid assigning too many tasks to a staff member. The reason for this is, as mentioned previously, to let the staff member have some time to allot for inner services during weekdays. The equation below models this preference.
\begin{equation} \label{constr:four_weekly_shifts_at_most}
\sum_{d \in D_5}\sum_{s \in S_d}\sum_{j \in J_d} x_{iwdsj} \leq 4, \;   i\in I, w \in W
\end{equation}
Equation \ref{constr:four_weekly_shifts_at_most} allows a staff member at most four weekday shifts per week. The exception, which is one of the BokB staff members, is left out of this equation for simplicity reasons. To model that, a new subset of \textit{I} needs to be created where that staff member is left out. 
%Ändra till max 4 tasks per vecka diff {36}?

\subsection{Weekend and rotation constraints} \label{section:weekend_rot_constraints}
Every staff member's schedule can be rotated up to nine times, where the decision variable $r_{iv}$ decides the rotation. Therefore, the parameter $qualavail_{iwdsj}$ has to align with the rotation so that all staff members are assigned tasks only when available. This is provided in the following equation.
\begin{equation} \label{constr:qualavail}
x_{iwdsj} \leq \sum_{v \in V} r_{iv}*qualavail_{i\omega(v,w)dsj}, \;   i \in I, w \in W, d \in D, s \in S_d, j \in J_d
\end{equation}
The function $\omega(v,w)$ represents the modulus function $mod_{10}(w-v+10)+1$ and is a function which takes rotations into account when calculating the week to look for in the availability matrix. Table \ref{tab:mod} gives a better understanding of how this modulus function is used. 

The variable \textit{w}, in Table \ref{tab:mod}, represents the current week in the initial schedule, and \textit{v} represents \textit{v-1} rotations to the right from the initial schedule. With initial schedule, we refer to the schedule, where all staff members' weekends occur in the first week. The resulting week, when rotations have been taken into account, can be seen inside the cells. When scheduling, it is the resulting week $\omega$ which is of interest when checking if a staff member is available for a task assignment. 
\begin{table}[H]
\centering
\caption{Resulting table of the function $\omega(v,w) = mod_{10}(w-v+10)+1$. Here, $w$ is the week in the initial schedule, $v$ is the rotation variable. Inside the cells, the resulting week is shown when rotations have been made.}
\label{tab:mod}
\begin{tabular}{llllllllllll}
    &                         &                         &                         &                         &                         & \multicolumn{2}{l}{w =}                           &                         &                         &                         &                         \\
    &                         & 1                       & 2                       & 3                       & 4                       & 5                       & 6                       & 7                       & 8                       & 9                       & 10                      \\ \cline{3-12} 
    & \multicolumn{1}{l|}{1}  & \multicolumn{1}{l|}{1}  & \multicolumn{1}{l|}{2}  & \multicolumn{1}{l|}{3}  & \multicolumn{1}{l|}{4}  & \multicolumn{1}{l|}{5}  & \multicolumn{1}{l|}{6}  & \multicolumn{1}{l|}{7}  & \multicolumn{1}{l|}{8}  & \multicolumn{1}{l|}{9}  & \multicolumn{1}{l|}{10} \\ \cline{3-12} 
    & \multicolumn{1}{l|}{2}  & \multicolumn{1}{l|}{10} & \multicolumn{1}{l|}{1}  & \multicolumn{1}{l|}{2}  & \multicolumn{1}{l|}{3}  & \multicolumn{1}{l|}{4}  & \multicolumn{1}{l|}{5}  & \multicolumn{1}{l|}{6}  & \multicolumn{1}{l|}{7}  & \multicolumn{1}{l|}{8}  & \multicolumn{1}{l|}{9}  \\ \cline{3-12} 
    & \multicolumn{1}{l|}{3}  & \multicolumn{1}{l|}{9}  & \multicolumn{1}{l|}{10} & \multicolumn{1}{l|}{1}  & \multicolumn{1}{l|}{2}  & \multicolumn{1}{l|}{3}  & \multicolumn{1}{l|}{4}  & \multicolumn{1}{l|}{5}  & \multicolumn{1}{l|}{6}  & \multicolumn{1}{l|}{7}  & \multicolumn{1}{l|}{8}  \\ \cline{3-12} 
    & \multicolumn{1}{l|}{4}  & \multicolumn{1}{l|}{8}  & \multicolumn{1}{l|}{9}  & \multicolumn{1}{l|}{10} & \multicolumn{1}{l|}{1}  & \multicolumn{1}{l|}{2}  & \multicolumn{1}{l|}{3}  & \multicolumn{1}{l|}{4}  & \multicolumn{1}{l|}{5}  & \multicolumn{1}{l|}{6}  & \multicolumn{1}{l|}{7}  \\ \cline{3-12} 
v = & \multicolumn{1}{l|}{5}  & \multicolumn{1}{l|}{7}  & \multicolumn{1}{l|}{8}  & \multicolumn{1}{l|}{9}  & \multicolumn{1}{l|}{10} & \multicolumn{1}{l|}{1}  & \multicolumn{1}{l|}{2}  & \multicolumn{1}{l|}{3}  & \multicolumn{1}{l|}{4}  & \multicolumn{1}{l|}{5}  & \multicolumn{1}{l|}{6}  \\ \cline{3-12} 
    & \multicolumn{1}{l|}{6}  & \multicolumn{1}{l|}{6}  & \multicolumn{1}{l|}{7}  & \multicolumn{1}{l|}{8}  & \multicolumn{1}{l|}{9}  & \multicolumn{1}{l|}{10} & \multicolumn{1}{l|}{1}  & \multicolumn{1}{l|}{2}  & \multicolumn{1}{l|}{3}  & \multicolumn{1}{l|}{4}  & \multicolumn{1}{l|}{5}  \\ \cline{3-12} 
    & \multicolumn{1}{l|}{7}  & \multicolumn{1}{l|}{5}  & \multicolumn{1}{l|}{6}  & \multicolumn{1}{l|}{7}  & \multicolumn{1}{l|}{8}  & \multicolumn{1}{l|}{9}  & \multicolumn{1}{l|}{10} & \multicolumn{1}{l|}{1}  & \multicolumn{1}{l|}{2}  & \multicolumn{1}{l|}{3}  & \multicolumn{1}{l|}{4}  \\ \cline{3-12} 
    & \multicolumn{1}{l|}{8}  & \multicolumn{1}{l|}{4}  & \multicolumn{1}{l|}{5}  & \multicolumn{1}{l|}{6}  & \multicolumn{1}{l|}{7}  & \multicolumn{1}{l|}{8}  & \multicolumn{1}{l|}{9}  & \multicolumn{1}{l|}{10} & \multicolumn{1}{l|}{1}  & \multicolumn{1}{l|}{2}  & \multicolumn{1}{l|}{3}  \\ \cline{3-12} 
    & \multicolumn{1}{l|}{9}  & \multicolumn{1}{l|}{3}  & \multicolumn{1}{l|}{4}  & \multicolumn{1}{l|}{5}  & \multicolumn{1}{l|}{6}  & \multicolumn{1}{l|}{7}  & \multicolumn{1}{l|}{8}  & \multicolumn{1}{l|}{9}  & \multicolumn{1}{l|}{10} & \multicolumn{1}{l|}{1}  & \multicolumn{1}{l|}{2}  \\ \cline{3-12} 
    & \multicolumn{1}{l|}{10} & \multicolumn{1}{l|}{2}  & \multicolumn{1}{l|}{3}  & \multicolumn{1}{l|}{4}  & \multicolumn{1}{l|}{5}  & \multicolumn{1}{l|}{6}  & \multicolumn{1}{l|}{7}  & \multicolumn{1}{l|}{8}  & \multicolumn{1}{l|}{9}  & \multicolumn{1}{l|}{10} & \multicolumn{1}{l|}{1}  \\ \cline{3-12} 
\end{tabular}
\end{table}
An example to better understand the function: imagine that we are looking at an unrotated initial schedule where everyone is assumed to work weekend the first week. Say that we look at a staff member's first week, $w=1$. If this schedule is rotated two times to the right ($v=3$), then the first week in the new rotated schedule represents the previous ninth week, that is $w=9$ when no rotation exists. The availability to look at, in the initial schedule is, therefore, week nine.

Initially, all staff members are assigned weekend work the first week. Then, a week rotation variable is introduced that rotates the staff members' availability matrices so that the demand constraints on evenings and weekends can be met. The most basic constraints, regarding which week a staff member's weekend work occurs and its week rotation, can be seen below.

\begin{equation} \label{constr:one_rot}
\sum_{w \in W} r_{iw} = 1, \;   i\in I
\end{equation}
\begin{equation} \label{constr:max_one_weekend}
\sum_{w \in W} H_{iwh} \leq 1, \;   i\in I, h \in \{1,2\}
\end{equation}
\begin{equation} \label{constr:rot_weekend}
r_{iw} \geq H_{iw1}, \;   i\in I, w \in W
%	\begin{cases}
% 		1, & \text{if $H_{iwh=1} = 1$} \\
% 		0/1, & \text{otherwise}
%	\end{cases}
%	\;   i\in I, w \in W \;
\end{equation}

Equation \ref{constr:one_rot} provides all staff members with a rotation of their schedule regardless if they are working weekends or not. Equation \ref{constr:max_one_weekend} allows a staff member a maximum of two weekends, for $h=1$ and $h=2$, per ten weeks.

Once a staff member is assigned a new rotation, the availability matrix has to be correctly rotated. This is done using Equation \ref{constr:rot_weekend}. In case a staff member is never due for weekend work, that is, $H_{iwh} = 0, \; \forall i\in I, w \in W, h \in \{1,2\}$, then the rotation $r_{iw}$ is free, as its availability matrix is identical for all weeks. 

A staff member is supposed to work weekends with a five-week interval. However, in case there are enough staff members to satisfy the demand on weekends, it may be enough for some to only work one weekend per ten weeks. To avoid problems with the rotation if only the second weekend is assigned to a staff member, the following equation is used.
\begin{equation} \label{constr:five_week_interval}
r_{i(mod_{10}(w+4)+1)} \geq H_{iw2}, \;   i\in I, w \in W
\end{equation}
Worth noting is that if a staff member is assigned both weekends, then Equation \ref{constr:five_week_interval} provides the same information as Equation \ref{constr:rot_weekend}.

A staff member is supposed to work with the same task both Saturday and Sunday when working a weekend. This can be modelled with the following constraints.
\begin{equation} \label{constr:consecutive_days}
\sum_{j \in J_d} x_{iw61j} + \sum_{j \in J_d} x_{iw71j} = 2*\sum_{h = 1}^{2} H_{iwh}, \;   i\in I, w \in W
\end{equation}
\begin{equation} \label{constr:same_tasks}
x_{iw61j} = x_{iw71j}, \;   i\in I, w \in W, j \in J_d
\end{equation}
Equation \ref{constr:consecutive_days} ensures that the staff member will work Saturday and Sunday consecutively, whenever he or she is due for weekend work. To make sure it is the same task as well, Equation \ref{constr:same_tasks} is implemented.

Friday evening is also included in an extension to working Saturday and Sunday, unless the staff member is assigned to HB. It is, however, not a necessity to perform the same task Friday evening as during the weekend, thus Fridays are not included in Equation \ref{constr:same_tasks}. Equation \ref{constr:friday_added} below adds Fridays to the weekend.
\begin{equation} \label{constr:friday_added}
\sum_{j \in J \backslash \{L\}}x_{iw54j} = f_{iw}, \;   i \in I, w \in W
\end{equation}
Here, "L" refers to the task type library on wheels. The variable $f_{iw}$ is, as mentioned in the variable declaration, equal to one if and only if a staff member is working weekend as well as not being assigned to HB.

It is of interest to implement a constraint to prevent staff members from being assigned to HB more than once every ten weeks in order to avoid unfairness. Why this is preferable is described in Section \ref{section:library_tasks}. This constraint can be modelled as.
\begin{equation} \label{constr:max_one_hb}
\sum_{w \in W}\sum_{d = 6}^{7}x_{iwd1B} \leq 2, \;   i \in I
\end{equation}
Here, "B" refers to HB. Equation \ref{constr:same_tasks} and \ref{constr:max_one_hb} ensure both that a staff member only can be assigned to HB two days per ten weeks, and that those days are consecutive. The exception is in case a staff member only works in HB, which is disregarded in this simplified model.

\subsection{Objective function constraints} \label{section:obj_fcn_constraints}
The variable $\delta$ in the objective function, Equation \ref{objfcn}, is defined by the following equation.
\begin{equation} \label{constr:delta}
\delta = \sum_{i \in I} \sum_{w \in W_5} \sum_{d \in D_5} \sum_{s \in S_3} d_{iwds}
\end{equation}
This equation gives an integer value, which represents the total number of shift differences that occur for all workers, through all weekdays, in the library. The variable $d_{iwds}$, defined in Equation \ref{constr:obj_fcn_shifts} below, compares task assignments between two five-week separated weeks.

To calculate the variable $s^{min}$ used in the objective function, \ref{objfcn}, the following equation is used.
\begin{equation} \label{constr:s_min}
s^{min} \leq L\cdot \sum_{i \in I_{lib}} l_{iwd} + A\cdot \sum_{i \in I_{ass}} a_{iwd}, \;   w \in W, d \in D_5
\end{equation}
%\begin{equation} \label{constr:l_min}
%l^{min} \leq \sum_{i \in I_{lib}} l_{iwd}, \;   w \in W, d \in D_5
%\end{equation}
%\begin{equation} \label{constr:a_min}
%a^{min} \leq \sum_{i \in I_{ass}} a_{iwd}, \;   w \in W, d \in D_5
%\end{equation}

Here, $l_{iwd}$ and $a_{iwd}$ are binary variables stating whether a staff member is a stand-in on a day or not. Since $s^{min}$ is being maximized in the objective function, it will assume the lowest value of the sum of stand-in librarians and assistants for any day during the ten weeks. It will, therefore, represent the worst solution found regarding stand-ins. Just as for previous equations, auxiliary constraints have been left out for simplicity reasons. Here, it is left out how $l_{iwd}$ and $a_{iwd}$ are determined.

If $L < A$ holds in the model, the solver would prioritize assistants over librarians as stand-ins. Librarians are, however, more desired as stand-ins, due to their ability to perform all types of tasks. Therefore, it is desired to let $L \geq A$.

As stated in Section \ref{section:obj}, two weeks with a five-week interval should be as similar as possible. Equation \ref{constr:obj_fcn_shifts}, together with Equation \ref{constr:delta} and the objective function equation, \ref{objfcn}, provides this preference.
\begin{equation} \label{constr:obj_fcn_shifts}
d_{iwds} = \abs{y_{iwds} - y_{i(w+5)ds}}, \;   i \in I, w \in W_5, d \in D_5, s \in S_3
\end{equation}
The decision variable $d_{iwds}$ states if there is a difference in assignment between two tasks at the same hour and day for the two weeks, $w$ and $w+5$. The differences occur if, say, an Exp task is assigned on Monday week one at 8 a.m. to 10 a.m. and no task is assigned the same shift Monday week six. Thus, a variable that is minimized in the objective function.

\subsection{Meeting constraints} \label{section:meeting_constraints}
At the library there are both library meetings and group meetings, which both occur with a five-week interval. Library meetings are set to take place from 8 a.m. to 10 a.m. on Mondays, whereas group meetings are more freely distributed. A few staff members are not assigned library meetings, as they are needed in the library to keep it running. The set $I_{big}$ in the equation below consists of all staff members who are to be assigned library meetings. The constraints modelling library meetings are as follows.
\begin{equation} \label{constr:library_meetings}
\sum_{w \in W_5} M_{w11} = 1
\end{equation}
\begin{equation} \label{constr:library_meetings2}
M_{(w+5)11} = M_{w11}, \;   w \in W_5
\end{equation}
\begin{equation} \label{constr:library_meetings3}
\sum_{s=1}^{3} \sum_{j \in J_1 \backslash \{L\}} x_{iw1sj} \leq 1-M_{w11}, \;   i \in I \backslash I_{big}, w \in W
\end{equation}
Equation \ref{constr:library_meetings} and \ref{constr:library_meetings2} assign two library meetings with a five-week interval during the ten-week scheduling period. Equation \ref{constr:library_meetings3} makes sure that the staff members, that do not attend library meetings, are not assigned any other task during the day of the meeting. This implicitly force the staff members, that do not attend the library meetings, to meet the library demand constraints during the meeting. 

The constraints added to model the group meetings are somewhat similar to the library meeting constraints. Just as for library meetings, they take place two times during the ten weeks with a five-week interval, which is described by Equations \ref{constr:dep_meetings} and \ref{constr:dep_meetings2}.
\begin{equation} \label{constr:dep_meetings}
\sum_{w \in W_5}\sum_{d \in D_5}\sum_{s \in S_3} m_{wdsg} = 1, \; g \in G
\end{equation}
\begin{equation} \label{constr:dep_meetings2}
m_{(w+5)dsg} = m_{wdsg}, \;   g \in G, w \in W_5, d \in D_5, s \in S_3
\end{equation}
\begin{equation} \label{constr:dep_meetings3}
m_{wdsg} + x_{iwdsj} \leq 1, \;   g \in G, i \in I_{G}\{g\}, w \in W, d \in D_5, s \in S_3, j \in J_d
\end{equation}
\begin{equation} \label{constr:dep_meetings4}
m_{wdsg} \leq \sum_{v \in V} r_{iv}*qualavail_{i\omega(v,w)dsE}, \;   g \in G, i \in I_{G}\{g\}, w \in W, d \in D_5, s \in S_3
\end{equation}
Equation \ref{constr:dep_meetings3} prohibits a staff member from multitasking, that is, to attend a meeting and work with a task simultaneously. Equation \ref{constr:dep_meetings4} enables group meetings only when everyone in that group is available. The rotation is also taken into the account.

In Equation \ref{constr:dep_meetings4} $\omega$ is the week calculation when rotations are taken into the account, see Section \ref{section:weekend_rot_constraints}. The task type "E" stands for Exp, and is used due to it is the only task type everyone is available for, as \textit{qualavail} is dependent on the task variable \textit{j}. A better way to model this would be to create a new parameter, say $avail_{iwds}$, that is not dependent on staff member qualifications. As this way of modelling works, this was not implemented in order to to save us some time.


\inputencoding{\enc}
\chapter{Results from mathematical model}

\inputencoding{\enc}
%Results from running the mathematical model in CPELX.

\section{Evaluation of performance}

Running the mathematical model in CPLEX 12.5.0.0 finds an optimum solution in approximately 17 minutes (1031). In the objective function, the parameters L = 4000, A = 2000 and N = 5 were used, thus having twice as much priority on finding librarian stand-ins than assistant stand-ins. 

\iffalse
Other parameters include:

\begin{itemize}
\item the number of allowed PL per week, which was one
\item the number of allowed PL in 10 weeks, which was three
\item the number of tasks which can be placed on the same shift time during a week, which was two
\end{itemize}
\fi


%Solution time: 
%Objective function parameters chosen: L = 4000, A = 2000, N = 5
%Constraint parameters: num_PL_week = 1,num_PL_week_10_weeks = 3, num_shifts_per_time = 2

\section{Evaluation of results}

%Number of stand ins - vem och hur


\inputencoding{\enc}

\chapter{Herustic: task distribution approach}

\inputencoding{\enc}

\section{Introduction}

The first approach, where fixed weeks are distributed to workers, is discussed in the previous chapter. The second approach, which is presented in this chapter, instead distributes individual tasks to workers. This method greatly resembles the process of manually placing tasks as is typically done in many practical situations. 

The objective of the scheduling process is not only to schedule a number of tasks to the workers but also to place them optimally with respect to stand ins. Thus, a method for moving through the solution space and a way of distinguishing between good and bad solutions is of great importance.  

The primary method used in this approach is a large neighbourhood search (LNS) together with a simulated annhealing (SA) accept function. Destroying and repairing the solution, as is customary in LNS, helps leading the solution out of local optima or plateaus. Similarly, SA is used in order to allow the solution to move in a less favourable directions to avoid these local optima. The 
search is guided by a continuously updated schedule cost.

Write: two phases implemented.

\section{Objective functions}
Distinguishing good schedules from bad schedules means there is a need for a way of seting a cost to different schedules. In the original mathematical model, the cost of a schedule consists of two terms: a weighted sum of the number of stand ins on the worst day and the number of different shifts present in the schedule. The rest of the model consists of hard constraints which cannot be violated. However, in the heuristic approach these hard constraints are divided into hard and soft constraints, as is illustrated in Figure (TODO).

During the scheduling process in the implementation, three different objective functions are used. The objective functions are illustrated in Table (TODO) and consist of a weekend objective function, a worker objective function and a weekday objective function.


\begin{table}[]
\centering
\caption{Objective functions used in the implementation.}
\label{tab: task objective functions}
\begin{tabular}{|p{3cm}|l|}
\hline
\multirow{5}{*}{\begin{tabular}[t]{@{}l@{}}\textbf{Weekend} \\ \textbf{Objective} \\ \textbf{Function}\end{tabular}} & \\
 & Min stand in cost + average num stand ins\\  
 & Min shift availability cost + average shift avail\\ 
 & Min day availability cost + average day avail\\  
 & \\ 
\hline

\multirow{7}{*}{\begin{tabular}[t]{@{}l@{}}\textbf{Worker}\\ \textbf{Objective}\\ \textbf{Function}\end{tabular}}    & \\
& Num tasks per day cost \\ 
& Num tasks per week cost \\ 
& Num PL per week cost \\ 
& Total num of PL cost \\ 
& Num tasks at same shift per week  cost \\ 
& \\
\hline

\multirow{4}{*}{\begin{tabular}[t]{@{}l@{}}\textbf{Weekday}\\ \textbf{Objective}\\ \textbf{Function}\end{tabular}}   & \\ & Min stand in cost   \\ 
 & \\ & \\ 
\hline
\end{tabular}
\end{table}

The weekday objective function is associated with the weekend distribution phase of the problem. In the weekend objective function, a stand in cost is defined as the weighted sum between the number of librarians and assistants which are stand ins at a certain day. The minimum stand in cost is defined as the cost at the worst av all days throughout all weeks. 

The min shift availability cost refers to the minimum number of workers available at a shift throughout all shifts, days and weeks. Similarly, the min day availability cost is the lowest number of workers available any shift throughout the whole schedule. In addition to these costs, the average of all three cost types is also used to distinguish solutions which have the same cost.

The worker objective function is used during the weekday task distribution phase of the problem. The function value is a combination of the relaxed constraints of the problem, as described at the beginning of this chapter. Thus, the worker objective function must always be reduced to zero before the schedule will be considered feasible.

The objective function consists of five different costs. All of them are calculated as the total cost for all workers. For example, the first cost described in Table (todo) is the total number of tasks per day exceeding the maximum of one daily task. In the objective function, the sum of excess tasks is taken over all workers. 

The second cost is the number of tasks exceeding four per week. The third is the number of Fetch list tasks performed by a worker exceeding the max limit of one per week. In addition to this, there is a specified max limit for the number of fetch lists that can be performed by a worker throughout all weeks and which is different for different workers. This is the fourth cost. Finally, the number of tasks performed at the same shift in a week should not exceed two, which is the final cost in the worker objective function.

A feasible solution, where the worker objective function is zero, is evaluated using the weekday objective function described lastly in the table. This cost has only one cost and corresponds to the objective function in the mathematical model. The min stand in cost is calculated in the same way as in the weekend objective function. 

\section{Weekend phase}

The first phase in the scheduling process is the weekend phase. In this phase the weekends of all workers are placed and optimized before placing the remaining tasks. The reason for implementing such a phase rather than placing all tasks at once was the big impact of the weekend structure on the entire schedule. First and foremost, the location of a worker's weekend affects the availability of the worker in the following week where the week rest is placed. The worker is unavailable during week rest and if such week rests are combined in an unfortunate way, the scheduling can result in an uneven distribution of workers during days affected by week rest.

In order to measure what a good distribution of weekends is, the weekend objective function described in the previous section is used during the search process. Although the overall objective is to maximize the number of stand ins at the most critical day in the schedule, that is, even out the stand ins over the days, it is not trivial to measure this in a schedule with no tasks placed. Evening work reduces potential stand ins as well as tasks distributed during the days. Furthermore, limits on how many tasks per week a person is allowed to take further complicates the measurement of stand ins.

Because of the difficulty in measuring stand ins, certain tasks are placed already in the weekend phase. This includes all Library on Wheel tasks and all evening tasks. For the Library on Wheels, there is not much choice or variability, and thus a fixed structure is applied in order to simplify its distribution. Similarly, the workers who are available at the evening tasks are in most part, equal to the number of workers needed, thus leaving little or no choice in scheduling evenings. Using this as a basis for approximating the objective function value, a more accurate measurement can be performed.


\begin{table}[!h]
\caption{Worker availability placing evening tasks and BokB.}
\centering
\begin{tabular}{|C{1.2cm}
|C{0.6cm}|C{0.6cm}|C{0.6cm}|C{0.6cm}|C{0.6cm}|C{0.6cm}|C{0.6cm}|}
\hline
&\multicolumn{7}{l|}{\textbf{Num available assistants}} \\ \hline
& Mo & Tu & We & Th & Fr & Sa & Su \\ \hline
Shift 1: &8 & 10 & 10 & 10 & 6 & 0 & 0  \\ \hline
Shift 2: &8 & 9 & 9 & 9 & 6 & 0 & 0 \\ \hline
Shift 3: &9 & 8 & 9 & 7 & 5 & 0 & 0 \\ \hline
Shift 4: & 3 & 2 & 2 & 3 & 0 & 0 & 0 \\ \hline
\hline
&\multicolumn{7}{l|}{\textbf{Num available librarians}} \\ \hline
& Mo & Tu & We & Th & Fr & Sa & Su \\ \hline
Shift 1: & 16 & 15 & 16 & 13 & 12 & 0 & 0 \\ \hline
Shift 2: & 18 & 15 & 17 & 14 & 13 & 0 & 0  \\ \hline
Shift 3: &17 & 14 & 18 & 18 & 13 & 0 & 0  \\ \hline
Shift 4: &3 & 4 & 4 & 3 & 1 & 0 & 0 \\ \hline
\hline
&\multicolumn{7}{l|}{\textbf{Num available BBlib}} \\ \hline
& Mo & Tu & We & Th & Fr & Sa & Su \\ \hline
Shift 1:  & 2 & 0 & 1 & 1 & 1 & 0 & 0 \\ \hline
Shift 2: & 0 & 0 & 0 & 0 & 0 & 0 & 0 \\ \hline
Shift 3: &0 & 0 & 0 & 0 & 0 & 0 & 0 \\ \hline
Shift 4: &1 & 0 & 2 & 2 & 0 & 0 & 0 \\ \hline
\end{tabular}
\end{table}

\begin{table}[!h]
\centering
\caption{Worker availability after placing evening tasks and BokB for an example week.}
\begin{tabular}{|C{1.2cm}
|C{0.6cm}|C{0.6cm}|C{0.6cm}|C{0.6cm}|C{0.6cm}|C{0.6cm}|C{0.6cm}|}
\hline
&\multicolumn{7}{l|}{\textbf{Num available assistants}} \\ \hline
& Mo & Tu & We & Th & Fr & Sa & Su \\ \hline
Shift 1: & 7 & 10 & 9 & 9 & 8 & 0 &  0 \\ \hline   
Shift 2: &7 & 9 & 8 & 8 & 8 & 0 & 0 \\ \hline
Shift 3: & 7 & 7 & 8 & 6 & 6 & 0 & 0 \\ \hline 
Shift 4: & 0 & 0 & 0 & 0 & 0 & 0 & 0 \\ \hline
\hline 
&\multicolumn{7}{l|}{\textbf{Num available librarians}} \\ \hline
& Mo & Tu & We & Th & Fr & Sa & Su \\ \hline
Shift 1: & 14 & 11 & 13 & 11 & 12 & 0 & 0 \\ \hline  
Shift 2: &14 & 11 & 14 & 11 & 12 & 0 & 0 \\ \hline  
Shift 3: &13 & 11 & 14 & 13 & 13 & 0 & 0 \\ \hline       
Shift 4: &0 & 0 & 0 & 1 & 1 & 0 & 0 \\ \hline
\hline   
&\multicolumn{7}{l|}{\textbf{Num available BBlib}} \\ \hline
& Mo & Tu & We & Th & Fr & Sa & Su \\ \hline
Shift 1: & 0 & 0 & 0 & 0 & 0 & 0 & 0 \\ \hline
Shift 2: & 0 & 0 & 0 & 0 & 0 & 0 & 0 \\ \hline
Shift 3: & 0 & 0 & 0 & 0 & 0 & 0 & 0 \\ \hline
Shift 4: & 0 & 0 & 1 & 0 & 0 & 0 & 0 \\ \hline
\end{tabular}
\end{table}


Initial solution: placing tasks at random
Destroy: Destroying a certain number of weekends at random.
Repair: Repairing same weekends for same workers. Prioritizing weekends according to 1. qualification 2. avail demand diff. Almost random worker placed, although making sure that HB is placed correctly.

Infeasibility check: Does the current assignment of weekends generate a schedule where there are not enough workers at the shifts?

Placing Library on Wheels and evenings: Considered as having very little degree of freedom. 

Evenings: based on worker costs.

Heuristic methods: SA on LNS with random destroy and repair. SA accept function, accepting with exponential cooling. Tuning parameters T and alpha.

\section{Weekday phase}
Evenings,weekends and BokB already placed. 

Concept: destroy worst worker until all workers have feasible schedules. Record the library cost of the solution. 

Destroy: weekday tasks for workers with highest cost.
Repair: 1. qualification, 2. avail demand diff. Place cheapest worker.

Infeasibility: when a feasible worker cost is not found for a large number of iterations. 

\section{Simplifications of Mathematical Model}
-10 Week scheduling
-Objective function term about similar weeks
-BokB fixed weeks for every other week workers.
-Even odd weeks. How to handle?
(-lower limit PL)

\section{Implementation}
C++, object orientation, run on a linux operating system. Reading availability of workers into the program and outputing a result file, which can be read by Excel. Results are to be visualized in Excel (write this in another part?)

\inputencoding{\enc}

%##########################################################
% -------------------END CHAPTERS -------------------------
%##########################################################

\bibliography{Chapters/References}


%%% .................. appendix ...................................

\appendix
\chapter{Problem definitions} \label{definitions}
\section{Sets}
\itab{I} \tab{Set of workers}\\
\itab{I\_lib} \tab{Set of librarians (I\_lib $\subseteq$ I)} \\
\itab{I\_ass}	 \tab{Set of assistants (I\_ass $\subseteq$ I)}	\\
\itab{W}                 \tab{Set of weeks}                                               \\
\itab{D}                 \tab{Set of days in a week}                                      \\
\itab{$S_d$}           \tab{Set of shifts day \textit{d}}                                        \\
\itab{$J_d$}            \tab{Set of task types day \textit{d}}                                    \\
\itab{I\_LOW}	 \tab{Set of librarians available to work in library on wheels}	\\
\itab{I\_free\_day}	 \tab{Set of workers that shall be assigned a free weekday per week}	\\
\itab{I\_odd\_even}	 \tab{Set of all workers with odd or even weeks}	\\
\itab{I\_weekend\_avail}	 \tab{Set of workers available for weekend work}	\\ 
\newpage
\section{Variables} \label{variabs}
\begin{align}
    x_{iwdsj}&=
    \begin{cases}
      1, & \text{if worker \textit{i} is assigned in week \textit{w}, day \textit{d}, shift \textit{s} to a task \textit{j}}\\
      0, & \text{otherwise}
    \end{cases}
    \\
    H_{iwh}&=
    \begin{cases}
      1, & \text{if worker \textit{i} works weekend h in week \textit{w}}\\
      0, & \text{otherwise}
    \end{cases}
	\\
	r_{iw}&=
	\begin{cases}
		1, & \text{if worker \textit{i} has its scheduled rotated \textit{w-1 steps}}\\
		0, & \text{otherwise}
	\end{cases}
	\\
	lib_{iwd}&=
	\begin{cases}
	  1, & \text{if librarian \textit{i} is a stand-in week \textit{w} day \textit{d}} \\
	  0, & \text{otherwise}
	\end{cases}
	\\
	ass_{iwd}&=
	\begin{cases}
 		1, & \text{if assistant \textit{i} is a stand-in week \textit{w} day \textit{d}} \\
 		0, & \text{otherwise}
	\end{cases}
	\\
	y_{iwds}&=
	\begin{cases}
 		1, & \text{if worker \textit{i} is working \textit{w} day \textit{d} regardless of task type} \\
 		0, & \text{otherwise}
	\end{cases}
	\\
	hb_{iw}&=
	\begin{cases}
 		1, & \text{if assistant \textit{i} is a stand-in week \textit{w} day \textit{d}} \\
 		0, & \text{otherwise}
	\end{cases}
	\\
	friday\_evening_{iw}&=
	\begin{cases}
 		1, & \text{if assistant \textit{i} is a stand-in week \textit{w} day \textit{d}} \\
 		0, & \text{otherwise}
	\end{cases}	
	\\
	lib\_min&= \text{lowest number of stand-in librarians found (integer)} \\
	ass\_min&= \text{lowest number of stand-in assistants found (integer)}
\end{align}
\newpage
\section{Parameters} \label{params}
\begin{align}
	N1l&= \text{a value to prioritize the amount of stand-in librarians}
	\\
	N1a&= \text{a value to prioritize the amount of stand-in assistants}
	\\
	N2&= \text{a value to prioritize similar weeks}
	\\
	avail\_day_{iwd}&=
	\begin{cases}
		1, & \text{if worker \textit{i} is available for work week \textit{w}, day \textit{d}} \\
		0, & \text{otherwise}
	\end{cases}
	\\
	task\_demand_{dsj}&= \text{number of workers required day \textit{d}, shift \textit{s} for task type \textit{j}}
	\\
	qualavail_{iwdsj}&=
	\begin{cases}
		1, & \text{if worker \textit{i} is qualified and available week \textit{w}, day \textit{d}, shift \textit{s} for task type \textit{j}} \\
		0, & \text{otherwise}
	\end{cases}
	\\
	LOW\_demand_{wds}&= \text{number of workers required day \textit{d}, shift \textit{s} at the library on wheels}
\end{align}




% this is the copyright page, I did not touch it much.

\cleardoublepage

\thispagestyle{plain}

\begin{picture}(20,20)(0,0)
% \put(0,10){\includegraphics[width=3.5cm]{epresslogo}}
\put(280,0){\includegraphics[width=2.5cm]{univlogo.eps}}
\end{picture}
 
\null
\vspace{0cm} 

Copyright
\medskip

The publishers will keep this document online on the Internet - or its possible replacement - for a period of 25 years from the date of publication barring exceptional circumstances.
The online availability of the document implies a permanent permission for anyone to read, to download, to print out single copies for your own use and to use it unchanged for any non-commercial research and educational purpose. Subsequent transfers of copyright cannot revoke this permission. All other uses of the document are conditional on the consent of the copyright owner. The publisher has taken technical and administrative measures to assure authenticity, security and accessibility.
According to intellectual property law the author has the right to be mentioned when his/her work is accessed as described above and to be protected against infringement.
For additional information about the Linköping University Electronic Press and its procedures for publication and for assurance of document integrity, please refer to its WWW home page: http://www.ep.liu.se/
\bigskip

Upphovsr\"att
\medskip

Detta dokument hålls tillgängligt på Internet - eller dess framtida ersättare - under 25 år från publiceringsdatum under förutsättning att inga extraordi\-nära omständigheter uppstår.
Tillgång till dokumentet innebär tillstånd för var och en att läsa, ladda ner, skriva ut enstaka kopior för enskilt bruk och att använda det oförändrat för ickekommersiell forskning och för undervisning. Överföring av upphovsrätten vid en senare tidpunkt kan inte upphäva detta tillstånd. All annan användning av dokumentet kräver upphovsmannens medgivande. För att garantera äktheten, säkerheten och tillgängligheten finns det lösningar av teknisk och administrativ art.
Upphovsmannens ideella rätt innefattar rätt att bli nämnd som upphovsman i den omfattning som god sed kräver vid användning av dokumentet på ovan beskrivna sätt samt skydd mot att dokumentet ändras eller presenteras i sådan form eller i sådant sammanhang som är kränkande för upphovsmannens litterära eller konstnärliga anseende eller egenart.
För ytterligare information om Linköping University Electronic Press se förlagets hemsida http://www.ep.liu.se/

\medskip

\copyright\phantom{.} \putshortdate, \putauthor




%----------------END OF DOCUMENT----------------------------------------------
%----------------END OF DOCUMENT----------------------------------------------

\end{document}
 
