% Master Thesis Report
% Created by: Claes Arvidson and Emelie Karlsson 
% Linköping University Spring 2016
% First version: 2016-01-27

\documentclass[a4paper, 10pt, twoside, openright]{book}
\pagestyle{headings}

% some of the packages used are
\usepackage[round, authoryear]{natbib}
\bibliographystyle{unsrtnat}
\usepackage{xcolor, colortbl}
\usepackage{tabularx}
\usepackage{latexsym}
\usepackage{eepic}
\usepackage{makeidx}
\usepackage{booktabs}
\usepackage{graphicx}
\usepackage[utf8]{inputenc}  %för ä ö å ?
\usepackage[english]{babel}
%\usepackage{times}
\usepackage{amssymb}
%\usepackage{fancybox} 
%\usepackage{textcomp}
\usepackage{float}
%\usepackage{amsmath}
\usepackage{mathtools}
\usepackage{multirow}
\usepackage{footnote}
%\makesavenoteenv{tabular}
%\makesavenoteenv{table}
\usepackage{tikz}
\usetikzlibrary{shapes,arrows}
\usepackage{caption} 
\captionsetup[table]{skip=5pt}

% the fancy header/footer
% consult http://research.cmis.csiro.au/gjw/tex/docs/fancyhdr.pdf
% or some other fancy header documenatation for more info
\usepackage{fancyhdr}
\pagestyle{fancy}
\lfoot{\nouppercase{\rightmark}}
\rfoot{\nouppercase{\leftmark}}
\fancyfoot{} 
\fancyhead[LO]{\nouppercase{\small{\rightmark}}}
\fancyhead[RE]{\nouppercase{\small{\leftmark}}}
\fancyhead[LE,RO]{\small{\thepage}}
\fancyhead[C]{}
\renewcommand{\footrulewidth}{0pt}
\renewcommand{\headrulewidth}{0.2pt}
\fancypagestyle{plain}{ \fancyhf{}  
  \fancyfoot[RE,LO]{\nouppercase{\small{\putegotext}}}
  \fancyfoot[LE,RO]{\small{\thepage}}
  \renewcommand{\headrulewidth}{0pt}  \renewcommand{\footrulewidth}{0.2pt} }

\newcommand\mydots{\hbox to 1em{.\hss.\hss.}}
% ------------------------------------------------------------------------
% BEGINNING of abstract, author, examiner, etc
% here you must fill in your own text. Remember to respect the { and the }.
%
% the *-marked % in 
%                                   *
%      \newcommand{\putabstract}[0]{% 
%      some text}
%
% is made to avoid a blank space in the beginning of the abstract,
% author, date etc.
%
\newcommand{\putabstract}[0]{%
The distribution of tasks to an inhomogeneous work force at libraries and other service institutions is time consuming for manual shcedulers, but well suited for optimization software. Such a problem is studied. The problem concerns five different task types, two types of workers and around 100 tasks to be scheduled in total weekly. Also weekends are to be scheduled and worker satisfaction is taken into account. The objective of the scheduling is to create a ten week rotating schedule in which the stand-in staff members are evenly distributed.

A mathematical model is formulated for the problem, which is solved using the commercial CPLEX solver and by using two different LNS heuristic implementations. The first heuristic schedules week blocks to the workers, while the second distributes one task at a time. The latter heuristic works better than the former and achieves results comparable to those of the commercial solver. Our conclusion is that
the second heuristic works better since it focuses on finding a good weekend distribution before creating the rest of the schedule.}

\newcommand{\putkeywords}[0]{%
Optimization, Scheduling, Task distribution, LNS, Weekend Scheduling, Inhomogeneous workforce}

\newcommand{\putthemonth}[0]{June }

\newcommand{\putshortdate}[0]{2016}

\newcommand{\putmydate}[0]{\putthemonth \putshortdate}

\newcommand{\putauthor}[0]{Claes Arvidson and Emelie Karlsson}

\newcommand{\putegotext}[0]{Arvidson, Karlsson, 2016.}

\newcommand{\puttitle}[0]{Work Distribution for a Heterogeneous Library Staff - \\  A Personnel Task Scheduling Problem}

% some kind of administrator will give you this number
\newcommand{\putregnumber}[0]{LiTH - MAT - EX - - 04 / 04 - - SE}

\newcommand{\putexaminer}[0]{Elina Rönnberg}

\newcommand{\putsupervisor}[0]{Torbjörn Larsson}

% for example Applied Mathematics
\newcommand{\putdepartment}[0]{Optimeringslära}

% send an email to ep.liu.se to make sure it is correct before you print
% this number is the one I had
\newcommand{\putmyurl}[0]{http://www.ep.liu.se/exjobb/mai/2004/tm/004/}

% Please do not change into University of Linköping or something, the
% international name of Liu is Linköpings Universitet
\newcommand{\putliu}[0]{Linköpings Universitet}

%Create the command \tab
\newcommand{\tab}[1]{\hspace{.2\textwidth}\rlap{#1}}
\newcommand{\itab}[1]{\hspace{0em}\rlap{#1}}
\DeclarePairedDelimiter\abs{\lvert}{\rvert}
\DeclarePairedDelimiter\norm{\lVert}{\rVert}
%
% END of abstract, author, examiner, etc
% ------------------------------------------------------------------------

\begin{document} 

\frontmatter
\raggedbottom
% some new commands I used, you won't need them I guess- just erase it
\newcommand{\ala}[0]{{\alpha}_{1}}
\newcommand{\alb}[0]{{\alpha}_{2}}
\newcommand{\nc}[0]{{N}_{C}}
\newcommand{\tCh}[0]{the Chemostat }
\newcommand{\Ch}[0]{Chemostat }
\newcommand{\ch}[0]{chemostat }
\newcommand{\MMk}[0]{Michaelis-Menten kinetics }
\newcommand{\KI}[0]{{K}_{I} }
\newcommand{\KN}[0]{{K}_{N} }
\newcommand{\mmu}[0]{{\mu}_{max} }
\newcommand{\KMAX}[0]{\mmu}
\newcommand{\AAA}[0]{\mathbf{A} }
% stop erasing here

%\usepackage[first=0,last=9]{lcg}
%\newcommand{\ra}{\rand0.\arabic{rand}}

% Colors
\newcommand{\enc}[0]{utf8}
\definecolor{bluegray}{rgb}{0.4, 0.6, 0.8}
\definecolor{darkcyan}{rgb}{0.0, 0.55, 0.55}
\definecolor{corn}{rgb}{0.98, 0.93, 0.36}
\definecolor{coralred}{rgb}{1.0, 0.25, 0.25}
\definecolor{Gray}{gray}{0.85}
\definecolor{Darkgray}{gray}{0.9}
\definecolor{maroon}{cmyk}{0,0.87,0.68,0.32}
\newcommand{\colcell}{\cellcolor{Gray}}
\newcommand{\colcelltwo}{\cellcolor{corn}}
\newcommand{\colcellthree}{\cellcolor{darkcyan}}
\newcolumntype{g}{>{\columncolor{corn}}c}

% Special cells
\newcommand{\specialcell}[2][c]{%
  \begin{tabular}[#1]{@{}l@{}}#2\end{tabular}}
\newcommand{\specialcelltwo}[2][c]{%
   \begin{tabular}[#1]{@{}l@{}l@{}}#2\end{tabular}}
\newcolumntype{L}[1]{>{\raggedright\let\newline\\\arraybackslash\hspace{0pt}}m{#1}}
\newcolumntype{C}[1]{>{\centering\let\newline\\\arraybackslash\hspace{0pt}}m{#1}}
\newcolumntype{R}[1]{>{\raggedleft\let\newline\\\arraybackslash\hspace{0pt}}m{#1}}

\makeatletter% Set distance from top of page to first float
\setlength{\@fptop}{5pt}
\makeatother


% don't mind these
\title{\puttitle}
\author{\putauthor}

% this command makes the first page, and the second page, handle with care
% You really don't have to change anything here I guess.
\renewcommand{\maketitle}{%
              \newpage%
              \pagestyle{empty}
              \null%
              \hfil\hspace*{-3mm}
              \begin{minipage}{150mm}
                \center 
                {\Large{Master thesis}} \\\vspace*{4mm}
                {\vbox to 22mm{\vfil\Large\textbf{\puttitle}}}
                \vspace*{5mm}
                 {\Large{\putauthor}}       \\ \vspace*{4mm}
                  \putregnumber\\
                \end{minipage}\hfil
              \clearpage%\hspace*{0mm}\thispagestyle{empty}
              \clearpage%\thispagestyle{empty}
              \null%
              \hfil\hspace*{-3mm}
              \begin{minipage}{150mm}
                \center
                {\vbox to 48mm{\vfil\Large\textbf{\puttitle}}}
                \vspace*{5mm}
                  \putdepartment, \putliu \\ \vspace*{4mm}
                  \textbf{\putauthor}       \\ \vspace*{4mm}
                  \putregnumber
                  \vspace*{100mm}\\
                \flushleft
                
                Exam work:\hspace*{3pt}
                \begin{minipage}[t]{70mm}
% but you might want to change this 30
                  \textbf{30 hp}
                \end{minipage} \\ \vspace*{4mm}
                
                Level:\hspace*{3pt}
                \begin{minipage}[t]{70mm}
% and you might want to change this A 
                  \textbf{A} 
                \end{minipage} \\ \vspace*{4mm}
                
                Supervisor:\hspace*{3pt}
                \begin{minipage}[t]{120mm}
                  \textbf{\putsupervisor}, \\\putdepartment, \putliu
                \end{minipage} \\ \vspace*{4mm}
                
                Examiner:\hspace*{3pt}
                \begin{minipage}[t]{120mm}
                  \textbf{\putexaminer}, \\\putdepartment, \putliu
                \end{minipage} \\ \vspace*{4mm}
                
                Linköping:
                \begin{minipage}[t]{70mm}
                  \textbf{\putmydate}
                \end{minipage} \\ \vspace*{4mm}
              \end{minipage} \\ \hfill
}

% now the \maketitle command is changed and we use the new one
\maketitle

% I'm not sure this phantom is needed, but I leave to make sure
% a phantom leaves empty space corresponding to the size of what is in it.
\phantom{crap}
\thispagestyle{empty}
\pagestyle{empty}

\setlength{\unitlength}{1mm}


\setlength{\unitlength}{1pt}

% now turn on fancy headers
\pagestyle{fancy}

% ..................chapter ...............................................
\chapter*{Abstract\index{Abstract}}
% your abstract
\putabstract

% enters keywords
\begin{description}
\item[Keywords:]{%
\putkeywords
}
\item[URL for electronic version: ]{\hfill%\quad\\%
%\phantom{\qquad} http://urn.kb.se/resolve?urn=urn:nbn:se:liu:diva-77777\\
%where 77777 should be replacd by an appropriate number.
\begin{center}
http://urn.kb.se/resolve?urn=urn:nbn:se:liu:diva-77777
\end{center}
%where 77777 should be replaced by an appropriate number.
}
\end{description}


% here or on the next page you put an abstract in swedish
%\section*{Abstract in Swedish: Sammanfattning}
%Det finns många typer av bioreaktorer som tillämpas\ldots

\chapter*{Acknowledgements\index{Acknowledgements}}

The authors of this thesis would like to express our deepest gratitude to our supervisor Torbjörn Larsson at Linköping University, who has helped us find our way in moments of uncertainty and who has been guiding us through the whole project. Thank you for the time and effort you have spent on our thesis. We would also like to thank Elina Rönnberg, who has provided encouraging words and valuable insights throughout the project.

Furthermore, we would like to thank Elisabeth Cserhalmi and Ingrid Loeld  Rasch at Norrköpings Stadsbibliotek for providing us with an interesting thesis topic and for answering all our questions many times.

We thank our opponents, Akdas Hossain and Emma Miléus, for their comments and thoughts on this report.

Lastly, we would like to thank our families and loved ones, who have supported us until the end. 



% optional I guess
\chapter*{Nomenclature\index{Nomenclature}}

Most of the reoccurring terms and abbreviations are described here.

%----------------NEW_SECTION--------------------------------------------------
\section*{Optimization terms\index{Optimization terms}}
\begin{tabular}{lp{10cm}}

Heuristic & An algorithm designed to find sufficiently good, but not necessarily feasible solutions. \\

%Rostering & \\
%Matheuristic & Text\\
\end{tabular}

\section*{Library terms\index{Library terms}}
\begin{tabular}{lp{10cm}}

Fetch list & The library task of collecting books from shelves, to be delivered elsewhere. \\
Library on wheels & The task of driving a library bus with books to remote areas of town. \\
%Rostering & \\
%Matheuristic & Text\\
\end{tabular}

%\begin{tabular}{ll}
%$Y_0$      & The amount of the variable $Y$ inserted into a system.\\
%$\hat Y$& The unit-dimension of the variable $Y$, for example $\hat t=1s$ .\\
%$\bar Y_i$ & A steady state (number $i$) value of Y.\\
%\phantom{a}& \phantom{b} \\
%$K_i$ & Constants used in kinetic expressions, for example $K_I$.\\
%\phantom{a}& \phantom{b} \\
%$\AAA$     & The system matrix. \\
%\end{tabular}

\section*{Abbreviations\index{Abbreviations}}

\begin{tabular}{ll}
Exp		& Service counter (sv. expeditionsdisken) \\
Info	& Information counter (sv. informationsdisken)\\
PL		& Fetch list (sv. plocklistan)\\
BokB    & Library on wheels (sv. bokbussen)\\
HB      & Hageby library \\
\end{tabular}


\tableofcontents 
\begin{samepage}
\listoffigures
\let\clearpage\relax
\listoftables
\end{samepage}


\mainmatter

%##########################################################
% -------------------CHAPTERS -----------------------------
%##########################################################

\chapter{Introduction}\label{chap:intro}

\inputencoding{\enc}
% Introduction to Master Thesis


%\section{Background} 
%Schemaläggning av personal vid Norrköpings bibliotek.
\section{Background}
At a library absence can cause problems, since the qualifications required to perform tasks varies. If a worker were to be unavailable a day due to a meeting or simply being ill it would require for a stand-in to fill the vacancy. Therefore, it is of great interest to have a schedule with as many skilled stand-ins as possible to overcome such disturbances. 


\section{Problem description}
The goal of this thesis is to distribute given tasks to the heterogeneous workforce at the library of Norrköping. Each task is either classified as an outer or an inner service where an outer service is when a librarian needs to interact with visitors. Inner services can in some rare cases require a predetermined person to be assigned to a specified time or day.

Demands and requests are to be fulfilled to the furthest extent possible. Weekends are included in the scheduling problem, which adds more constraints regarding the number of contiguous working days. However, the librarians are permitted a few exceptions from these laws regarding days of rest.

The main purpose of the thesis is to create a schedule robust enough to withstand absence, such that outer services always are assigned to a qualified and available worker. This is visualized as having a list of available stand-ins for each shift. 

There are a limited number of workers at the library and they make the resources that are to be distributed. Each individual has a set of \textit{skills} and \textit{competences}. Competences refer to the capability of being assigned the different outer services; Expedition, Norpan, Information desk, Library on wheels and Hageby as well as different inner services. The set of skills an individual can possess are described in Table \ref{int:1}. In total there are 39 workers available.

The outer services can be seen as assignments which requires available workers to be assigned to them. Each outer service is specified to a certain station, time and date. They also have a fix length and occur on a regular basis every ten weeks, which makes it possible to create a periodic schedule with a period of ten weeks. 


%Examensarbetet går ut på att lägga ett arbetsschema för personalen vid Norrköpings bibliotek. Problemet går i grund och botten ut på att fylla alla uppgifter på de stationer som tillhör bibliotekets utåtriktade verksamhet (så kallade yttre tjänst)  med personal av rätt kompetens samt samtidigt som personalen får tid över till övriga uppgifter (så kallad inre tjänst). Schemat som tas fram ska även uppfylla de regelverk och önskemål som finns kring personalens individuella scheman, till exempel de arbetstider som ingår i de olika tjänsterna. Då det även ingår helgarbete i personalens arbetsuppgifter ska lediga dagar fördelas enligt arbetsmiljölagen och de undantag från dessa lagar gällande veckovilan.

%Utöver detta ska schemat även medföra en robusthet så att störningar i den yttre tjänsten, i form av att personal blir sjuk eller uppbokad annonstädes, ska gå att avhjälpa med en reservlista. Denna reservlista består av de bibliotekarier och assistenter som inte har något pass tilldelat sig samt är tillgängliga under dagen. 

%Personalen på biblioteket är begränsad och utgör de resurser som finns att tillgå. Varje enskild personal har en uppsättning \textit{egenskaper} och \textit{kompetenser}. Kompetenser syftar på personalens förmåga att arbeta vid någon av de yttre stationerna; Expedition, Norpan, Informationsdisk, Bokbuss och Hageby samt några av de inre stationerna; inköp, katalogisering med mera. De egenskaper som identifierats hos en personal finns beskrivna i tabell \ref{int:1}. Totala arbetskraften består av 39 stycken arbetare på biblioteket.

%De yttre och inre uppgifterna kan ses som behov av personal som måste täckas av den personal som finns att tillgå. De olika yttre uppgifterna som behöver utföras inkluderar arbete vid olika stationer vid olika tidpunker och datum. Varje uppgift har en bestämd längd och återkommer regelbundet inom ett 10-veckorsinterval, vilket gör att ett rullande schema kan skapas med en period om tio veckor. 

\begin{table}[h]
\centering
\caption{Personnel}
\label{int:1}
\begin{tabular}{|l|l|}
%-------------------------------------------------------------------
\hline 
\textbf{Skills} & \textbf{Description} \\ \hline
%-------------------------------------------------------------------
Work degree & 0-100 \% 
\\ \hline 
%-------------------------------------------------------------------
Type of employment & Librarian/Assistant
\\ \hline 
%-------------------------------------------------------------------
Competence & Inner and outer services the worker is qualified for  
\\ \hline 
%-------------------------------------------------------------------
Weekly rest & Which days the worker has requested after working a weekend
\\ \hline 
%-------------------------------------------------------------------
Other requests & Does not work evenings etc.
\\ \hline 
%-------------------------------------------------------------------
\end{tabular}
\end{table} 

Furthermore, outer and a few inner services can be characterized by different properties, which are represented in Table \ref{int:2}. \\

\begin{table}[!h]
\caption{Outer and inner services}
\label{int:2}
\begin{tabular}{|l|l|}
%-------------------------------------------------------------------
\hline
\textbf{Outer service} & \textbf{Property} \\ \hline
%-------------------------------------------------------------------
 & Start time, end time, week and duration \
\\ \hline 
%-------------------------------------------------------------------
 & Station
\\ \hline 
%-------------------------------------------------------------------
 & Number of qualified librarians demanded
\\ \hline 
%-------------------------------------------------------------------
 & Number of qualified assistants demanded
\\ \hline 
%-------------------------------------------------------------------

\textbf{Inner service} & \textbf{Property} \\ \hline
%-------------------------------------------------------------------
 & Start time, end time, week and duration \
\\ \hline 
%-------------------------------------------------------------------
 & Type
\\ \hline 
%-------------------------------------------------------------------
 & Number of qualified librarians demanded
\\ \hline 
%-------------------------------------------------------------------
 & Number of qualified assistants demanded
\\ \hline 
%-------------------------------------------------------------------
\end{tabular}
\end{table}

In addition to the properties mentioned above, there are several requirements that have to be met. These can be divided into job, robust and other requirements and are listed in Table \ref{int:3} below. 

%Utöver de ovan nämnda resurserna och behoven, finns ett antal krav som ställs på hur schemat får utformas. Dessa kan delas upp i arbetsvillkor, robusthetskrav samt övriga krav och finns representerade i tabell \ref{int:3}.

\begin{table}[H]
\caption{Requirements}
\label{int:3}
\begin{tabular}{|l|l|}
%-------------------------------------------------------------------
\hline
\textbf{Job requirements} & \textbf{Description} \\ \hline
%-------------------------------------------------------------------
& A maximum of one outer service is to be distributed to each person and day.
\\ \hline 
%-------------------------------------------------------------------
& Remaining work time is individually distributed on assignments such as Övrig arbetstid distribueras självständig med uppgifter såsom exempelvis bokplock eller bokuppsättning.
\\ \hline
%-------------------------------------------------------------------
 & Helgarbete ska fördelas rättvist mellan de i personalen som är tillgängliga för helgarbete. 
\\ \hline 
%-------------------------------------------------------------------
 & Helgarbete innefattar arbete under fredag kväll, påföljande lördag och söndag.
\\ \hline 
%-------------------------------------------------------------------
 & Högst ett kvällspass per personal i veckan bortsett från den vecka helgarbete ska utföras.
\\ \hline 
%-------------------------------------------------------------------
 & Schemat ska upprepa sig var 10e vecka.
\\ \hline 
%-------------------------------------------------------------------
 & Varje arbetsvecka ska ha liknande struktur i största möjliga mån.
\\ \hline 
%-------------------------------------------------------------------

\textbf{Robust requirements} & \textbf{Description} \\ \hline
%-------------------------------------------------------------------
 & För varje yttre uppgift ska det finnas minst en reserv.
\\ \hline 
%-------------------------------------------------------------------
 & Reserverna ska vara av rätt kompetens för uppgiften de är reserver till.
\\ \hline 
%-------------------------------------------------------------------
 & Prioritet ligger i att det lägsta antalet reserver för en uppgift ska vara så hög som möjligt.
\\ \hline 
%-------------------------------------------------------------------

\textbf{Other requirements} & \textbf{Description} \\ \hline
%-------------------------------------------------------------------
 & Stormöte och avdelningsmöten ska vardera hållas en gång var femte vecka.
\\ \hline 
%-------------------------------------------------------------------
\end{tabular}
\end{table}
\medskip

%Den optimala schemat ska inte endast uppfylla kraven ovan, utan även 



\inputencoding{\enc}

\chapter{Literature review}\label{chap:lit}

\inputencoding{\enc}
% Review of previous work in the field
Emelie

The scheduling problem has been studied since the 1950's as a mathematical optimization problem and involves creating a feasible and satisfactory schedule for either workers or machines performing tasks. According to Ernst et al. the complexity of the scheduling problem has not in itself become more advanced with time. However, the mathematical models used to solve the scheduling problems have become more realistic and refined. This together with more powerful computational methods, makes it possible today to solve scheduling problems in a more satisfactory way, taking into account softer values such as worker satisfaction and worker fatigue \cite{Ernst_2004}.

In the paper \cite{Ernst_2004} the scheduling problem is classified into different subcategories. A few relevant areas for our work include task based demand scheduling, days off scheduling, shift scheduling, tour scheduling and task assignment. 

Task based demand scheduling involves the process of distributing a fixed number of tasks which need to be performed over a workforce. The workforce can either be fixed, as in our case, or subject to minimization in the objective function.
	
Days off scheduling involves scheduling staff and assigning a days off as required by work time regulations. This problem is often found together with shift scheduling which involves choosing the most suitable shifts for a workforce. The combination of the two is called tour scheduling and will be discussed later in this report. The big difference between our problem and tour scheduling is that in our problem we are not allowed to choose the free days of the workers, only in what week to assign them, and there is no shift work.

The problem which is most similar to our problem is, however, task assignment. This problem and different variations of it will be discussed in section \ref{PTSP}.



\section{Tour Scheduling Problem with a heterogenous work force}\label{TSP}
Emelie

The Tour Scheduling Problem (TSP) involves creating work shifts with days off for a work force. According to Loucks and Jacobs, the vast majority of all tour scheduling problems up to 1991 were with a homogeneous workforce, that is under the assumption that any worker could perform any assigned task \cite{loucks_1991}. The authors discuss a tour scheduling problem where the objective is to construct weekly schedules for each worker which also show the specific task assignments. The problem is studied in the context of fast food restaurants, where certain personnel is qualified only for certain stations in the restaurant. In such industries, the demand of staff differs between different weekdays and different times of the day. Two things differ between workers, their availability for work and their qualification. Furthermore, the workers working times are not fixed, but are composed of a number of consecutive tasks assigned during a block of time during a day.

The representative problem studied in the article involves creating a one-week schedule for 40 workers in a fast food restaurant, available for eight different tasks with a seven-day, 128-hour workweek. Several synthetic problems are studied in the article, all, however with minimum shift lenght three hours, maximum shift length eight hours and five maximum number of work days.

A similar problem to the one descibed by Loucks and Jacobs is studied by \cite{choi_hwang_park_2009}. The article focuses on a particular fast food restaurant in Korea, which is made a representative of fast food chains in general. One big difference between this study and the previous one is the identification of part-time and full-time workers, between which the ratio of scheduled personnel should always be 6:4. Although a tour is scheduled also in this problem, the properties of the tour is different from Louck and Jacobs as the shifts are already divided into periods 8:00-12:00, 12:00-14:00, 14:00-21:00 and 21:00-23:00 while the latter schedules on an hourly basis. Further, a tour is defined as working five consecutive days, as opposed to the previous article. The task assignment dimension is lacking in this article, making it less similar to the problem described in this report.

In both articles the main objective is to minimize overstaffing and understaffing, which will both have severe consequences for the fast food chain. This is not relevant to our problem as we have a fixed work force. In the example studied by Loucks and Jacobs there is also a goal to meet staff demand on total working hours. This is modeled as a secondary goal and is similar to our goal of creating even and fair schedules.


The greatest difference between the problem studied by Loucks and Jacobs and our problem is the composition of shifts. Both problems have heterogeneous worker qualification and availability and both deal with task assignment for schedules with a fluctuating worker demand. Since our problem only concerns librarians and assistants, there are fewer skill groups. Compared to the problem studied by \cite{choi_hwang_park_2009}, there is more similarity in the shift design as the library also has four different shifts. However, our problem is a task asssignment problem and does not affect working times. 

In some cases, a problem can be a combined tour scheduling and task assignment problem or can be divided into these two solution stages, as is the case in \cite{keylist}. "An integer linear programming-based heuristic for scheduling heterogeneous, part-time service employees" , 2011


\section{Personnel Task Scheduling Problem} \label{PTSP}
Claes

In many practical instances production managers will face the Personnel Task Scheduling Problem (PTSP) while scheduling plant operations. It occurs when the rosterer or shift supervisor need to allocate tasks with specified start and end times to available personnel who have the required qualifications. Furthermore, it also occurs in situations where tasks of fixed times have been assigned to machines. Decisions will then have to be made regarding the amount of maintenance workers needed and which machine the workers are assigned to. \cite{krishnamoorthy_2001}

There are several variants to the PTSP. These have been studied in an article by \cite{krishnamoorthy_2001} who gives a list of attributes that commonly appear in a PTSP and are listed in Table \ref{PTSP} below. There are furthermore traits that always appear in a PTSP; tasks with fixed start and end time are to be distributed to staff members that possesses certain skills, allowing them to perform a subset of the available tasks. The start and end time of their shifts are also predetermined for each day.

One variant, which also is the most simple, is mentioned in \cite{krishnamoorthy_2001} and is called the \textit{Feasibility Problem} where the aim is to just find a feasible solution. This requires that each task is allocated to a qualified and available worker. It is also required that a worker can not be assigned more than one task simultaneously as well as tasks can not be pre-empted, meaning that each task has to be completed by one and the same worker.

In Table \ref{PTSP} one can see attributes of PTSP variants. The nomenclature of the attributes T, S, Q, O refer to the \textit{Task type}, \textit{Shift type}, \textit{Qualifications} and \textit{Objective function} respectively. 
\begin{table}[H]
\caption{PTSP variants}
\label{PTSP}
\begin{tabular}{|c|c|l|}
%-------------------------------------------------------------------
\hline
\textbf{Attribute} & \textbf{Type} & \textbf{Explanation} \\ \hline
%-------------------------------------------------------------------
T & F & Fixed contiguous tasks \\
& V & Variable task durations \\
& S & Split (non-contiguous) tasks \\
& C & Changeover times between consecutive tasks \\
\hline 
%-------------------------------------------------------------------
S & F & Fixed, given shift lengths \\
& I & Identical shifts which are effectively of infinite duration \\
& D & Maximum duration without given start or end times \\
& U & Unlimited number of shifts of each type available \\
\hline 
%-------------------------------------------------------------------
Q & I & Identical qualification for all staff (homogeneous workforce) \\
& H & Heterogeneous workforce \\
\hline 
%-------------------------------------------------------------------
O & F & No objective, just find a feasible schedule \\
& A & Minimise assignment cost \\
& T & Worktime costs including overtime \\
& W & Minimise number of workers \\
& U & Minimise unallocated tasks \\
\hline  

%-------------------------------------------------------------------
\end{tabular}
\end{table}

With this definition of PTSP attributes many of the most basic problems and a few more complex ones can be described. It is, however, not possible to describe all of the numerous types of PTSP using these nomenclatures.

By combining attributes it is possible to obtain more complex variants of the PTSP. An example would be the PTSP[F;F;H;A-T-W] mentioned in \cite{krishnamoorthy_2001} where multiple objectives are used. This problem has fixed contiguous tasks, fixed shift lengths, heterogeneous workforce and three objective functions; assigment costs, work time with overtime included and requirements to minimize the number of workers respectively. This objective function is then a linear combination with different parameters used to prioritize them against each other.

Our problem would be most related to the PTSP[F;F;H;F]. The difference is the objective function, since we are looking to maximize the number of qualified stand-ins each day as well as maximize employee satisfaction by meeting their recommendations. This can not be described with the type of attributes given in Table \ref{PTSP} above because we have no costs, a fix number of workers and no unallocated tasks when a feasible solution is found. 

Different variants of PTSP are given names in the literature. One example is when the shifts and qualifications are identical (S=I and Q=I) and the objective function is to minimize the number of workers that are used (O=W). This variant, PTSP[F;I;I;W], has been published as the \textit{"fixed job schedule problem"} and is described in Section \ref{FJSP} below \cite{krishnamoorthy_2001}.

\subsection{Applications}
This type of problem can be found when developing a rostering solution for ground personnel at an airport. Such a problem can be dealt with by first assigning workers to days to satisfy all the labour constraints, followed by assigning the tasks to the scheduled workers.

Similar problems of type PTSP related to airplanes can also be found when scheduling for either airport mainteance staff (leading to either PTSP[F;I;H;U-A] or PTSP[F;I-U;H;W]), staff that do not stay in one location, such as airline stewards, or planes to gates. 

Another application, which has been frequently studied, can be found in classroom assignments. Based on demands such as the amount of students in a class or the duration of the class, different classrooms have to be considered. Requirements of certain equipment, e.g. for a laboratory, may also greatly limit the available rooms to choose from.

For classroom assignment there are no start or end times for the shifts, as they represent the rooms. The aim would be to find a feasible assignment of classrooms and therefore the type of problem would be PTSP[S;I;H;F] with the possibility of adding preferences to the objective function. An example of a preference would be to assign the lessons as close to each other as possible on a day, preventing travel distances between classes for teachers and students.



%Papers of interest:
%"The Personnel Task Scheduling Problem", Mohan Krishnamoorthy, Andreas T. Ernst (2001) - probably the most fundamental article
%
%"Task assignement for maintenance personnel": Roberts and Escudero, 1983a, 1983b
%
%"A stochastic programming model for scheduling maintenance personnel" Duffuaa and Al-Sultan, 1999

\section{Shift Minimisation Personnel Task Scheduling Problem}\label{SMTSP}
Claes

A close relative to the PTSP is the Shift Minimisation Personnel Task Scheduling Problem (SMPTSP) and is a special case in which the aim is to minimize the cost occuring due to the number of personnel (shifts) that are used. The same traits are valid in this problem as in the PTSP; workers with fixed work hours are to be assigned tasks, with specified start and end times, they are qualified for.

In article \cite{krishnamoorthy_2011} they "concentrate mainly on a variant of the PTSP in which the number of personnel (shifts) required is to be minimised."


Difference: "The only cost incurred is due to the number of personnel (shifts) that are used."

Papers of interest:
"Algorithms for large scale Shift Minimisation Personnel Task Scheduling Problems" Krishnamoorthy, Ernst, Baatar (2011)

"The shift minimisation personnel task scheduling problem: A new hybrid approach and computational insights" Smet, Wauters, Mihaylov, Berghe (2014)

"Fast local search and guided local search and their application to British Telecom's workforce scheduling problem" Tsang and Voudouris, 1997 - also with travelling costs, investigates two methods.

"A Triplet-Based Exact Method for the Shift Minimisation Personnel Task Scheduling Problem" Baatar et al., 2015

\section{Other similar problems}\label{other}

Variations of the task assignment problem relevant for our problem include the fixed job schedule problem and the flexible job scheduling problem. The fixed job schedule problem (FJSP) has been studied since the 1970s in the context of task assignment in processors. The problem concerns the distribution of tasks with fixed starting and ending times over a workforce with identical skills, such as processing units \cite{krishnamoorthy_2011}. Such problmes have been solved by I. Gertsbakh, H.I. Stern \cite{Gertsbakh_1977} and Fischetti et al. \cite{fischetti_1992}. In the article by Gertsbakh, a situation where n jobs need to be scheduled over an unlimited number of procesors is studied. The jobs have a specified starting time and duration. The objective of such a problem becomes the minimization of the number of machines needed to perform all tasks. Fischetti solves a similar problem, but adds time constraints, saying that no processor is allowed to work for more than a fixed time \textit{T} during a day as well as a spread time constraint forcing tasks to tasks to spread out with time gap \textit{s} over a processor.	

Another type of problem is the tactical fixed interval scheduling problem. This is a problem very closely related to the SMPTSP problem with the only difference being that the TFISP concerns workers which are always available, such as industrial machines or processors. The problem is studied by for example Kroon et al. \cite{kroon_1997}. As opposed to the FJSP, this problem deals with a heterogeneous workforce. Two different contexts are studied by Kroon et al. One of them concerns the handling of arriving aircraft passengers at an earport. Two modes of transport from the aeroplane to the airport are investigated; directly by gate or by bus. The two transportation modes thus correspond processing units which can take only a number of jobs at the same time.

OFISP


Problem: "A metaheuristic for the fixed job scheduling problem under spread time constraints" André Rossi, http://www.sciencedirect.com/science/article/pii/S0305054809002251 (Fixed job)

\section{Work load allocation and worker satisfaction} \label{WLA}
For most scheduling problems, the main objective is to reduce worker-related costs by reducing the number of workers needed to perform a task, or by reducing the working time for part-time employees. Equivalently, the goal in production industries is to reduce the number of machines needed. However, what has been studied more in recent years is also scheduling problems which take into account worker satisfaction. In an article by Akbari from 2012 a scheduling problem for part-time workers with different preferences, seniority level and productivity is investigated. \cite{akbari_2012}

Trötthet och uttråkad. Något vi borde ta med i litteraturen enligt Torbjörn, fast inte leta källor på det. Mer källor?

Source: "Employee positioning and workload allocation", Eiselt, Marianov, 2006


"Scheduling part-time personnel with availability restrictions and preferences to maximize employee satisfaction" Srimathy Mohan 2008

\section{Methods}
\subsection{TSP with inhom workforce}

Solution methods to compare (similar problems):

"Task assignment and tour scheduling": Loucks and Jacobs, 1991


"Scheduling Restaurant Workers to Minimize Labor Cost and Meet Service Standards" Choi, Hwang and Park, 2009

"An integer linear programming-based heuristic for scheduling heterogeneous, part-time service employees" Heterogenous work force, tour scheduling. Using two objective functions Hojati and Patil, 2010

for another definition as PTSP[F;I;I;W], see "The Personnel Task Scheduling Problem" by Krishnamoorty and Ernst, 2001




\inputencoding{\enc}

\chapter{The mathematical model}\label{chap:mathmod}

\inputencoding{\enc}
%A thorough description of the mathematical model in words and in math. More significant and general constraints in equations and less significant in words.
In this chapter the mathematical model implemented to solve the ten week scheduling problem described in Section \ref{problem_description} will be presented. Prior to the objective function and constraints, the most significant sets and variables will be provided. Section \ref{section:obj} presents the objective function and gives a short description of what it represents. In Section \ref{constraints}, the essential constraints will be presented and explained. A complete model with all definitions and the full set of constraints can be found in Appendix \ref{appendix:mathmod}. It should also be noted that the abbreviations for the different tasks in the mathematical model are: E for Exp, I for Info, P for PL, H for HB and B for BokB.%To get a full view of the mathematical problem the reader  A list of the defined parameters can be found in Appendix \ref{definitions}. 
\section{Set and variable definitions} \label{variables}
To solve the problem, many sets and variables have to be declared. As mentioned before, there are many unique and individual requirements given by the library that have to be met. An example is that some staff members have two alternating schedules for odd and even weeks. The specific cases have to be modelled and result in a variety of sets and variable definitions. Below, we focus on the common and general cases in order to keep the presentation simple and easily accessible. A complete list of the definitions can be found in Appendix \ref{appendix:mathmod}. The most important sets include: \\
\itab{$I$} \tab{Set of staff members}\\
\itab{$I_{lib}$} \tab{Set of librarians ($I_{lib} \subseteq I$)} \\
\itab{$I_{ass}$}	 \tab{Set of assistants ($I_{ass} \subseteq I$)}	\\
\itab{$I_{G}\{g\}$}	 \tab{Set of workers in the group g}	\\
\itab{$W$}                 \tab{Set of all ten weeks} \\
\itab{$W_5$}	\tab{Set of the first five weeks} \\
\itab{$D$}                 \tab{Set of all days in a week}           \\
\itab{$D_5$}	\tab{Set of all five weekdays} \\
\itab{$S_d$}           \tab{Set of shifts available on day \textit{d}}         \\
\itab{$S_3$}           \tab{Set of the first three shifts on a weekday}     \\
\itab{$J_d$}            \tab{Set of task types available on day \textit{d}}   \\
\itab{$G$}	 \tab{Set of groups}	\\
\itab{$V$}	 \tab{Set of possible week rotations (shifts the week by 0-9 steps forwards)}	\\

In order to further define the problem we introduce the following variables.
\begin{align}
    x_{iwdsj}&=
    \begin{cases}
      1, & \text{if staff member \textit{i} is assigned to a task \textit{j} in week \textit{w}, day \textit{d}, shift \textit{s}}\\
      0, & \text{otherwise}
    \end{cases}
    \\
    H_{iwh}&=
    \begin{cases}
      1, & \text{if staff member \textit{i} works weekend \textit{h} in week \textit{w}}\\
      0, & \text{otherwise}
    \end{cases}
	\\
	r_{iw}&=
	\begin{cases}
		1, & \text{if staff member \textit{i} has its schedule rotated \textit{w-1} steps forwards}\\
		0, & \text{otherwise}
	\end{cases}
	\\
	l_{iwd}&=
	\begin{cases}
	  1, & \text{if librarian \textit{i} is a stand-in week \textit{w}, day \textit{d}} \\
	  0, & \text{otherwise}
	\end{cases}
	\\
	a_{iwd}&=
	\begin{cases}
 		1, & \text{if assistant \textit{i} is a stand-in week \textit{w}, day \textit{d}} \\
 		0, & \text{otherwise}
	\end{cases}
	\\
	y_{iwds}&=
	\begin{cases}
 		1, & 
 		\parbox[t]{.7\textwidth}{if staff member \textit{i} works at task type E, I or P in week \textit{w}, day \textit{d}, shift \textit{s}} \\
 		0, & \text{otherwise}
	\end{cases}
	\\
	W_{iwd}&=
	\begin{cases}
	 	1, & \text{if a staff member \textit{i} is working a shift in week \textit{w}, day \textit{d}} \\
	 	0, & \text{otherwise}
	\end{cases}
	\\
	b_{iw}&=
	\begin{cases}
 		1, & \text{if staff member \textit{i} works at HB in week \textit{w}} \\
 		0, & \text{otherwise}
	\end{cases}
	\\
	f_{iw}&=
	\begin{cases}
 		1, & \text{if staff member \textit{i} is assigned to work Friday evening in week \textit{w}} \\
 		0, & \text{otherwise}
	\end{cases}	
	\\
	M_{wds}&=
	\begin{cases}
	 	1, & \text{if a big meeting is placed in week \textit{w}, day \textit{d}, shift \textit{s}} \\
	 	0, & \text{otherwise}
	\end{cases}
	\\
	m_{wdsg}&=
	\begin{cases}
	 	1, & \text{if a meeting is assigned for group \textit{g} in week \textit{w}, day \textit{d}, shift \textit{s}} \\
	 	0, & \text{otherwise}
	\end{cases}
	\\
	d_{iwds}&=
	\begin{cases}
	 	1, & \text{if there is a difference for a staff member \textit{i} in assignment of}\\
	 		& \text{ tasks at shift \textit{s}, day \textit{d} between weeks \textit{w} and \textit{w+5}} \\
	 	0, & \text{otherwise}
	\end{cases}
	\\
	l^{min}&= \text{lowest number of stand-in librarians found at a day (integer)} \\
	a^{min}&= \text{lowest number of stand-in assistants found at a day(integer)} \\
	s^{min}&= \text{weighted sum of numbers of stand-in librarians and assistants} \\
	\delta&= \text{total number of shifts that differ for all staff members (integer)}
\end{align}

Based on the variables defined above it has been possible to model and solve our scheduling problem. The variables \textit{$l^{min}$} and \textit{$a^{min}$} are the ones of most significance, as they represent the number of stand-ins found at the worst day after a run. 

\section{Objective function} \label{section:obj}
As the problem contains multiple objective functions, it has been necessary to weigh them against each other using parameters. These weights are shown in Equation \ref{objfcn} as $M$ and $N$.

\begin{equation} \label{objfcn}
\text{maximize} \hspace{0.3cm} M\cdot s^{min} - N \cdot \delta
\end{equation}
The first part of the objective function represents the lowest amount of stand-in librarians and assistants found at a day, while the second part minimizes the difference between the first and last five weeks. This means that week 1 and 6, 2 and 7 and so on should be as similar as possible. The parameter $N$ prioritizes the similarity of weeks compared to the number of stand-ins. Based on the information given by the library, there is a much higher priority to have many stand-ins. Hence, $M \gg N$, where $N > 0$ holds in our case. The exact relation between $M$ and $N$ is not of high importance, as we have seen that multiple optimal solutions exist, regarding the stand-ins. In our case, we set $M$ to be a hundred times bigger than $N$.




% % % % GO THROUGH THE REFERENCES TO CONSTRAINT SECTION IN THE DOCUMENT
\section{Constraints} \label{constraints}
In order to model the problem it is of relevance to divide many of the constraints into weekend and weekday constraints. Several auxiliary constraints and variables are also added to avoid multiplication of two variables, which would make the problem non-linear. These auxiliary constraints are needed as the solver CPLEX can not handle non-linear constraints, as it displays error messages in case they occur. The auxiliary constraints are left out of this chapter for simplicity reasons. Instead, they can be seen in Appendix \ref{appendix:mathmod}.

\subsection{Demand and assignment constraints} \label{section:demand_ass_constraints}
The most crucial constraint is to ensure that the demand of staff members is met each day. This can be modelled as
\begin{equation} \label{eq:demand}
\sum_{i \in I} x_{iwdsj} = demand_{wdsj}, \;   w\in W,d\in D,s\in S,j\in J_d
\end{equation}
Here, $demand_{wdsj}$ is an integer representing the number of staff members required in week \textit{w}, day \textit{d}, shift \textit{s} for a task \textit{j}.

The following constraint says whether or not a staff member is assigned a task during a weekday. Evening shifts and library on wheels tasks are excluded.
\begin{equation} \label{constr:y_assign}
y_{iwds} = \sum_{j \in J_d\backslash \{L\}} x_{iwdsj}, \;   i \in I, w \in W, d \in D_5, s \in S_3
\end{equation}
This variable assignment is used to simplify a couple of constraints later on.

To ensure that no staff members are assigned more than one task the following constraint is implemented.
\begin{equation} \label{constr:one_task_constraint}
\sum_{s\in S}\sum_{j\in J_d} x_{iwdsj} \leq 1, \;   i\in I, w \in W, d\in D
\end{equation}
However, if we allow a person to have two shifts at the library on wheels on a day, Equation \ref{constr:one_task_constraint} has to be slightly modified. This is left out of this chapter, but is allowed in the complete model.

For most staff members, it is preferred to allow only one PL per week and a maximum of three PL per ten weeks. These are easily modelled with the following constraints.
\begin{equation} \label{constr:one_PL}
\sum_{s \in S_d}\sum_{d \in D} x_{iwdsP} \leq 1, \;   i\in I, w \in W
\end{equation}
\begin{equation} \label{constr:three_PL}
\sum_{w \in W}\sum_{s \in S_d}\sum_{d \in D} x_{iwdsP} \leq 3, \;   i\in I
\end{equation}
The long duration of a PL (in the equations named "P"), is the reason for these preferences. Some staff members are required to have some time free from outer tasks, to be able to perform their inner tasks.

Another preference is to have varying work hours when it comes to the assignment of tasks. Then the more and less desired shifts will be fairly distributed. The following equation is implemented to meet such requirements.
\begin{equation} \label{constr:various_start_times}
\sum_{d \in D_5} y_{iwds} \leq 2, \;   i\in I, w \in W, s \in S_3
\end{equation}
Equation \ref{constr:various_start_times} allows a staff member to have at most two tasks starting at the same hour every week. Worth noting is that library on wheels is disregarded in this constraint as the variable $y_{iwds}$ is used, see Equation \ref{constr:y_assign} for the definition.

It is desirable to avoid assigning too many tasks to a staff member. The reason for this is, as mentioned previously, to let the staff member have some time to allot for inner services during weekdays. The equation below models this preference.
\begin{equation} \label{constr:four_weekly_shifts_at_most}
\sum_{d \in D_5}\sum_{s \in S_d}\sum_{j \in J_d} x_{iwdsj} \leq 4, \;   i\in I, w \in W
\end{equation}
Equation \ref{constr:four_weekly_shifts_at_most} allows a staff member at most four weekday shifts per week. The exception, which is one of the BokB staff members, is left out of this equation for simplicity reasons. To model that, a new subset of \textit{I} needs to be created where that staff member is left out. 
%Ändra till max 4 tasks per vecka diff {36}?

\subsection{Weekend and rotation constraints} \label{section:weekend_rot_constraints}
Every staff member's schedule can be rotated up to nine times, where the decision variable $r_{iv}$ decides the rotation. Therefore, the parameter $qualavail_{iwdsj}$ has to align with the rotation so that all staff members are assigned tasks only when available. This is provided in the following equation.
\begin{equation} \label{constr:qualavail}
x_{iwdsj} \leq \sum_{v \in V} r_{iv}*qualavail_{i\omega(v,w)dsj}, \;   i \in I, w \in W, d \in D, s \in S_d, j \in J_d
\end{equation}
The function $\omega(v,w)$ represents the modulus function $mod_{10}(w-v+10)+1$ and is a function which takes rotations into account when calculating the week to look for in the availability matrix. Table \ref{tab:mod} gives a better understanding of how this modulus function is used. 

The variable \textit{w}, in Table \ref{tab:mod}, represents the current week in the initial schedule, and \textit{v} represents \textit{v-1} rotations to the right from the initial schedule. With initial schedule, we refer to the schedule, where all staff members' weekends occur in the first week. The resulting week, when rotations have been taken into account, can be seen inside the cells. When scheduling, it is the resulting week $\omega$ which is of interest when checking if a staff member is available for a task assignment. 
\begin{table}[H]
\centering
\caption{Resulting table of the function $\omega(v,w) = mod_{10}(w-v+10)+1$. Here, $w$ is the week in the initial schedule, $v$ is the rotation variable. Inside the cells, the resulting week is shown when rotations have been made.}
\label{tab:mod}
\begin{tabular}{llllllllllll}
    &                         &                         &                         &                         &                         & \multicolumn{2}{l}{w =}                           &                         &                         &                         &                         \\
    &                         & 1                       & 2                       & 3                       & 4                       & 5                       & 6                       & 7                       & 8                       & 9                       & 10                      \\ \cline{3-12} 
    & \multicolumn{1}{l|}{1}  & \multicolumn{1}{l|}{1}  & \multicolumn{1}{l|}{2}  & \multicolumn{1}{l|}{3}  & \multicolumn{1}{l|}{4}  & \multicolumn{1}{l|}{5}  & \multicolumn{1}{l|}{6}  & \multicolumn{1}{l|}{7}  & \multicolumn{1}{l|}{8}  & \multicolumn{1}{l|}{9}  & \multicolumn{1}{l|}{10} \\ \cline{3-12} 
    & \multicolumn{1}{l|}{2}  & \multicolumn{1}{l|}{10} & \multicolumn{1}{l|}{1}  & \multicolumn{1}{l|}{2}  & \multicolumn{1}{l|}{3}  & \multicolumn{1}{l|}{4}  & \multicolumn{1}{l|}{5}  & \multicolumn{1}{l|}{6}  & \multicolumn{1}{l|}{7}  & \multicolumn{1}{l|}{8}  & \multicolumn{1}{l|}{9}  \\ \cline{3-12} 
    & \multicolumn{1}{l|}{3}  & \multicolumn{1}{l|}{9}  & \multicolumn{1}{l|}{10} & \multicolumn{1}{l|}{1}  & \multicolumn{1}{l|}{2}  & \multicolumn{1}{l|}{3}  & \multicolumn{1}{l|}{4}  & \multicolumn{1}{l|}{5}  & \multicolumn{1}{l|}{6}  & \multicolumn{1}{l|}{7}  & \multicolumn{1}{l|}{8}  \\ \cline{3-12} 
    & \multicolumn{1}{l|}{4}  & \multicolumn{1}{l|}{8}  & \multicolumn{1}{l|}{9}  & \multicolumn{1}{l|}{10} & \multicolumn{1}{l|}{1}  & \multicolumn{1}{l|}{2}  & \multicolumn{1}{l|}{3}  & \multicolumn{1}{l|}{4}  & \multicolumn{1}{l|}{5}  & \multicolumn{1}{l|}{6}  & \multicolumn{1}{l|}{7}  \\ \cline{3-12} 
v = & \multicolumn{1}{l|}{5}  & \multicolumn{1}{l|}{7}  & \multicolumn{1}{l|}{8}  & \multicolumn{1}{l|}{9}  & \multicolumn{1}{l|}{10} & \multicolumn{1}{l|}{1}  & \multicolumn{1}{l|}{2}  & \multicolumn{1}{l|}{3}  & \multicolumn{1}{l|}{4}  & \multicolumn{1}{l|}{5}  & \multicolumn{1}{l|}{6}  \\ \cline{3-12} 
    & \multicolumn{1}{l|}{6}  & \multicolumn{1}{l|}{6}  & \multicolumn{1}{l|}{7}  & \multicolumn{1}{l|}{8}  & \multicolumn{1}{l|}{9}  & \multicolumn{1}{l|}{10} & \multicolumn{1}{l|}{1}  & \multicolumn{1}{l|}{2}  & \multicolumn{1}{l|}{3}  & \multicolumn{1}{l|}{4}  & \multicolumn{1}{l|}{5}  \\ \cline{3-12} 
    & \multicolumn{1}{l|}{7}  & \multicolumn{1}{l|}{5}  & \multicolumn{1}{l|}{6}  & \multicolumn{1}{l|}{7}  & \multicolumn{1}{l|}{8}  & \multicolumn{1}{l|}{9}  & \multicolumn{1}{l|}{10} & \multicolumn{1}{l|}{1}  & \multicolumn{1}{l|}{2}  & \multicolumn{1}{l|}{3}  & \multicolumn{1}{l|}{4}  \\ \cline{3-12} 
    & \multicolumn{1}{l|}{8}  & \multicolumn{1}{l|}{4}  & \multicolumn{1}{l|}{5}  & \multicolumn{1}{l|}{6}  & \multicolumn{1}{l|}{7}  & \multicolumn{1}{l|}{8}  & \multicolumn{1}{l|}{9}  & \multicolumn{1}{l|}{10} & \multicolumn{1}{l|}{1}  & \multicolumn{1}{l|}{2}  & \multicolumn{1}{l|}{3}  \\ \cline{3-12} 
    & \multicolumn{1}{l|}{9}  & \multicolumn{1}{l|}{3}  & \multicolumn{1}{l|}{4}  & \multicolumn{1}{l|}{5}  & \multicolumn{1}{l|}{6}  & \multicolumn{1}{l|}{7}  & \multicolumn{1}{l|}{8}  & \multicolumn{1}{l|}{9}  & \multicolumn{1}{l|}{10} & \multicolumn{1}{l|}{1}  & \multicolumn{1}{l|}{2}  \\ \cline{3-12} 
    & \multicolumn{1}{l|}{10} & \multicolumn{1}{l|}{2}  & \multicolumn{1}{l|}{3}  & \multicolumn{1}{l|}{4}  & \multicolumn{1}{l|}{5}  & \multicolumn{1}{l|}{6}  & \multicolumn{1}{l|}{7}  & \multicolumn{1}{l|}{8}  & \multicolumn{1}{l|}{9}  & \multicolumn{1}{l|}{10} & \multicolumn{1}{l|}{1}  \\ \cline{3-12} 
\end{tabular}
\end{table}
An example to better understand the function: imagine that we are looking at an unrotated initial schedule where everyone is assumed to work weekend the first week. Say that we look at a staff member's first week, $w=1$. If this schedule is rotated two times to the right ($v=3$), then the first week in the new rotated schedule represents the previous ninth week, that is $w=9$ when no rotation exists. The availability to look at, in the initial schedule is, therefore, week nine.

Initially, all staff members are assigned weekend work the first week. Then, a week rotation variable is introduced that rotates the staff members' availability matrices so that the demand constraints on evenings and weekends can be met. The most basic constraints, regarding which week a staff member's weekend work occurs and its week rotation, can be seen below.

\begin{equation} \label{constr:one_rot}
\sum_{w \in W} r_{iw} = 1, \;   i\in I
\end{equation}
\begin{equation} \label{constr:max_one_weekend}
\sum_{w \in W} H_{iwh} \leq 1, \;   i\in I, h \in \{1,2\}
\end{equation}
\begin{equation} \label{constr:rot_weekend}
r_{iw} \geq H_{iw1}, \;   i\in I, w \in W
%	\begin{cases}
% 		1, & \text{if $H_{iwh=1} = 1$} \\
% 		0/1, & \text{otherwise}
%	\end{cases}
%	\;   i\in I, w \in W \;
\end{equation}

Equation \ref{constr:one_rot} provides all staff members with a rotation of their schedule regardless if they are working weekends or not. Equation \ref{constr:max_one_weekend} allows a staff member a maximum of two weekends, for $h=1$ and $h=2$, per ten weeks.

Once a staff member is assigned a new rotation, the availability matrix has to be correctly rotated. This is done using Equation \ref{constr:rot_weekend}. In case a staff member is never due for weekend work, that is, $H_{iwh} = 0, \; \forall i\in I, w \in W, h \in \{1,2\}$, then the rotation $r_{iw}$ is free, as its availability matrix is identical for all weeks. 

A staff member is supposed to work weekends with a five-week interval. However, in case there are enough staff members to satisfy the demand on weekends, it may be enough for some to only work one weekend per ten weeks. To avoid problems with the rotation if only the second weekend is assigned to a staff member, the following equation is used.
\begin{equation} \label{constr:five_week_interval}
r_{i(mod_{10}(w+4)+1)} \geq H_{iw2}, \;   i\in I, w \in W
\end{equation}
Worth noting is that if a staff member is assigned both weekends, then Equation \ref{constr:five_week_interval} provides the same information as Equation \ref{constr:rot_weekend}.

A staff member is supposed to work with the same task both Saturday and Sunday when working a weekend. This can be modelled with the following constraints.
\begin{equation} \label{constr:consecutive_days}
\sum_{j \in J_d} x_{iw61j} + \sum_{j \in J_d} x_{iw71j} = 2*\sum_{h = 1}^{2} H_{iwh}, \;   i\in I, w \in W
\end{equation}
\begin{equation} \label{constr:same_tasks}
x_{iw61j} = x_{iw71j}, \;   i\in I, w \in W, j \in J_d
\end{equation}
Equation \ref{constr:consecutive_days} ensures that the staff member will work Saturday and Sunday consecutively, whenever he or she is due for weekend work. To make sure it is the same task as well, Equation \ref{constr:same_tasks} is implemented.

Friday evening is also included in an extension to working Saturday and Sunday, unless the staff member is assigned to HB. It is, however, not a necessity to perform the same task Friday evening as during the weekend, thus Fridays are not included in Equation \ref{constr:same_tasks}. Equation \ref{constr:friday_added} below adds Fridays to the weekend.
\begin{equation} \label{constr:friday_added}
\sum_{j \in J \backslash \{L\}}x_{iw54j} = f_{iw}, \;   i \in I, w \in W
\end{equation}
Here, "L" refers to the task type library on wheels. The variable $f_{iw}$ is, as mentioned in the variable declaration, equal to one if and only if a staff member is working weekend as well as not being assigned to HB.

It is of interest to implement a constraint to prevent staff members from being assigned to HB more than once every ten weeks in order to avoid unfairness. Why this is preferable is described in Section \ref{section:library_tasks}. This constraint can be modelled as.
\begin{equation} \label{constr:max_one_hb}
\sum_{w \in W}\sum_{d = 6}^{7}x_{iwd1B} \leq 2, \;   i \in I
\end{equation}
Here, "B" refers to HB. Equation \ref{constr:same_tasks} and \ref{constr:max_one_hb} ensure both that a staff member only can be assigned to HB two days per ten weeks, and that those days are consecutive. The exception is in case a staff member only works in HB, which is disregarded in this simplified model.

\subsection{Objective function constraints} \label{section:obj_fcn_constraints}
The variable $\delta$ in the objective function, Equation \ref{objfcn}, is defined by the following equation.
\begin{equation} \label{constr:delta}
\delta = \sum_{i \in I} \sum_{w \in W_5} \sum_{d \in D_5} \sum_{s \in S_3} d_{iwds}
\end{equation}
This equation gives an integer value, which represents the total number of shift differences that occur for all workers, through all weekdays, in the library. The variable $d_{iwds}$, defined in Equation \ref{constr:obj_fcn_shifts} below, compares task assignments between two five-week separated weeks.

To calculate the variable $s^{min}$ used in the objective function, \ref{objfcn}, the following equation is used.
\begin{equation} \label{constr:s_min}
s^{min} \leq L\cdot \sum_{i \in I_{lib}} l_{iwd} + A\cdot \sum_{i \in I_{ass}} a_{iwd}, \;   w \in W, d \in D_5
\end{equation}
%\begin{equation} \label{constr:l_min}
%l^{min} \leq \sum_{i \in I_{lib}} l_{iwd}, \;   w \in W, d \in D_5
%\end{equation}
%\begin{equation} \label{constr:a_min}
%a^{min} \leq \sum_{i \in I_{ass}} a_{iwd}, \;   w \in W, d \in D_5
%\end{equation}

Here, $l_{iwd}$ and $a_{iwd}$ are binary variables stating whether a staff member is a stand-in on a day or not. Since $s^{min}$ is being maximized in the objective function, it will assume the lowest value of the sum of stand-in librarians and assistants for any day during the ten weeks. It will, therefore, represent the worst solution found regarding stand-ins. Just as for previous equations, auxiliary constraints have been left out for simplicity reasons. Here, it is left out how $l_{iwd}$ and $a_{iwd}$ are determined.

If $L < A$ holds in the model, the solver would prioritize assistants over librarians as stand-ins. Librarians are, however, more desired as stand-ins, due to their ability to perform all types of tasks. Therefore, it is desired to let $L \geq A$.

As stated in Section \ref{section:obj}, two weeks with a five-week interval should be as similar as possible. Equation \ref{constr:obj_fcn_shifts}, together with Equation \ref{constr:delta} and the objective function equation, \ref{objfcn}, provides this preference.
\begin{equation} \label{constr:obj_fcn_shifts}
d_{iwds} = \abs{y_{iwds} - y_{i(w+5)ds}}, \;   i \in I, w \in W_5, d \in D_5, s \in S_3
\end{equation}
The decision variable $d_{iwds}$ states if there is a difference in assignment between two tasks at the same hour and day for the two weeks, $w$ and $w+5$. The differences occur if, say, an Exp task is assigned on Monday week one at 8 a.m. to 10 a.m. and no task is assigned the same shift Monday week six. Thus, a variable that is minimized in the objective function.

\subsection{Meeting constraints} \label{section:meeting_constraints}
At the library there are both library meetings and group meetings, which both occur with a five-week interval. Library meetings are set to take place from 8 a.m. to 10 a.m. on Mondays, whereas group meetings are more freely distributed. A few staff members are not assigned library meetings, as they are needed in the library to keep it running. The set $I_{big}$ in the equation below consists of all staff members who are to be assigned library meetings. The constraints modelling library meetings are as follows.
\begin{equation} \label{constr:library_meetings}
\sum_{w \in W_5} M_{w11} = 1
\end{equation}
\begin{equation} \label{constr:library_meetings2}
M_{(w+5)11} = M_{w11}, \;   w \in W_5
\end{equation}
\begin{equation} \label{constr:library_meetings3}
\sum_{s=1}^{3} \sum_{j \in J_1 \backslash \{L\}} x_{iw1sj} \leq 1-M_{w11}, \;   i \in I \backslash I_{big}, w \in W
\end{equation}
Equation \ref{constr:library_meetings} and \ref{constr:library_meetings2} assign two library meetings with a five-week interval during the ten-week scheduling period. Equation \ref{constr:library_meetings3} makes sure that the staff members, that do not attend library meetings, are not assigned any other task during the day of the meeting. This implicitly force the staff members, that do not attend the library meetings, to meet the library demand constraints during the meeting. 

The constraints added to model the group meetings are somewhat similar to the library meeting constraints. Just as for library meetings, they take place two times during the ten weeks with a five-week interval, which is described by Equations \ref{constr:dep_meetings} and \ref{constr:dep_meetings2}.
\begin{equation} \label{constr:dep_meetings}
\sum_{w \in W_5}\sum_{d \in D_5}\sum_{s \in S_3} m_{wdsg} = 1, \; g \in G
\end{equation}
\begin{equation} \label{constr:dep_meetings2}
m_{(w+5)dsg} = m_{wdsg}, \;   g \in G, w \in W_5, d \in D_5, s \in S_3
\end{equation}
\begin{equation} \label{constr:dep_meetings3}
m_{wdsg} + x_{iwdsj} \leq 1, \;   g \in G, i \in I_{G}\{g\}, w \in W, d \in D_5, s \in S_3, j \in J_d
\end{equation}
\begin{equation} \label{constr:dep_meetings4}
m_{wdsg} \leq \sum_{v \in V} r_{iv}*qualavail_{i\omega(v,w)dsE}, \;   g \in G, i \in I_{G}\{g\}, w \in W, d \in D_5, s \in S_3
\end{equation}
Equation \ref{constr:dep_meetings3} prohibits a staff member from multitasking, that is, to attend a meeting and work with a task simultaneously. Equation \ref{constr:dep_meetings4} enables group meetings only when everyone in that group is available. The rotation is also taken into the account.

In Equation \ref{constr:dep_meetings4} $\omega$ is the week calculation when rotations are taken into the account, see Section \ref{section:weekend_rot_constraints}. The task type "E" stands for Exp, and is used due to it is the only task type everyone is available for, as \textit{qualavail} is dependent on the task variable \textit{j}. A better way to model this would be to create a new parameter, say $avail_{iwds}$, that is not dependent on staff member qualifications. As this way of modelling works, this was not implemented in order to to save us some time.


\inputencoding{\enc}

%\chapter{Two heuristics}\label{chap:heur}
%
%\inputencoding{\enc}
%%The two heuristical approaches


In the AMPL implementation, the model was identical to the one described in Chapter \ref{chap:mathmod}. However, in order to implement the heuristics, the model's constraints had to be relaxed. Figure \ref{fig:AMPL_vs_heur} illustrates this process as some constraints considered hard in the original model are softened in the heuristics. The alterations of the model for the two heuristics is described in Tables \ref{tab:task_constraints} and \ref{tab:weekly_task_constraints}. The reason for relaxing the model is to create a larger neighbourhood for the heuristic to search through so that it can move more freely between solutions and increase the chance of finding the optimal solution.

% Define block styles
\tikzstyle{block} = [rectangle, draw=none, fill=white,
    text width=8em, text ragged, rounded corners, minimum height=0em, node distance =1.5cm]
\tikzstyle{line} = [draw, -latex']
    
\newcommand*{\h}{\hspace{18pt}}% for indentation
\newcommand*{\hh}{\hspace{24pt}}% double indentation

\begin{figure}[!h]
\caption{Illustration of the difference between the model used in AMPL and the model  for a heuristic}
\label{fig:AMPL_vs_heur}
\begin{center}
\begin{tikzpicture}[node distance = 2cm , auto, scale=0.7, every node/.style={scale=0.7}]
	%Row 1
	\node[block, node distance= 6cm, text width=4em] (AMPL){\Large \textbf{AMPL}};
	\node[block, right of=AMPL, node distance= 5cm] (Mid){};
	\node[block, right of=Mid, node distance= 6cm] (Heur){\Large \textbf{HEURISTIC}};
	%Row 2
	\node[block, below of = AMPL, text width=15em, node distance =1.5cm](ObjfA) {\hh AMPL Objective Function};
	\node[block, below of = Heur, text width=15em, node distance =1.5cm](ObjfH) { \h Heuristic Objective Function};
	%Row 3	
	\node[block, below of = ObjfA, text width=10em](ConstA) {AMPL Constraints};
	\node[block, below of = Mid, text width=10em, node distance=2.5cm](SConst) {\h Soft Constraints};
	%Row 4
	\node[block, below of = SConst, text width=10em, node distance=1cm](HConst) {\h Hard Constraints};
	\node[block, right of = HConst, text width=15em, node distance=6cm](ConstH) {\h Heuristic Constraints};
	%Invisible node
	\node[block, right of=SConst,scale=0.05, node distance=4cm](inv){};
	%\node at (ObjfH)[block, left of=ObjfH,scale=0.05,node distance=2.7cm](inv2){};
	\node at (8.4,-1.6)(inv2){};
	\begin{scope}[every path/.style=line]
		\path (ObjfA) -- (ObjfH);
		\path (ConstA.east) -- (HConst.west);
		\path (ConstA.east) -- (SConst.west);
		\path (SConst.east) -- (inv2.west);
		\path (HConst.east) -- (ConstH.west);
%		\path [-,draw](SConst) -- (inv);
%		\path (inv) |- (ObjfH);
	\end{scope}
	\draw [color=gray!70,thick](15,1) rectangle (-3,-5);
\end{tikzpicture}
\end{center}
\end{figure}


\begin{table}[!h]
\centering
\caption{Week block scheduling approach; soft and relaxed constraints}
\label{tab:weekly_task_constraints}
\begin{tabular}{|p{4cm}|p{7cm}|}
\hline
% ---------------------------------------------------------
\multicolumn{2}{|l|}{\cellcolor{gray!90} \textbf{Soft Constraints}} \\
\hline 
\rowcolor{Gray} Affecting constraints & Constraint description \\ \hline
\ref{eq:demand} & The amount of workers needed for every shift and task type in the library.  \\ \hline
\ref{constr:three_PL} & Number of PL per ten weeks restriction. \\ \hline
% ---------------------------------------------------------
\multicolumn{2}{|l|}{\cellcolor{gray!90} \textbf{Relaxed Constraints}} \\
\hline 
\rowcolor{Gray} Affecting constraints & Constraint alteration \\ \hline
Several. $W = W_5$ in all constraints. & Five week scheduling instead of ten week scheduling. \\ \hline
\ref{constr:obj_fcn_shifts} & Any two weeks \textit{w} and \textit{w+5} shall be as similar as possible. \\ \hline
BokB-constraints & BokB placed according to constraints, but the schedule is fixed. \\ \hline
Availability data & Even and odd week workers have availability at each shift according to the stricter of the two sets. \\ \hline
\ref{constr:library_meetings} - \ref{constr:dep_meetings4} & Meetings are not implemented. \\ \hline
\end{tabular}
\end{table}


\begin{table}[!h]
\centering
\caption{Task distribution approach; soft and relaxed constraints}
\label{tab:task_constraints}
\begin{tabular}{|p{4cm}|p{7cm}|}
\hline
% ---------------------------------------------------------
\multicolumn{2}{|l|}{\cellcolor{gray!90} \textbf{Soft Constraints}} \\
\hline 
\rowcolor{Gray} Affecting constraints & Constraint description \\ \hline
\ref{constr:one_task_constraint} & One task per day restriction.  \\ \hline
\ref{constr:four_weekly_shifts_at_most} & Four tasks per week restriction. \\ \hline
\ref{constr:one_PL} & One PL per week restriction. \\ \hline
\ref{constr:three_PL} & Number of PL per ten weeks restriction. \\ \hline
\ref{constr:various_start_times} & Not more than two tasks at the same shift in a week restriction.  \\ \hline
% ---------------------------------------------------------
\multicolumn{2}{|l|}{\cellcolor{gray!90} \textbf{Relaxed Constraints}} \\
\hline 
\rowcolor{Gray} Affecting constraints & Constraint alteration \\ \hline
Several. $W = W_5$ in all constraints. & Five week scheduling instead of ten week scheduling. \\ \hline
\ref{constr:obj_fcn_shifts} & Any two weeks \textit{w} and \textit{w+5} shall be as similar as possible. \\ \hline
BokB-constraints & BokB placed according to constraints, but the schedule is fixed. \\ \hline
Availability data & Even and odd week workers have availability at each shift according to the stricter of the two sets. \\ \hline
\ref{constr:library_meetings} - \ref{constr:dep_meetings4} & Meetings are not implemented. \\ \hline
\end{tabular}
\end{table}

%\inputencoding{\enc}

\chapter{Two heuristics}\label{chap:weekly}

\inputencoding{\enc}


In Appendix \ref{appendix:flow_charts}, Figure \ref{flow_chart}, is a flow chart of the implemented heuristic presented.

For every worker their information such as availability and qualification is inserted into a table template in Excel. It is then written to a text file using Visual Basic code, which in turn is read by the heuristic.

\section{Block creation} \label{block_creation}
A big part of this heuristic is to create the big pool of unique week appearances. These are then filtered for each of the worker based on its availability. The workers availability are generalized into three categories: \textit{Weekend week}, \textit{weekrest week} and \textit{weekday week}, where weekday week occurs three times during a five week period, see Table \ref{tab:Bob_avail}. 


One week block contains seven days and up to four shifts where each shift can contain up to three task types. Table \ref{Generalized weekblock} below is a representation of a general week block with all possible tasks for each day \textit{d} and shift \textit{s}. \textit{I} in the table represents "No task" is assigned that day, \textit{D} represents "Desk task" meaning either Exp or Info, \textit{PL} represents "Fetch list" and \textit{HB} represents "Hageby". 

\begin{table}[!h]
\centering
\caption{A generalized weekblock with all existing tasks}
\label{Generalized weekblock}
\begin{tabular}{cccccccc}
                         & Mon                         & Tue                         & Wed                         & Thu                         & Fri                         & Sat                         & Sun                         \\ \cline{2-8} 
\multicolumn{1}{c|}{08:00-10:00} & \multicolumn{1}{c|}{I,D,PL} & \multicolumn{1}{c|}{I,D,PL} & \multicolumn{1}{c|}{I,D,PL} & \multicolumn{1}{c|}{I,D,PL} & \multicolumn{1}{c|}{I,D,PL} & \multicolumn{1}{c|}{I,D,HB} & \multicolumn{1}{c|}{I,D,HB} \\ \cline{2-8} 
\multicolumn{1}{c|}{10:00-13:00} & \multicolumn{1}{c|}{D}      & \multicolumn{1}{c|}{D}      & \multicolumn{1}{c|}{D}      & \multicolumn{1}{c|}{D}      & \multicolumn{1}{c|}{D}      &       \\ \cline{2-6} 
\multicolumn{1}{c|}{13:00-16:00} & \multicolumn{1}{c|}{D}      & \multicolumn{1}{c|}{D}      & \multicolumn{1}{c|}{D}      & \multicolumn{1}{c|}{D}      & \multicolumn{1}{c|}{D}      &       \\ \cline{2-6} 
\multicolumn{1}{c|}{16:00-20:00} & \multicolumn{1}{c|}{D}      & \multicolumn{1}{c|}{D}      & \multicolumn{1}{c|}{D}      & \multicolumn{1}{c|}{D}      & \multicolumn{1}{c|}{D}      &       \\ \cline{2-6} 
\end{tabular}
\end{table}

Every day must contain exactly \textit{one} task from either of the four shifts when creating a week block. The tasks \textit{I} and \textit{PL} are ranging more than one shift. The duration of a PL is three shift and I refers to the entire day. Hence, they are both placed in the first shift to simplify the complete task representation. When creating the combinations of block appearances there are additional conditions that have to be met. These are:
\begin{enumerate}  
\item At most two tasks can be assigned the same shift and week.\label{first_item}
\item No more than two evenings are allowed each week, one of which is required to be a Friday. \label{second_item}
\item At most one PL is allowed in a weekblock. \label{third_item}
\item Saturday and Sunday shall always contain the same task type.\label{fourth_item}
\item If Saturday and Sunday contain Desk tasks, then so shall Friday afternoon (fourth shift). \label{friday_as_weekend}
\item No more than four tasks are allowed during the weekdays leaving at least one day without tasks. \label{fifth_item}
\end{enumerate}

%23,328 -> 9072 when item 4 applied
Too see the growth of the problem when more tasks are added, consider the following example: If item \ref{first_item}, \ref{second_item}, \ref{third_item}, \ref{friday_as_weekend} and \ref{fifth_item} are disregarded there exists $6^5*3 = 23,328$ unique weekblocks. In contrast, if Exp and Info were to be considered separately, instead of the combination of the two, the possible combinations would be $10^5*4 = 400,000$.  By applying all conditions the total amount of unique block appearances for this implementation are 4,175. 

An illustration of one of the 4,175 existing block can be seen in Table \ref{block_example} below.
\begin{table}[!h]
\centering
\caption{Illustration of one of the unique block appearances}
\label{block_example}
\begin{tabular}{cccccccc}
                           & Mon                                            & Tue                                             & Wed                    & Thu                                            & Fri                    & Sat                                             & Sun                                             \\ \cline{2-8} 
\multicolumn{1}{c|}{08:00-10:00}  & \multicolumn{1}{c|}{}                          & \multicolumn{1}{c|}{\cellcolor[HTML]{FCFF2F}PL} & \multicolumn{1}{c|}{I} & \multicolumn{1}{c|}{}                          & \multicolumn{1}{c|}{I} & \multicolumn{1}{c|}{\cellcolor[HTML]{FCFF2F}HB} & \multicolumn{1}{c|}{\cellcolor[HTML]{FCFF2F}HB} \\ \cline{2-8} 
\multicolumn{1}{c|}{10:00-13:00} & \multicolumn{1}{c|}{}                          & \multicolumn{1}{c|}{\cellcolor[HTML]{FCFF2F}}   & \multicolumn{1}{c|}{}  & \multicolumn{1}{c|}{}                          & \multicolumn{1}{c|}{}  &                                                 &                                                 \\ \cline{2-6}
\multicolumn{1}{c|}{13:00-16:00} & \multicolumn{1}{c|}{}                          & \multicolumn{1}{c|}{\cellcolor[HTML]{FCFF2F}}   & \multicolumn{1}{c|}{}  & \multicolumn{1}{c|}{\cellcolor[HTML]{FCFF2F}D} & \multicolumn{1}{c|}{}  &                                                 &                                                 \\ \cline{2-6}
\multicolumn{1}{c|}{16:00-20:00} & \multicolumn{1}{c|}{\cellcolor[HTML]{FCFF2F}D} & \multicolumn{1}{c|}{}                           & \multicolumn{1}{c|}{}  & \multicolumn{1}{c|}{}                          & \multicolumn{1}{c|}{}  &                                                 &                                                 \\ \cline{2-6}
\end{tabular}
\end{table}

This block contains five tasks; two of them are weekend tasks and three are weekday tasks. Which week this block can be assigned to is dependent on the worker's rotation. Since Hageby is assigned to the weekblock one can conclude that only a librarian can have this block assigned to itself. Due to this fact, the Desk tasks can imply either Exp or Info desk work as librarians are qualified for both.


\section{Block percolation}
After creating all existing week combinations they are percolated to each of the workers based on their availability in each of the three categories mentioned in Section \ref{block_creation}. Table \ref{blocks_available_per_worker} in Appendix \ref{appendix:weekblock} shows the results after this percolation has been made for all workers.


All of the values in the table are a subset of the total amount of 4,175 blocks. Knowing the structure of the problem, one can deduce that the amount of available weekrest blocks are always less than or equal to the available weekday blocks, as it should be. The only difference between the two mentioned block categories is when a worker is free from work due to its weekrest. Therefore, using this information one can interpret that they are equal when the worker is never working weekends.
\begin{table}[!h]
\centering
\caption{Typical availability for a generalized worker. Yellow signifies that the worker is available. In parenthesis, the weekend shift hours.}
\label{typical_availability}
\begin{tabularx}{\textwidth}{|X|l|l|l|l|l|l|l|X|}
\hline
%-------------------------------------------------------------------
\textbf{Weekend week}& \colcell \textbf{Mon} & \colcell \textbf{Tue} & \colcell \textbf{Wed} & \colcell \textbf{Thu} & \colcell \textbf{Fri} & \colcell \textbf{Sat} & \colcell \textbf{Sun}
\\ \hline 
%%------------------------------------------------------------------- 
%\rowcolor{Gray} 
\colcell 08:00-10:00 (11:00-16:00) & \colcelltwo & \colcelltwo & \colcelltwo & \colcelltwo & \colcelltwo & \colcelltwo & \colcelltwo
\\ \hline 
%%-------------------------------------------------------------------
%\rowcolor{Gray} 
\colcell 10:00-13:00 & \colcelltwo & \colcelltwo & \colcelltwo & \colcelltwo & \colcelltwo &   & 
\\ \hline 
%%-------------------------------------------------------------------
%\rowcolor{Gray} 
\colcell 13:00-16:00 & \colcelltwo & \colcelltwo & \colcelltwo & \colcelltwo & \colcelltwo & &
\\ \hline 
%%-------------------------------------------------------------------
%\rowcolor{Gray} 
\colcell 16:00-20:00 & & & \colcelltwo & & \colcelltwo & &
\\ \hline 
%%-------------------------------------------------------------------
\end{tabularx}
\begin{tabularx}{\textwidth}{|X|l|l|l|l|l|l|l|X|}
\hline
%-------------------------------------------------------------------
\textbf{Weekrest week}& \colcell \textbf{Mon} & \colcell \textbf{Tue} & \colcell \textbf{Wed} & \colcell \textbf{Thu} & \colcell \textbf{Fri} & \colcell \textbf{Sat} & \colcell \textbf{Sun}
\\ \hline 
%%------------------------------------------------------------------- 
%\rowcolor{Gray} 
\colcell 08:00-10:00 (11:00-16:00) & \colcelltwo & \colcelltwo & \colcelltwo & & & & 
\\ \hline 
%%-------------------------------------------------------------------
%\rowcolor{Gray} 
\colcell 10:00-13:00 & \colcelltwo & \colcelltwo & \colcelltwo & & & & 
\\ \hline 
%%-------------------------------------------------------------------
%\rowcolor{Gray} 
\colcell 13:00-16:00 & \colcelltwo & \colcelltwo & \colcelltwo & & & &
\\ \hline 
%%-------------------------------------------------------------------
%\rowcolor{Gray} 
\colcell 16:00-20:00 & & & \colcelltwo & & & &
\\ \hline 
%%-------------------------------------------------------------------
\end{tabularx}
\begin{tabularx}{\textwidth}{|X|l|l|l|l|l|l|l|X|}
\hline
%-------------------------------------------------------------------
\textbf{Weekday week}& \colcell \textbf{Mon} & \colcell \textbf{Tue} & \colcell \textbf{Wed} & \colcell \textbf{Thu} & \colcell \textbf{Fri} & \colcell \textbf{Sat} & \colcell \textbf{Sun}
\\ \hline 
%%------------------------------------------------------------------- 
%\rowcolor{Gray} 
\colcell 08:00-10:00 (11:00-16:00) & \colcelltwo & \colcelltwo & \colcelltwo & \colcelltwo & \colcelltwo & & 
\\ \hline 
%%-------------------------------------------------------------------
%\rowcolor{Gray} 
\colcell 10:00-13:00 & \colcelltwo & \colcelltwo & \colcelltwo & \colcelltwo & \colcelltwo &   & 
\\ \hline 
%%-------------------------------------------------------------------
%\rowcolor{Gray} 
\colcell 13:00-16:00 & \colcelltwo & \colcelltwo & \colcelltwo & \colcelltwo & \colcelltwo & &
\\ \hline 
%%-------------------------------------------------------------------
%\rowcolor{Gray} 
\colcell 16:00-20:00 & & & \colcelltwo & & & &
\\ \hline 
%%-------------------------------------------------------------------
\end{tabularx}
\begin{tabularx}{\textwidth}{|X|l|l|l|l|l|l|l|X|}
\hline
%-------------------------------------------------------------------
\textbf{Weekday week}& \colcell \textbf{Mon} & \colcell \textbf{Tue} & \colcell \textbf{Wed} & \colcell \textbf{Thu} & \colcell \textbf{Fri} & \colcell \textbf{Sat} & \colcell \textbf{Sun}
\\ \hline 
%%------------------------------------------------------------------- 
%\rowcolor{Gray} 
\colcell 08:00-10:00 (11:00-16:00) & \colcelltwo & \colcelltwo & \colcelltwo & \colcelltwo & \colcelltwo & & 
\\ \hline 
%%-------------------------------------------------------------------
%\rowcolor{Gray} 
\colcell 10:00-13:00 & \colcelltwo & \colcelltwo & \colcelltwo & \colcelltwo & \colcelltwo &   & 
\\ \hline 
%%-------------------------------------------------------------------
%\rowcolor{Gray} 
\colcell 13:00-16:00 & \colcelltwo & \colcelltwo & \colcelltwo & \colcelltwo & \colcelltwo & &
\\ \hline 
%%-------------------------------------------------------------------
%\rowcolor{Gray} 
\colcell 16:00-20:00 & & & \colcelltwo & & & &
\\ \hline 
%%-------------------------------------------------------------------
\end{tabularx}
\begin{tabularx}{\textwidth}{|X|l|l|l|l|l|l|l|X|}
\hline
%-------------------------------------------------------------------
\textbf{Weekday week}& \colcell \textbf{Mon} & \colcell \textbf{Tue} & \colcell \textbf{Wed} & \colcell \textbf{Thu} & \colcell \textbf{Fri} & \colcell \textbf{Sat} & \colcell \textbf{Sun}
\\ \hline 
%%------------------------------------------------------------------- 
%\rowcolor{Gray} 
\colcell 08:00-10:00 (11:00-16:00) & \colcelltwo & \colcelltwo & \colcelltwo & \colcelltwo & \colcelltwo & & 
\\ \hline 
%%-------------------------------------------------------------------
%\rowcolor{Gray} 
\colcell 10:00-13:00 & \colcelltwo & \colcelltwo & \colcelltwo & \colcelltwo & \colcelltwo &   & 
\\ \hline 
%%-------------------------------------------------------------------
%\rowcolor{Gray} 
\colcell 13:00-16:00 & \colcelltwo & \colcelltwo & \colcelltwo & \colcelltwo & \colcelltwo & &
\\ \hline 
%%-------------------------------------------------------------------
%\rowcolor{Gray} 
\colcell 16:00-20:00 & & & \colcelltwo & & & &
\\ \hline 
%%-------------------------------------------------------------------
\end{tabularx}
\end{table} 

Looking at a generalized worker's availability shown in Table \ref{typical_availability} one might think that there shall be more weekend blocks available than weekday blocks as the availability, almost in all cases, are higher for weekend blocks. This is not the case as all blocks without weekend tasks are removed in the percolation. This means that all combinations with "No task" on weekends are removed, as well as the case that is shown in Table \ref{Friday_percolation}. The case in Table \ref{Friday_percolation} occurs when a worker is assigned weekend Desk tasks and therefore can not be assigned any other task that Friday.





\begin{table}[!h]
\centering
\caption{A weekend block with Desk tasks preventing any other tasks on Fridays.}
\label{Friday_percolation}
\begin{tabular}{cccccccc}
                                 & Mon                    & Tue                    & Wed                    & Thu                    & Fri                                            & Sat                                            & Sun                                            \\ \cline{2-8} 
\multicolumn{1}{c|}{08:00-10:00} & \multicolumn{1}{c|}{X} & \multicolumn{1}{c|}{X} & \multicolumn{1}{c|}{X} & \multicolumn{1}{c|}{X} & \multicolumn{1}{c|}{\cellcolor[HTML]{000000}}  & \multicolumn{1}{c|}{\cellcolor[HTML]{FCFF2F}D} & \multicolumn{1}{c|}{\cellcolor[HTML]{FCFF2F}D} \\ \cline{2-8} 
\multicolumn{1}{c|}{10:00-13:00} & \multicolumn{1}{c|}{}  & \multicolumn{1}{c|}{}  & \multicolumn{1}{c|}{}  & \multicolumn{1}{c|}{}  & \multicolumn{1}{c|}{\cellcolor[HTML]{000000}}  &                                                &                                                \\ \cline{2-6}
\multicolumn{1}{c|}{13:00-16:00} & \multicolumn{1}{c|}{}  & \multicolumn{1}{c|}{}  & \multicolumn{1}{c|}{}  & \multicolumn{1}{c|}{}  & \multicolumn{1}{c|}{\cellcolor[HTML]{000000}}  &                                                &                                                \\ \cline{2-6}
\multicolumn{1}{c|}{16:00-20:00} & \multicolumn{1}{c|}{}  & \multicolumn{1}{c|}{}  & \multicolumn{1}{c|}{}  & \multicolumn{1}{c|}{}  & \multicolumn{1}{c|}{\cellcolor[HTML]{FCFF2F}D} &                                                &                                                \\ \cline{2-6}
\end{tabular}
\end{table}

 "X" in Table \ref{Friday_percolation} can represent any task and shifts colored in black means that no other task can be assigned that shift.

\section{Rotation assignment} \label{rotation}
There are 35 weekend workers available of which 21 are librarians and 14 are assistants. The demand of weekend workers each week is seven, i.e. the demand for five weeks is exactly $7*5 = 35$ workers. Another requirement in excess to seven workers each weekend is that at least four of them have to be librarians due to three librarians are needed in Information desks and one in Hageby. Therefore, it deems reasonable to swap rotations between workers in the destroy so that it always remains feasible. 

A random generator is used in the assignment of rotations that always makes sure that the two mentioned requirements are met. Furthermore, all of the LoW-workers have fixed weekends and hence are not given new rotations in the destroy/repair loop and is more thoroughly described in Section \ref{LoW_assignment} below. 

Table \ref{rotation_assignment} shows a destroy/repair iteration regarding rotation assignments. The amount of workers being destroyed in each iteration is three.

\begin{table}[!h]
\centering
\caption{An iteration in the destroy/repair loop showing a swap of weekends when three workers are destroyed}
\label{rotation_assignment}
Initial assignment:\\
\begin{tabular}{l|llllll}
\rowcolor[HTML]{C0C0C0}
Week       & 1 & 2 & 3 & 4 & 5  \\ \hline
Librarians & 4 & 4 & 5 & 4 & 4  \\ \hline
Assistants & 3 & 3 & 2 & 3 & 3 
\end{tabular}\\
After destroy:\\
\begin{tabular}{l|llllll}
\rowcolor[HTML]{C0C0C0}
Week       & 1                         & 2                         & 3                         & 4                         & 5                          \\ \hline
Librarians & \cellcolor[HTML]{FFFE65}3 & 4 & \cellcolor[HTML]{FFFE65}4 & 4 & 4  \\ \hline
Assistants & 3 & \cellcolor[HTML]{FFFE65}2 & 2 & 3 & 3
\end{tabular}\\
After repair:\\
\begin{tabular}{l|llllll}
\rowcolor[HTML]{C0C0C0}
Week       & 1 & 2 & 3 & 4 & 5  \\ \hline
Librarians & \cellcolor[HTML]{9AFF99}4 & \cellcolor[HTML]{9AFF99}5 & 4 & 4 & 4  \\ \hline
Assistants & 3 & 2 & \cellcolor[HTML]{9AFF99}3 & 3 & 3 
\end{tabular}
\end{table}

Yellow indicates that a worker, either librarian or assistant, has been destroyed that week and green indicates that a worker has been repaired. Comparing the "Initial assignment" with "After repair" a swap can be seen between week 2 and 3. Worth noting is that a swap can occur even if the amount of qualified workers remains the same after a repair. To understand this, imagine that two librarians with different rotations are destroyed, then two cases can occur: Either they are assigned the same rotation as before or they swap weekends. 


\section{Assignment of Library on Wheels} \label{LoW_assignment}
In order to avoid creating enormous amounts of block combinations, LoW tasks are assigned manually. To assign LoW manually also lead to a fix weekend rotation for the five LoW-workers. This slightly reduces the degrees of freedom of the problem as the LoW-workers' rotation will remain unchanged. However, as there are only a couple different set of feasible LoW assignments it shall not be the deciding factor for the quality of the solution.

Without a manual assignment of LoW the number of existing tasks in a weekblock would increase significantly. The number of tasks would increase from 36 to 43, as there are seven LoW tasks during a week, resulting in a lot more than 4,175 unique week appearances. % %ändra sista

\section{Initial solution} \label{initial_solution}
The initial solution is created in a similar fashion as the repair function in this heuristic. Based on a greedy heuristic the best weekblock for a random worker and week is found and inserted. This is done until every worker have one weekend block, one weekrest block and three weekday blocks assigned to them.

To find the best weekblock several costs have been introduced to measure whether a block is good or bad to assign a worker. Say, if the library demands two librarians at the Information desk Monday at 08:00 and currently there are one assigned, then it would be good to assign another one. Such an assignment will, therefore, be rewarded using a cost. Good assignments will provide negative costs and bad assignments will provide positive costs to the objective function value.

Table \ref{block_to_evaluate} together with Figure \ref{flow_chart_cost} shows an increment where different cost parameters have to be considered when evaluating a block before assigning it to a worker. In order to calculate demand costs for the PL in the library, assume that this block which is being evaluated is to be inserted at week three.

% Please add the following required packages to your document preamble:
% \usepackage[table,xcdraw]{xcolor}
% If you use beamer only pass "xcolor=table" option, i.e. \documentclass[xcolor=table]{beamer}
\begin{table}[!h]
\centering
\caption{A block example to be evaluated using costs}
\label{block_to_evaluate}
\begin{tabular}{cccccccc}
                                 & Mon                                             & Tue                    & Wed                                            & Thu                    & Fri                                            & Sat                    & Sun                    \\ \cline{2-8} 
\multicolumn{1}{c|}{08:00-10:00} & \multicolumn{1}{c|}{\cellcolor[HTML]{FCFF2F}PL} & \multicolumn{1}{c|}{I} & \multicolumn{1}{c|}{}                          & \multicolumn{1}{c|}{I} & \multicolumn{1}{c|}{}                          & \multicolumn{1}{c|}{I} & \multicolumn{1}{c|}{I} \\ \cline{2-8} 
\multicolumn{1}{c|}{10:00-13:00} & \multicolumn{1}{c|}{\cellcolor[HTML]{FCFF2F}}   & \multicolumn{1}{c|}{}  & \multicolumn{1}{c|}{}                          & \multicolumn{1}{c|}{}  & \multicolumn{1}{c|}{\cellcolor[HTML]{FCFF2F}D} &                        &                        \\ \cline{2-6}
\multicolumn{1}{c|}{13:00-16:00} & \multicolumn{1}{c|}{\cellcolor[HTML]{FCFF2F}}   & \multicolumn{1}{c|}{}  & \multicolumn{1}{c|}{\cellcolor[HTML]{FCFF2F}D} & \multicolumn{1}{c|}{}  & \multicolumn{1}{c|}{}                          &                        &                        \\ \cline{2-6}
\multicolumn{1}{c|}{16:00-20:00} & \multicolumn{1}{c|}{}                           & \multicolumn{1}{c|}{}  & \multicolumn{1}{c|}{}                          & \multicolumn{1}{c|}{}  & \multicolumn{1}{c|}{}                          &                        &                        \\ \cline{2-6}
                                 &                                                 &                        &                                                &                        &                                                &                        &                        \\
\multicolumn{3}{c}{Task to be evaluated}                                                                    &                                                &                        &                                                &                        &                        \\
                                 & Mon                                             &                        &                                                &                        &                                                &                        &                        \\ \cline{2-2}
\multicolumn{1}{c|}{08:00-10:00} & \multicolumn{1}{c|}{\cellcolor[HTML]{FCFF2F}PL} &                        &                                                &                        &                                                &                        &                        \\ \cline{2-2}
\multicolumn{1}{c|}{10:00-13:00} & \multicolumn{1}{c|}{\cellcolor[HTML]{FCFF2F}}   &                        &                                                &                        &                                                &                        &                        \\ \cline{2-2}
\multicolumn{1}{c|}{13:00-16:00} & \multicolumn{1}{c|}{\cellcolor[HTML]{FCFF2F}}   &                        &                                                &                        &                                                &                        &                        \\ \cline{2-2}
\end{tabular}
\end{table}

% Define block styles
\tikzstyle{decision} = [diamond, draw, fill=blue!20, 
    text width=4.5em, text badly centered, inner sep=0pt]
\tikzstyle{block} = [rectangle, draw, fill=blue!20, 
    text width=5em, text centered, rounded corners, minimum height=4em]
\tikzstyle{line} = [draw, -latex']
\tikzstyle{cloud} = [draw, ellipse,fill=red!20, node distance=3cm,
    minimum height=2em]


% % FLOW CHART of COSTS
\begin{figure}[!h]
  \caption{A flow chart of appearing costs when a PL is assigned a block on a Monday, third week relative to the library schedule.}
  \centering
	\scalebox{0.85}{ \label{flow_chart_cost}
		\begin{tikzpicture}[node distance = 2cm, auto]
		    % Place nodes
		    \node [decision] (demand) {Need for another PL-worker Monday week 3?};
		    %Invisible node
		    \node[below of= demand, node distance=3.5cm,scale=0.01](inv){};
		    
		    \node [decision, left of= inv, node distance = 3.5cm] (whoami1) {Am I ass/lib?};
		    \node [decision, right of= inv, node distance = 3.5cm] (whoami2) {Am I ass/lib?};
		    
		    %Invisible nodes
		    \node[below of= whoami1, node distance=3cm,scale=0.01](inv2){};
		    \node[below of= whoami2, node distance=3cm,scale=0.01](inv3){};
		    
		    \node[block, left of= inv2] (add_ass_cost1) {Add positive assistant demand cost};
		    \node[block, right of= inv2] (add_lib_cost1) {Add positive librarian demand cost};
		    \node[block, left of= inv3] (add_ass_cost2) {Add negative assistant demand cost};
		    \node[block, right of= inv3] (add_lib_cost2) {Add negative librarian demand cost};
			\node[decision, below of= demand, node distance = 11cm] (too_many) {Have I too many PL already?};
			
			%Invisible node
			\node[below of= too_many, node distance=3cm,scale=0.01](inv4){};
			\node[block, left of= inv4, node distance=2cm] (pl_good) {Add negative PL amount cost};
			\node[block, right of= inv4, node distance=2cm] (pl_violate) {Add positive PL amount cost};
			
		    % Draw edges
		    \path [line] (demand) -- node[left]{no}(whoami1);
		    \path [line] (demand) -- node[right]{yes}(whoami2);
		    \path [line] (whoami1) -- node[left]{ass}(add_ass_cost1);
		    \path [line] (whoami1) -- node[right]{lib}(add_lib_cost1);
		    \path [line] (whoami2) -- node[left]{ass}(add_ass_cost2);
		    \path [line] (whoami2) -- node[right]{lib}(add_lib_cost2);
		    \path [line] (add_ass_cost1) -- (too_many.north);
		    \path [line] (add_lib_cost1) -- (too_many.north);
		    \path [line] (add_ass_cost2) -- (too_many.north);
		    \path [line] (add_lib_cost2) -- (too_many.north);
		    \path [line] (too_many) -- node[left]{no}(pl_good);
		    \path [line] (too_many) -- node[right]{yes}(pl_violate);
		    
		    %First cost
		    \draw [color=gray!70,thick](-8,-9) rectangle(8,2);
		    \node[draw] at (-6.5,1.5) {PL demand cost};
		    %Second cost
		    \draw [color=gray!70,thick](-4,-16.5) rectangle(4,-9);
   		    \node[draw] at (-2.5,-9.5) {PL amount cost};
		    
		\end{tikzpicture}
		}
\end{figure}

    


\section{Costs}
In order to find a feasible solution many of the implemented costs must be carefully chosen. Presently, there are 16 existing costs where some of them are correlated with each other. The solution behaves differently depending on the relation between the costs changes.

Six of the costs are presented in Figure \ref{flow_chart_cost}. They have all been assigned unique values so that, for example, positive assistant demand cost differs from negative assistant demand cost in absolute value. To explain why, one can imagine a case where four workers are demanded of which two have to be librarians, see Table \ref{library_solutions}. The first case, where one assistant and three librarians have been assigned, is more desirable than the case with three assistants and one librarian. In the first case it is still a feasible solution as the qualification of assistants are a subset of the librarians. The second case is feasible, however, not optimal due to the exceeding use of librarians. The optimal solution is shown in green as Case 3.

\begin{table}[!h]
\centering
\caption{Library demand at a shift and solution qualities.}
\label{library_solutions}
\begin{tabular}{|l|l|l|}
\hline
\rowcolor[HTML]{C0C0C0} 
Demand:                         & \multicolumn{2}{l|}{\cellcolor[HTML]{C0C0C0}4 workers ($\geq 2$ librarians)} \\ \hline
\rowcolor[HTML]{FD6864} 
\cellcolor[HTML]{C0C0C0}Case 1: & 3 assistants, 1 librarian                  & (infeasible)                 \\ \hline
\rowcolor[HTML]{FFFE65} 
\cellcolor[HTML]{C0C0C0}Case 2: & 1 assistant, 3 librarians                  & (feasible)                     \\ \hline
\rowcolor[HTML]{34FF34} 
\cellcolor[HTML]{C0C0C0}Case 3:  & 2 assistants, 2 librarians                 & (optimal)                      \\ \hline
\end{tabular}
\end{table}

The complete list of costs with description can be seen in Table \ref{tab:all_costs}. 

\begin{table}[!h]
\centering
\caption{List of all costs with description.}
\label{tab:all_costs}
\begin{tabular}{|l|l|}
\hline
\rowcolor[HTML]{C0C0C0} 
Cost name                                      & Description       \\ \hline
\rowcolor[HTML]{FD6864} 
\multicolumn{2}{|c|}{\cellcolor[HTML]{FD6864}Demand costs}    \\ \hline
Demand\_few\_assistants                        & In need of more assistants to fill quota.                  \\ \hline
Demand\_few\_librarians                        & In need of more librarians to fill quota.                 \\ \hline
Demand\_many\_assistants                       & More assistants assigned than needed (redundancy).           \\ \hline
Demand\_many\_librarians                       & More librarians assigned than reference value.                  \\ \hline
Demand\_few\_total                             & Incorrect amount of workers assigned a task.                  \\ \hline
Demand\_many\_total                            & Incorrect amount of workers assigned a task.                  \\ \hline
Demand\_evening\_cost         & Incorrect amount of workers assigned an evening task (more crucial).\\ \hline
Demand\_PL\_good\_assistant        & Assigning a PL to assistant, empty before assignment.            \\ \hline
Demand\_PL\_good\_librarian        & Assigning a PL to librarian, empty before assignment.           \\ \hline
Demand\_PL\_bad\_assistant         & Assigning a PL to assistant, others already assigned.           \\ \hline
Demand\_PL\_bad\_librarian         & Assigning a PL to librarian, others already assigned.             \\ \hline
\rowcolor[HTML]{FD6864} 
\multicolumn{2}{|c|}{\cellcolor[HTML]{FD6864}PL amount costs} \\ \hline
PL\_good\_amount                  & Assigning PL when in need of more for feasibility.                  \\ \hline
PL\_violate\_amount             & Assigning more PL than allowed to that worker.                  \\ \hline
\rowcolor[HTML]{FD6864} 
\multicolumn{2}{|c|}{\cellcolor[HTML]{FD6864}Weekend costs}   \\ \hline
HB\_amount                       & No or more than one HB workers assigned the same weekend.   \\ \hline
No\_weekend                &                   \\ \hline
\rowcolor[HTML]{FD6864} 
\multicolumn{2}{|c|}{\cellcolor[HTML]{FD6864}Stand-in costs}  \\ \hline
Stand\_in\_cost                     & Occuring when a possible stand-in is ruined due to assignment.    \\ \hline
\end{tabular}
\end{table}

\section{Destroy}
The destroy consists of two contiguous steps: Choosing three workers to destroy and then destroying their assigned blocks. The LoW tasks remain unchanged even if a worker assigned those tasks is destroyed. 
\section{Repair}
Initially in the repair function the destroyed workers are assigned new rotations, see Table \ref{rotation_assignment}, followed by five new week blocks. Which order the blocks are inserted in is randomly generated. For each iteration is one of the three workers randomly chosen followed by one block that has not been reassigned until all 15 destroyed weeks have been repaired. Which week block that is chosen is based on the costs listed in Table \ref{tab:all_costs}. 
\section{Evaluation of solution}
After each destroy/repair loop the new solution is evaluated. It is common for the new solution to be worse than the previous one. However, all new solutions are accepted regardless.

All costs are regarded in the evaluation function. If there are any demand differences in the library or any differences in amount of PL assigned to a worker these are given the respective costs. Furthermore, the amount of stand-ins is evaluated here provided with a negative cost showing that these are desired.

If there are no infeasibilities in the solution the evaluation function will return the value zero, unless there are stand-ins. In case there are stand-ins a negative value can be returned. 
\section{Final phase}
At the end of a run there are mostly a few infeasibilities left that do not get solved. To fix this, a final phase was implemented. Whenever a run is "relatively close" to a solution and it is "theoretically possible" to find a solution the final phase is initiated. "Relatively close" means in this case that the only infeasibility is the lack of workers at a handful of weekday shifts. "Theoretically possible" means that there are more stand-ins available all days where the infeasibility occurs.

The final phase destroys one random week for a random worker and immediately repairs it, resulting in the same or better solution until it is feasible. % %In discussion: Sometimes it does not find a feasible solution due to the worker had already four tasks assigned, i.e. the required block does not exist.

Due to well chosen cost parameter values the run will always eventually get relatively close and reach the final phase. If there are fewer stand-ins available a day where workers are lacking the solution will be discarded before entering the final phase. 

\inputencoding{\enc}

%\chapter{Task distribution approach}\label{chap:taskdist}
%
%\inputencoding{\enc}
%
\section{Introduction}

The first approach, where fixed weeks are distributed to workers, is discussed in the previous chapter. The second approach, which is presented in this chapter, instead distributes individual tasks to workers. This method greatly resembles the process of manually placing tasks as is typically done in many practical situations. 

The objective of the scheduling process is not only to schedule a number of tasks to the workers but also to place them optimally with respect to stand ins. Thus, a method for moving through the solution space and a way of distinguishing between good and bad solutions is of great importance.  

The primary method used in this approach is a large neighbourhood search (LNS) together with a simulated annhealing (SA) accept function. Destroying and repairing the solution, as is customary in LNS, helps leading the solution out of local optima or plateaus. Similarly, SA is used in order to allow the solution to move in a less favourable directions to avoid these local optima. The 
search is guided by a continuously updated schedule cost.

Write: two phases implemented.

\section{Objective functions}
Distinguishing good schedules from bad schedules means there is a need for a way of seting a cost to different schedules. In the original mathematical model, the cost of a schedule consists of two terms: a weighted sum of the number of stand ins on the worst day and the number of different shifts present in the schedule. The rest of the model consists of hard constraints which cannot be violated. However, in the heuristic approach these hard constraints are divided into hard and soft constraints, as is illustrated in Figure (TODO).

During the scheduling process in the implementation, three different objective functions are used. The objective functions are illustrated in Table (TODO) and consist of a weekend objective function, a worker objective function and a weekday objective function.


\begin{table}[]
\centering
\caption{Objective functions used in the implementation.}
\label{tab: task objective functions}
\begin{tabular}{|p{3cm}|l|}
\hline
\multirow{5}{*}{\begin{tabular}[t]{@{}l@{}}\textbf{Weekend} \\ \textbf{Objective} \\ \textbf{Function}\end{tabular}} & \\
 & Min stand in cost + average num stand ins\\  
 & Min shift availability cost + average shift avail\\ 
 & Min day availability cost + average day avail\\  
 & \\ 
\hline

\multirow{7}{*}{\begin{tabular}[t]{@{}l@{}}\textbf{Worker}\\ \textbf{Objective}\\ \textbf{Function}\end{tabular}}    & \\
& Num tasks per day cost \\ 
& Num tasks per week cost \\ 
& Num PL per week cost \\ 
& Total num of PL cost \\ 
& Num tasks at same shift per week  cost \\ 
& \\
\hline

\multirow{4}{*}{\begin{tabular}[t]{@{}l@{}}\textbf{Weekday}\\ \textbf{Objective}\\ \textbf{Function}\end{tabular}}   & \\ & Min stand in cost   \\ 
 & \\ & \\ 
\hline
\end{tabular}
\end{table}

The weekday objective function is associated with the weekend distribution phase of the problem. In the weekend objective function, a stand in cost is defined as the weighted sum between the number of librarians and assistants which are stand ins at a certain day. The minimum stand in cost is defined as the cost at the worst av all days throughout all weeks. 

The min shift availability cost refers to the minimum number of workers available at a shift throughout all shifts, days and weeks. Similarly, the min day availability cost is the lowest number of workers available any shift throughout the whole schedule. In addition to these costs, the average of all three cost types is also used to distinguish solutions which have the same cost.

The worker objective function is used during the weekday task distribution phase of the problem. The function value is a combination of the relaxed constraints of the problem, as described at the beginning of this chapter. Thus, the worker objective function must always be reduced to zero before the schedule will be considered feasible.

The objective function consists of five different costs. All of them are calculated as the total cost for all workers. For example, the first cost described in Table (todo) is the total number of tasks per day exceeding the maximum of one daily task. In the objective function, the sum of excess tasks is taken over all workers. 

The second cost is the number of tasks exceeding four per week. The third is the number of Fetch list tasks performed by a worker exceeding the max limit of one per week. In addition to this, there is a specified max limit for the number of fetch lists that can be performed by a worker throughout all weeks and which is different for different workers. This is the fourth cost. Finally, the number of tasks performed at the same shift in a week should not exceed two, which is the final cost in the worker objective function.

A feasible solution, where the worker objective function is zero, is evaluated using the weekday objective function described lastly in the table. This cost has only one cost and corresponds to the objective function in the mathematical model. The min stand in cost is calculated in the same way as in the weekend objective function. 

\section{Weekend phase}

The first phase in the scheduling process is the weekend phase. In this phase the weekends of all workers are placed and optimized before placing the remaining tasks. The reason for implementing such a phase rather than placing all tasks at once was the big impact of the weekend structure on the entire schedule. First and foremost, the location of a worker's weekend affects the availability of the worker in the following week where the week rest is placed. The worker is unavailable during week rest and if such week rests are combined in an unfortunate way, the scheduling can result in an uneven distribution of workers during days affected by week rest.

In order to measure what a good distribution of weekends is, the weekend objective function described in the previous section is used during the search process. Although the overall objective is to maximize the number of stand ins at the most critical day in the schedule, that is, even out the stand ins over the days, it is not trivial to measure this in a schedule with no tasks placed. Evening work reduces potential stand ins as well as tasks distributed during the days. Furthermore, limits on how many tasks per week a person is allowed to take further complicates the measurement of stand ins.

Because of the difficulty in measuring stand ins, certain tasks are placed already in the weekend phase. This includes all Library on Wheel tasks and all evening tasks. For the Library on Wheels, there is not much choice or variability, and thus a fixed structure is applied in order to simplify its distribution. Similarly, the workers who are available at the evening tasks are in most part, equal to the number of workers needed, thus leaving little or no choice in scheduling evenings. Using this as a basis for approximating the objective function value, a more accurate measurement can be performed.


\begin{table}[!h]
\caption{Worker availability placing evening tasks and BokB.}
\centering
\begin{tabular}{|C{1.2cm}
|C{0.6cm}|C{0.6cm}|C{0.6cm}|C{0.6cm}|C{0.6cm}|C{0.6cm}|C{0.6cm}|}
\hline
&\multicolumn{7}{l|}{\textbf{Num available assistants}} \\ \hline
& Mo & Tu & We & Th & Fr & Sa & Su \\ \hline
Shift 1: &8 & 10 & 10 & 10 & 6 & 0 & 0  \\ \hline
Shift 2: &8 & 9 & 9 & 9 & 6 & 0 & 0 \\ \hline
Shift 3: &9 & 8 & 9 & 7 & 5 & 0 & 0 \\ \hline
Shift 4: & 3 & 2 & 2 & 3 & 0 & 0 & 0 \\ \hline
\hline
&\multicolumn{7}{l|}{\textbf{Num available librarians}} \\ \hline
& Mo & Tu & We & Th & Fr & Sa & Su \\ \hline
Shift 1: & 16 & 15 & 16 & 13 & 12 & 0 & 0 \\ \hline
Shift 2: & 18 & 15 & 17 & 14 & 13 & 0 & 0  \\ \hline
Shift 3: &17 & 14 & 18 & 18 & 13 & 0 & 0  \\ \hline
Shift 4: &3 & 4 & 4 & 3 & 1 & 0 & 0 \\ \hline
\hline
&\multicolumn{7}{l|}{\textbf{Num available BBlib}} \\ \hline
& Mo & Tu & We & Th & Fr & Sa & Su \\ \hline
Shift 1:  & 2 & 0 & 1 & 1 & 1 & 0 & 0 \\ \hline
Shift 2: & 0 & 0 & 0 & 0 & 0 & 0 & 0 \\ \hline
Shift 3: &0 & 0 & 0 & 0 & 0 & 0 & 0 \\ \hline
Shift 4: &1 & 0 & 2 & 2 & 0 & 0 & 0 \\ \hline
\end{tabular}
\end{table}

\begin{table}[!h]
\centering
\caption{Worker availability after placing evening tasks and BokB for an example week.}
\begin{tabular}{|C{1.2cm}
|C{0.6cm}|C{0.6cm}|C{0.6cm}|C{0.6cm}|C{0.6cm}|C{0.6cm}|C{0.6cm}|}
\hline
&\multicolumn{7}{l|}{\textbf{Num available assistants}} \\ \hline
& Mo & Tu & We & Th & Fr & Sa & Su \\ \hline
Shift 1: & 7 & 10 & 9 & 9 & 8 & 0 &  0 \\ \hline   
Shift 2: &7 & 9 & 8 & 8 & 8 & 0 & 0 \\ \hline
Shift 3: & 7 & 7 & 8 & 6 & 6 & 0 & 0 \\ \hline 
Shift 4: & 0 & 0 & 0 & 0 & 0 & 0 & 0 \\ \hline
\hline 
&\multicolumn{7}{l|}{\textbf{Num available librarians}} \\ \hline
& Mo & Tu & We & Th & Fr & Sa & Su \\ \hline
Shift 1: & 14 & 11 & 13 & 11 & 12 & 0 & 0 \\ \hline  
Shift 2: &14 & 11 & 14 & 11 & 12 & 0 & 0 \\ \hline  
Shift 3: &13 & 11 & 14 & 13 & 13 & 0 & 0 \\ \hline       
Shift 4: &0 & 0 & 0 & 1 & 1 & 0 & 0 \\ \hline
\hline   
&\multicolumn{7}{l|}{\textbf{Num available BBlib}} \\ \hline
& Mo & Tu & We & Th & Fr & Sa & Su \\ \hline
Shift 1: & 0 & 0 & 0 & 0 & 0 & 0 & 0 \\ \hline
Shift 2: & 0 & 0 & 0 & 0 & 0 & 0 & 0 \\ \hline
Shift 3: & 0 & 0 & 0 & 0 & 0 & 0 & 0 \\ \hline
Shift 4: & 0 & 0 & 1 & 0 & 0 & 0 & 0 \\ \hline
\end{tabular}
\end{table}


Initial solution: placing tasks at random
Destroy: Destroying a certain number of weekends at random.
Repair: Repairing same weekends for same workers. Prioritizing weekends according to 1. qualification 2. avail demand diff. Almost random worker placed, although making sure that HB is placed correctly.

Infeasibility check: Does the current assignment of weekends generate a schedule where there are not enough workers at the shifts?

Placing Library on Wheels and evenings: Considered as having very little degree of freedom. 

Evenings: based on worker costs.

Heuristic methods: SA on LNS with random destroy and repair. SA accept function, accepting with exponential cooling. Tuning parameters T and alpha.

\section{Weekday phase}
Evenings,weekends and BokB already placed. 

Concept: destroy worst worker until all workers have feasible schedules. Record the library cost of the solution. 

Destroy: weekday tasks for workers with highest cost.
Repair: 1. qualification, 2. avail demand diff. Place cheapest worker.

Infeasibility: when a feasible worker cost is not found for a large number of iterations. 

\section{Simplifications of Mathematical Model}
-10 Week scheduling
-Objective function term about similar weeks
-BokB fixed weeks for every other week workers.
-Even odd weeks. How to handle?
(-lower limit PL)

\section{Implementation}
C++, object orientation, run on a linux operating system. Reading availability of workers into the program and outputing a result file, which can be read by Excel. Results are to be visualized in Excel (write this in another part?)

%\inputencoding{\enc}

\chapter{Results and discussion}\label{chap:res}
\inputencoding{\enc}
\section{Results} \label{section:results}
In the following sections, results from running the AMPL implementation and both heuristics will be presented. The parameter values used in the runs will also be given. 

\subsection{AMPL implementation}\label{sec:ampl_res}

By implementing the given mathematical model in AMPL and solving it with CPLEX 12.5, provides the values displayed in Table \ref{tab:CPLEX_res}. The weights in Equation \ref{objfcn} and \ref{constr:s_min} were set to $L = 2$, $A = 1$, $M =100$ and $N=1$, thus making librarians twice as valuable as assistants and stand-ins 100 times more valuable than no shift changes. 

\begin{table}[!h]
\centering
\caption{Results from solving the mathematical model with CPLEX solver.}
\label{tab:CPLEX_res}
\begin{tabular}{|l|p{3cm}|p{3cm}|l|}
\hline
\rowcolor{Gray} & \textbf{Min num stand-ins (lib, ass)} & \textbf{Stand-in cost} & \textbf{Solution time} \\ \hline
\cellcolor{Gray} \textbf{Result} & \multicolumn{1}{c|}{(3,0)} & \multicolumn{1}{c|}{6} & \multicolumn{1}{c|}{19 min} \\
\hline
\end{tabular}
\end{table}

Only the stand-in part of the objective function is presented, since this value can be used as a benchmark in measuring the performance of the heuristics. The number of shift changes, which is the second part of the objective function in Equation \ref{objfcn}, is zero.

\subsection{Week block scheduling approach}
As most of the costs are correlated with each other, some parameter tuning was required. The final result of the parameter tuning are shown in Table \ref{tab:cost_parameters}. Their respective descriptions can be seen in Table \ref{tab:all_costs}. 

\begin{table}[!h]
\centering
\caption{List of all costs used in the week block scheduling approach and their respective values.}
\label{tab:cost_parameters}
\begin{tabular}{|c|c|}
\hline
\rowcolor[HTML]{FD6864} 
\multicolumn{2}{|l|}{\cellcolor{corn} \textbf{Demand costs}} \\ \hline
%\multicolumn{2}{|c|}{\cellcolor[HTML]{FD6864}Demand costs}    \\ \hline
\rowcolor[HTML]{C0C0C0} 
Cost name                                      & Cost value       \\ \hline
Demand\_few\_ass                        & 350         \\ \hline
Demand\_few\_lib                        & 300         \\ \hline
Demand\_many\_ass                       & 200         \\ \hline
Demand\_many\_lib                       & 40          \\ \hline
Demand\_few\_total                             & 800         \\ \hline
Demand\_many\_total                            & 700         \\ \hline
Demand\_evening\_cost         & 20,000 				\\ \hline
Demand\_PL\_good\_ass        & 1,200            \\ \hline
Demand\_PL\_good\_lib        & 800           \\ \hline
Demand\_PL\_bad\_ass         & 1,200           \\ \hline
Demand\_PL\_bad\_lib         & 1,600             \\ \hline
\rowcolor[HTML]{FD6864} 
\multicolumn{2}{|l|}{\cellcolor{corn} \textbf{PL costs}} \\ \hline
\rowcolor[HTML]{C0C0C0} 
Cost name                                      & Cost value       \\ \hline
PL\_good\_amount                  & 1,000                   \\ \hline
PL\_violate\_amount             & 1,500                  \\ \hline
\rowcolor[HTML]{FD6864} 
\multicolumn{2}{|l|}{\cellcolor{corn} \textbf{Weekend costs}} \\ \hline
\rowcolor[HTML]{C0C0C0} 
Cost name                                      & Cost value       \\ \hline
HB\_amount                       & 15,000    \\ \hline
No\_weekend                & 5,000                   \\ \hline
\rowcolor[HTML]{FD6864} 
\multicolumn{2}{|l|}{\cellcolor{corn} \textbf{Stand-in costs}} \\ \hline
\rowcolor[HTML]{C0C0C0} 
Cost name                                      & Cost value       \\ \hline
Stand\_in\_cost                     & 5     \\ \hline
\end{tabular}
\end{table}

Worth noting is that these values are most certainly not the best possible, as they have mostly been assigned using intuition. However, a few relations were determined by either running tests or using reasoning. An example is if $PL\_good\_cost > Demand\_PL\_good\_ass$ or $PL\_good\_cost > Demand\_PL\_good\_lib$. If one of those holds, all staff members that is able to be assigned another PL, without violating the individual PL assigment constraint, will be. Thereby, leaving the library overstaffed with staff members assigned to PL. The reason is because the net of $-PL\_good\_cost + Demand\_PL\_good\_ass/lib$ will be negative, meaning it is preferable to assign yet another PL to a staff member, regardless of how many there already are assigned. This leads to an overstaffed library on some days, and lack of staff members on other days.

Table \ref{successful_iter} shows the success rate of finding a solution during a script execution. A run is considered to fail, if no feasible solution is found during the final phase. This script execution is using the cost parameters shown in Table \ref{tab:cost_parameters}. 
\begin{table}[!h]
\centering
\caption{Number of successful runs. A failed run is when no feasible solution is found during the final phase.}
\label{successful_iter}
\begin{tabular}{|c|c|}
\hline
Successful runs         & 420      \\ \hline
Runs (total) & 638      \\ \hline
\%                 & $\sim$66 \\ \hline
\end{tabular}
\end{table}

Table \ref{tab:stand_in_spread} shows statistics of how many stand-ins were found after 420 successful runs. In case "No stand-ins" are found after a run, then a feasible solution is found, however, there are no stand-ins on the worst day (or days). The results "1 assistant" or "1 librarian" mean that one stand-in was found at best on the worst day (or days).
\begin{table}[!h]
\centering
\caption{Number of stand-ins found in 420 successful runs.}
\label{tab:stand_in_spread}
\begin{tabular}{|c|c|}
\hline
\rowcolor[HTML]{D2D2D2} 
Stand-in spread & Times \\ \hline
No stand-ins    & 351                      \\ \hline
1 assistant     & 33                      \\ \hline
1 librarian 	& 36 \\ \hline
\end{tabular}
\end{table}

Figure \ref{fig:feasibleRerun} shows a plot of the objective function value after destroy and repair iterations. When the value on the y-axis reaches zero, a feasible solution is found. Whenever an iteration is below the final phase boundary, the entering criterion is checked. If the final phase is entered, the few existing demand violations will be taken care of, one by one, by assigning the stand-ins where the violations occur. A feasible solution is found whenever an objective function value of zero is reached. The final phase boundary is set as $1\cdot10^4$.

Every time the value skyrockets, a solution has been discarded in the final phase. Empty blocks are, thereafter, assigned to all staff members and the run is once again entering the destroy and repair loop. Destroy and repair iterations are being executed until a solution is likely to be found, that is, there are enough stand-ins available to take care of the demand violations. 

%\includegraphics[scale = 0.3, width = 15cm]{Rplot}
%\includegraphics[scale = 0.3, width = 15cm]{1Plot}
\begin{figure}[!h]
\centering
%\includegraphics[scale = 0.3]{Chapters/ImagesClaes/Rfeasible.png}
\includegraphics[scale = 0.8]{Chapters/ImagesClaes/1phase1iter1ReRun.png}
\caption{Plot of the objective function value after destroy and repair iterations; one time without enough stand-ins, when trying to enter the final phase.}
\label{fig:feasibleRerun}
\end{figure}

As seen in Figure \ref{fig:feasibleRerun} there are not enough stand-ins one time during the run. What happens is that the objective function gets below the final phase boundary, but does not meet the stand-in criterion and, therefore, destroys every workers' schedule. 

Figure \ref{fig:feasibleNoRerun} illustrates a run when there are enough stand-ins to enter the final phase.

\begin{figure}[!h]
	\centering
	\includegraphics[scale = 0.8]{Chapters/ImagesClaes/1phase1iterNoReRun.png}
	\caption{Plot of the objective function value after destroy and repair iterations with enough stand-ins when trying to enter the final phase.}
	\label{fig:feasibleNoRerun}
\end{figure}

To illustrate why the solution gets worse after a destroy and repair iteration; consider the case when two librarians with the same rotation are being repaired, where one only can work at HB during weekends and the other one can work at either HB or with Desk tasks. If the more versatile librarian is repaired first and being assigned to HB, due to the cost parameters, then it will be impossible for the second librarian to be assigned any weekend tasks. This results in a high cost contribution to the objective function value.

\subsection{Task allocation approach}\label{sec:task_dist_res}

In order to find a good value for the weights described in Section \ref{subsection:tasks_cost}, different tuning tests were performed. These tests resulted in the weights presented in Table \ref{tab:tasks_weight}. These weights are used in all test in this section unless stated otherwise.

\begin{table}[!h]
\centering
\caption{Weights used in the implementation.}
\label{tab:tasks_weight}
\begin{tabular}{|l|l|}
\hline
\rowcolor{gray!90} \textbf{Weight} & \textbf{Weight value} \\ \hline
\multicolumn{2}{|l|}{\cellcolor{Gray} \textbf{Weekend Objective Function Weights}} \\ \hline
$W_{SI\_m}$ & 0.1 \\ \hline
$W_{S\_m}$ & 0.1 \\ \hline
$W_{D\_m}$ & 10 \\ \hline
$W_{SI\_a}$ & 0.01 \\ \hline
$W_{S\_a}$ & 0.01 \\ \hline
$W_{D\_a}$ & 1 \\ \hline
\multicolumn{2}{|l|}{\cellcolor{Gray} \textbf{Staff Member/Staff Member Objective Function Weights}} \\ \hline
$W_{w\_SI}$ & 
\begin{tabular} [x]{@{}c@{}}
	2 \text{if librarian} \\ 
	1 \text{if assistant}
\end{tabular} \\ \hline
$W_{Task\_D}$ / $W_{w\_Task\_D}$ & 100\\ \hline
$W_{Task\_W}$ / $W_{w\_Task\_W}$ & 	10 \\ \hline
$W_{PL\_W}$ / $W_{w\_PL\_W}$ & 5 \\ \hline
$W_{PL\_Tot}$ / $W_{w\_PL\_Tot}$ & 5 \\ \hline
$W_{SShift\_W}$ / $W_{w\_SShift\_W}$ & 4 \\ \hline
\end{tabular}
\end{table}



The first six weights, belonging to the weekend objective function, were most experimented with and will be discussed further in this section. The first three, the min weights, are scaled relatively to each other and decide how much weight is placed on maximizing the number of stand-ins during the worst day, maximizing the number of staff members at the worst day and maximizing the number of staff members at the worst shift, respectively. Only maximizing the number of stand-ins at the worst day during the weekend phase would not give a complete estimation of the number of stand-ins in the final schedule, since this number also depends on how many workers are present in total during that day and during the shifts of the days. Thus, all three aspects were measured.

The next three weights, refered to as the average weights, belong to the three costs which give the average of each of the three aspects described above. They are simply calculated as one tenth of their equivalent min weights. The weights in Equation \ref{eq:wend_cost_calc}, $W_{lib}$ and $W_{ass}$ were set to $2$ and $1$, respectively, since it can be argued that one librarian can perform twice as many tasks as one assistant.

The individual staff member cost and the staff member objective function costs, which measure the same aspects of the schedule, have been given the same weights, as is illustrated in Table \ref{tab:tasks_weight}. This is due to the fact that these weights, relative to each other, decide what constraints are most strict. It is, for example more severe to break the rule of one task per day, than to break the rule of having too many shifts at the same time in a week, which is reflected in the weights. The stand-in weight $W_{w\_SI}$ is twice as high for librarians as for assistant.

The two phases of the implementation are run a specified number of iterations, referred to as $It_{wend}$ and $It_{wday}$. These parameters were also trimmed and the results from different iteration values are displayed in Tables \ref{tab:taskdist_res_wendit} and \ref{tab:taskdist_res_wdayit}. The results for weight tuning are displayed in Table \ref{tab:taskdist_weights_res}.

\begin{table}[!h]
\centering
\caption{Results from the task allocation heuristic when varying $It_{wend}$}
\label{tab:taskdist_res_wendit}
\begin{tabular}{|l|l|l|}
\hline
\rowcolor{Gray} \textbf{$It_{wend}/It_{wday}$} &  \textbf{Heuristic cost} &  \textbf{AMPL cost} \\ \hline
\cellcolor{Gray} \textbf{10/10} & \multicolumn{1}{c|}{4.40} & \multicolumn{1}{c|}{4.78} \\
\cellcolor{Gray} \textbf{50/10} & \multicolumn{1}{c|}{5.00} & \multicolumn{1}{c|}{5.34} \\
\cellcolor{Gray} \textbf{100/10} & \multicolumn{1}{c|}{5.26} & \multicolumn{1}{c|}{5.52} \\
\cellcolor{Gray} \textbf{500/10} & \multicolumn{1}{c|}{5.66} & \multicolumn{1}{c|}{5.78} \\
\cellcolor{Gray} \textbf{1000/10} & \multicolumn{1}{c|}{5.54} & \multicolumn{1}{c|}{5.66}  \\
\hline
\end{tabular}
\end{table}

\begin{table}[!h]
\centering
\caption{Results from the task allocation heuristic when varying $It_{wday}$}
\label{tab:taskdist_res_wdayit}
\begin{tabular}{|l|l|l|}
\hline
\rowcolor{Gray} \textbf{$It_{wend}/It_{wday}$} &  \textbf{Heuristic cost} &  \textbf{AMPL cost} \\ \hline
\cellcolor{Gray} \textbf{500/5} & \multicolumn{1}{c|}{5.47} & \multicolumn{1}{c|}{5.74} \\
\cellcolor{Gray} \textbf{500/10} & \multicolumn{1}{c|}{5.66} & \multicolumn{1}{c|}{5.78} \\
\cellcolor{Gray} \textbf{500/15} & \multicolumn{1}{c|}{5.6} & \multicolumn{1}{c|}{5.74} \\
\cellcolor{Gray} \textbf{500/20} & \multicolumn{1}{c|}{5.57} & \multicolumn{1}{c|}{5.74} \\
\hline
\end{tabular}
\end{table}


\begin{table}[!h]
\centering
\caption{Results from the task allocation heuristic when varying weights. Here, $It_{wend} = 1000$ and $It_{wday} = 20$ in all runs.}
\label{tab:taskdist_weights_res}
\begin{tabular}{|l|l|l|}
\hline
\rowcolor{Gray} \textbf{$W_{SI\_m}/W_{S\_m}/W_{D\_m}$} & \textbf{Heuristic cost} &  \textbf{AMPL cost} \\ \hline
\cellcolor{Gray} \textbf{0/0/0 }& \multicolumn{1}{c|}{2.85} & \multicolumn{1}{c|}{3.66}  \\
\cellcolor{Gray} \textbf{1/1/1} & \multicolumn{1}{c|}{5.43} & \multicolumn{1}{c|}{5.54}  \\
\cellcolor{Gray} \textbf{10/0.1/0.1} & \multicolumn{1}{c|}{5.11} & \multicolumn{1}{c|}{5.36}  \\
\cellcolor{Gray} \textbf{0.1/10/0.1} & \multicolumn{1}{c|}{5.43} & \multicolumn{1}{c|}{5.60}  \\
\cellcolor{Gray} \textbf{0.1/0.1/10} & \multicolumn{1}{c|}{5.57} & \multicolumn{1}{c|}{5.80} \\
\hline
\end{tabular}
\end{table}

The tables display two costs for each test, the heuristic cost and the AMPL cost. These are calculated as the average result of 100 runs. Each run takes between a few seconds and a few minutes to perform. The heuristic cost is the actual result of the heuristic, measuring how well the heuristic solves the problem. The AMPL cost is the result obtained when solving the problem to optimality in CPLEX, using the weekend work pattern found in the heuristic. Thus, this value measures how good the weekend phase works. Both figures should be compared to the optimal stand-in value for the problem, which is 6 (see Table \ref{tab:CPLEX_res}).

Studying Table \ref{tab:taskdist_res_wendit}, it can be seen that the AMPL cost increases with more weekend iterations. However, after 500 iterations, the cost stops to improve and it is actually deteriorating slightly. This is probably indicating that the weekend objective function value can only give a rough measure of a good schedule, and the result does not only depend on the number of iterations but also on chance. 

Similarly, in Table \ref{tab:taskdist_res_wdayit}, the heuristic cost increases until a certain threshold, when the solutions are not improving any more. This might indicate that the second phase is not random enough, so the same schedules are found multiple times. This might be solved through implementing varying weights or introducing more randomness in task placement.

Table \ref{tab:taskdist_weights_res} shows the results from using a few extreme weight combinations. It is is clear from the table that, although the performance does not vary to any larger extent when shifting the focus in the weights, using a completely random search with no weights produces poor results. The last combination of weights seems to be performing slightly better than the others, and thus these weights were chosen in the implementation.

The weekend objective function value during a typical run is shown in Figure \ref{fig:obj_fun_vals}. The effects of the SA accept function are visible in the plot as smaller or larger dives in the objective function value. In the runs, the SA parameters $T_0 = 0.4$ and $\alpha = 0.985$. $It_{wend} = 1000$ were used. The graph suggests that a high weekend objective function value is found already within the 200 first iterations, which is consistent with the results in Table \ref{tab:taskdist_res_wendit}.

\begin{figure}[!htbp]
\centering
\includegraphics[width=0.9\textwidth, trim = 100px 0px 100px 20px, clip]{Chapters/ImagesEmelie/Plot_1000_20.png}
\caption{The weekend objective function for 1000 iterations}
\label{fig:obj_fun_vals}
\end{figure}


\begin{figure}[!htbp]
\centering
\includegraphics[width=0.9\textwidth, trim = 100px 0px 100px 20px, clip]{Chapters/ImagesEmelie/Components_1000_20.png}
\caption{The weekend objective function minimum cost components for 1000 iterations}
\label{fig:obj_fun_comp}
\end{figure}

\begin{figure}[!h]
\centering
\includegraphics[width=0.9\textwidth, trim = 100px 0px 100px 20px, clip]{Chapters/ImagesEmelie/Components_av_1000_20.png}
\caption{The weekend objective function average costs components for 1000 iterations}
\label{fig:obj_fun_comp_aver}
\end{figure}


In Figure \ref{fig:obj_fun_comp} and Figure \ref{fig:obj_fun_comp_aver}, the three different cost components of the weekend objective function for the same run are displayed. In the first plot, the minimum costs are displayed. These seem to move between a few discrete values, although a small improvement over the iterations can be seen, at least in the day avail cost component. The average values are almost constant throughout the iterations, as can be seen in the second plot. This is probably due to the fact that the total number of available staff members barely changes when shifting their weeks.

Attempts were made to correlate the min component values of the weekend objective function to each other. However, the results were fluctuating between different runs so no clear correlations could be identified. This suggests that they are measuring three independent aspects.

\section{Discussion}
In this section, the heuristic results, presented in Section \ref{section:results}, will be discussed. Pros and cons for both methods are given, so that they can be more easily compared.


\subsection{Week block scheduling approach}
Table \ref{tab:pros_cons_weekly_scheduling} lists pros and cons with the implemented week block scheduling approach. Some pros and cons were considered before this heuristic was chosen. Mainly, the con concerning the exponential growth of week blocks was taken into consideration, and an estimate of the upper limit of the problem size was done.

The pro regarding the same amount of week blocks was taken into consideration, as the heuristic was thought of having firstly, a block construction phase and secondly, an assignment phase. However, the most significant pro of this approach is that many of the constraints are already met, implicitly, when the blocks are created. Constraints like maximum of one task per day, one evening per week, one PL per week, two shifts at the same hour per week and more are already taken care of. 

Cons like rotations, solution time and costs are no major issues, as they can be avoided by a few smarter implementations. For instance, rotations can be improved by assigning a chosen value whenever a staff member is available as stand-in on a day. From the sum of those values an even distribution of possible stand-ins can be acquired by, for instance, always letting the difference between the lowest and the highest sum of stand-ins, through the days, be as small as possible. This improvement should be done instead of randomly generating the new rotations in the destroy and repair iteration. 

The improvement mentioned above should also implicitly decrease the solution time, as less iterations would be required to find a feasible solution.

In contrast to the cons mentioned above, a major issue is to always be able to create a pool of possible week block appearances, regardless of the problem size. Just by adding meetings and the assignment of the library on wheels tasks would make the number of possible week block appearances grow considerably.

\begin{table}[H]
\caption{Pros and cons with the implemented week block scheduling approach.}
\label{tab:pros_cons_weekly_scheduling}
\begin{tabularx}{\linewidth}{>{\parskip1ex}X@{\kern4\tabcolsep}>{\parskip1ex}X}
\toprule
\hfil\bfseries Pros
&
\hfil\bfseries Cons
\\\cmidrule(r{3\tabcolsep}){1-1}\cmidrule(l{-\tabcolsep}){2-2}

%% PROS, seperated by empty line or \par
Many constraints are already met, implicitly, when the blocks are created. \par
The same number of week block appearances will exist for five and ten weeks.\par
Quick iterations when destroying and repairing.\par

&

%% CONS, seperated by empty line or \par
Rotations need to be assigned in a systematic way in order to achieve reasonable results, regarding lowest number of stand-ins through the days.\par
The number of unique week block appearances grows exponentially in case more task types are added, such as meetings.\par
The solution time can vary considerably, as several random number generators have been used.\par
More cost parameters are needed for this heuristic, where every one of them affect the solution procedure.

\\\bottomrule
\end{tabularx}
\end{table}

 

\subsection{Task allocation approach}
The results from the previous section point to the fact that finding a good weekend allocation is essential in order to produce a good final schedule. Thus, when $It_{wend}$ is increased, the results improve, up to a limit of around 500 iterations. The results are in general comparable to those of the AMPL implementation. This suggests that the weekend objective function used gives a weekend schedule which, with a high probability, can provide a good basis for creating the complete schedule. However, as the results seem to converge to a value slightly lower than the AMPL result the weekend objective function must be seen only as a fairly good an optimal weekend schedule. 

Table \ref{tab:taskdist_weights_res} suggests that the weights used in the weekend objective function are important for how well the implementation works, but only to a certain degree. More testing could be done on the weights in order to see if the result can be further improved. However, since the objective function is an incomplete measure of a good schedule the results will never be completely optimal. This is listed as the main drawback in the cons column in Table \ref{tab:pros_cons_task_scheduling}. 

\begin{table}[!ht]
\caption{Pros and cons with the implemented task allocation approach}
\label{tab:pros_cons_task_scheduling}
\begin{tabularx}{\linewidth}{>{\parskip1ex}X@{\kern4\tabcolsep}>{\parskip1ex}X}
\toprule
\hfil\bfseries Pros
&
\hfil\bfseries Cons
\\\cmidrule(r{3\tabcolsep}){1-1}\cmidrule(l{-\tabcolsep}){2-2}

%% PROS, seperated by empty line or \par
In most cases, the method finds an optimal solution. \par
The method is very fast compared to the CPLEX solver. Only a few iterations in each phase gives good results. \par
The method can be used to find a good weekend schedule. The rest of the schedule can then be created using another method such as CPLEX, in a short time. \par

&

%% CONS, seperated by empty line or \par
The weekend objective function is not an exact measurement of a good weekend schedule. \par
There are many weights and parameters to take into consideration in the implementaion. \par


\\\bottomrule
\end{tabularx}
\end{table}

\iffalse
\begin{table}[!h]
\centering
\label{tab:taskdist_res}
\caption{Results from the task allocation heuristic. Weights from Table \ref{tab:tasks_weight}}
\begin{tabular}{|l|l|l|l|}
\hline
\rowcolor{Gray} \textbf{$It_{wend}/It_{wday}$} &  \textbf{Heuristic cost} &  \textbf{AMPL cost} & \textbf{$\rho_{O\_A}$} \\ \hline
\cellcolor{Gray} \textbf{100/20} & \multicolumn{1}{c|}{5.22} & \multicolumn{1}{c|}{5.48} & 0.58 \\
\cellcolor{Gray} \textbf{500/20} & \multicolumn{1}{c|}{5.64} & \multicolumn{1}{c|}{5.74} & -0.20 \\
\cellcolor{Gray} \textbf{1000/10} & \multicolumn{1}{c|}{5.54} & \multicolumn{1}{c|}{5.66} & 0.03 \\
\cellcolor{Gray} \textbf{1000/20} & \multicolumn{1}{c|}{5.57} & \multicolumn{1}{c|}{5.80} & 0.08 \\
\cellcolor{Gray} \textbf{1000/30} & \multicolumn{1}{c|}{5.50} & \multicolumn{1}{c|}{5.68} & -0.06 \\
\hline
\end{tabular}
\end{table}
\fi

\inputencoding{\enc}

\chapter{Further development}\label{chap:further}
\inputencoding{\enc}
Although good results were obtained both by using the AMPL model and the second heuristic, there is room for further development in the modeling as well as the implementation of the heuristics. One of the things which could be done in a more robust way is the modeling of exceptions and preferences among the staff members. In the current model, these are modeled as hard constraints or by costs which are pushed down to zero for feasibility. If these could be modeled in a more general way or in separation from the core problem, the model would not be so sensitive to changes in exceptions or staff preferences.


Several components of the model could be simplified. For example, The library on wheels could be modeled separately from the rest of the problem. Another suggestion is to simplify every other week staff memebers' schedules so that the person gets two five week schedules instead. Also, a more general way of describing meetings would simplify their distribution. One way of doing this could be to produce schedules which are not entirely feasible according to the problem description, but which can be used as a basis for manual scheduling. One interesting topic to study in order to resolve this problem could be fuzzy goal programming.

Regarding the implementations, there are several components missing in order for the heuristics to fully correspond to the mathematical model. Although the features are not crucial, these could be implemented in order to improve the heuristics. Furthermore, a more systematic framework for testing and evaluating heuristics could be used for measuring the results of the implementations.


\iffalse
Although the constraints were difficult to model, the actual problem was not very difficult to solve, neither for CPLEX nor for the second heuristic.  


For both of the heuristics the first step is to generalize the problem to ten weeks instead of five. In doing so, the heuristics will have the same time span as the AMPL implementation. Furthermore, if meetings are implemented and five-week separated work weeks look alike the heuristics will solve the same problem as the AMPL implementation. 

By looking at the results from the heuristics it deems possible to solve the complete problem by spending more time implementing. However, regardless if such results were reached it would only work for this specific library. A general solver would not be a feature that could be implemented. The reason is due to the uniqueness of requirements in the library. In case they were similar someone would, most likely, already have created a general solver for libraries or any other service institution.
\fi

\inputencoding{\enc}

\chapter{Conclusion}\label{chap:concl}
\inputencoding{\enc}
In this thesis work, a mathematical model for the staff scheduling problem at the Library of Norrköping was developed according to the demands given by the library. The model was solved by the commercial optimization solver CPLEX, which generated an optimal schedule, where three librarian stand-ins and zero assistant stand-ins was found on the worst day (or days). This is better than the current schedule at the library, where only one librarian and zero assistants are assigned as stand-ins on the worst day.

When developing the two heuristics for solving the problem, the conclusion was reached that the weekend distribution is essential for the stand-in distribution. The heuristic that focused on making weekend scheduling separately from the rest of the problem, thus performed better. Thus, the first heuristic would probably work better if it took weekend allocation into consideration.

Although good results were obtained both by using the AMPL model and in one of the heuristics, there is room for improvement in the modeling of the problem. One thing which could be done in a more robust way, is the modelling of exceptions and preferences among the staff members. In the current model, these are simply modelled as hard constraints. If these could be modelled in a more general way or separately from the rest of the problem, the model would not be so sensitive to changes in exceptions or in the staff preferences.

Several components of the model could be simplified. For example, the library on wheels could be modelled separately from the rest of the problem. Furthermore, staff members schedules which differ between odd and even weeks also add complexity to the problem. One way of handling this could be to produce schedules which are not entirely feasible according to the problem description, but which can be used as basic schedules for which manual adjustment is needed.

In the heuristics, there are several components from the full mathematical model which are missing. Although these features are not crucial, they could be implemented in order to make the heuristics more accurate. Furthermore, a more systematic framework for testing and evaluating the heuristics could be developed, in order to be able to measure the results more fairly. If one heuristic would be chosen for further development it would be the second one, since, in the first method, a very large number of blocks would have to be created if more task types were added.

The conclusion is that the mathematical model provides the best results, followed by the second heuristic. Both meetings and a ten week schedule are included in the mathematical model, which are left out in both heuristics, due to lack of time. Also, the first heuristic needs a better rotation distribution in order to provide better results.


\inputencoding{\enc}


%##########################################################
% -------------------END CHAPTERS -------------------------
%##########################################################

\bibliography{Chapters/References}


%%% .................. appendix ...................................
\appendix
\chapter{Problem definitions} \label{definitions}
\inputencoding{\enc}
\section{Sets}
\itab{I} \tab{Set of workers}\\
\itab{I\_lib} \tab{Set of librarians (I\_lib $\subseteq$ I)} \\
\itab{I\_ass}	 \tab{Set of assistants (I\_ass $\subseteq$ I)}	\\
\itab{W}                 \tab{Set of weeks}                                               \\
\itab{$W_5$}	\tab{Set of first five weeks} \\
\itab{D}                 \tab{Set of days in a week}                                      \\
\itab{$D_5$}	\tab{Set of all five weekdays} \\
\itab{$S_d$}           \tab{Set of shifts day \textit{d}}                                        \\
\itab{$S_3$}           \tab{Set of first three shifts on a weekday}     \\
\itab{$J_d$}            \tab{Set of task types day \textit{d}}                                    \\
\itab{I\_LOW}	 \tab{Set of librarians available to work in library on wheels}	\\
\itab{I\_free\_day}	 \tab{Set of workers that shall be assigned a free weekday per week}	\\
\itab{I\_odd\_even}	 \tab{Set of all workers with odd or even weeks}	\\
\itab{I\_weekend\_avail}	 \tab{Set of workers available for weekend work}	\\
\itab{I\_big\_meeting}	 \tab{Set of all workers that attend big meetings}	\\
\itab{I\_no\_PL}	 \tab{Set of workers not working on PL}	\\
\itab{I\_weekend\_avail}	 \tab{Set of workers working 3-4 times PL in 10 weeks}	\\
\itab{Dep}	 \tab{Set of departments}	\\
\itab{I\_dep\{Dep\}}	 \tab{Set of workers in departments}	\\
\itab{V}	 \tab{Set of possible week rotations (shift the week by 1-10 steps)}	\\


\section{Variables} \label{variabs}
\begin{align}
    x_{iwdsj}&=
    \begin{cases}
      1, & \text{if worker \textit{i} is assigned in week \textit{w}, day \textit{d}, shift \textit{s} to a task \textit{j}}\\
      0, & \text{otherwise}
    \end{cases}
    \\
    H_{iwh}&=
    \begin{cases}
      1, & \text{if worker \textit{i} works weekend \textit{h} (= 1, 2) in week \textit{w}}\\
      0, & \text{otherwise}
    \end{cases}
	\\
	r_{iw}&=
	\begin{cases}
		1, & \text{if worker \textit{i} has its schedule rotated \textit{w-1} steps}\\
		0, & \text{otherwise}
	\end{cases}
	\\
	l_{iwd}&=
	\begin{cases}
	  1, & \text{if librarian \textit{i} is a stand-in week \textit{w}, day \textit{d}} \\
	  0, & \text{otherwise}
	\end{cases}
	\\
	a_{iwd}&=
	\begin{cases}
 		1, & \text{if assistant \textit{i} is a stand-in week \textit{w}, day \textit{d}} \\
 		0, & \text{otherwise}
	\end{cases}
	\\
	y_{iwds}&=
	\begin{cases}
 		1, & \text{if worker \textit{i} works week \textit{w}, day \textit{d}, shift \textit{s} at task type E, I or P} \\
 		0, & \text{otherwise}
	\end{cases}
	\\
	W_{iwd}&=
	\begin{cases}
	 	1, & \text{if a worker \textit{i} is working a shift week \textit{w}, day \textit{d}} \\
	 	0, & \text{otherwise}
	\end{cases}
	\\
	b_{iw}&=
	\begin{cases}
 		1, & \text{if worker \textit{i} works at HB week \textit{w}} \\
 		0, & \text{otherwise}
	\end{cases}
	\\
	f_{iw}&=
	\begin{cases}
 		1, & \text{if worker \textit{i} is assigned to work friday evening week \textit{w}} \\
 		0, & \text{otherwise}
	\end{cases}	
	\\
	M_{wds}&=
	\begin{cases}
	 	1, & \text{if a big meeting is placed on week \textit{w}, day \textit{d}, shift \textit{s}} \\
	 	0, & \text{otherwise}
	\end{cases}
	\\
	m_{wdsD}&=
	\begin{cases}
	 	1, & \text{if a meeting is placed on week \textit{w}, day \textit{d}, shift \textit{s} at department \textit{D}} \\
	 	0, & \text{otherwise}
	\end{cases}
	\\
	d_{iwds}&=
	\begin{cases}
	 	1, & \text{if there is a difference in assignment of tasks at a shift}\\
	 		& \text{ \textit{s}, for a worker \textit{i}, day \textit{d} between week \textit{w} and \textit{w+5}} \\
	 	0, & \text{otherwise}
	\end{cases}
	\\
	l^{min}&= \text{lowest number of stand-in librarians found (integer)} \\
	a^{min}&= \text{lowest number of stand-in assistants found (integer)} \\
	s^{min}&= \text{weighted sum with number of stand-in librarians and assistants}.
\end{align}



\section{Parameters} \label{params}
\begin{align}
	N1l&= \text{a value to prioritize the amount of stand-in librarians}\\
	N1a&= \text{a value to prioritize the amount of stand-in assistants}\\
	N1&= \text{a value to prioritize total number of stand ins}\\
	N2&= \text{a value to prioritize similar weeks}\\
	avail\_day_{iwd}&=
	\begin{cases}
		1, & \text{if worker \textit{i} is available for work week \textit{w}, day \textit{d}} \\
		0, & \text{otherwise}
	\end{cases}\\
	demand_{dsj}&= \text{number of workers required day \textit{d}, shift \textit{s} for task type \textit{j}}\\
	qualavail_{iwdsj}&=
	\begin{cases}
		1, & \text{if worker \textit{i} is qualified and available week \textit{w}, day \textit{d}, shift \textit{s} for task type \textit{j}} \\
		0, & \text{otherwise}
	\end{cases}\\
	LOW\_demand_{wds}&= \text{number of workers required day \textit{d}, shift \textit{s} at the library on wheels}
\end{align}

\section{Objective function and constraints} \label{app:constr}
\begin{equation} \label{app:objfcn}
maximize \hspace{0.3cm} M\cdot s^{min} - N \cdot \sum_{i \in I}\sum_{w \in 1..5}\sum_{d \in 1..5}\sum_{s \in 1..3} d_{iwds}
\end{equation}

\begin{equation}
\sum_{i \in I} x_{iwdsj} = (1-M_{wds})demand_{dsj}, \; \forall w\in W,d\in 1..5,s\in S_d,j\in J_d\backslash \{L\}
\end{equation}

\begin{equation}
\sum_{i \in I} x_{iwdsj} = demand_{dsj}, \; \forall w\in W,d\in 6..7,s\in S_d,j\in J_d
\end{equation}

\begin{equation}
\sum_{i \in I} x_{iwdsL} = LOW\_demand_{wds}, \; \forall w\in W,d\in 1..5,s\in S_d
\end{equation}

\begin{equation}
\sum_{d \in 1..5}\sum_{s \in S_d}\sum_{j in J_d} x_{iwdsj} \leq 4, \; \forall i\in I\backslash\{36\},w\in W
\end{equation}

\begin{equation}
\sum_{w \in 1..5} M_{w11} = 1
\end{equation}

\begin{equation}
M_{(w+5)11} = M_{w11}
\end{equation}

\begin{equation}
\sum_{w \in W}\sum_{d \in 1..5}\sum_{s \in S_d} M_{wds} = 2
\end{equation}

\begin{equation}
\sum_{s \in 2..3}\sum_{j \in J_d\backslash\{L\}} x_{iw1sj} \leq 1-M_{w11}, \; \forall i \in I\backslash I\_big, w \in W
\end{equation}

\begin{equation}
\sum_{w \in 1..5}\sum_{d \in 1..5}\sum_{s \in 1..3} m_{wdsD} = 1, \; \forall D \in Dep
\end{equation}

\begin{equation}
m_{(w+5)dsD} = m_{wdsD}, \; \forall D \in Dep, w \in 1..5, d \in 1..5, s \in 1..3
\end{equation}

\begin{equation}
m_{wdsD} + x_{iwdsj} \leq 1, \; \forall D \in Dep, i \in I\_dep\{Dep\}, w \in W, d \in 1..5, s \in 1..3, j \in J_d
\end{equation}

\begin{equation}
m_{wd2D} + m_{wd3D} + x_{iwd1P} \leq 1, \; \forall D \in Dep, i \in I\_dep\{Dep\}, w \in W, d \in 1..5, s \in 1..3
\end{equation}

\begin{equation}
m_{wdsD} \leq \sum_{v \in V}r_{iv}\cdot qualavail_{i(mod_{10}(w-v+10)+1)dsE}, \; \forall D \in Dep, i \in I\_dep\{Dep\} \backslash \{39\}, w \in W, d \in 1..5, s \in 1..3
\end{equation}


%subject to max_one_meeting_per_week{dep in 1..3, i in I_big_meeting union I_dep[dep], w in W}:
%	M_big[w,1,1] + sum{d in 1..5}(sum{s in 1..3} meeting[w,d,s,dep]) <= 1;
%
%######################## Maximum one task per day #####################################
%#Stating that a worker can at maximum perform one task per shift
%subject to only_LOW{i in I, w in W, d in D, s in S[d]}:
%	sum {j in J[d]} x[i,w,d,s,j] <= 1;
%
%#Stating that a worker performing library on wheels cannot perform another task that day, Mon-Thur
%subject to only_one_task_per_day{i in I, w in W, d in 1..4, s2 in S[d]}:
%	sum{s in S[d]}(sum {j in {'Exp','Info','PL'}} x[i,w,d,s,j]) <= 1 - x[i,w,d,s2,'LOW'];
%
%#Stating that a worker performing can only have one library task at Fridays.
%subject to only_LOW_friday{i in I, w in W}:
%	sum{s in S[5]}(sum {j in {'Exp','Info','PL'}} x[i,w,5,s,j]) <= 1;
%
%#Stating that a worker performing library on Friday morning can only have a task in the evning.
%subject to two_tasks_if_LOW_friday_morning{i in I, w in W, s in 1..3}:
%	(sum {j in {'Exp','Info','PL'}} x[i,w,5,s,j])  <= 1 - x[i,w,5,1,'LOW'];
%
%subject to max_one_task_per_day_weekend{i in I, w in W, d in 6..7}:
%
%	sum{s in S[d]}(sum {j in J[d]} x[i,w,d,s,j]) <= 1;
%
%subject to max_one_task_per_evening_not_fridays{i in I diff {36}, w in W}:
%	sum{d in 1..4}(sum{j in J[d]} x[i,w,d,4,j]) <= 1;



\inputencoding{\enc}






% % %New chapter
\chapter{Flow charts} \label{appendix:flow_charts}

% Define block styles
\tikzstyle{decision} = [diamond, draw, fill=red!30,
    text width=3.5em, text badly centered, node distance=3cm, inner sep=0.1pt]
\tikzstyle{block} = [rectangle, draw, fill=blue!30,
    text width=5em, text centered, rounded corners, minimum height=4em]
\tikzstyle{line} = [draw, -latex']
\tikzstyle{cloud} = [draw, ellipse,fill=red!20, node distance=3cm,
    minimum height=2em]


\section{Weekly scheduling approach}
% % FLOW CHART
\begin{figure}[!h]
  \caption{A flow chart of the implemented heuristic with weekly scheduling}
  \centering
	\scalebox{0.7}{ \label{flow_chart}
		\begin{tikzpicture}[node distance = 2cm, auto]
		    % Place nodes
			\node [block] (blocks) {Create all blocks};
			\node [block,below of= blocks] (workers) {Create all workers};
			\node [block,below of= workers] (sort) {Sort blocks to workers};
			\node [block,below of= sort] (assign_rot) {Assign rotations};
			\node [block,below of= assign_rot] (assign_low) {Assign LoW schedule};
			\node [block,below of= assign_low] (init_sol) {Create initial solution};
			\node [decision, right of = init_sol, node distance= 4cm] (feasible) {Is solution feasible?};
			\node [block,above of= feasible, node distance = 3cm] (repair) {Repair workers};
			\node [block,above of= repair, node distance = 2cm] (destroy) {Destroy workers};
			
			
			\node [block,below of= feasible, node distance = 6cm] (print) {Print solution to file};
			\node[decision,right of= feasible, node distance=3.5cm] (close) {Close enough?};
			\node[block,below of= close, node distance = 3cm] (final) {Enter final phase};
			\node[decision, right of= final, node distance = 4cm] (solved) {Solved in given number of iterations?};
			\node[block,right of= assign_low, node distance = 11.5cm] (failed) {Failed iteration, run again};
			
			
			%Invisible node, useful later
			\node[right of= sort, node distance=5.5cm,scale=0.01](invLNS){};
			\node[above of= blocks, node distance=1.3cm,scale=0.01](inv){};
			
			\node[above of= invLNS, node distance = 0.7cm] (text) {LNS};
			
		    % Draw edges
		    \path [line] (inv) -- (blocks);
		    \path [line] (blocks) -- (workers);
		    \path [line] (workers) -- (sort);
		    \path [line] (sort) -- (assign_rot);
		    \path [line] (assign_rot) -- (assign_low);
		    \path [line] (assign_low) -- (init_sol);
		    \path [line] (init_sol) |- (feasible);
		    \path [line] (destroy) -- (repair);
		    \path [line] (repair) -- (feasible);
		    \path [line] (feasible) -- node[right]{yes}(print);
		    \path [line] (feasible) -- node[above]{no}(close);
		    \path [line] (close) |- node[right]{no}(destroy);
		    \path [line] (close) -- node[right]{yes}(final);
		    \path [line] (final) -- (solved);
		    \path [line] (solved) |- node[right]{yes}(print);
		    \path [line] (solved) -- node[right]{no}(failed);
		    \path [line] (failed) |- (blocks);
%			\path[-,draw] (feasible) -| node[right]{no} (inv.north);
%		    \path[line]{} (inv.north) |- node{} (destroy);
			\draw [color=gray!70,thick](2.3,-3.5) rectangle(6,-12);
		\end{tikzpicture}
		}
\end{figure}


\section{Task distribution approach}

\begin{figure}[h]
\centering
\caption{Algorithm for distributing weekends.}
\label{fig:weekend_alg}
\scalebox{0.6}{
\begin{tikzpicture}[node distance = 2cm, auto]
    % Place nodes
    %Right row
    \node [scale=0.01,node distance=0cm] (invinit) {};
    \node [block, below of=invinit] (distribute) {Distribute all weekends randomly};
    \node [decision, below of=distribute, node distance=2.5cm] (evaluate) {Solution feasible?};
    \node [block, left of=distribute, node distance=4cm] (redistribute) {Remove all weekends};
     \node [block, below of=evaluate, node distance=3.5cm] (setobjf) {Set weekend obj fun value};
     \node [block, below of=setobjf, node distance=3cm] (save) {Save temp solution};
     \node [block, below of=save, node distance=3cm] (dr) {Destroy and repair solution};
     \node [decision, below of=dr] (evaluate2) {Solution feasible?};
     \node [block, right of=dr, node distance=3cm] (reload) {Reload temp solution};
     
     %Middle row
     \node [block, left of=evaluate2, node distance=5cm] (setobjf2) {Set weekend obj fun value};
     \node [decision, above of=setobjf2, node distance=2.5cm] (compare) {Better than saved?};
     \node [decision, above of=compare, node distance=2.5cm] (SA) {Accept anyway?};
     \node [decision, left of=SA, node distance=4cm] (best) {Best solution?};
     \node  [block, above of=SA, node distance =2.5cm] (reloadsaved) {Reload saved solution}; 
     \node [block, above of=best, minimum height=3em, node distance = 2cm] (savebest) {Save solution as best};
     \node [decision, above of=reloadsaved, node distance=2cm] (done) {Iteration \textgreater max?};
     \node[above of=savebest, left of=savebest, node distance=2.5cm, scale=0.1](inv){};
     
     %Bottom nodes
     \node [decision, below of=best, node distance=10cm] (isbest) {Current better than best?};
     \node [block, right of=isbest, minimum height=3em, node distance=3cm] (reload2) {Reload best};
      \node [block, below of=reload2, minimum height=3em, node distance=2.5cm] (finished) {Search finished!};

    % Draw edges
    \path [line] (invinit) -- (distribute);
    \path [line] (distribute) -- (evaluate);
    \path [line] (evaluate) -| node {no} (redistribute);
    \path [line] (redistribute) |- (distribute);
    \path [line] (evaluate) -- node {yes}(setobjf);
    \path [line] (setobjf) -- (save);
    \path [line] (save) -- (dr);
    \path [line] (dr) -- (evaluate2);
    \path [line] (evaluate2) -| node {no} (reload);
    \path [line] (reload) -- (dr);
    \path [line] (evaluate2) -- node {yes} (setobjf2);
    \path [line] (setobjf2) -- (compare);
    \path [line] (compare) -| node {yes} (best);
    \path [line] (compare) -- node {no} (SA);
    \path [line] (SA) -- node {yes} (best);
    \path [line] (SA) -- node {no} (reloadsaved);
    \path [line] (reloadsaved) -- (done);
    \path [line] (best) -- node {yes} (savebest);
    \path [line] (best) -- node {no} (done);
    \path [line] (savebest) -- (done);
    \path [line] (done) -- node {no}(save);
    \path [-,draw] (done) -- node {yes}(inv);
    \path [line] (inv) |- (isbest);
    \path [line] (isbest) -- node {no} (reload2);
    \path [line] (isbest) -- node {yes} (finished);
    \path [line] (reload2) -- (finished);
   
    %Feasiblity loop
    \node at (-2,-19) [above=4mm, right=2mm] {\textsc{destroy and repair loop}};
    \draw [color=gray!70,thick](-2,-19) rectangle (5,-12);
    %Weekend loop
    \node at (-10.5,-19.5) [above=4mm, right=2mm] {\textsc{weekend distribution loop}};
    \draw [color=gray!70,thick](-10.5,-19.5) rectangle (5,-6);
    %SA componenet
    \node at (-7.5,-13.5) [above=4mm, right=2mm] {\textsc{SA}};
    \draw [color=gray!70,thick](-7.5,-13.5) rectangle (-3,-10.5);
\end{tikzpicture}
}
\end{figure}


\begin{figure}[!h]
\centering
\caption{Algorithm for distributing weekday tasks.}
\label{fig:weekday_alg}
\scalebox{0.8}{
\begin{tikzpicture}[node distance = 2cm , auto, every node/.style={scale=0.8}]
    %Right column
    \node [scale=0.01,node distance=0cm] (invinit) {};
    \node [block, below of=invinit] (distribute) {Try distribute all tasks};
    \node [block, below of=distribute] (workercost) {Set worker obj fun};
    \node [decision, below of=workercost, node distance=2.5cm] (evaluatewc) {Worker cost = 0?};
    \node [decision, below of=evaluatewc, node distance=3cm] (evaluatetasks) {All tasks distributed?};
    \node [decision, below of=evaluatetasks, node distance=3cm] (feasible) {Solution feasible?};
    \node [block, below of=feasible, node distance=2.5cm] (destamount) {Set destroy amount};
    \node [block, below of=destamount] (dr) {Destroy and repair solution};

    %Middle column
     \node [block, left of=evaluatetasks, node distance = 4cm] (taskcost) {Set weekday costs};
    \node [decision, above of=taskcost, node distance=2cm] (best) {Best solution?};
    \node [block, above of=best, minimum height=3em, node distance = 2cm] (savebest) {Save solution as best};
    
    %Left column
    \node [decision, left of=savebest, node distance=4cm] (done) {Iteration \textgreater max?};
    \node [block, above of=done, node distance=3.5cm] (remove) {Remove all tasks};
    
    %Get solution
    \node [decision, below of=done, node distance=15cm] (isbest) {Current better than best?};
    \node [block, right of=isbest, minimum height=3em, node distance=3cm] (reload2) {Reload best};
    \node [block, below of=reload2, minimum height=3em, node distance=2.5cm] (finished) {Search finished!};
    
	%Invisible nodes
    \node [right of=evaluatewc, scale=0.1,node distance=2cm] (inv) {};
    \node [right of=dr, scale=0.1,node distance=4cm] (inv2) {};
    \node [right of=workercost, scale=0.1,node distance=4cm] (inv3) {};
    \node[left of=isbest, node distance=4cm, scale=0.1](inv4){};
    
 
	%Draw edges
	\path [line] (invinit) -- (distribute);
    \path [line] (distribute) -- (workercost);
    \path [line] (workercost) -- (evaluatewc);
    \path [line] (evaluatewc) -- node [anchor=east]{yes} (evaluatetasks);    
    \path [-,draw] (evaluatewc) -- node [anchor=south]{no} (inv);    
    \path [line] (evaluatetasks) -- node [anchor=south]{yes} (taskcost);
    \path [line] (evaluatetasks) -- node [anchor=east]{no} (feasible);
    \path [line] (destamount) -- (dr);
    \path [-,draw] (dr) -- (inv2);
    \path [-,draw] (inv2) -- (inv3);
    \path [line] (inv3) -- (workercost);
     %To middle row   
	\path [line] (inv) |- (feasible);
    \path [line] (feasible) -- node [anchor=east]{yes} (destamount);
    \path [line] (feasible) -| node [anchor=east]{no} (done);
    %Middle row
    \path [line] (done) -- node [anchor=east]{no}(remove);    
    \path [line] (best) -- node [anchor=east]{yes}(savebest);
    \path [line] (savebest) -- (done);
    \path [line] (best) -- node [anchor=east]{no}(done);
    \path [line] (taskcost) -- (best);
    \path [line] (remove) -- (distribute);
    \path [-,draw] (done) -| node [anchor=east] {yes} (inv4);
    \path [line] (inv4) -- (isbest);
    \path [line] (isbest) -- node [anchor=south]{no} (reload2);
    \path [line] (isbest) -- node [anchor=east]{yes} (finished);
    \path [line] (reload2) -- (finished);

    %Feasiblity loop
    \node at (-2,-15) [above=4mm, right=2mm] {\textsc{destroy and repair loop}};
    \draw [color=gray!70,thick](-2,-15) rectangle (4,-2.5);
    %Weekend loop
    \node at (-8,-15) [above=4mm, right=2mm] {\textsc{weeday task distribution loop}};
    \draw [color=gray!70,thick](-8,-15) rectangle (4,-0.5);

\end{tikzpicture}
}
\end{figure}


\chapter{Weekblock table} \label{appendix:weekblock}
\begin{table}[!h]
\centering
\caption{Number of assignable unique blocks for the workers based on their availability and qualification.}
\label{blocks_available_per_worker}
\scalebox{0.75}{
	\begin{tabular}{|c|ccc|}
	\hline
	Worker & Weekend & Weekrest & Weekday \\ \hline
	1      & 532     & 347      & 1580    \\ \hline
	2      & 1580    & 1580     & 1580    \\ \hline
	3      & 1063    & 347      & 1580    \\ \hline
	4      & 557     & 165      & 779     \\ \hline
	5      & 261     & 130      & 531     \\ \hline
	6      & 532     & 130      & 1580    \\ \hline
	7      & 261     & 130      & 531     \\ \hline
	8      & 261     & 29       & 531     \\ \hline
	9      & 115     & 12       & 247     \\ \hline
	10     & 532     & 130      & 1580    \\ \hline
	11     & 9       & 8        & 8       \\ \hline
	12     & 1063    & 347      & 1580    \\ \hline
	13     & 771     & 92       & 1190    \\ \hline
	14     & 265     & 29       & 489     \\ \hline
	15     & 51      & 18       & 120     \\ \hline
	16     & 495     & 130      & 843     \\ \hline
	17     & 237     & 69       & 267     \\ \hline
	18     & 532     & 279      & 1580    \\ \hline
	19     & 495     & 47       & 843     \\ \hline
	20     & 532     & 130      & 1580    \\ \hline
	21     & 227     & 227      & 227     \\ \hline
	22     & 2       & 1        & 1       \\ \hline
	23     & 3       & 2        & 2       \\ \hline
	24     & 11      & 5        & 47      \\ \hline
	25     & 1063    & 279      & 1580    \\ \hline
	26     & 5       & 4        & 4       \\ \hline
	27     & 213     & 106      & 425     \\ \hline
	28     & 2       & 2        & 2       \\ \hline
	29     & 426     & 106      & 1281    \\ \hline
	30     & 127     & 126      & 455     \\ \hline
	31     & 495     & 130      & 843     \\ \hline
	32     & 261     & 29       & 531     \\ \hline
	33     & 72      & 45       & 306     \\ \hline
	34     & 425     & 425      & 425     \\ \hline
	35     & 91      & 20       & 221     \\ \hline
	36     & 55      & 27       & 101     \\ \hline
	37     & 1063    & 347      & 1580    \\ \hline
	38     & 3       & 1        & 1       \\ \hline
	39     & 2       & 1        & 1       \\ \hline
	\end{tabular}
	}
\end{table}

% this is the copyright page, I did not touch it much.

\cleardoublepage

\thispagestyle{plain}

\begin{picture}(20,20)(0,0)
% \put(0,10){\includegraphics[width=3.5cm]{epresslogo}}
\put(280,0){\includegraphics[width=2.5cm]{univlogo.eps}}
\end{picture}
 
\null
\vspace{0cm} 

Copyright
\medskip

The publishers will keep this document online on the Internet - or its possible replacement - for a period of 25 years from the date of publication barring exceptional circumstances.
The online availability of the document implies a permanent permission for anyone to read, to download, to print out single copies for your own use and to use it unchanged for any non-commercial research and educational purpose. Subsequent transfers of copyright cannot revoke this permission. All other uses of the document are conditional on the consent of the copyright owner. The publisher has taken technical and administrative measures to assure authenticity, security and accessibility.
According to intellectual property law the author has the right to be mentioned when his/her work is accessed as described above and to be protected against infringement.
For additional information about the Linköping University Electronic Press and its procedures for publication and for assurance of document integrity, please refer to its WWW home page: http://www.ep.liu.se/
\bigskip

Upphovsr\"att
\medskip

Detta dokument hålls tillgängligt på Internet - eller dess framtida ersättare - under 25 år från publiceringsdatum under förutsättning att inga extraordi\-nära omständigheter uppstår.
Tillgång till dokumentet innebär tillstånd för var och en att läsa, ladda ner, skriva ut enstaka kopior för enskilt bruk och att använda det oförändrat för ickekommersiell forskning och för undervisning. Överföring av upphovsrätten vid en senare tidpunkt kan inte upphäva detta tillstånd. All annan användning av dokumentet kräver upphovsmannens medgivande. För att garantera äktheten, säkerheten och tillgängligheten finns det lösningar av teknisk och administrativ art.
Upphovsmannens ideella rätt innefattar rätt att bli nämnd som upphovsman i den omfattning som god sed kräver vid användning av dokumentet på ovan beskrivna sätt samt skydd mot att dokumentet ändras eller presenteras i sådan form eller i sådant sammanhang som är kränkande för upphovsmannens litterära eller konstnärliga anseende eller egenart.
För ytterligare information om Linköping University Electronic Press se förlagets hemsida http://www.ep.liu.se/

\medskip

\copyright\phantom{.} \putshortdate, \putauthor




%----------------END OF DOCUMENT----------------------------------------------
%----------------END OF DOCUMENT----------------------------------------------

\end{document}
 
