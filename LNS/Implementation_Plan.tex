\documentclass{article}
\pagestyle{headings}

% some of the packages used are
%\usepackage[utf8]{inputenc}
\usepackage{latexsym}
\usepackage{eepic}
\usepackage{makeidx}
\usepackage[dvips]{graphicx}
\usepackage[utf8]{inputenc}  %för ä ö å ?
\usepackage[english]{babel}
%\usepackage{times}
\usepackage{amssymb}
\usepackage{fancybox} 
\usepackage{textcomp}
\usepackage{float}
\usepackage{xcolor,colortbl}
\usepackage{tabularx}
\usepackage{enumitem}

\newcommand{\mc}[2]{\multicolumn{#1}{c}{#2}}

% Colors
\newcommand{\enc}[0]{utf8}
\definecolor{bluegray}{rgb}{0.4, 0.6, 0.8}
\definecolor{darkcyan}{rgb}{0.0, 0.55, 0.55}
\definecolor{corn}{rgb}{0.98, 0.93, 0.36}
\definecolor{coralred}{rgb}{1.0, 0.25, 0.25}
\definecolor{Gray}{gray}{0.85}
\newcommand{\colcell}{\cellcolor{Gray}}
\newcommand{\colcelltwo}{\cellcolor{corn}}
\newcommand{\colcellthree}{\cellcolor{darkcyan}}
\newcolumntype{g}{>{\columncolor{corn}}c}

% the fancy header/footer
% consult http://research.cmis.csiro.au/gjw/tex/docs/fancyhdr.pdf
% or some other fancy header documenatation for more info
\usepackage{fancyhdr}
\pagestyle{fancy}




\usepackage{blindtext}
\usepackage[utf8]{inputenc}
 
\title{Implementation Plan for LNS heuristik}
\author{Claes Arvidson, Emelie Karlsson}
\date{\today}
\begin{document}
 
\maketitle
 
\section*{First version \index{IMPL1}}
	\subsection*{Requirements}
	\begin{itemize}
	\item Implement a 5 week schedule
	\item Library on Wheels not considered
	\item Softer values such as PL, free day or HB preferences not considered
	\item Meetings not considered
	\item Objective function:
		\begin{itemize}
		\item Minimize violated worker demand. The total number of tasks is constant, as well as the weekly number of tasks. However, days and shifts can violate demand constraints (relaxed hard constraint)
		\item Maximize number of stand in librarians
		\item Maximize number of stand in assistants
		\end{itemize}
	\item Hard constraints:
		\begin{itemize}
		\item 1 task/day for a worker
		\item worker must be available to perform task
		\item worker must be qualified for task (lib/ass)
		\item the total number of tasks to be performed is constant
		\end{itemize} 
	\end{itemize}
	\subsection*{Implementation: task based approach} 
	\begin{itemize}
	\item Initial solution. Firstly, distribute weekends to obtain a feasible weekend schedule. Secondly, distribute tasks according to some rule (for example by finding  the critical task hours). Initial solution not necessarily feasible with respect to the worker demand at each shift.
	\item Destroy and repair: 
		\begin{itemize}
		\item Weekend destroy: Destroy a subset of all weekends. Subsequently destroy the week follwing the new weekends to create a feasible schedule.
		\item Weekend repair: Repair by first placing new weekends, then, after second destroy, place tasks from week rest at deficit days.
		\item Task destroy: Identify the days with the most \textbf{surplus} of workers. Also being an unused stand in generates some cost. Destroy these days.
		\item Task repair: Repair by redistributing over the destroyed days and the days with \textbf{deficit} of workers. Create stand ins by placing tasks for workers who are not stand in avail.
		\end{itemize}
	\item Possible classes:
		\begin{itemize}
		\item Library. Contains:
		\begin{itemize}
			\item Demand schedule (int [w][d][s])
			\item Total\_cost
			\item Cost function for weekend destroy 
			\item Cost function summing workers in a week day (int[d])
			\item Cost function summing number of unused stand ins in a week day (int[d])
			
		\end{itemize} 
		\item Worker. Contains:
		\begin{itemize}
				\item Availability (int [[d][s], [d][s], ...])
				\item Stand in avail (int [w][d])
				\item Identity of worker (ID, name, position, preferences etc.)
				\item Current schedule variables (int tasks [w][d][s],int num\_tasks[w][d], int num\_PL[w][d], int weekend...)
				\item Cost function calculating worker satisfaction with schedule (int [w], int [d], implemented later...)
		\end{itemize} 
		\end{itemize}
	\end{itemize}

	
%\item Diff\_demand\_avail (int [w][d][s])
%\item Num\_stand\_in\_avail (int [w][d][s])
%\item Cost function for Diff\_demand\_avail
%\item Cost function for Num\_stand\_in\_avail
	
	\subsection*{Implementation: week block based approach} 

	\subsection*{Implementation: week block based approach}
	\begin{itemize}
	\item Initial solution. Generate a schedule by assigning a feasible solution for each worker. The resulting schedule will most likely violate the demand constraint that is relaxed into the objective function.

	
	\item Classes in consideration:
		\begin{itemize}
		\item $Library$ (Information about current demand, total demand, workers assigned at a day $d$ and objective functions such as number of stand-ins are stored here)
		\item $Worker$ (contains all personal information about demands, qualifications, assigned blocks, blocks available for assignment etc.)
		\item $Block$ (a week long block containing an ID, type of block (weekend, weekrest, normal), number of tasks, tasks with information of day, shift and tasktype etc)
		\item $Pair\_of\_blocks$ (used later when implementing for 10 week period, containts information of $shift\_differ$ and which two blocks that forms the pair)
		\end{itemize}
		
	\item Destroy:
		\begin{itemize}
		\item Destroy week 1 for all workers (10 \% destroy) - need more than one week to move weekends (and weekrest).
		\item Destroy five consecutive weeks for around a fifth of the workers
		\item Identify and order the most problematic weeks for all workers and destroy in the order. Add some randomization as well to prevent getting stuck on a plateau.
		\end{itemize}
	\item Repair:
		\begin{itemize}
		\item Generate a new schedule with regard to number of tasks destroyed. The repair function searches through the pool of blocks available for each person and creates a new solution using a heuristic.
		\item The heuristic takes the demand and availability for a person into account. If a worker is available for a whole day - minimize the number of tasks given that day in order to generate a stand-in. 
		\end{itemize}
	\end{itemize}
	\section*{Second version \index{IMPL2}}
	\begin{itemize}
		\item Implement a 10 week schedule
		\item Library on Wheels considered
		\item Meetings considered
		\item Objective function:
				\begin{itemize}
				\item Minimize unfulfilled worker demand (relaxed hard constraint)
				\item Maximize number of stand in librarians
				\item Maximize number of stand in assistants
				\item Maximize similar weeks for workers
				\end{itemize}
				
	\end{itemize}
	
	\end{document}

