% Introduction to Master Thesis


\section{Background} 
Schemaläggning av personal på biblioteket i Norrköping.

\section{Problembeskrivning}
Examensarbetet går ut på att lägga ett arbetsschema för personalen på Norrköpings bibliotek. Problemet är tvådelat då det dels går ut på att fylla alla positioner på de stationer som tillhör bibliotekets utåtriktade verksamhet (så kallade yttre tjänst) med personal av rätt kompetens. Denna del fyller de behov som bibliotekets verksamhet kräver.

Samtidigt ska schemat som tagits fram uppfylla de regulationer och önskemål som finns kring personalens arbetstider. Eftersom till exempel även helgarbete ingår i tjänsten så ska lediga dagar fördelas enligt arbetsmiljölagen. Denna del av problemet motsvarar arbetarnas krav.

I tredje hand så ska även schemat medföra en viss robusthet så att störningar i den yttre tjänsten, i form av att personal blir sjuk eller uppbokad annonstädes, ska gå att avhjälpa med en reservlista. 

Den personal som finns att tillgå på biblioteket är begränsad och utgör de resurser som finns att tillgå. Varje enskild personal har ett antal egenskaper och kompetenser. Dessa kompetenser syftar på personalens förmåga att arbeta vid någon av de yttre stationerna: Expedition, Norpan, Infodisk, Bokbuss och Hageby samt personalens förmåga att arbeta vid några av de inre stationerna: ???. Följande egenskaper kan idendifieras hos en personal: \\

\begin{table}
\caption{Personal}
\label{tab1}
\begin{tabular}[H]{|l|l|}
%-------------------------------------------------------------------
\hline 
\textbf{Egenskap} & \textbf{Beskrivning} \\ \hline
%-------------------------------------------------------------------
Arbetsomfattning & 0-100 \% 
\\ \hline 
%-------------------------------------------------------------------
Anställlningsform & Bibliotikarie/Assistent
\\ \hline 
%-------------------------------------------------------------------
Kompetens & Inre och yttre tjänster som personalen klarar av.  
\\ \hline 
%-------------------------------------------------------------------
Specialkompetens & Vilken tjänst personalen är specialicerad på.
\\ \hline 
%-------------------------------------------------------------------
Ledighetsform & Hur personalen tar ut sin veckovila efter helgarbete.
\\ \hline 
%-------------------------------------------------------------------
Övriga önskemål & Jobbar ej kväll etc
\\ \hline 
%-------------------------------------------------------------------
\end{tabular}
\end{table}

Totalt finns 40 personal att tillgå för biblioteket.

De yttre och inre uppgifterna kan ses som behov av personal som måste täckas av de den personal som finns att tillgå. De olika yttreuppgifterna som behöver utföras inkluderar arbete vid olika stationer vid olika tidpunket och datum. Uppgifterna återkommer inom ett 10-veckorsintervall vilket gör att de tillhör en viss tid under en viss dag en viss vecka. Dessutom så har varje uppgift en bestämd längd.   

Varje sådan uppgift som behöver utföras 

kan karatäriseras av vissa egenskaper: \\

\begin{table}
\caption{Yttre och inre uppgifter}
\label{tab2}
\begin{tabular}[H]{|l|l|}
%-------------------------------------------------------------------
\hline
\textbf{Yttre uppgift} & \textbf{Egenskap} \\ \hline
%-------------------------------------------------------------------
 & Starttid, sluttid, vecka och tidsåtgång \
\\ \hline 
%-------------------------------------------------------------------
 & Station
\\ \hline 
%-------------------------------------------------------------------
 & Krav på antal bibliotikarier av rätt kompetens.
\\ \hline 
%-------------------------------------------------------------------
 & Krav på antal assistenter av rätt kompetens.
\\ \hline 
%-------------------------------------------------------------------
 & Krav på totala antalet personal.
\\ \hline 
%-------------------------------------------------------------------

\textbf{Inre uppgift} & \textbf{Egenskap} \\ \hline
%-------------------------------------------------------------------
 & Vecka, Tidsåtgång \
\\ \hline 
%-------------------------------------------------------------------
 & Typ
\\ \hline 
%-------------------------------------------------------------------
 & Krav på antal bibliotikarier av rätt kompetens.
\\ \hline 
%-------------------------------------------------------------------
 & Krav på antal assistenter av rätt kompetens.
\\ \hline 
%-------------------------------------------------------------------
\end{tabular}
\end{table}

Utöver de ovan nämnda resurserna och behoven, finns ett antal krav som ställs på hur schemat får utformas. Dessa kan delas upp i arbetsvillkor, robusthetskrav samt övriga krav:

\begin{table}
\caption{Krav}
\label{tab3}
\begin{tabular}[H]{|l|l|}
%-------------------------------------------------------------------
\hline
\textbf{Arbetsmiljökrav} & \textbf{Beskrivning} \\ \hline
%-------------------------------------------------------------------
 & Rättvis fördelning av helgarbete \
\\ \hline 
%-------------------------------------------------------------------
 & Minst 36 timmar sammanhängande ledighet per 7 dagars arbete.
\\ \hline 
%-------------------------------------------------------------------
 & Högst ett kvällspass per personal i veckan.
\\ \hline 
%-------------------------------------------------------------------
 & Veckorna ska likna varandra i den mån det går.
\\ \hline 
%-------------------------------------------------------------------
 & Schemat ska variera tiden vid olika stationer.
\\ \hline 
%-------------------------------------------------------------------

\textbf{Robusthetskrav} & \textbf{Beskrivning} \\ \hline
%-------------------------------------------------------------------
 & För varje yttre uppgift ska det finnas minst en reserv.
\\ \hline 
%-------------------------------------------------------------------
 & Reserverna ska vara av rätt kompetens för uppgiften de är reserver till.
\\ \hline 
%-------------------------------------------------------------------
 & Reserverna ska vara lika fördelade över alla yttre uppgifter.
\\ \hline 
%-------------------------------------------------------------------

\textbf{Övriga krav} & \textbf{Beskrivning} \\ \hline
%-------------------------------------------------------------------
 & Personalen ska ha en dag utan yttre uppgifter som tillägnas plocklistan.
\\ \hline 
%-------------------------------------------------------------------
\end{tabular}
\end{table}
\medskip

\textbf{Frågor:} Går det att beskriva inre uppgifter som yttre uppgifter utan fixerad tid de måste utföras på? Eller måste de utföras inom ett visst tidsspan?

Hur kommer det sig att det är olika bemanning på Norpan/Bokbuss etc olika veckor?

Är det ett krav att man har ett helgpass var 5e vecka? Är det minst var 5e eller högst var 5e?

Går det att generalisera bibliotikarier till en kompetens?

