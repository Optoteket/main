% Introduction to Master Thesis

\section{Background}


At a library, staff absence can cause problems and result in a shortage of staff with a special competence. If a worker is unavailable a certain day due to a meeting or illness it would require a stand-in to fill the vacancy. Therefore, it is of great interest to create schedules with as many skilled stand-ins as possible to overcome such disturbances, when they occur. Furthermore, the library personnel have certain demands and preferences as to how a satisfactory working schedule should be. For instance, it is neither preferable to work more than one evening per week nor work more weekends than required. 

The problem addressed in this thesis work concerns the library staff at the Central Library of Norrköping. The library has more than 40 employees and the culturally important building from the 1950's is known to all inhabitants of Norrköping. The library is open weekdays between 08:00 and 20:00 and during weekends between 11:00 and 16:00. The generous opening times also creates a challenge for the library as it requires a large pool of well coordinated personnel to keep it running. In addition, the library also provides its services to one other smaller library. This creates a challenge for the library schedulers.

\section{Problem description}

In this section, the most important features of the worker scheduling problem studied in this thesis work will be explained.  

\subsection{Description of the daily tasks at the library}
The most important activity at a library is the activity directed towards the public. This includes book loan services as well as providing customers with helpful information about the resources at the disposal of the library. These activities are referred to as "outer tasks". In addition, the book collection must be maintained, the returned books must be sorted and put back to the shelves, the web page must be updated and so on. Such work is often referred to as "inner work" and this non-public work is an equally important part of the everyday activities at the library. 

Three main outer task types can be identified at the library of Norrköping, working in the service counter (sv. expiditionsdisken), working in the information counter (sv. informationsdiken)  and assembling books which are to be sent to other libraries according to the "fetch list" (sv. plocklista). The fetch list is a task type for which the worker is scheduled during a whole day, while the other task types have two to five hour shifts. The outer tasks can be performed by either librarians or both, as described in Table \ref{tab:Outer_Tasks}.

\begin{table}[h]
\centering
\caption{Outer tasks can be performed exclusively by librarians or by both librarians and assistants.}
\label{tab:Outer_Tasks}
\begin{tabularx}{\textwidth}{|l|l|X|}
\hline
%-------------------------------------------------------------------
\textbf{Task} & \textbf{Description} & \textbf{Qualification}\\ \hline 
%------------------------------------------------------------------- 
\specialcell[t]{Service counter \\ (Exp)}  & \specialcell[t]{Administring loans, library cards\\ and the loaning machine} & \specialcell[t]{Librarian or \\  assistant} 
%\begin{tabular}[x]{@{}c@{}}\\\end{tabular}  
\\ \hline
%-------------------------------------------------------------------
\specialcell[t]{Information counter \\ (Info)} & \specialcell[t]{Handling questions \\about the library's resources.} & Librarian
\\ \hline 
%-------------------------------------------------------------------
\specialcell[t]{Fetch list \\(PL)} & \specialcell[t]{Fetching books that are to be \\sent to other libraries.} & \specialcell[t]{Librarian  or \\  assistant}
\\ \hline 
%-------------------------------------------------------------------
\end{tabularx}
\end{table} 

Since the number of visitors at the library varies throughout the day and also during different days of the week, so does the demand for personnel for the three tasks. The demand of personnel for the different task types is illustrated in Table \ref{tab:Outer_Task_Demand}, according to figures given by the library. 



\begin{table}[h]
\centering
\caption{Demand of staff for the three daily tasks.}
\label{tab:Outer_Task_Demand}
\begin{tabularx}{\textwidth}{|X|l|l|l|l|X|}
\hline
%-------------------------------------------------------------------
\textbf{Day} & \textbf{Time} & \textbf{Exp demand} & \textbf{Info demand} & \textbf{PL demand}
\\ \hline 
%%------------------------------------------------------------------- 
Mon-Fri & 08:00-10:00 & 2 & 2 & 1
\\ \hline 
%%------------------------------------------------------------------- 
Mon-Fri & 10:00-13:00 & 3 & 3 & 1
\\ \hline 
%%-------------------------------------------------------------------
Mon-Fri & 13:00-16:00  & 3 & 3 & 1
\\ \hline 
%%-------------------------------------------------------------------
Mon-Fri & 16:00-20:00 & 3 & 3 & -
\\ \hline 
%%-------------------------------------------------------------------
Sat & 11:00-16:00  & 3 & 3 & -
\\ \hline 
%%------------------------------------------------------------------- 
Sun & 11:00-16:00  & 3 & 3 & -
\\ \hline 
\end{tabularx}
\end{table} 


As in the case with most libraries, the Central Library of Norrköping also has responsiblities that fall outside of its normal daily activities. One such resposibility is the running of a smaller library filial in Hageby, situated in a suburban area of Norrköping, during weekends. It is decided that only librarians are qualified for this task, as it implies some tasks that need certain knowledge.

Table \ref{tab:Hageby_Demand} shows the demand of staff at Hageby. Worth noting is that it is the same person working at Hageby a weekend due to a couple of reasons. Since a worker is supposed to work both saturday and sunday when due for a weekend, it is more desirable to let the worker focus on the same task both days. Furthermore, since Hageby is located in a suburban, some travel distance is required to reach it. This is compensated by letting the worker at Hageby be free from work friday evening, which otherwise is included in the weekend work. 

\begin{table}[H]
\centering
\caption{Demand of staff at Hageby}
\label{tab:Hageby_Demand}
\begin{tabularx}{0.5\textwidth}{|l|l|X|}
\hline
%-------------------------------------------------------------------
\textbf{Day} & \textbf{Time} & \textbf{HB}
\\ \hline 
%%------------------------------------------------------------------- 
Sat & 11:00-16:00  & 1 
\\ \hline 
%%-------------------------------------------------------------------
Sun & 11:00-16:00  & 1 
\\ \hline 
\end{tabularx}
\end{table} 

Similarly, only a handful librarians are qualified for the task known as the "Library on Wheels" (sv. Bokbussen), which is a library bus that provides citizens in remoter areas of the city with books and other library services. The Library on Wheels only operates a few times a week and the schedule differs between odd and even weeks.

\begin{table}[!h]
\centering
\caption{Demand of staff Library on Wheels}
\label{tab:LOW_Demand}
\begin{tabularx}{0.85\textwidth}{|l|l|l|X|}
\hline
%-------------------------------------------------------------------
\textbf{Day} & \textbf{Time} & \textbf{BokB - odd week} & \textbf{BokB - even week} 
\\ \hline 
%------------------------------------------------------------------- 
\rowcolor{Gray} 
Mon & 08:00-10:00 & 1 & 1
\\ \hline 
%-------------------------------------------------------------------
\rowcolor{Gray}  
Mon & 16:00-20:00 & 1 & -
\\ \hline 
%-------------------------------------------------------------------
Tue & 08:00-10:00 & - & -
\\ \hline 
%------------------------------------------------------------------- 
Tue & 16:00-20:00 & - & -
\\ \hline 
%-------------------------------------------------------------------
\rowcolor{Gray} 
Wed & 08:00-10:00 & 1 & 1
\\ \hline 
%------------------------------------------------------------------- 
\rowcolor{Gray} 
Wed & 16:00-20:00 & 1 & 1
\\ \hline 
%-------------------------------------------------------------------
Thu & 08:00-10:00 & 1 & 1
\\ \hline 
%-------------------------------------------------------------------
Thu & 16:00-20:00 & 1 & 1
\\ \hline 
%-------------------------------------------------------------------
\rowcolor{Gray}
Fri & 08:00-10:00 & 1 & -
\\ \hline 
%------------------------------------------------------------------- 
\rowcolor{Gray}
Fri & 16:00-20:00 & - & -
\\ \hline 
%--------------------------------------------------------------
\end{tabularx}
\end{table} 

\subsection{Personnel attributes}

At a library, the main group of workers are librarians and assistants (also janitors, cleaners, security guards etc.). The librarians make sure the library's resource collection is up to date, that the visitors find what they're looking for and perform a larger number of inner tasks. Assistants can perform most of theses tasks, while others require librarian competence.

The Central Library of Norrköping currently has 39 workers, 23 of which are librarians and 16 of which are assistants. All staff have different availability for performing tasks, depending on their working hours and the amount of inner work they are in charge of. In the standard case, each worker is assigned one evening per week and once per five weeks he/she is assigned to work during the weekend. The following week after a weekend is compensated by two free days, placed according to the wishes of the worker.

Let us consider a sample worker who is qualified for the same tasks as a librarian, works full time and is also assigned to work on Wednesday evenings. The worker is also assigned to work weekend on the fourth week and has chosen to take out its days off on Thursdays and Fridays on week five. The workers availability is thus illustrated in table \ref{tab:Bob_avail}. The schedule repeats itself after five weeks and illustrates only the availability for tasks but says nothing about whether the worker has been assigned any tasks or not.

All workers have a five week schedule in the same manner as the sample worker. However, in order to meet the weekend demands illustrated by Tables \ref{tab:Outer_Task_Demand}, \ref{tab:LOW_Demand} and \ref{tab:Hageby_Demand}, it is evident that not all workers can be assigned for weekend work at week 4. Instead, a worker's schedule should be seen as a \textit{relative schedule}, which can be shifted by whole weeks. We refer to this as the \textit{week rotation}.

The relative schedule is relative to the \textit{overall schedule} for the library. If the sample worker for example has week rotation 1, this means the weekend week in his relative schedule would be shifted to week 1 in the overall schedule. In Table \ref{tab:Bob_avail}, the weekend rotation is 4.



\begin{table}[!h]
\centering
\caption{Availability schedule for a sample worker. Yellow signifies that the worker is available. In parenthesis, the weekend shift. The week rotation is 4}
\label{tab:Bob_avail}
\begin{tabularx}{\textwidth}{|X|l|l|l|l|l|l|l|X|}
\hline
%-------------------------------------------------------------------
\textbf{Week 1}& \colcell \textbf{Mon} & \colcell \textbf{Tue} & \colcell \textbf{Wed} & \colcell \textbf{Thu} & \colcell \textbf{Fri} & \colcell \textbf{Sat} & \colcell \textbf{Sun}
\\ \hline 
%%------------------------------------------------------------------- 
%\rowcolor{Gray} 
\colcell 08:00-10:00 (11:00-16:00) & \colcelltwo & \colcelltwo & \colcelltwo & \colcelltwo & \colcelltwo & & 
\\ \hline 
%%-------------------------------------------------------------------
%\rowcolor{Gray} 
\colcell 10:00-13:00 & \colcelltwo & \colcelltwo & \colcelltwo & \colcelltwo & \colcelltwo &   & 
\\ \hline 
%%-------------------------------------------------------------------
%\rowcolor{Gray} 
\colcell 13:00-16:00 & \colcelltwo & \colcelltwo & \colcelltwo & \colcelltwo & \colcelltwo & &
\\ \hline 
%%-------------------------------------------------------------------
%\rowcolor{Gray} 
\colcell 16:00-20:00 & & & \colcelltwo & & & &
\\ \hline 
%%-------------------------------------------------------------------
\end{tabularx}
\begin{tabularx}{\textwidth}{|X|l|l|l|l|l|l|l|X|}
\hline
%-------------------------------------------------------------------
\textbf{Week 2}& \colcell \textbf{Mon} & \colcell \textbf{Tue} & \colcell \textbf{Wed} & \colcell \textbf{Thu} & \colcell \textbf{Fri} & \colcell \textbf{Sat} & \colcell \textbf{Sun}
\\ \hline 
%%------------------------------------------------------------------- 
%\rowcolor{Gray} 
\colcell 08:00-10:00 (11:00-16:00) & \colcelltwo & \colcelltwo & \colcelltwo & \colcelltwo & \colcelltwo & & 
\\ \hline 
%%-------------------------------------------------------------------
%\rowcolor{Gray} 
\colcell 10:00-13:00 & \colcelltwo & \colcelltwo & \colcelltwo & \colcelltwo & \colcelltwo &   & 
\\ \hline 
%%-------------------------------------------------------------------
%\rowcolor{Gray} 
\colcell 13:00-16:00 & \colcelltwo & \colcelltwo & \colcelltwo & \colcelltwo & \colcelltwo & &
\\ \hline 
%%-------------------------------------------------------------------
%\rowcolor{Gray} 
\colcell 16:00-20:00 & & & \colcelltwo & & & &
\\ \hline 
%%-------------------------------------------------------------------
\end{tabularx}
\begin{tabularx}{\textwidth}{|X|l|l|l|l|l|l|l|X|}
\hline
%-------------------------------------------------------------------
\textbf{Week 3}& \colcell \textbf{Mon} & \colcell \textbf{Tue} & \colcell \textbf{Wed} & \colcell \textbf{Thu} & \colcell \textbf{Fri} & \colcell \textbf{Sat} & \colcell \textbf{Sun}
\\ \hline 
%%------------------------------------------------------------------- 
%\rowcolor{Gray} 
\colcell 08:00-10:00 (11:00-16:00) & \colcelltwo & \colcelltwo & \colcelltwo & \colcelltwo & \colcelltwo & & 
\\ \hline 
%%-------------------------------------------------------------------
%\rowcolor{Gray} 
\colcell 10:00-13:00 & \colcelltwo & \colcelltwo & \colcelltwo & \colcelltwo & \colcelltwo &   & 
\\ \hline 
%%-------------------------------------------------------------------
%\rowcolor{Gray} 
\colcell 13:00-16:00 & \colcelltwo & \colcelltwo & \colcelltwo & \colcelltwo & \colcelltwo & &
\\ \hline 
%%-------------------------------------------------------------------
%\rowcolor{Gray} 
\colcell 16:00-20:00 & & & \colcelltwo & & & &
\\ \hline 
%%-------------------------------------------------------------------
\end{tabularx}
\begin{tabularx}{\textwidth}{|X|l|l|l|l|l|l|l|X|}
\hline
%-------------------------------------------------------------------
\textbf{Week 4}& \colcell \textbf{Mon} & \colcell \textbf{Tue} & \colcell \textbf{Wed} & \colcell \textbf{Thu} & \colcell \textbf{Fri} & \colcell \textbf{Sat} & \colcell \textbf{Sun}
\\ \hline 
%%------------------------------------------------------------------- 
%\rowcolor{Gray} 
\colcell 08:00-10:00 (11:00-16:00) & \colcelltwo & \colcelltwo & \colcelltwo & \colcelltwo & \colcelltwo & \colcelltwo & \colcelltwo
\\ \hline 
%%-------------------------------------------------------------------
%\rowcolor{Gray} 
\colcell 10:00-13:00 & \colcelltwo & \colcelltwo & \colcelltwo & \colcelltwo & \colcelltwo &   & 
\\ \hline 
%%-------------------------------------------------------------------
%\rowcolor{Gray} 
\colcell 13:00-16:00 & \colcelltwo & \colcelltwo & \colcelltwo & \colcelltwo & \colcelltwo & &
\\ \hline 
%%-------------------------------------------------------------------
%\rowcolor{Gray} 
\colcell 16:00-20:00 & & & \colcelltwo & & \colcelltwo & &
\\ \hline 
%%-------------------------------------------------------------------
\end{tabularx}

\begin{tabularx}{\textwidth}{|X|l|l|l|l|l|l|l|X|}
\hline
%-------------------------------------------------------------------
\textbf{Week 5}& \colcell \textbf{Mon} & \colcell \textbf{Tue} & \colcell \textbf{Wed} & \colcell \textbf{Thu} & \colcell \textbf{Fri} & \colcell \textbf{Sat} & \colcell \textbf{Sun}
\\ \hline 
%%------------------------------------------------------------------- 
%\rowcolor{Gray} 
\colcell 08:00-10:00 (11:00-16:00) & \colcelltwo & \colcelltwo & \colcelltwo & & & & 
\\ \hline 
%%-------------------------------------------------------------------
%\rowcolor{Gray} 
\colcell 10:00-13:00 & \colcelltwo & \colcelltwo & \colcelltwo & & & & 
\\ \hline 
%%-------------------------------------------------------------------
%\rowcolor{Gray} 
\colcell 13:00-16:00 & \colcelltwo & \colcelltwo & \colcelltwo & & & &
\\ \hline 
%%-------------------------------------------------------------------
%\rowcolor{Gray} 
\colcell 16:00-20:00 & & & \colcelltwo & & & &
\\ \hline 
%%-------------------------------------------------------------------
\end{tabularx}
\end{table} 

Considering again the sample worker's schedule, it may look as if the worker can be scheduled for tasks at any yellow shift. However, there are regulations set by the library controlling how the outer tasks are to be distributed among the workers. The purpose of the regulations is to guarantee that the worker schedules are not too unbalanced or uncomfortable. 

There four basic demands of the resulting schedule. Firstly, workers are only allowed to take at a maximum one task per day. Secondly, workers only work at most weekend per five weeks. During this weekend it is customary that you work Friday evening shift, Saturday and Sunday shift consecutively, unless you are scheduled for Hageby, which is far away and is thus compensated by not entailing Friday evening work. Thirdly, weekend work is to be compensated by days off. There are a few possible variations of the weekend compensation, but usually the worker takes one and a half or two days off the week following upon weekend work. Lastly, a worker is only allowed to work one evening per week, excluding the Friday which belongs to the weekend week.
%
%Regulations say that:
%
%\begin{enumerate}
%\item A worker can at a maximum perform one task per day.
%\item A worker works only one weekend in five weeks.
%\item A worker must work Friday evening shift, Saturday shift and Sunday shift consecutively. For HB-workers, no Friday evening shift.
%\item A worker is only allowed to work one evening per week, excluding the Friday evening of the weekend week.
%\end{enumerate}

An example of a feasible schedule for the sample worker is provided in Table \ref{tab:Lib_feas_sched}. It should be noted that this schedule is created for a general worker and does not necessarily apply to all workers. Among workers, special availability restrictions due to odd-and even weeks exist and only a subset of the librarians are available for the Library on Wheels and Hageby. Also, some workers never work evening or weekend shifts.  

\begin{table}[!h]
\centering
\caption{Example of a feasible first week for the sample worker.}
\label{tab:Lib_feas_sched}
\begin{tabularx}{\textwidth}{|X|l|l|l|l|l|l|l|X|}
\hline
%-------------------------------------------------------------------
\textbf{Week 1} & \colcell \textbf{Mon} & \colcell \textbf{Tue} & \colcell \textbf{Wed} & \colcell \textbf{Thu} & \colcell \textbf{Fri} & \colcell \textbf{Sat} & \colcell \textbf{Sun}
\\ \hline 
%%------------------------------------------------------------------- 
%\rowcolor{Gray} 
\small \colcell 08:00-10:00 (11:00-16:00)& \colcelltwo & \small \colcellthree Exp & \colcelltwo & \small \colcellthree PL & \colcelltwo & & 
\\ \hline 
%%-------------------------------------------------------------------
%\rowcolor{Gray} 
\small \colcell 10:00-13:00 & \small \colcellthree Exp & \colcelltwo & \colcelltwo & \small \colcellthree PL & \colcelltwo & & 
\\ \hline 
%%-------------------------------------------------------------------
%\rowcolor{Gray} 
\small \colcell 13:00-16:00 & \colcelltwo & \colcelltwo & \colcelltwo & \small \colcellthree PL & \small \colcellthree Info & &
\\ \hline 
%%-------------------------------------------------------------------
%\rowcolor{Gray} 
\small \colcell 16:00-20:00 & & & \small \colcellthree Info& & & &
\\ \hline 
%%-------------------------------------------------------------------
\end{tabularx}
\end{table} 


\subsection{Scheduling objectives: stand-in maximization and schedule variation}

What is investigated in this thesis is not primarily how to distribute the tasks according to the rules given above, but rather how to do this is a way so that the number of stand-in personnel is maximized. The emergency back-up system of stand-ins is crucial for the library in order to be able to keep the library desks open and fully staffed. Thus, the highest priority is to maximize the number of stand-in personnel.

A stand-in is defined as a worker who is available for outer tasks during the first three shifts, as well as not scheduled for any shift that day. Inner work sometimes prevents personnel from being scheduled as a stand-in for the outer tasks, and therefore also information about such work must be known to the scheduler. Both librarians and assistants can be scheduled as a stand-in, but only librarians are qualified to take emergency shifts at the Info desk. Since the regular library activity is the most crucial activity, there is no need for stand-ins during evening and weekends. Similarly, there are no assigned stand-ins for the Library on Wheels or for Hageby.

Apart from maximizing the number of stand-ins for each day at the library, another important measurement of a good schedule is the level of variation. It is desirable for the sake of the personnel to create schedules in which the weeks resemble eachother or are repeated according to some pattern. On the other hand, since some shifts are more attractive than other and in order to avoid too much repetitiveness it is desirable that the days during a work week are not too similar. The example schedule in Table \ref{tab:Lib_feas_sched} is an example of a sufficiently varied weekly schedule. 

\section{Method}



\section{Topics Covered}













\iffalse
\subsection{Main objectives}
Main objective is to create a schedule with as many stand-ins as possible, so that all days have a maximum amount of stand-ins. Diversity during the week and repetivity of the schedule each week is also desired.

\subsection{Requirements on weekday activities}
During weekdays, there is a worker demand at the stations Exp, Info, Plocklista and Library on wheels. <TABLE OF DEMAND>. Plocklista is unique in the sense that its duration is longer than one shift. It is modeled as the three first shifts of a weekday.

In addition there is the Library on wheels, which has a different demand of workers depending on odd and even weeks. 
\subsection{Requirements on weekend activities}
There are three different weekend stations: Exp, Info and Hageby. Working a weekend also means working friday unless you are scheduled to work in Hageby. 
\subsection{Diversity in skill}
There are essentially two types of workers in a library. Librarians and assistants. The competence of the workers looks as follows: <TABLE OF DIFFERENT COMPETENCES>

A subset of the librarians can handle the Library on wheels and all are expected to take shifts at Hageby.

\subsection{Personnel availability and task assignment limitations}
The staff at the library have certain times of availability. Every day, a person can perform at maximum one task. Plocklistan is limited to once a week.

Personnel is only allowed to work one weekend out of five. The following week, the worker is free on the times requested by the worker. Also, some workers have specific 

\subsection{Inner work} 

The goal of this thesis is to distribute given tasks to the heterogeneous workforce at the library of Norrköping. Each task is either classified as an outer or an inner service where an outer service is when a librarian needs to interact with visitors. Inner services can in some rare cases require a predetermined person to be assigned to a specified time or day.

Demands and requests are to be fulfilled to the furthest extent possible. Weekends are included in the scheduling problem, which adds more constraints regarding the number of contiguous working days. However, the librarians are permitted a few exceptions from these laws regarding days of rest.

The main purpose of the thesis is to create a schedule robust enough to withstand absence, such that outer services always are assigned to a qualified and available worker. This is visualized as having a list of available stand-ins for each shift. 

There are a limited number of workers at the library and they make the resources that are to be distributed. Each individual has a set of \textit{skills} and \textit{competences}. Competences refer to the capability of being assigned the different outer services; Expedition, Norpan, Information desk, Library on wheels and Hageby as well as different inner services. The set of skills an individual can possess are described in Table \ref{int:1}. In total there are 39 workers available.

The outer services can be seen as assignments which requires available workers to be assigned to them. Each outer service is specified to a certain station, time and date. They also have a fix length and occur on a regular basis every ten weeks, which makes it possible to create a periodic schedule with a period of ten weeks. 


%Examensarbetet går ut på att lägga ett arbetsschema för personalen vid Norrköpings bibliotek. Problemet går i grund och botten ut på att fylla alla uppgifter på de stationer som tillhör bibliotekets utåtriktade verksamhet (så kallade yttre tjänst)  med personal av rätt kompetens samt samtidigt som personalen får tid över till övriga uppgifter (så kallad inre tjänst). Schemat som tas fram ska även uppfylla de regelverk och önskemål som finns kring personalens individuella scheman, till exempel de arbetstider som ingår i de olika tjänsterna. Då det även ingår helgarbete i personalens arbetsuppgifter ska lediga dagar fördelas enligt arbetsmiljölagen och de undantag från dessa lagar gällande veckovilan.

%Utöver detta ska schemat även medföra en robusthet så att störningar i den yttre tjänsten, i form av att personal blir sjuk eller uppbokad annonstädes, ska gå att avhjälpa med en reservlista. Denna reservlista består av de bibliotekarier och assistenter som inte har något pass tilldelat sig samt är tillgängliga under dagen. 

%Personalen på biblioteket är begränsad och utgör de resurser som finns att tillgå. Varje enskild personal har en uppsättning \textit{egenskaper} och \textit{kompetenser}. Kompetenser syftar på personalens förmåga att arbeta vid någon av de yttre stationerna; Expedition, Norpan, Informationsdisk, Bokbuss och Hageby samt några av de inre stationerna; inköp, katalogisering med mera. De egenskaper som identifierats hos en personal finns beskrivna i tabell \ref{int:1}. Totala arbetskraften består av 39 stycken arbetare på biblioteket.

%De yttre och inre uppgifterna kan ses som behov av personal som måste täckas av den personal som finns att tillgå. De olika yttre uppgifterna som behöver utföras inkluderar arbete vid olika stationer vid olika tidpunker och datum. Varje uppgift har en bestämd längd och återkommer regelbundet inom ett 10-veckorsinterval, vilket gör att ett rullande schema kan skapas med en period om tio veckor. 

\begin{table}[h]
\centering
\caption{Personnel}
\label{int:1}
\begin{tabular}{|l|l|}
%-------------------------------------------------------------------
\hline 
\textbf{Skills} & \textbf{Description} \\ \hline
%-------------------------------------------------------------------
Work degree & 0-100 \% 
\\ \hline 
%-------------------------------------------------------------------
Type of employment & Librarian/Assistant
\\ \hline 
%-------------------------------------------------------------------
Competence & Inner and outer services the worker is qualified for  
\\ \hline 
%-------------------------------------------------------------------
Weekly rest & Which days the worker has requested after working a weekend
\\ \hline 
%-------------------------------------------------------------------
Other requests & Does not work evenings etc.
\\ \hline 
%-------------------------------------------------------------------
\end{tabular}
\end{table} 

Furthermore, outer and a few inner services can be characterized by different properties, which are represented in Table \ref{int:2}. \\

\begin{table}[!h]
\caption{Outer and inner services}
\label{int:2}
\begin{tabular}{|l|l|}
%-------------------------------------------------------------------
\hline
\textbf{Outer service} & \textbf{Property} \\ \hline
%-------------------------------------------------------------------
 & Start time, end time, week and duration \
\\ \hline 
%-------------------------------------------------------------------
 & Station
\\ \hline 
%-------------------------------------------------------------------
 & Number of qualified librarians demanded
\\ \hline 
%-------------------------------------------------------------------
 & Number of qualified assistants demanded
\\ \hline 
%-------------------------------------------------------------------

\textbf{Inner service} & \textbf{Property} \\ \hline
%-------------------------------------------------------------------
 & Start time, end time, week and duration \
\\ \hline 
%-------------------------------------------------------------------
 & Type
\\ \hline 
%-------------------------------------------------------------------
 & Number of qualified librarians demanded
\\ \hline 
%-------------------------------------------------------------------
 & Number of qualified assistants demanded
\\ \hline 
%-------------------------------------------------------------------
\end{tabular}
\end{table}

In addition to the properties mentioned above, there are several requirements that have to be met. These can be divided into job, robust and other requirements and are listed in Table \ref{int:3} below. 

%Utöver de ovan nämnda resurserna och behoven, finns ett antal krav som ställs på hur schemat får utformas. Dessa kan delas upp i arbetsvillkor, robusthetskrav samt övriga krav och finns representerade i tabell \ref{int:3}.

\begin{table}[H]
\caption{Requirements}
\label{int:3}
\begin{tabular}{|c|l|}
%-------------------------------------------------------------------
\hline
\textbf{Job requirements} & \textbf{Description} \\ \hline
%-------------------------------------------------------------------
1 & A maximum of one outer service is to be distributed to each person and day.
\\ \hline 
%-------------------------------------------------------------------
2 & Remaining work time is individually distributed on assignments such as reshelving books.
\\ \hline
%-------------------------------------------------------------------
3 & Weekend work are to be evenly distributed between the workers available on weekends. 
\\ \hline 
%-------------------------------------------------------------------
4 & Working a weekend includes work on saturday and sunday the same week.
\\ \hline 
%-------------------------------------------------------------------
5 & One evening shift on a weekday per person each week except when weekend work is required.
\\ \hline 
%-------------------------------------------------------------------
6 & Every ten weeks the schedule is to be repeated.
\\ \hline 
%-------------------------------------------------------------------
7 & It is recommended for each week to be as similar as possible.
\\ \hline 
%-------------------------------------------------------------------

\textbf{Robust requirements} & \textbf{Description} \\ \hline
%-------------------------------------------------------------------
1 & Each outer service require at least one stand-in.
\\ \hline 
%-------------------------------------------------------------------
2 & The stand-ins have to be qualified for the tasks they are stand-in for.
\\ \hline 
%-------------------------------------------------------------------
3 & Focus is to maximize the lowest number of stand-ins of any task.
\\ \hline 
%-------------------------------------------------------------------

\textbf{Other requirements} & \textbf{Description} \\ \hline
%-------------------------------------------------------------------
1 & Department and general meetings are to be held once per five weeks.
\\ \hline 
%-------------------------------------------------------------------
\end{tabular}
\end{table}
\medskip

There are also additional requirements of the resulting schedule made by the workers at the library. Two examples would be that a handful staff members are unable to work weekends as well as some personnel are unable to work in the evenings.

\section{Problem categorization}
The problem can be formulated as a personnel tasks scheduling problem for a heterogeneous workforce since the main objective is to distribute tasks to workers during their available times. The workforce is heterogeneous as certain tasks can only be performed by librarians or is restricted to a certain subset of the personnel. Another aspect of the problem is the cyclic nature of the personnel schedules, which gives a degree of freedom in availability of personnel.



\begin{table}[h]
\centering
\caption{Overall schedule for week 1.}
\label{tab:General_schedule}
\begin{tabularx}{\textwidth}{|X|l|l|l|l|X|}
\hline
%-------------------------------------------------------------------
\textbf{Schedule week 1} & & & & &  
\\ \hline 
%%------------------------------------------------------------------- 
\textbf{Monday}& \colcell \textbf{Exp} & \colcell \textbf{Info} & \colcell \textbf{PL} & \colcell \textbf{LoW - odd} & \colcell \textbf{LoW- even} 
\\ \hline 
%%------------------------------------------------------------------- 

 08:00 - 10:00 & \small W1, W2 & \small W1,W2 & \small W1 & \small W1 & \small W1
\\ \hline 
%%------------------------------------------------------------------- 
 10:00 - 13:00 & \small W1, W2, W3 & \small W1, W2, W3 & \small W1 & - & -
\\ \hline 
%%------------------------------------------------------------------- 
 13:00 - 16:00 & \small W1, W2, W3 & \small W1, W2, W3 & \small W1 & - & -
\\ \hline 
%%------------------------------------------------------------------- 
 16:00 - 20:00 & \small W1, W2, W3 & \small W1, W2, W3 & \small W1 & - & -
\\ \hline 
%%------------------------------------------------------------------- 
\textbf{Tuesday}& \colcell \textbf{Exp} & \colcell \textbf{Info} & \colcell \textbf{PL} & \colcell \textbf{LoW - odd } & \colcell \textbf{LoW- even} 
\\ \hline 
%%------------------------------------------------------------------- 
\textbf{Wednesday}& \colcell \textbf{Exp} & \colcell \textbf{Info} & \colcell \textbf{PL} & \colcell \textbf{LoW - odd } & \colcell \textbf{LoW- even} 
\\ \hline 
%%-------------------------------------------------------------------
\textbf{Thursday}& \colcell \textbf{Exp} & \colcell \textbf{Info} & \colcell \textbf{PL} & \colcell \textbf{LoW - odd } & \colcell \textbf{LoW- even} 
\\ \hline 
%%-------------------------------------------------------------------
\textbf{Friday}& \colcell \textbf{Exp} & \colcell \textbf{Info} & \colcell \textbf{PL} & \colcell \textbf{LoW - odd } & \colcell \textbf{LoW- even} 
\\ \hline 
%%-------------------------------------------------------------------
\end{tabularx}
\end{table}

\fi
