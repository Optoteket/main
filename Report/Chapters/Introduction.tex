% Introduction to Master Thesis


%\section{Background} 
%Schemaläggning av personal vid Norrköpings bibliotek.
\section{Background}
At a library absence can cause problems, both due to lack of personnel as well as due to the qualifications required to perform a task varies. If a worker were to be unavailable a day on a meeting or being ill it would require for a stand-in to fill the vacancy. Therefore, it is of great interest to have a schedule with as many skilled stand-ins as possible to overcome such disturbances. Furthermore, the library personnel have certain demands and preferences as to how a satisfactory working schedule should be. For instance, it is neither preferable to work more than one evening each week nor work more weekends than required. 

\section{Problem description}
\subsection{Main objectives}
\subsection{Requirements on weekday activities in outer stations}
\subsection{Requirements on weekend activities: library and filial}
The goal of this thesis is to distribute given tasks to the heterogeneous workforce at the library of Norrköping. Each task is either classified as an outer or an inner service where an outer service is when a librarian needs to interact with visitors. Inner services can in some rare cases require a predetermined person to be assigned to a specified time or day.

Demands and requests are to be fulfilled to the furthest extent possible. Weekends are included in the scheduling problem, which adds more constraints regarding the number of contiguous working days. However, the librarians are permitted a few exceptions from these laws regarding days of rest.

The main purpose of the thesis is to create a schedule robust enough to withstand absence, such that outer services always are assigned to a qualified and available worker. This is visualized as having a list of available stand-ins for each shift. 

There are a limited number of workers at the library and they make the resources that are to be distributed. Each individual has a set of \textit{skills} and \textit{competences}. Competences refer to the capability of being assigned the different outer services; Expedition, Norpan, Information desk, Library on wheels and Hageby as well as different inner services. The set of skills an individual can possess are described in Table \ref{int:1}. In total there are 39 workers available.

The outer services can be seen as assignments which requires available workers to be assigned to them. Each outer service is specified to a certain station, time and date. They also have a fix length and occur on a regular basis every ten weeks, which makes it possible to create a periodic schedule with a period of ten weeks. 


%Examensarbetet går ut på att lägga ett arbetsschema för personalen vid Norrköpings bibliotek. Problemet går i grund och botten ut på att fylla alla uppgifter på de stationer som tillhör bibliotekets utåtriktade verksamhet (så kallade yttre tjänst)  med personal av rätt kompetens samt samtidigt som personalen får tid över till övriga uppgifter (så kallad inre tjänst). Schemat som tas fram ska även uppfylla de regelverk och önskemål som finns kring personalens individuella scheman, till exempel de arbetstider som ingår i de olika tjänsterna. Då det även ingår helgarbete i personalens arbetsuppgifter ska lediga dagar fördelas enligt arbetsmiljölagen och de undantag från dessa lagar gällande veckovilan.

%Utöver detta ska schemat även medföra en robusthet så att störningar i den yttre tjänsten, i form av att personal blir sjuk eller uppbokad annonstädes, ska gå att avhjälpa med en reservlista. Denna reservlista består av de bibliotekarier och assistenter som inte har något pass tilldelat sig samt är tillgängliga under dagen. 

%Personalen på biblioteket är begränsad och utgör de resurser som finns att tillgå. Varje enskild personal har en uppsättning \textit{egenskaper} och \textit{kompetenser}. Kompetenser syftar på personalens förmåga att arbeta vid någon av de yttre stationerna; Expedition, Norpan, Informationsdisk, Bokbuss och Hageby samt några av de inre stationerna; inköp, katalogisering med mera. De egenskaper som identifierats hos en personal finns beskrivna i tabell \ref{int:1}. Totala arbetskraften består av 39 stycken arbetare på biblioteket.

%De yttre och inre uppgifterna kan ses som behov av personal som måste täckas av den personal som finns att tillgå. De olika yttre uppgifterna som behöver utföras inkluderar arbete vid olika stationer vid olika tidpunker och datum. Varje uppgift har en bestämd längd och återkommer regelbundet inom ett 10-veckorsinterval, vilket gör att ett rullande schema kan skapas med en period om tio veckor. 

\begin{table}[h]
\centering
\caption{Personnel}
\label{int:1}
\begin{tabular}{|l|l|}
%-------------------------------------------------------------------
\hline 
\textbf{Skills} & \textbf{Description} \\ \hline
%-------------------------------------------------------------------
Work degree & 0-100 \% 
\\ \hline 
%-------------------------------------------------------------------
Type of employment & Librarian/Assistant
\\ \hline 
%-------------------------------------------------------------------
Competence & Inner and outer services the worker is qualified for  
\\ \hline 
%-------------------------------------------------------------------
Weekly rest & Which days the worker has requested after working a weekend
\\ \hline 
%-------------------------------------------------------------------
Other requests & Does not work evenings etc.
\\ \hline 
%-------------------------------------------------------------------
\end{tabular}
\end{table} 

Furthermore, outer and a few inner services can be characterized by different properties, which are represented in Table \ref{int:2}. \\

\begin{table}[!h]
\caption{Outer and inner services}
\label{int:2}
\begin{tabular}{|l|l|}
%-------------------------------------------------------------------
\hline
\textbf{Outer service} & \textbf{Property} \\ \hline
%-------------------------------------------------------------------
 & Start time, end time, week and duration \
\\ \hline 
%-------------------------------------------------------------------
 & Station
\\ \hline 
%-------------------------------------------------------------------
 & Number of qualified librarians demanded
\\ \hline 
%-------------------------------------------------------------------
 & Number of qualified assistants demanded
\\ \hline 
%-------------------------------------------------------------------

\textbf{Inner service} & \textbf{Property} \\ \hline
%-------------------------------------------------------------------
 & Start time, end time, week and duration \
\\ \hline 
%-------------------------------------------------------------------
 & Type
\\ \hline 
%-------------------------------------------------------------------
 & Number of qualified librarians demanded
\\ \hline 
%-------------------------------------------------------------------
 & Number of qualified assistants demanded
\\ \hline 
%-------------------------------------------------------------------
\end{tabular}
\end{table}

In addition to the properties mentioned above, there are several requirements that have to be met. These can be divided into job, robust and other requirements and are listed in Table \ref{int:3} below. 

%Utöver de ovan nämnda resurserna och behoven, finns ett antal krav som ställs på hur schemat får utformas. Dessa kan delas upp i arbetsvillkor, robusthetskrav samt övriga krav och finns representerade i tabell \ref{int:3}.

\begin{table}[H]
\caption{Requirements}
\label{int:3}
\begin{tabular}{|c|l|}
%-------------------------------------------------------------------
\hline
\textbf{Job requirements} & \textbf{Description} \\ \hline
%-------------------------------------------------------------------
1 & A maximum of one outer service is to be distributed to each person and day.
\\ \hline 
%-------------------------------------------------------------------
2 & Remaining work time is individually distributed on assignments such as reshelving books.
\\ \hline
%-------------------------------------------------------------------
3 & Weekend work are to be evenly distributed between the workers available on weekends. 
\\ \hline 
%-------------------------------------------------------------------
4 & Working a weekend includes work on saturday and sunday the same week.
\\ \hline 
%-------------------------------------------------------------------
5 & One evening shift on a weekday per person each week except when weekend work is required.
\\ \hline 
%-------------------------------------------------------------------
6 & Every ten weeks the schedule is to be repeated.
\\ \hline 
%-------------------------------------------------------------------
7 & It is recommended for each week to be as similar as possible.
\\ \hline 
%-------------------------------------------------------------------

\textbf{Robust requirements} & \textbf{Description} \\ \hline
%-------------------------------------------------------------------
1 & Each outer service require at least one stand-in.
\\ \hline 
%-------------------------------------------------------------------
2 & The stand-ins have to be qualified for the tasks they are stand-in for.
\\ \hline 
%-------------------------------------------------------------------
3 & Focus is to maximize the lowest number of stand-ins of any task.
\\ \hline 
%-------------------------------------------------------------------

\textbf{Other requirements} & \textbf{Description} \\ \hline
%-------------------------------------------------------------------
1 & Department and general meetings are to be held once per five weeks.
\\ \hline 
%-------------------------------------------------------------------
\end{tabular}
\end{table}
\medskip

There are also additional requirements of the resulting schedule made by the workers at the library. Two examples would be that a handful staff members are unable to work weekends as well as some personnel are unable to work in the evenings.

\section{Problem categorization}
The problem can be formulated as a personnel tasks scheduling problem for a heterogeneous workforce since the main objective is to distribute tasks to workers during their available times. The workforce is heterogeneous as certain tasks can only be performed by librarians or is restricted to a certain subset of the personnel. Another aspect of the problem is the cyclic nature of the personnel schedules, which gives a degree of freedom in availability of personnel.


