% Introduction to Master Thesis


%\section{Background} 
%Schemaläggning av personal vid Norrköpings bibliotek.
\section{Background}
At a library absence can cause problems, both due to lack of personnel as well as due to the qualifications required to perform a task varies. If a worker were to be unavailable a day because of a meeting or being ill it would require for a stand-in to fill the vacancy. Therefore, it is of great interest to have a schedule with as many skilled stand-ins as possible to overcome such disturbances. Furthermore, the library personnel have certain demands and preferences as to how a satisfactory working schedule should be. For instance, it is neither preferable to work more than one evening each week nor work more weekends than required.

The central library of Norrköping is more than 100 years old. The library has more than 40 employees. These consist of both librarians and assistants, who handle simpler tasks.  The library is open both weekdays until 20:00 and during weekend days until 16, which is challenge for the schedulers of the library since this requires time compensation for the staff. The library also provides its services to more than five other smaller libraries.% Beskrivande text om Norrköpings bibliotek! 

\section{Problem description}

\subsection{Description of the daily tasks at the library}
The most important activity at a library is the activity directed towards the public. This includes lending books services as well as providing customers with helpful information about the resources at their disposal. These are referred to as "outer tasks". At the same time, the uppsättning of books must be maintained, the returned books must be sorted and put back, the web page must be up to date and so on. Such work is often referred to as "inner work" and is equally part of the everyday tasks of a librarian. 

At the library of Norrköping, three main outer tasks can be identified as working in the service counter (sv. expiditionsdisken), working in the information counter (sv. informationsdiken)  and assembling books according to the "fetch list" (sv. plocklista). These tasks can be performed by either librarians or assistants, as descirbed in able \ref{tab:Outer_Tasks}.

\begin{table}[h]
\centering
\caption{Outer tasks can be performed exclusively by librarians or by assistants also.}
\label{tab:Outer_Tasks}
\begin{tabularx}{\textwidth}{|X|l|X|}
\hline
%-------------------------------------------------------------------
\large{\textbf{Task}} & \large{\textbf{Description}} & \large{\textbf{Qualification}}\\ \hline 
%------------------------------------------------------------------- 
Service Counter (Exp)  & \specialcell[t]{Administring loans, library cards\\ and the loaning machine} & Assistants, Librarians
%\begin{tabular}[x]{@{}c@{}}\\\end{tabular}  
\\ \hline
%-------------------------------------------------------------------
Information Counter (Info) & \specialcell[t]{Handling questions \\about the library's resources.} & Librarians
\\ \hline 
%-------------------------------------------------------------------
Fetch List (FL) & \specialcell[t]{Fetching books that are to be \\sent to other libraries.} & Assistants, Librarians
\\ \hline 
%-------------------------------------------------------------------
\end{tabularx}
\end{table} 

As the number of visitors in the library differs at different times of the day and during different days so does the demand for people for the three tasks. The demand of people at for the different tasks is illustrated in table \ref{tab:Outer_Task_Demand}. 



\begin{table}[h]
\centering
\caption{Demand of staff for the three daily tasks.}
\label{tab:Outer_Task_Demand}
\begin{tabularx}{\textwidth}{|X|l|l|l|l|X|}
\hline
%-------------------------------------------------------------------
\textbf{Day} & \textbf{Time} & \textbf{Exp demand} & \textbf{Info demand} & \textbf{FL demand}
\\ \hline 
%%------------------------------------------------------------------- 
Mon-Fri & 08:00-10:00 & 2 & 2 & 1
\\ \hline 
%%------------------------------------------------------------------- 
Mon-Fri & 10:00-13:00 & 3 & 3 & 1
\\ \hline 
%%-------------------------------------------------------------------
Mon-Fri & 13:00-16:00  & 3 & 3 & 1
\\ \hline 
%%-------------------------------------------------------------------
Mon-Fri & 16:00-20:00 & 3 & 3 & -
\\ \hline 
%%-------------------------------------------------------------------
Sat & 11:00-16:00  & 3 & 3 & -
\\ \hline 
%%------------------------------------------------------------------- 
Sun & 11:00-16:00  & 3 & 3 & -
\\ \hline 
\end{tabularx}
\end{table} 


As is the case with most libraries, the Central Library of Norrköping also has responsiblities that fall outside of it's normal daily activities. One such resposibility is the running of a smaller library filial in Hageby, situated in a suburbian area of Norrköping, during weekends. For this tasks only librarians are qualified as the placement implies all types of library tasks.

Similarly, only librarians are qualified for the task known as "Library on Wheels" (sv. Bokbussen), which is a type of library bus, providing citizens in remoter areas of the city with books and other library services. The Library on Wheels has only operates a few times a week and the schedule differs between even and odd weeks.

\begin{table}[h]
\centering
\caption{Demand of staff  Library on Wheels}
\label{tab:LOW_Demand}
\begin{tabularx}{0.85\textwidth}{|l|l|l|X|}
\hline
%-------------------------------------------------------------------
\textbf{Day} & \textbf{Time} & \textbf{LoW - odd week} & \textbf{LoW - even week} 
\\ \hline 
%%------------------------------------------------------------------- 
\rowcolor{Gray} 
Mon & 08:00-10:00 & 1 & 1
\\ \hline 
%%-------------------------------------------------------------------
\rowcolor{Gray}  
Mon & 16:00-20:00 & 1 & -
\\ \hline 
%%-------------------------------------------------------------------
Tue & 08:00-10:00 & - & -
\\ \hline 
%%------------------------------------------------------------------- 
Tue & 16:00-20:00 & - & -
\\ \hline 
%%-------------------------------------------------------------------
\rowcolor{Gray} 
Wed & 08:00-10:00 & 1 & 1
\\ \hline 
%%------------------------------------------------------------------- 
\rowcolor{Gray} 
Wed & 16:00-20:00 & 1 & 1
\\ \hline 
%%-------------------------------------------------------------------
Thu & 08:00-10:00 & 1 & 1
\\ \hline 
%%-------------------------------------------------------------------
Thu & 16:00-20:00 & 1 & 1
\\ \hline 
%%-------------------------------------------------------------------
\rowcolor{Gray}
Fri & 08:00-10:00 & 1 & -
\\ \hline 
%%------------------------------------------------------------------- 
\rowcolor{Gray}
Fri & 16:00-20:00 & - & -
\\ \hline 
%%%%-------------------------------------------------------------------
%Sat & 11:00-16:00 & - & -
%\\ \hline 
%%%-------------------------------------------------------------------
%Sun & 11:00-16:00 & - & -
%\\ \hline 
\end{tabularx}
\end{table} 

\begin{table}[h]
\centering
\caption{Demand of staff at Hageby}
\label{tab:Hageby_Demand}
\begin{tabularx}{0.5\textwidth}{|l|l|X|}
\hline
%-------------------------------------------------------------------
\textbf{Day} & \textbf{Time} & \textbf{Hageby}
\\ \hline 
%%------------------------------------------------------------------- 
%%-------------------------------------------------------------------
Sat & 11:00-16:00  & 1 
\\ \hline 
%%-------------------------------------------------------------------
Sun & 11:00-16:00  & 1 
\\ \hline 
\end{tabularx}
\end{table} 

\subsection{Personnel attributes}

At a library, the most important workers are the librarians. They make sure the library's resource collection is up to date, that the visitors find what they're looking for and perform a larger number of inner tasks. Simpler tasks can also be performed by a group of workers called assistants and they help the librarians in keeping the library running in many ways.

The Central Library of Norrköping currently has 39 workers, 23 of which are librarians and 16 of which are assistants. All staff have different availability for performing tasks, depending on their working hours and the amount of inner work they are in charge of. Generally, each worker is assigned one evening per week and once per five weeks, he/she is asssigned to weekend work. The following week is compensated by two free days, placed according to the wishes of the worker.

Let's consider a sample worker Bob. Bob works full time and is also assigned to work on Wednesday evenings. Bob works weekend on the fourth week and has chosen to take out his days off on Thursdays and Fridays on week five. His availability is thus illustrated in table \ref{tab:Bob_avail}. The availability schedule repeats itself after five weeks (Note: Whether or not Bob is assigned any tasks during these weeks is not illustrated in the table).

All workers have a five week schedule in the same manner as sample worker Bob. However, in order to meet the weekend demands as illustrated by Tables \ref{tab:Outer_Task_Demand} and \ref{tab:Hageby_LOW_Demand}, it is evident that not all workers can be assigned for weekend work at week 4. Instead, Bob's schedule should be seen as a \textit{relative schedule}, which can be shifted by whole weeks. We refer to this as the \textit{week rotation}. The relative schedule is relative to the \textit{overall schedule} for the library, which is illustrated by Table \tab{tab:}. If Bob for example has week rotation 1, this means the week 4 in his relative schedule would be placed on week 1 in the overall schedule.



\begin{table}[h]
\centering
\caption{Schedule for sample worker Bob. Yellow signifies availability. In parenthesis, the weekend shift. Week rotation 4.}
\label{tab:Bob_avail}
\begin{tabularx}{\textwidth}{|X|l|l|l|l|l|l|l|X|}
\hline
%-------------------------------------------------------------------
\textbf{Week 1}& \colcell \textbf{Mon} & \colcell \textbf{Tue} & \colcell \textbf{Wed} & \colcell \textbf{Thu} & \colcell \textbf{Fri} & \colcell \textbf{Sat} & \colcell \textbf{Sun}
\\ \hline 
%%------------------------------------------------------------------- 
%\rowcolor{Gray} 
\colcell 08:00-10:00 (11:00-16:00) & \colcelltwo & \colcelltwo & \colcelltwo & \colcelltwo & \colcelltwo & & 
\\ \hline 
%%-------------------------------------------------------------------
%\rowcolor{Gray} 
\colcell 10:00-13:00 & \colcelltwo & \colcelltwo & \colcelltwo & \colcelltwo & \colcelltwo &   & 
\\ \hline 
%%-------------------------------------------------------------------
%\rowcolor{Gray} 
\colcell 13:00-16:00 & \colcelltwo & \colcelltwo & \colcelltwo & \colcelltwo & \colcelltwo & &
\\ \hline 
%%-------------------------------------------------------------------
%\rowcolor{Gray} 
\colcell 16:00-20:00 & & & \colcelltwo & & & &
\\ \hline 
%%-------------------------------------------------------------------
\end{tabularx}
\begin{tabularx}{\textwidth}{|X|l|l|l|l|l|l|l|X|}
\hline
%-------------------------------------------------------------------
\textbf{Week 2}& \colcell \textbf{Mon} & \colcell \textbf{Tue} & \colcell \textbf{Wed} & \colcell \textbf{Thu} & \colcell \textbf{Fri} & \colcell \textbf{Sat} & \colcell \textbf{Sun}
\\ \hline 
%%------------------------------------------------------------------- 
%\rowcolor{Gray} 
\colcell 08:00-10:00 (11:00-16:00) & \colcelltwo & \colcelltwo & \colcelltwo & \colcelltwo & \colcelltwo & & 
\\ \hline 
%%-------------------------------------------------------------------
%\rowcolor{Gray} 
\colcell 10:00-13:00 & \colcelltwo & \colcelltwo & \colcelltwo & \colcelltwo & \colcelltwo &   & 
\\ \hline 
%%-------------------------------------------------------------------
%\rowcolor{Gray} 
\colcell 13:00-16:00 & \colcelltwo & \colcelltwo & \colcelltwo & \colcelltwo & \colcelltwo & &
\\ \hline 
%%-------------------------------------------------------------------
%\rowcolor{Gray} 
\colcell 16:00-20:00 & & & \colcelltwo & & & &
\\ \hline 
%%-------------------------------------------------------------------
\end{tabularx}
\begin{tabularx}{\textwidth}{|X|l|l|l|l|l|l|l|X|}
\hline
%-------------------------------------------------------------------
\textbf{Week 3}& \colcell \textbf{Mon} & \colcell \textbf{Tue} & \colcell \textbf{Wed} & \colcell \textbf{Thu} & \colcell \textbf{Fri} & \colcell \textbf{Sat} & \colcell \textbf{Sun}
\\ \hline 
%%------------------------------------------------------------------- 
%\rowcolor{Gray} 
\colcell 08:00-10:00 (11:00-16:00) & \colcelltwo & \colcelltwo & \colcelltwo & \colcelltwo & \colcelltwo & & 
\\ \hline 
%%-------------------------------------------------------------------
%\rowcolor{Gray} 
\colcell 10:00-13:00 & \colcelltwo & \colcelltwo & \colcelltwo & \colcelltwo & \colcelltwo &   & 
\\ \hline 
%%-------------------------------------------------------------------
%\rowcolor{Gray} 
\colcell 13:00-16:00 & \colcelltwo & \colcelltwo & \colcelltwo & \colcelltwo & \colcelltwo & &
\\ \hline 
%%-------------------------------------------------------------------
%\rowcolor{Gray} 
\colcell 16:00-20:00 & & & \colcelltwo & & & &
\\ \hline 
%%-------------------------------------------------------------------
\end{tabularx}
\begin{tabularx}{\textwidth}{|X|l|l|l|l|l|l|l|X|}
\hline
%-------------------------------------------------------------------
\textbf{Week 4}& \colcell \textbf{Mon} & \colcell \textbf{Tue} & \colcell \textbf{Wed} & \colcell \textbf{Thu} & \colcell \textbf{Fri} & \colcell \textbf{Sat} & \colcell \textbf{Sun}
\\ \hline 
%%------------------------------------------------------------------- 
%\rowcolor{Gray} 
\colcell 08:00-10:00 (11:00-16:00) & \colcelltwo & \colcelltwo & \colcelltwo & \colcelltwo & \colcelltwo & \colcelltwo & \colcelltwo
\\ \hline 
%%-------------------------------------------------------------------
%\rowcolor{Gray} 
\colcell 10:00-13:00 & \colcelltwo & \colcelltwo & \colcelltwo & \colcelltwo & \colcelltwo &   & 
\\ \hline 
%%-------------------------------------------------------------------
%\rowcolor{Gray} 
\colcell 13:00-16:00 & \colcelltwo & \colcelltwo & \colcelltwo & \colcelltwo & \colcelltwo & &
\\ \hline 
%%-------------------------------------------------------------------
%\rowcolor{Gray} 
\colcell 16:00-20:00 & & & \colcelltwo & & \colcelltwo & &
\\ \hline 
%%-------------------------------------------------------------------
\end{tabularx}

\begin{tabularx}{\textwidth}{|X|l|l|l|l|l|l|l|X|}
\hline
%-------------------------------------------------------------------
\textbf{Week 5}& \colcell \textbf{Mon} & \colcell \textbf{Tue} & \colcell \textbf{Wed} & \colcell \textbf{Thu} & \colcell \textbf{Fri} & \colcell \textbf{Sat} & \colcell \textbf{Sun}
\\ \hline 
%%------------------------------------------------------------------- 
%\rowcolor{Gray} 
\colcell 08:00-10:00 (11:00-16:00) & \colcelltwo & \colcelltwo & \colcelltwo & & & & 
\\ \hline 
%%-------------------------------------------------------------------
%\rowcolor{Gray} 
\colcell 10:00-13:00 & \colcelltwo & \colcelltwo & \colcelltwo & & & & 
\\ \hline 
%%-------------------------------------------------------------------
%\rowcolor{Gray} 
\colcell 13:00-16:00 & \colcelltwo & \colcelltwo & \colcelltwo & & & &
\\ \hline 
%%-------------------------------------------------------------------
%\rowcolor{Gray} 
\colcell 16:00-20:00 & & & \colcelltwo & & & &
\\ \hline 
%%-------------------------------------------------------------------
\end{tabularx}
\end{table} 


\begin{table}[h]
\centering
\caption{Overall schedule for week 1.}
\label{tab:General_schedule}
\begin{tabularx}{\textwidth}{|X|l|l|l|l|X|}
\hline
%-------------------------------------------------------------------
\textbf{Schedule week 1} & & & & &  
\\ \hline 
%%------------------------------------------------------------------- 
\textbf{Monday}& \colcell \textbf{Exp} & \colcell \textbf{Info} & \colcell \textbf{FL} & \colcell \textbf{LoW - odd} & \colcell \textbf{LoW- even} 
\\ \hline 
%%------------------------------------------------------------------- 

 08:00 - 10:00 & \small W1, W2 & \small W1,W2 & \small W1 & \small W1 & \small W1
\\ \hline 
%%------------------------------------------------------------------- 
 10:00 - 13:00 & \small W1, W2, W3 & \small W1, W2, W3 & \small W1 & - & -
\\ \hline 
%%------------------------------------------------------------------- 
 13:00 - 16:00 & \small W1, W2, W3 & \small W1, W2, W3 & \small W1 & - & -
\\ \hline 
%%------------------------------------------------------------------- 
 16:00 - 20:00 & \small W1, W2, W3 & \small W1, W2, W3 & \small W1 & - & -
\\ \hline 
%%------------------------------------------------------------------- 
\textbf{Tuesday}& \colcell \textbf{Exp} & \colcell \textbf{Info} & \colcell \textbf{FL} & \colcell \textbf{LoW - odd } & \colcell \textbf{LoW- even} 
\\ \hline 
%%------------------------------------------------------------------- 
\textbf{Wednesday}& \colcell \textbf{Exp} & \colcell \textbf{Info} & \colcell \textbf{FL} & \colcell \textbf{LoW - odd } & \colcell \textbf{LoW- even} 
\\ \hline 
%%-------------------------------------------------------------------
\textbf{Thursday}& \colcell \textbf{Exp} & \colcell \textbf{Info} & \colcell \textbf{FL} & \colcell \textbf{LoW - odd } & \colcell \textbf{LoW- even} 
\\ \hline 
%%-------------------------------------------------------------------
\textbf{Friday}& \colcell \textbf{Exp} & \colcell \textbf{Info} & \colcell \textbf{FL} & \colcell \textbf{LoW - odd } & \colcell \textbf{LoW- even} 
\\ \hline 
%%-------------------------------------------------------------------
%Mon-Fri & 08:00-10:00 & 2 & 2 & 1
%\\ \hline 
%%%------------------------------------------------------------------- 
%Mon-Fri & 10:00-13:00 & 3 & 3 & 1
%\\ \hline 
%%%-------------------------------------------------------------------
%Mon-Fri & 13:00-16:00  & 3 & 3 & 1
%\\ \hline 
%%%-------------------------------------------------------------------
%Mon-Fri & 16:00-20:00 & 3 & 3 & -
%\\ \hline 
%%%-------------------------------------------------------------------
%Sat & 11:00-16:00  & 3 & 3 & -
%\\ \hline 
%%%------------------------------------------------------------------- 
%Sun & 11:00-16:00  & 3 & 3 & -
%\\ \hline 
\end{tabularx}
\end{table}

Personnel works whole weekends!

\subsection{Main objective: increase number of stand in personnel}

\subsection{Secondary objectives: repetitiveness of the schedule}


\iffalse
\subsection{Main objectives}
Main objective is to create a schedule with as many stand-ins as possible, so that all days have a maximum amount of stand-ins. Diversity during the week and repetivity of the schedule each week is also desired.

\subsection{Requirements on weekday activities}
During weekdays, there is a worker demand at the stations Exp, Info, Plocklista and Library on wheels. <TABLE OF DEMAND>. Plocklista is unique in the sense that its duration is longer than one shift. It is modeled as the three first shifts of a weekday.

In addition there is the Library on wheels, which has a different demand of workers depending on odd and even weeks. 
\subsection{Requirements on weekend activities}
There are three different weekend stations: Exp, Info and Hageby. Working a weekend also means working friday unless you are scheduled to work in Hageby. 
\subsection{Diversity in skill}
There are essentially two types of workers in a library. Librarians and assistants. The competence of the workers looks as follows: <TABLE OF DIFFERENT COMPETENCES>

A subset of the librarians can handle the Library on wheels and all are expected to take shifts at Hageby.

\subsection{Personnel availability and task assignment limitations}
The staff at the library have certain times of availability. Every day, a person can perform at maximum one task. Plocklistan is limited to once a week.

Personnel is only allowed to work one weekend out of five. The following week, the worker is free on the times requested by the worker. Also, some workers have specific 

\subsection{Inner work} 

The goal of this thesis is to distribute given tasks to the heterogeneous workforce at the library of Norrköping. Each task is either classified as an outer or an inner service where an outer service is when a librarian needs to interact with visitors. Inner services can in some rare cases require a predetermined person to be assigned to a specified time or day.

Demands and requests are to be fulfilled to the furthest extent possible. Weekends are included in the scheduling problem, which adds more constraints regarding the number of contiguous working days. However, the librarians are permitted a few exceptions from these laws regarding days of rest.

The main purpose of the thesis is to create a schedule robust enough to withstand absence, such that outer services always are assigned to a qualified and available worker. This is visualized as having a list of available stand-ins for each shift. 

There are a limited number of workers at the library and they make the resources that are to be distributed. Each individual has a set of \textit{skills} and \textit{competences}. Competences refer to the capability of being assigned the different outer services; Expedition, Norpan, Information desk, Library on wheels and Hageby as well as different inner services. The set of skills an individual can possess are described in Table \ref{int:1}. In total there are 39 workers available.

The outer services can be seen as assignments which requires available workers to be assigned to them. Each outer service is specified to a certain station, time and date. They also have a fix length and occur on a regular basis every ten weeks, which makes it possible to create a periodic schedule with a period of ten weeks. 


%Examensarbetet går ut på att lägga ett arbetsschema för personalen vid Norrköpings bibliotek. Problemet går i grund och botten ut på att fylla alla uppgifter på de stationer som tillhör bibliotekets utåtriktade verksamhet (så kallade yttre tjänst)  med personal av rätt kompetens samt samtidigt som personalen får tid över till övriga uppgifter (så kallad inre tjänst). Schemat som tas fram ska även uppfylla de regelverk och önskemål som finns kring personalens individuella scheman, till exempel de arbetstider som ingår i de olika tjänsterna. Då det även ingår helgarbete i personalens arbetsuppgifter ska lediga dagar fördelas enligt arbetsmiljölagen och de undantag från dessa lagar gällande veckovilan.

%Utöver detta ska schemat även medföra en robusthet så att störningar i den yttre tjänsten, i form av att personal blir sjuk eller uppbokad annonstädes, ska gå att avhjälpa med en reservlista. Denna reservlista består av de bibliotekarier och assistenter som inte har något pass tilldelat sig samt är tillgängliga under dagen. 

%Personalen på biblioteket är begränsad och utgör de resurser som finns att tillgå. Varje enskild personal har en uppsättning \textit{egenskaper} och \textit{kompetenser}. Kompetenser syftar på personalens förmåga att arbeta vid någon av de yttre stationerna; Expedition, Norpan, Informationsdisk, Bokbuss och Hageby samt några av de inre stationerna; inköp, katalogisering med mera. De egenskaper som identifierats hos en personal finns beskrivna i tabell \ref{int:1}. Totala arbetskraften består av 39 stycken arbetare på biblioteket.

%De yttre och inre uppgifterna kan ses som behov av personal som måste täckas av den personal som finns att tillgå. De olika yttre uppgifterna som behöver utföras inkluderar arbete vid olika stationer vid olika tidpunker och datum. Varje uppgift har en bestämd längd och återkommer regelbundet inom ett 10-veckorsinterval, vilket gör att ett rullande schema kan skapas med en period om tio veckor. 

\begin{table}[h]
\centering
\caption{Personnel}
\label{int:1}
\begin{tabular}{|l|l|}
%-------------------------------------------------------------------
\hline 
\textbf{Skills} & \textbf{Description} \\ \hline
%-------------------------------------------------------------------
Work degree & 0-100 \% 
\\ \hline 
%-------------------------------------------------------------------
Type of employment & Librarian/Assistant
\\ \hline 
%-------------------------------------------------------------------
Competence & Inner and outer services the worker is qualified for  
\\ \hline 
%-------------------------------------------------------------------
Weekly rest & Which days the worker has requested after working a weekend
\\ \hline 
%-------------------------------------------------------------------
Other requests & Does not work evenings etc.
\\ \hline 
%-------------------------------------------------------------------
\end{tabular}
\end{table} 

Furthermore, outer and a few inner services can be characterized by different properties, which are represented in Table \ref{int:2}. \\

\begin{table}[!h]
\caption{Outer and inner services}
\label{int:2}
\begin{tabular}{|l|l|}
%-------------------------------------------------------------------
\hline
\textbf{Outer service} & \textbf{Property} \\ \hline
%-------------------------------------------------------------------
 & Start time, end time, week and duration \
\\ \hline 
%-------------------------------------------------------------------
 & Station
\\ \hline 
%-------------------------------------------------------------------
 & Number of qualified librarians demanded
\\ \hline 
%-------------------------------------------------------------------
 & Number of qualified assistants demanded
\\ \hline 
%-------------------------------------------------------------------

\textbf{Inner service} & \textbf{Property} \\ \hline
%-------------------------------------------------------------------
 & Start time, end time, week and duration \
\\ \hline 
%-------------------------------------------------------------------
 & Type
\\ \hline 
%-------------------------------------------------------------------
 & Number of qualified librarians demanded
\\ \hline 
%-------------------------------------------------------------------
 & Number of qualified assistants demanded
\\ \hline 
%-------------------------------------------------------------------
\end{tabular}
\end{table}

In addition to the properties mentioned above, there are several requirements that have to be met. These can be divided into job, robust and other requirements and are listed in Table \ref{int:3} below. 

%Utöver de ovan nämnda resurserna och behoven, finns ett antal krav som ställs på hur schemat får utformas. Dessa kan delas upp i arbetsvillkor, robusthetskrav samt övriga krav och finns representerade i tabell \ref{int:3}.

\begin{table}[H]
\caption{Requirements}
\label{int:3}
\begin{tabular}{|c|l|}
%-------------------------------------------------------------------
\hline
\textbf{Job requirements} & \textbf{Description} \\ \hline
%-------------------------------------------------------------------
1 & A maximum of one outer service is to be distributed to each person and day.
\\ \hline 
%-------------------------------------------------------------------
2 & Remaining work time is individually distributed on assignments such as reshelving books.
\\ \hline
%-------------------------------------------------------------------
3 & Weekend work are to be evenly distributed between the workers available on weekends. 
\\ \hline 
%-------------------------------------------------------------------
4 & Working a weekend includes work on saturday and sunday the same week.
\\ \hline 
%-------------------------------------------------------------------
5 & One evening shift on a weekday per person each week except when weekend work is required.
\\ \hline 
%-------------------------------------------------------------------
6 & Every ten weeks the schedule is to be repeated.
\\ \hline 
%-------------------------------------------------------------------
7 & It is recommended for each week to be as similar as possible.
\\ \hline 
%-------------------------------------------------------------------

\textbf{Robust requirements} & \textbf{Description} \\ \hline
%-------------------------------------------------------------------
1 & Each outer service require at least one stand-in.
\\ \hline 
%-------------------------------------------------------------------
2 & The stand-ins have to be qualified for the tasks they are stand-in for.
\\ \hline 
%-------------------------------------------------------------------
3 & Focus is to maximize the lowest number of stand-ins of any task.
\\ \hline 
%-------------------------------------------------------------------

\textbf{Other requirements} & \textbf{Description} \\ \hline
%-------------------------------------------------------------------
1 & Department and general meetings are to be held once per five weeks.
\\ \hline 
%-------------------------------------------------------------------
\end{tabular}
\end{table}
\medskip

There are also additional requirements of the resulting schedule made by the workers at the library. Two examples would be that a handful staff members are unable to work weekends as well as some personnel are unable to work in the evenings.

\section{Problem categorization}
The problem can be formulated as a personnel tasks scheduling problem for a heterogeneous workforce since the main objective is to distribute tasks to workers during their available times. The workforce is heterogeneous as certain tasks can only be performed by librarians or is restricted to a certain subset of the personnel. Another aspect of the problem is the cyclic nature of the personnel schedules, which gives a degree of freedom in availability of personnel.

\fi


\section{Method}

\section{Topics Covered}

"The thesis is divided into..."



\iffalse
\begin{table}[!h]
\centering
\caption{Week 4 for sample worker Bob. Yellow signifies availability. In parenthesis, the weekend shift.}
\label{tab:Lib_avail2}
\begin{tabularx}{\textwidth}{|X|l|l|l|l|l|l|l|X|}
\hline
%-------------------------------------------------------------------
& \colcell \textbf{Mon} & \colcell \textbf{Tue} & \colcell \textbf{Wed} & \colcell \textbf{Thu} & \colcell \textbf{Fri} & \colcell \textbf{Sat} & \colcell \textbf{Sun}
\\ \hline 
%%------------------------------------------------------------------- 
%\rowcolor{Gray} 
\colcell 08:00-10:00 (11:00-16:00) & \colcelltwo & \colcelltwo & \colcelltwo & \colcelltwo & \colcelltwo & \colcelltwo & \colcelltwo
\\ \hline 
%%-------------------------------------------------------------------
%\rowcolor{Gray} 
\colcell 10:00-13:00 & \colcelltwo & \colcelltwo & \colcelltwo & \colcelltwo & \colcelltwo &   & 
\\ \hline 
%%-------------------------------------------------------------------
%\rowcolor{Gray} 
\colcell 13:00-16:00 & \colcelltwo & \colcelltwo & \colcelltwo & \colcelltwo & \colcelltwo & &
\\ \hline 
%%-------------------------------------------------------------------
%\rowcolor{Gray} 
\colcell 16:00-20:00 & & & \colcelltwo & & \colcelltwo & &
\\ \hline 
%%-------------------------------------------------------------------
\end{tabularx}
\end{table} 

\begin{table}[!h]
\centering
\caption{Week 5 for sample worker Bob. Yellow signifies availability. In parenthesis, the weekend shift.}
\label{tab:Lib_avail3}
\begin{tabularx}{\textwidth}{|X|l|l|l|l|l|l|l|X|}
\hline
%-------------------------------------------------------------------
& \colcell \textbf{Mon} & \colcell \textbf{Tue} & \colcell \textbf{Wed} & \colcell \textbf{Thu} & \colcell \textbf{Fri} & \colcell \textbf{Sat} & \colcell \textbf{Sun}
\\ \hline 
%%------------------------------------------------------------------- 
%\rowcolor{Gray} 
\colcell 08:00-10:00 (11:00-16:00) & \colcelltwo & \colcelltwo & \colcelltwo & & & & 
\\ \hline 
%%-------------------------------------------------------------------
%\rowcolor{Gray} 
\colcell 10:00-13:00 & \colcelltwo & \colcelltwo & \colcelltwo & & & & 
\\ \hline 
%%-------------------------------------------------------------------
%\rowcolor{Gray} 
\colcell 13:00-16:00 & \colcelltwo & \colcelltwo & \colcelltwo & & & &
\\ \hline 
%%-------------------------------------------------------------------
%\rowcolor{Gray} 
\colcell 16:00-20:00 & & & \colcelltwo & & & &
\\ \hline 
%%-------------------------------------------------------------------
\end{tabularx}
\end{table} 
\fi