% Review of previous work in the field
"An Annotated Bibliography of Personnel Scheduling
and Rostering" Ernst et al. 2004

\section{Tour scheduling problem (TSP)}
In the paper review of different scheduling problem written by Ernst et al. (2004) one can see a number of references to tour scheduling problems. However, there are significantly lower number of references regarding heterogeneous workforce.

Papers of interest:
"Task assignment and tour scheduling": Loucks and Jacobs, 1991

"Scheduling Restaurant Workers to Minimize Labor Cost and Meet Service Standards" Choi, Hwang and Park, 2009

"An integer linear programming-based heuristic for scheduling heterogeneous, part-time service employees" Heterogenous work force, tour scheduling. Using two objective functions Hojati and Patil, 2010


\section{Personnel task scheduling problem (PTSP)}
Most likely our problem. Definition is "in which a set of tasks with fixed start and finish times have to be allocated to a heterogeneous workforce". The objective of these problems is to minimise the overall cost of personnel required to perform all tasks.

Papers of interest:
"The Personnel Task Scheduling Problem", Mohan Krishnamoorthy, Andreas T. Ernst (2001) - probably the most fundamental article

"Task assignement for maintenance personnel": Roberts and Escudero, 1983a, 1983b

"A stochastic programming model for scheduling maintenance personnel" Duffuaa and Al-Sultan, 1999

\section{Shift minimisation personnel task scheduling problem (SMPTSP)}
Difference: "The only cost incurred is due to the number of personnel (shifts) that are used."

Papers of interest:
"Algorithms for large scale Shift Minimisation Personnel Task Scheduling Problems" Krishnamoorthy, Ernst, Baatar (2011)

"The shift minimisation personnel task scheduling problem: A new hybrid approach and computational insights" Smet, Wauters, Mihaylov, Berghe (2014)

"Fast local search and guided local search and their application to British Telecom's workforce scheduling problem" Tsang and Voudouris, 1997 - also with travelling costs, investigates two methods.

"A Triplet-Based Exact Method for the Shift Minimisation Personnel Task Scheduling Problem" Baatar et al., 2015

\section{Fixed/flexible job scheduling problem (FJSP)}
Identical skill of the workers/machines and indentical skill requirements of the operations to execute.

Problem defined in: "Algorithms for large scale Shift Minimisation Personnel Task Scheduling Problems" M. Krishnamoorthy
http://www.sciencedirect.com/science/article/pii/S0377221711010435

Problem: "A metaheuristic for the fixed job scheduling problem under spread time constraints" André Rossi, http://www.sciencedirect.com/science/article/pii/S0305054809002251 (Fixed job)

Cemal Özgüven
http://www.sciencedirect.com/science/article/pii/S0307904X11004173 (Flexible job)
Processors with a ready time, due date etc. (??)

\section{Work load allocation and worker satisfaction}
Trötthet och uttråkad. Något vi borde ta med i litteraturen enligt Torbjörn, fast inte leta källor på det.

Source: "Employee positioning and workload allocation", Eiselt, Marianov, 2006
"Scheduling part-time and mixed-skilled workers to maximize employee satisfaction" Mohammad Akbari 2012

"Scheduling part-time personnel with availability restrictions and preferences to maximize employee satisfaction" Srimathy Mohan 2008



