%A thorough description of the mathematical model in words and in math. More significant and general constraints in equations and less significant in words.
In this chapter the mathematical model implemented to solve this problem will be presented. Prior to the objective function and constraints, the most significant variables and sets will be stated to give the reader some basic knowledge of the implemented model. In Section \ref{constraints} the essential constraints will be presented and explained. A complete model with all definitions and the full set of constraints can be found in Attachment \ref{definitions}. %To get a full view of the mathematical problem the reader  A list of the defined parameters can be found in Appendix \ref{definitions}. 
\section{Set and variable definitions} \label{variables}
To solve the problem many sets and variables had to be declared, as there are many unique and individual requirements that have to be met. An example is that some personnel want a day free from outer tasks to be able to focus on other assignments or attend meetings. Another example is that some have two alternating schedules for odd and even weeks. These specific cases have to be modeled and result in a variety of set and variable definitions. Hence, only the most important ones are listed below. A complete list of the definitions can be found in Appendix \ref{definitions}. \\
\itab{$I$} \tab{Set of workers}\\
\itab{$I_{lib}$} \tab{Set of librarians ($I_{lib} \subseteq I$)} \\
\itab{$I_{ass}$}	 \tab{Set of assistants ($I_{ass} \subseteq I$)}	\\
\itab{$W$}                 \tab{Set of weeks}                                               \\
\itab{$D$}                 \tab{Set of days in a week}                                      \\
\itab{$S_d$}           \tab{Set of shifts day \textit{d}}                                        \\
\itab{$J_d$}            \tab{Set of task types day \textit{d}}                                    \\

In order to further define the problem we introduce the following variables. Let,
\begin{align}
    x_{iwdsj}&=
    \begin{cases}
      1, & \text{if worker \textit{i} is assigned in week \textit{w}, day \textit{d}, shift \textit{s} to a task \textit{j}}\\
      0, & \text{otherwise}
    \end{cases}
    \\
    H_{iwh}&=
    \begin{cases}
      1, & \text{if worker \textit{i} works weekend \textit{h} in week \textit{w}}\\
      0, & \text{otherwise}
    \end{cases}
	\\
	r_{iw}&=
	\begin{cases}
		1, & \text{if worker \textit{i} has its scheduled rotated \textit{w-1} steps}\\
		0, & \text{otherwise}
	\end{cases}
	\\
	l_{iwd}&=
	\begin{cases}
	  1, & \text{if librarian \textit{i} is a stand-in week \textit{w}, day \textit{d}} \\
	  0, & \text{otherwise}
	\end{cases}
	\\
	a_{iwd}&=
	\begin{cases}
 		1, & \text{if assistant \textit{i} is a stand-in week \textit{w}, day \textit{d}} \\
 		0, & \text{otherwise}
	\end{cases}
	\\
	y_{iwds}&=
	\begin{cases}
 		1, & \text{if worker \textit{i} works week \textit{w}, day \textit{d}, shift \textit{s} regardless of task type} \\
 		0, & \text{otherwise}
	\end{cases}
	\\
	b_{iw}&=
	\begin{cases}
 		1, & \text{if worker \textit{i} works at HB week \textit{w}} \\
 		0, & \text{otherwise}
	\end{cases}
	\\
	f_{iw}&=
	\begin{cases}
 		1, & \text{if worker \textit{i} is assigned to work friday evening week \textit{w}} \\
 		0, & \text{otherwise}
	\end{cases}	
	\\
	l_{min}&= \text{lowest number of stand-in librarians found (integer)} \\
	a_{min}&= \text{lowest number of stand-in assistants found (integer)} \\
	s_{iwds}&= \text{number of shifts \textit{s} that differ for a worker \textit{i} between week \textit{w} and \textit{w+5}, day \textit{d} (integer)}
\end{align}

%All decision variables are not presented. Instead, it is up to the reader to see the complete list of variables in Appendix \ref{definitions}.

Based on the variables defined above it has been possible to solve our scheduling problem. \textit{$l_{min}$} and \textit{$a_{min}$} are the variables of most significance as they represent the number of stand-ins found after a run. 

\section{Objective function}
Due to multiple objective functions it has been necessary to weigh them against each other using parameters. These are shown in Equation \ref{objfcn} below as \textit{N1l}, \textit{N1a} and \textit{N2}.

\begin{equation} \label{objfcn}
\begin{split}
max &N1l*l_{min}\\
 + &N1a*a_{min} \\
 - &N2*\sum_{i \in I}\sum_{w = 1}^{5}\sum_{d = 1}^{5}\sum_{s = 1}^{3} s_{iwds}
\end{split}
\end{equation}

If $\textit{N1l} < \textit{N1a}$ then the function were to prioritize assistants as stand-ins rather than librarians. Librarians are, however, more desired as stand-ins, due to their ability to perform all task types. Therefore, it is desired to set $\textit{N1l} \geq \textit{N1a}$.

\section{Constraints} \label{constraints}
To model this problem it has been of relevance to divide many of the constraints into weekend- and weekday constraints. Several help constraints have also been added to avoid multiplication of two variables, making the problem non-linear.

The most crucial constraint is to ensure that the demand of workers is met each day. This can be modeled as,
\begin{equation}
\sum_{i \in I} x_{iwdsj} = demand_{wdsj}, \; \forall w\in W,d\in D,s\in S,j\in J_d
\end{equation}

where $demand_{wdsj}$ is the number of workers required week \textit{w}, day \textit{d}, shift \textit{s} for a task \textit{j}.

To ensure that no worker is assigned more than one task the following constraint is implemented,
\begin{equation} \label{constr:one_task_constraint}
\sum_{s\in S}\sum_{j\in J_d} x_{iwdsj} \leq 1, \; \forall i\in I, w \in W, d\in D
\end{equation}
However, if we allow a person to have two shifts at the library on wheels in a day Equation \ref{constr:one_task_constraint} has to be slightly modified, which is shown in the complete model in Appendix \ref{definitions}.

A preference is to only allow one PL per week and a maximum of three PL per ten weeks. These are easily modeled with the following constraints,
\begin{equation} \label{constr:one_PL}
\sum_{s \in S_d}\sum_{d \in D} x_{iwdsPL} \leq 1, \; \forall i\in I, w \in W
\end{equation}
\begin{equation} \label{constr:three_PL}
\sum_{w \in W}\sum_{s \in S_d}\sum_{d \in D} x_{iwdsPL} \leq 3, \; \forall i\in I
\end{equation}

To further model the problem weekends and week rotations have to be considered. The following three constraints are the most basic constraints regarding weekends and week rotations:
\begin{equation} \label{constr:one_rot}
\sum_{w \in W} r_{iw} = 1, \; \forall i\in I
\end{equation}
\begin{equation} \label{constr:max_one_weekend}
\sum_{w \in W} H_{iwh} \leq 1, \; \forall i\in I, h = 1,2
\end{equation}
\begin{equation} \label{constr:rot_weekend}
r_{iw} \geq H_{iw1}, \; \forall i\in I, w \in W
%	\begin{cases}
% 		1, & \text{if $H_{iwh=1} = 1$} \\
% 		0/1, & \text{otherwise}
%	\end{cases}
%	\; \forall i\in I, w \in W \;
\end{equation}

Equation \ref{constr:one_rot} provides all workers a rotation of their schedule regardless if they are working weekends or not. Equation \ref{constr:max_one_weekend} allows a worker to work a maximum of two weekends $(h = 1, h = 2)$ per ten weeks. Lastly, Equation \ref{constr:rot_weekend} in combination with Equation \ref{constr:one_rot} ensures that the schedule rotation is aligned with the first weekend, if the worker is due for weekend work. In case the worker is not due for weekend work, the rotation of the schedule is free.

A worker is supposed to work weekends with a five-week interval. However, in case there are enough workers to satisfy the demand on weekends it may only be necessary to work one weekend per ten weeks for some. To avoid problems with the rotation when only the second weekend is assigned to a worker, the following equation was implemented:
\begin{equation} \label{constr:five_week_interval}
r_{i(mod_{10}(w+4)+1)} \geq H_{iw2}, \; \forall i\in I, w \in W
\end{equation}
Worth noting is that if a worker is assigned both weekends then Equation \ref{constr:five_week_interval} says the same as Equation \ref{constr:rot_weekend}.

A worker is supposed to work Saturday and Sunday with the same task when working a weekend. This can be modeled with the following constraints:
\begin{equation} \label{constr:consecutive_days}
\sum_{j \in J_d} x_{iw61j} + \sum_{j \in J_d} x_{iw71j} = 2*\sum_{h = 1}^{2} H_{iwh}, \; \forall i\in I, w \in W
\end{equation}
\begin{equation} \label{constr:same_tasks}
x_{iw61j} = x_{iw71j}, \; \forall i\in I, w \in W, j \in J_d
\end{equation}

Friday evening is also included in extension to working Saturday and Sunday, unless the worker is assigned to HB. It does, however, not require to be the same task as during the weekend, thus Fridays not included in Equation \ref{constr:same_tasks}. Equation \ref{constr:friday_added} below adds Fridays to the weekend:
\begin{equation} \label{constr:friday_added}
\sum_{j \in J \backslash 'LoW'}x_{iw54j} = f_{iw}, \; \forall i \in I, w \in W
\end{equation}
The variable $f_{iw}$ is, as mentioned previously, one if and only if a worker is working weekend as well as not being assigned to HB. The help constraints implemented to avoid multiplication of the two variables $b_{iw}$ and $H_{iwh}$ are left out of this section and can instead be seen in Appendix \ref{definitions}.

It is of interest to implement a constraint preventing workers to be assigned HB more than once every ten weeks to avoid unfairness. The constraint can be modeled as:
\begin{equation} \label{constr:max_one_hb}
\sum_{w \in W}\sum_{d = 6}^{7}x_{iwd1HB} \leq 2, \; \forall i \in I
\end{equation}
The combination of Equations \ref{constr:max_one_hb} and \ref{constr:same_tasks} ensure that once a worker is assigned HB it will be for two consecutive weekend days. Why this is preferable is described in Section \ref{section:library_tasks}.

To calculate the variable $l_{min}$ and $a_{min}$ used in the objective function (\ref{objfcn}) the following equations were implemented:
\begin{equation} \label{constr:l_min}
l_{min} \leq \sum_{i \in I_{lib}} l_{iwd}, \; \forall w \in W, d=1, \ldots, 5
\end{equation}
\begin{equation} \label{constr:a_min}
a_{min} \leq \sum_{i \in I_{ass}} a_{iwd}, \; \forall w \in W, d=1, \ldots, 5
\end{equation}

%a_{iwd} 0/1, so sum of the I_ass gives all the stand-ins. a_min <= #stand_ins that still shall be maximized in the obj function.

%subject to find_lowest_stand_in_amount_no_weekends_no_evenings_lib{w in W, d in 1..5}: #RHS: number of qualified workers at work that is available & not assigned to any task.
%
%	stand_in_lib_min <= sum{i in I_lib} stand_in_lib[i,w,d]; 