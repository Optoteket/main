% Review of previous work in the field

\section{Tour scheduling problem (TSP)}
In the paper review of different scheduling problem written by Ernst et al. (2004) one can see a number of references to tour scheduling problems. However, there are significantly lower number of references regarding heterogeneous workforce

\section{Personnel task scheduling problem (PTSP)}
Most likely our problem. Definition is "in which a set of tasks with fixed start and finish times have to be allocated to a heterogeneous workforce". The objective of these problems is to minimise the overall cost of personnel required to perform all tasks.
Papers of interest:
The Personnel Task Scheduling Problem, Mohan Krishnamoorthy, Andreas T. Ernst (2001)

\section{Shift minimisation personnel task scheduling problem (SMPTSP)}
Difference: "The only cost incurred is due to the number of personnel (shifts) that are used."
Papers of interest:
"Algorithms for large scale Shift Minimisation Personnel Task Scheduling Problems" Krishnamoorthy, Ernst, Baatar (2011)
"The shift minimisation personnel task scheduling problem: A new hybrid approach and computational insights" Smet, Wauters, Mihaylov, Berghe (2014)

\section{Fixed/flexible job scheduling problem (FJSP)}
Identical skill of the workers/machines and indentical skill requirements of the operations to execute.
Problem defined in: http://www.sciencedirect.com/science/article/pii/S0377221711010435
Problem: http://www.sciencedirect.com/science/article/pii/S0305054809002251 (Fixed job)
http://www.sciencedirect.com/science/article/pii/S0307904X11004173 (Flexible job)
Processors with a ready time, due date etc. (??)

\section{work load allocation}
Tr�tthet och uttr�kad. N�got vi borde ta med i litteraturen enligt Torbj�rn, fast inte leta k�llor p� det.
Source: "Employee positioning and workload allocation", Eiselt, Marianov, 2006

