% Review of previous work in the field
Hej
The scheduling problem has been studied since the 1950's as a mathematical optimization problem and concerns the ability of creating a satisfactory schedule, considering a number of constraints and objectives. According to Ernst et al. the complexity of the rostering problem has not in itself become more advanced since then. However, the mathematical models used to solve these scheduling problems have become more realistic and refined. This together with more powerful computational methods, makes it possible today to solve scheduling problems in a more satisfactory way \cite{Ernst_2004}.

In the paper \cite{Ernst_2004} the scheduling problem is classified into different subcategories. A few relevant areas for our work include task based demand scheduling, days off scheduling, shift scheduling, tour scheduling and task assignment. Task based demand scheduling involves the process of finding the demand from a list of tasks which need to be performed. This problem resembles our problem since also in our problem we require a number of tasks to be performed. However, in our case the pool of workers is fixed and thus the objective differs. 
	
Days off scheduling involves scheduling staff and assigning a day off. This problem is often found together with shift scheduling which involves choosing the most suitable shifts for a workforce. The combination of the two is called tour scheduling and will be discussed more later in this report. The big difference between our problem and tour scheduling is that in our problem we are not allowed to choose the free days of the workers, only in what week to assign them.

The problem which is most similar to our problem is, however, task assignment. This problem and different variations of it will be discussed in section \ref{PTSP}.

\iffalse
 (The Use of Mathematical Models in Plant Maintenance Decision Making, 1967) In surveys such as "An Annotated Bibliography of Personnel Scheduling
and Rostering" Ernst et al. 2004,  covers papers from 1954. The bibliography covers the most important contributions to the area of scheduling up to the year 2004 and divides the area into different subcategories. Below, we have identified the most relevant areas of scheduling in relation to the problem at hand.
\fi

\section{Tour scheduling problem with a heterogenous work force}\label{TSP}
The tour scheduling problem (TSP) involves creating work shifts with days off for a work force. 
In the paper review of different scheduling problem written by Ernst et al. (2004) one can see a number of references to tour scheduling problems. However, there are significantly lower number of references regarding heterogeneous workforce which is of interest to us.

Papers of interest:
"Task assignment and tour scheduling": Loucks and Jacobs, 1991
----- Inhomogeneous work force

"Scheduling Restaurant Workers to Minimize Labor Cost and Meet Service Standards" Choi, Hwang and Park, 2009

"An integer linear programming-based heuristic for scheduling heterogeneous, part-time service employees" Heterogenous work force, tour scheduling. Using two objective functions Hojati and Patil, 2010


\section{Personnel task scheduling problem} \label{PTSP}
In many practical instances production managers will face the personnel task scheduling problem (PTSP) while scheduling plant operations. It occurs when the rosterer or shift supervisor need to allocate tasks with specified start and end times to available personnel who have the required qualifications. Furthermore, it also occurs in situations where tasks of fixed times have been assigned to machines. Decisions will then have to be made regarding the amount of maintenance workers needed and which machine the workers are assigned to.

There are several variants to the Personnel Task Scheduling Problem. The common attributes are to assign a set of tasks with fixed start and end time to staff members that possesses certain skills, allowing them to perform a subset of the available tasks. The start and end time of their shifts are also predetermined for each day.

One variant, which also is the most simple, is called the \textit{Feasibility Problem} where the aim is to just find a feasible solution. This requires that each task is allocated to a qualified and available worker. It is also required that a worker can not be assigned more than one task simultaneously as well as tasks can not be pre-empted, meaning that each task has to be completed by one and the same worker.

In Table \ref{PTSP} one can see attributes of PTSP variants. The nomenclature of the attributes T, S, Q, O refer to the \textit{Task type}, \textit{Shift type}, \textit{Qualifications} and \textit{Objective function} respectively. 
\begin{table}[!h]
\caption{PTSP variants}
\label{PTSP}
\begin{tabular}{|c|c|l|}
%-------------------------------------------------------------------
\hline
\textbf{Attribute} & \textbf{Type} & \textbf{Explanation} \\ \hline
%-------------------------------------------------------------------
T & F & Fixed contiguous tasks \\
& V & Variable task durations \\
& S & Split (non-contiguous) tasks \\
& C & Changeover times between consecutive tasks \\
\hline 
%-------------------------------------------------------------------
S & F & Fixed, given shift lengths \\
& I & Identical shifts which are effectively of infinite duration \\
& D & Maximum duration without given start or end times \\
& U & Unlimited number of shifts of each type available \\
\hline 
%-------------------------------------------------------------------
Q & I & Identical qualification for all staff (homogeneous workforce) \\
& H & Heterogeneous workforce \\
\hline 
%-------------------------------------------------------------------
O & F & No objective, just find a feasible schedule \\
& A & Minimise assignment cost \\
& T & Worktime costs including overtime \\
& W & Minimise number of workers \\
& U & Minimise unallocated tasks \\
\hline  

%-------------------------------------------------------------------
\end{tabular}
\end{table}

Our problem would be most related to the PTSP[F;F;H;F]. The difference is the objective function as we are looking to maximize the number of qualified stand-ins each day as well as maximize employee satisfaction by meeting their recommendations. 

Most likely our problem. Definition is "in which a set of tasks with fixed start and finish times have to be allocated to a heterogeneous workforce". The objective of these problems is to minimise the overall cost of personnel required to perform all tasks.

Papers of interest:
"The Personnel Task Scheduling Problem", Mohan Krishnamoorthy, Andreas T. Ernst (2001) - probably the most fundamental article

"Task assignement for maintenance personnel": Roberts and Escudero, 1983a, 1983b

"A stochastic programming model for scheduling maintenance personnel" Duffuaa and Al-Sultan, 1999

\section{Shift minimisation personnel task scheduling problem (SMPTSP)}\label{SMTSP}
Difference: "The only cost incurred is due to the number of personnel (shifts) that are used."

Papers of interest:
"Algorithms for large scale Shift Minimisation Personnel Task Scheduling Problems" Krishnamoorthy, Ernst, Baatar (2011)

"The shift minimisation personnel task scheduling problem: A new hybrid approach and computational insights" Smet, Wauters, Mihaylov, Berghe (2014)

"Fast local search and guided local search and their application to British Telecom's workforce scheduling problem" Tsang and Voudouris, 1997 - also with travelling costs, investigates two methods.

"A Triplet-Based Exact Method for the Shift Minimisation Personnel Task Scheduling Problem" Baatar et al., 2015

\section{Fixed/flexible job scheduling problem (FJSP)}\label{FJSP}
Identical skill of the workers/machines and indentical skill requirements of the operations to execute.

Problem defined in: "Algorithms for large scale Shift Minimisation Personnel Task Scheduling Problems" M. Krishnamoorthy
http://www.sciencedirect.com/science/article/pii/S0377221711010435

Problem: "A metaheuristic for the fixed job scheduling problem under spread time constraints" André Rossi, http://www.sciencedirect.com/science/article/pii/S0305054809002251 (Fixed job)

Cemal Özgüven
http://www.sciencedirect.com/science/article/pii/S0307904X11004173 (Flexible job)
Processors with a ready time, due date etc. (??)

\section{Work load allocation and worker satisfaction} \label{WLA}
Trötthet och uttråkad. Något vi borde ta med i litteraturen enligt Torbjörn, fast inte leta källor på det.

Source: "Employee positioning and workload allocation", Eiselt, Marianov, 2006
"Scheduling part-time and mixed-skilled workers to maximize employee satisfaction" Mohammad Akbari 2012

"Scheduling part-time personnel with availability restrictions and preferences to maximize employee satisfaction" Srimathy Mohan 2008



