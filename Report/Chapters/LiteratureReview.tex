% Review of previous work in the field
Emelie

The scheduling problem has been studied since the 1950's as a mathematical optimization problem and involves creating a feasible and satisfactory schedule for either workers or machines performing tasks. According to Ernst et al. the complexity of the scheduling problem has not in itself become more advanced with time. However, the mathematical models used to solve the scheduling problems have become more realistic and refined. This together with more powerful computational methods, makes it possible today to solve scheduling problems in a more satisfactory way, taking into account softer values such as worker satisfaction and worker fatigue \cite{Ernst_2004}.

In the paper \cite{Ernst_2004} the scheduling problem is classified into different subcategories. A few relevant areas for our work include task based demand scheduling, days off scheduling, shift scheduling, tour scheduling and task assignment. 

Task based demand scheduling involves the process of distributing a fixed number of tasks which need to be performed over a workforce. The workforce can either be fixed, as in our case, or subject to minimization in the objective function.
	
Days off scheduling involves scheduling staff and assigning a days off as required by work time regulations. This problem is often found together with shift scheduling which involves choosing the most suitable shifts for a workforce. The combination of the two is called tour scheduling and will be discussed later in this report. The big difference between our problem and tour scheduling is that in our problem we are not allowed to choose the free days of the workers, only in what week to assign them, and there is no shift work.

The problem which is most similar to our problem is, however, task assignment. This problem and different variations of it will be discussed in section \ref{PTSP}.



\section{Tour scheduling problem with a heterogenous work force}\label{TSP}
Emelie

The tour scheduling problem (TSP) involves creating work shifts with days off for a work force. According to Loucks and Jacobs, the vast majority of all tour scheduling problems up to 1991 were with a homogeneous workforce, that is under the assumption that any worker could perform any assigned task \cite{loucks_1991}. The authors discuss a tour scheduling problem where the objective is to construct weekly schedules for each worker which also show the specific task assignments. The problem is studied in the context of fast food restaurants, where certain personnel is qualified only for certain stations in the restaurant. In such industries, the demand of staff differs between different weekdays and different times of the day. Two things differ between workers, their availability for work and their qualification. Furthermore, the workers working times are not fixed, but are composed of a number of consecutive tasks assigned during a block of time during a day.

The representative problem studied in the article involves creating a one-week schedule for 40 workers in a fast food restaurant, available for eight different tasks with a seven-day, 128-hour workweek. Several synthetic problems are studied in the article, all, however with minimum shift lenght three hours, maximum shift length eight hours and five maximum number of work days.

A similar problem to the one descibed by Loucks and Jacobs is studied by \cite{choi_hwang_park_2009}. The article focuses on a particular fast food restaurant in Korea, which is made a representative of fast food chains in general. One big difference between this study and the previous one is the identification of part-time and full-time workers, between which the ratio of scheduled personnel should always be 6:4. Although a tour is scheduled also in this problem, the properties of the tour is different from Louck and Jacobs as the shifts are already divided into periods 8:00-12:00, 12:00-14:00, 14:00-21:00 and 21:00-23:00 while the latter schedules on an hourly basis. Further, a tour is defined as working five consecutive days, as opposed to the previous article. The task assignment dimension is lacking in this article, making it less similar to the problem described in this report.

In both articles the main objective is to minimize overstaffing and understaffing, which will both have severe consequences for the fast food chain. This is not relevant to our problem as we have a fixed work force. In the example studied by Loucks and Jacobs there is also a goal to meet staff demand on total working hours. This is modeled as a secondary goal and is similar to our goal of creating even and fair schedules.


The greatest difference between the problem studied by Loucks and Jacobs and our problem is the composition of shifts. Both problems have heterogeneous worker qualification and availability and both deal with task assignment for schedules with a fluctuating worker demand. Since our problem only concerns librarians and assistants, there are fewer skill groups. Compared to the problem studied by \cite{choi_hwang_park_2009}, there is more similarity in the shift design as the library also has four different shifts. However, our problem is a task asssignment problem and does not affect working times. 

In some cases, a problem can be a combined tour scheduling and task assignment problem or can be divided into these two solution stages, as is the case in \cite{keylist}. "An integer linear programming-based heuristic for scheduling heterogeneous, part-time service employees" , 2011


\section{Personnel task scheduling problem} \label{PTSP}
Claes

In many practical instances production managers will face the personnel task scheduling problem (PTSP) while scheduling plant operations. It occurs when the rosterer or shift supervisor need to allocate tasks with specified start and end times to available personnel who have the required qualifications. Furthermore, it also occurs in situations where tasks of fixed times have been assigned to machines. Decisions will then have to be made regarding the amount of maintenance workers needed and which machine the workers are assigned to.

There are several variants to the Personnel Task Scheduling Problem. The common attributes are to assign a set of tasks with fixed start and end time to staff members that possesses certain skills, allowing them to perform a subset of the available tasks. The start and end time of their shifts are also predetermined for each day.

One variant, which also is the most simple, is called the \textit{Feasibility Problem} where the aim is to just find a feasible solution. This requires that each task is allocated to a qualified and available worker. It is also required that a worker can not be assigned more than one task simultaneously as well as tasks can not be pre-empted, meaning that each task has to be completed by one and the same worker.

In Table \ref{PTSP} one can see attributes of PTSP variants. The nomenclature of the attributes T, S, Q, O refer to the \textit{Task type}, \textit{Shift type}, \textit{Qualifications} and \textit{Objective function} respectively. 
\begin{table}[H]
\caption{PTSP variants}
\label{PTSP}
\begin{tabular}{|c|c|l|}
%-------------------------------------------------------------------
\hline
\textbf{Attribute} & \textbf{Type} & \textbf{Explanation} \\ \hline
%-------------------------------------------------------------------
T & F & Fixed contiguous tasks \\
& V & Variable task durations \\
& S & Split (non-contiguous) tasks \\
& C & Changeover times between consecutive tasks \\
\hline 
%-------------------------------------------------------------------
S & F & Fixed, given shift lengths \\
& I & Identical shifts which are effectively of infinite duration \\
& D & Maximum duration without given start or end times \\
& U & Unlimited number of shifts of each type available \\
\hline 
%-------------------------------------------------------------------
Q & I & Identical qualification for all staff (homogeneous workforce) \\
& H & Heterogeneous workforce \\
\hline 
%-------------------------------------------------------------------
O & F & No objective, just find a feasible schedule \\
& A & Minimise assignment cost \\
& T & Worktime costs including overtime \\
& W & Minimise number of workers \\
& U & Minimise unallocated tasks \\
\hline  

%-------------------------------------------------------------------
\end{tabular}
\end{table}

With this definition of PTSP attributes many of the most basic problems and a few more complex ones can be described. It is, however, not possible to describe all of the numerous combinations using these nomenclatures.

By combining attributes it is possible to obtain more complex variants of the PTSP. An example would be the PTSP[F;F;H;A-T-W] where multiple objectives are used. This problem has fixed contiguous tasks, fixed shift lengths, heterogeneous workforce and three objective functions; assigment costs, work time with overtime included and requirements to minimize the number of workers respectively. This objective function is then a linear combination with different parameters used to prioritize them against each other.

Our problem would be most related to the PTSP[F;F;H;F]. The difference is the objective function, since we are looking to maximize the number of qualified stand-ins each day as well as maximize employee satisfaction by meeting their recommendations. This can not be described with the type of attributes given in Table \ref{PTSP} above because we have no costs, a fix number of workers and no unallocated tasks when a feasible solution is found. 

Different variants of PTSP are given names in the literature. One example is when the shifts and qualifications are identical (S=I and Q=I) and the objective function is to minimize the number of workers that are used (O=W). This variant, PTSP[F;I;I;W], has been published as the \textit{"fixed job schedule problem"} and is described in Section \ref{FJSP} below.

\subsection{Applications}
This type of problem can be found when developing a rostering solution for ground personnel at an airport. Such a problem can be dealt with by first assigning workers to days to satisfy all the labour constraints, followed by assigning the tasks to the scheduled workers.

Similar problems of type PTSP related to airplanes can also be found when scheduling for either airport mainteance staff (leading to either PTSP[F;I;H;U-A] or PTSP[F;I-U;H;W]), planes to gates or staff that do not stay in one location, such as airline stewards. 

Another application, which has been frequently studied, can be found in classroom assignments. Based on demands such as the amount of students in a class or the duration of the class, different classrooms have to be considered. Requirements of certain equipment, e.g. for a laboratory, may also greatly limit the available rooms to choose from.

For classroom assignment there are no start or end times for the shifts, as they represent the rooms. The aim would be to find a feasible assignment of classrooms and therefore the type of problem would be PTSP[S;I;H;F] with the possibility of adding preferences to the objective function. An example of a preference would be to assign the lessons as close to each other as possible on a day, preventing travel distances between classes for teachers and students.



Papers of interest:
"The Personnel Task Scheduling Problem", Mohan Krishnamoorthy, Andreas T. Ernst (2001) - probably the most fundamental article

"Task assignement for maintenance personnel": Roberts and Escudero, 1983a, 1983b

"A stochastic programming model for scheduling maintenance personnel" Duffuaa and Al-Sultan, 1999

\section{Shift minimisation personnel task scheduling problem (SMPTSP)}\label{SMTSP}
Claes

Difference: "The only cost incurred is due to the number of personnel (shifts) that are used."

Papers of interest:
"Algorithms for large scale Shift Minimisation Personnel Task Scheduling Problems" Krishnamoorthy, Ernst, Baatar (2011)

"The shift minimisation personnel task scheduling problem: A new hybrid approach and computational insights" Smet, Wauters, Mihaylov, Berghe (2014)

"Fast local search and guided local search and their application to British Telecom's workforce scheduling problem" Tsang and Voudouris, 1997 - also with travelling costs, investigates two methods.

"A Triplet-Based Exact Method for the Shift Minimisation Personnel Task Scheduling Problem" Baatar et al., 2015

\section{Work load allocation and worker satisfaction} \label{WLA}
Emelie

Trötthet och uttråkad. Något vi borde ta med i litteraturen enligt Torbjörn, fast inte leta källor på det.

Source: "Employee positioning and workload allocation", Eiselt, Marianov, 2006
"Scheduling part-time and mixed-skilled workers to maximize employee satisfaction" Mohammad Akbari 2012

"Scheduling part-time personnel with availability restrictions and preferences to maximize employee satisfaction" Srimathy Mohan 2008

\section{Fixed/flexible job scheduling problem (FJSP)}\label{FJSP}
Maybe...

Identical skill of the workers/machines and indentical skill requirements of the operations to execute.

Problem defined in: "Algorithms for large scale Shift Minimisation Personnel Task Scheduling Problems" M. Krishnamoorthy
http://www.sciencedirect.com/science/article/pii/S0377221711010435

Problem: "A metaheuristic for the fixed job scheduling problem under spread time constraints" André Rossi, http://www.sciencedirect.com/science/article/pii/S0305054809002251 (Fixed job)

Cemal Özgüven
http://www.sciencedirect.com/science/article/pii/S0307904X11004173 (Flexible job)
Processors with a ready time, due date etc. (??)

\section{Methods}
\subsection{TSP with inhom workforce}

Solution methods to compare (similar problems):

"Task assignment and tour scheduling": Loucks and Jacobs, 1991


"Scheduling Restaurant Workers to Minimize Labor Cost and Meet Service Standards" Choi, Hwang and Park, 2009

"An integer linear programming-based heuristic for scheduling heterogeneous, part-time service employees" Heterogenous work force, tour scheduling. Using two objective functions Hojati and Patil, 2010

for another definition as PTSP[F;I;I;W], see "The Personnel Task Scheduling Problem" by Krishnamoorty and Ernst, 2001



