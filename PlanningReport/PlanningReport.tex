\documentclass{article}
\pagestyle{headings}

% some of the packages used are
%\usepackage[utf8]{inputenc}
\usepackage{latexsym}
\usepackage{eepic}
\usepackage{makeidx}
\usepackage[dvips]{graphicx}
\usepackage[utf8]{inputenc}  %för ä ö å ?
\usepackage[english]{babel}
%\usepackage{times}
\usepackage{amssymb}
\usepackage{fancybox} 
\usepackage{textcomp}
\usepackage{float}

% the fancy header/footer
% consult http://research.cmis.csiro.au/gjw/tex/docs/fancyhdr.pdf
% or some other fancy header documenatation for more info
\usepackage{fancyhdr}
\pagestyle{fancy}




\usepackage{blindtext}
\usepackage[utf8]{inputenc}
 
\title{Scheduling of library staff with respect to demands and staff competence}
\author{Claes Arvidson, Emelie Karlsson}
\date{\today}
\begin{document}
 
\maketitle
\pagebreak
 
\section*{Problem formulation\index{Problem formulation}}
The problem at hand requires a solution to an integer linear tour scheduling problem with a heterogeneous work force. Papers like Loucks and Jacobs, 1991 focused on a similar problem with a slight difference; the demand for staff increased as the workload grew at different times of the day. Their approach was to, at a certain hour, assign the tasks one at the time to the workers based on their qualifications and availabilities. This, however, does not have to be considered in our case since we have more of a pure task assignment problem at hand. The reason for this is because our demand for staff each day is fix.  


Examensarbetet går ut på att lägga ett arbetsschema för personalen vid Norrköpings bibliotek. Problemet går i grund och botten ut på att fylla alla uppgifter på de stationer som tillhör bibliotekets utåtriktade verksamhet (så kallade yttre tjänst) med personal av rätt kompetens samtidigt som peronalen får tid över till övriga uppgifter (så kallad inre tjänst). Schemat som tas fram ska även uppfylla de regelverk och önskemål som finns kring personalens individuella scheman, till exempel de arbetstider som ingår i de olika tjänsterna. Då det även ingår helgarbete i personalens arbetsuppgifter ska lediga dagar fördelas enligt arbetsmiljölagen gällande veckovilan.

Utöver detta ska schemat även medföra en robusthet så att störningar i den yttre tjänsten, i form av att personal blir sjuk eller uppbokad annonstädes, ska gå att avhjälpa med en reservlista. Denna reservlista består av de bibliotekarier och assistenter som är tillgängliga under arbetsdagen. 

Personalen på biblioteket är begränsad och utgör de resurser som finns att tillgå. Varje enskild personal har en uppsättning \textit{egenskaper} och \textit{kompetenser}. Kompetenser syftar på personalens förmåga att arbeta vid någon av de yttre stationerna; Expedition, Norpan, Informationsdisk, Bokbuss och Hageby samt några av de inre stationerna; inköp, katalogisering med mera. De egenskaper som identifierats hos en personal finns beskrivna i tabell \ref{int:1}. Totala arbetskraften består av 40 stycken arbetare på biblioteket.

\pagebreak
\section*{Approach\index{Approach}}
The plan for the project is to 


\end{document}

